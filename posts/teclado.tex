\subsection*{anti-yoga del teclado}
%\addcontentsline{toc}{section}{Anti-yoga del teclado}

Cuando pienso en las manos sobre las teclas imagino una exhibición de
contorsionismo, como si se jugara con los dedos al \textit{twister}, estirándose
sobre los colores lejanos desparramados sobre el suelo. Siempre falta
una distancia nueva para estirarse más, perder balance y caer dolorido.
Me acuerdo de una serie de tele en la que un hacker describe su relación
con su mascota --pero creo que en realidad se refiere a su teclado%--
: ``si no fuera por Qwerty estaría completamente solo''.
\smallskip

Ya que las teclas se vuelven algo muy íntimo y cotidiano más vale tomar
el tema bien de cerca. Yo las arranqué de su lugar \sout{original} y las cambié de %\href{http://www.dvzine.org/}{orden} 
\textit{orden}
por \texttt{dvorak}. Parece mucho más simple la
solución de poner las vocales de un lado y las consonantes del otro.
%Puedo 
Se pueden
golpear las letras con manos alternadas a cada vez. ¡Y siento que
tengo el ritmo potencial de un batero! Hace una década que hice esto
pero en este detalle no muchos confluímos. Somos animales de costumbres
y no entiendo casi nunca por qué tanto así. %\\[2\baselineskip]

\cleardoublepage
