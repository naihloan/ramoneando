\documentclass[]{article}
\usepackage{lmodern}
\usepackage{amssymb,amsmath}
\usepackage{ifxetex,ifluatex}
\usepackage{fixltx2e} % provides \textsubscript
\ifnum 0\ifxetex 1\fi\ifluatex 1\fi=0 % if pdftex
  \usepackage[T1]{fontenc}
  \usepackage[utf8]{inputenc}
\else % if luatex or xelatex
  \ifxetex
    \usepackage{mathspec}
  \else
    \usepackage{fontspec}
  \fi
  \defaultfontfeatures{Ligatures=TeX,Scale=MatchLowercase}
\fi
% use upquote if available, for straight quotes in verbatim environments
\IfFileExists{upquote.sty}{\usepackage{upquote}}{}
% use microtype if available
\IfFileExists{microtype.sty}{%
\usepackage{microtype}
\UseMicrotypeSet[protrusion]{basicmath} % disable protrusion for tt fonts
}{}
\usepackage[unicode=true]{hyperref}
\hypersetup{
            pdfborder={0 0 0},
            breaklinks=true}
\urlstyle{same}  % don't use monospace font for urls
\IfFileExists{parskip.sty}{%
\usepackage{parskip}
}{% else
\setlength{\parindent}{0pt}
\setlength{\parskip}{6pt plus 2pt minus 1pt}
}
\setlength{\emergencystretch}{3em}  % prevent overfull lines
\providecommand{\tightlist}{%
  \setlength{\itemsep}{0pt}\setlength{\parskip}{0pt}}
\setcounter{secnumdepth}{0}
% Redefines (sub)paragraphs to behave more like sections
\ifx\paragraph\undefined\else
\let\oldparagraph\paragraph
\renewcommand{\paragraph}[1]{\oldparagraph{#1}\mbox{}}
\fi
\ifx\subparagraph\undefined\else
\let\oldsubparagraph\subparagraph
\renewcommand{\subparagraph}[1]{\oldsubparagraph{#1}\mbox{}}
\fi

\date{}

\begin{document}

\section{Entre Todos Hacemos Tu
Carroza}\label{entre-todos-hacemos-tu-carroza}

\subsection{La Fiesta Nacional de los
Estudiantes}\label{la-fiesta-nacional-de-los-estudiantes}

\begin{quote}
\emph{``Qué mejor que los jóvenes para que nos marquen el camino hacia
una nueva era y una nueva forma de hacer las cosas. Son justamente
ellos, seres idealistas, rebeldes por naturaleza y propulsores
apasionados del cambio los que podrán darle siempre un nuevo sentido a
nuestra querida Fiesta. Para continuarla y cuidarla pero también
enaltecerla y honrarla.'' -} \textbf{Lucy Vilte}
\end{quote}

Cuando llega septiembre, todo se viste de primavera, pero es en Jujuy en
donde esta estación llega no sólo a colorear el paisaje con los lapachos
multicolores en flor, sino que desde hace más de 60 años, la primavera
se vive en Jujuy muy intensamente.

Este marcado acontecimiento, lo viven fundamentalmente los adolescentes
de la provincia, y del país, en lo que se llama \emph{la Fiesta Nacional
de Los Estudiantes {[}FNE{]}}. Como su nombre lo indica, es una fiesta y
es de los estudiantes secundarios de todos los colegios de la provincia
y también de los estudiantes del país que envían representantes durante
el mes de septiembre para los acontecimientos que en Jujuy suceden.

La FNE nació en los 1950' y desde ese momento solo ha crecido en
dimensión e impacto en los estudiantes.

Considerando el año de su nacimiento, hoy podemos decir que \emph{todos
los jujeños} adultos la hemos vivido y acompañamos ahora a nuestros
hijos, sobrinos, vecinos a que la vivan alegremente.

Bien, ¿de qué se trata la FNE? Comienza oficialmente en la segunda
quincena de septiembre con múltiples eventos, pero los estudiantes de la
provincia se preparan para ella durante todo el año.

El principal evento es el \emph{Desfile de Carrozas:} Cada colegio
construye una carroza alegórica a un tema de su interés. Hay carrozas
técnicas (con movimiento) y no técnicas. El desfile se realiza en la
segunda quincena de septiembre y se les da un puntaje a cada una, en
función de ciertos parámetros. Hay premiación para todos los colegios.

Es en el Desfile de Carrozas en donde se desarrolla esta experiencia de
\emph{triple impacto}. Históricamente construidas con estructuras de
hierro, forradas en tela y cubiertas con flores de papel crepe y
celofán, iluminadas con un laberinto de circuitos eléctricos con focos
incandescentes; las carrozas \emph{¡son obras de arte!} Los alumnos allí
aprenden de \emph{manualidades} (realización de tipos diferentes de
flores), \emph{herrería} (soldar, empalmar, doblar, hierros) y
\emph{electricidad} (confección de largas líneas de iluminación y
construcción del tablero para el control del ``juego'' de luces).

Desde siempre el dinero para su construcción se reúne con eventos como
\emph{El Baile de la Elección de Reinas}, venta de comida, concursos que
realizan algunas empresas que entregan premios en papel, hierro o
dinero. También el gobierno de la provincia colabora con hierro para
todos los colegios, se canjean las facturas de compras por dinero. Es
decir, todos los sectores de la sociedad se unen en pos del objetivo
común de tener y vivir \emph{la Fiesta de los Estudiantes}.

Con el correr de los años, la coyuntura, los precios y la mayor
conciencia de los jóvenes en el cuidado del ambiente; algunos colegios
comenzaron a incluir en su carroza materiales de descarte. Generalmente
cartón y PETs. No era común esto, al contrario, desde el mismo
reglamento de la construcción de carrozas, se indicaba que el 80\% de la
misma debía estar cubierta de flores de papel.

Cada vez fueron más colegios sumándose a esta ``movida'' pero sin lograr
impactar en grande a la FNE. Eran iniciativas dentro del colegio, sin
lograr trascender y movilizar a la sociedad.

Nosotras, Lucy Vilte, Iris Civardi y yo, Claudia Lamas, ya veníamos
trabajando y tratando de cambiar la FNE vinculándonos con el Ente
Autárquico {[}Ente organizador y gestor de la FNE{]} para lograr que el
cambio hacia una Fiesta más amigable con el medio ambiente surja con
ellos, derramando a todos los participantes. Así es que pasó un año y no
logramos lo que queríamos, desde la FNE: \emph{promover el cambio de
materiales en las carrozas, promover el cuidado ambiental en la sociedad
civil}.

Así fue que en julio de 2018, unos meses antes que llegara otra FNE, se
nos ocurrió una idea tan simple, que de tan simple es \emph{efectiva}.
Creamos \href{https://www.facebook.com/groups/234634360665192/}{un grupo
de Facebook}: \emph{Entre Todos hacemos tu carroza}. Y declaramos el uso
que podían darle al mismo con esta breve descripción:

\begin{quote}
\emph{Si sos carrocero --\textgreater{} Posteá que materiales
reciclables }necesitas* para tu carroza.\emph{ }Si sos
empresa/organismo/particular --\textgreater{} Posteá que materiales
\emph{tenes} para descartar y ayudar en la construcción de una carroza.*
\end{quote}

Lo compartimos a través de nuestras propias redes, comenzamos a buscar
otros grupos donde hubieren estudiantes para lograr \emph{atraerlos}
hacia el nuestro. ``Stalkeamos'' a periodistas locales para que
\href{https://www.eltribuno.com/jujuy/nota/2018-8-12-20-14-0-impulsan-el-uso-de-reciclado-en-carrozas}{den
a conocer la iniciativa}, buscamos a profesores de los colegios para que
conozcan el grupo. ¡Muchos de ellos no tenían ni un perfil de Facebook!.
Preguntamos en los colegios qué necesitaban para su carroza. Lo
posteamos nosotros y así el grupo comenzó a moverse, a crecer, a
autogestionarse, los integrantes a entender de qué se trataba esto de
\emph{ayudar a la construcción de una carroza con ``la basura''}. Así,
los mismos padres de los carroceros, comenzaron a postear lo que
necesitaban los colegios. Y así logramos ``mover'' a todos los sectores
en pos del mismo objetivo: \emph{vivir la FNE cuidando el ambiente.}

\begin{itemize}
\item
  El impacto es ambiental. Kilos y Kilos de descarte (Papel, cartón,
  latas, PETs, Plásticos inflados, tapitas, latinchas, rollos de cartón,
  medias de nylon, bijouterie, CDs/DVDS, envolturas de golosinas -de
  celofán o plateadas, espejos rotos, y la lista es muy larga) \emph{no
  llegan a los basurales, se transforman en obras de arte.}
\item
  El impacto es económico. Este año un diario local muy importante hizo
  una investigación del impacto del reciclaje en el costo de las
  carrozas y obtuvo como resultado que el ahorro aproximado es del 40\%
  del costo que es arriba de los 200 mil pesos.
\item
  El impacto es social. A través del grupo se fomenta que
  \emph{particulares separen los residuos} con el fin de ayudar a un
  colegio. Se sumaron hasta el momento casi mil personas al grupo, que
  postean qué es lo que tienen para dar a algún colegio. Esto permite
  obtener \emph{volumen}, ya que para el reciclaje sabemos que se
  necesitan grandes cantidades de productos para transformarlos y crear
  uno nuevo.
\end{itemize}

El grupo obtuvo buena repercusión local y nacional, sigue creciendo.
Este año se sumaron varias empresas dispuestas a separar el descarte y
entregarlo al colegio que lo necesite.

Los chicos hoy no solo aprenden a hacer flores con papel sino con latas,
con rollitos de carton, con medias de nylon, con hilo de PET. Aprenden
diferentes técnicas porque los materiales con los que que trabajan son
variados.

Queremos lograr basura cero en la construcción y sobre todo
``destrucción'' de las carrozas. Este año también surgió la posibilidad
de que una asociación civil recibiera ladrillos ecológicos para
remodelación de plazas y espacios públicos y estamos invitando a los
colegios a que se sumen a la iniciativa.

Averiguamos sobre la construcción con bambú para lograr reemplazar parte
de la estructura de hierro por este noble material. Es un camino a
recorrer.

Hoy existe un premio a ``la Carroza más sustentable''. Desde el año
pasado se entregó y lo ganó una escuela del interior de la Puna, que usó
materiales como lana de llama, oveja, cáscaras de naranjas en su
carruaje.

Hay mucho para hacer, pero la semilla ya está puesta. Hay que cuidarla
cada día del año para que florezca y multiplique en la Primavera de
Jujuy.

\end{document}
