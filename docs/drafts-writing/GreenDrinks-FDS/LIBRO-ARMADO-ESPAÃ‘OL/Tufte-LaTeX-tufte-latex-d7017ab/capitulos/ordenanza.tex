  \end{fullwidth} 

\chapter{DE UN TRABAJO PRÁCTICO A \\ UNA ORDENANZA MUNICIPAL}\label{de-un-trabajo-pruxe1ctico-a-una-ordenanza-municipal}
  \begin{fullwidth}

% \begin{figure}[htbp]
% \centering
% \includegraphics[width=1.77292in,height=1.76250in]{media/image10.jpeg}
% \caption{}
% \end{figure}

\textbf{Matías Roldán }

Técnico Superior en Gestión Ambiental del Colegio Universitario IES.
Actual docente de la misma carrera, en la Cátedra de Evaluación de
Impacto Ambiental, Sistemas de Gestión ISO 14000 entre otras. Asesor
Empresarial y de organismos públicos. Actualmente se especializa en
comunicación ambiental empresaria y RSE.

Proyecto Consumo Responsable de Bolsas Plásticas Córdoba

~

\emph{Apodado por algunos como ``el chico de las bolsas'', nunca pensé
que un trabajo práctico pensado para una materia o como temática pensada
para mi tesis de tecnicatura pudiese terminar como una ordenanza modelo
que inspiró a ciudades de la provincia de Córdoba, otras provincias y/o
ciudades de países vecinos, a implementar el mismo modelo de proyecto.
Desde el año 2010, inicié un relevamiento y propuesta de normativa que
regule el consumo de bolsas plásticas en Córdoba y 5 años después, de la
mano de un proceso consensuado entre distintos actores basados en mi
proyecto, se logró que el consumo disminuyera casi en un 90\%.}

\textbf{Embolsando la idea}

Agosto del año 2010, todavía en papel de alumno de la carrera de Gestión
Ambiental del Colegio Universitario IES, buscaba una temática que
sirviera para un trabajo práctico de una materia que me formó en
planificación estratégica, y como siempre me llamaban la atención las
temáticas complejas, elegí el planificar la gestión de un basural
informal que sirviera como modelo para otros basurales, problemática que
contempla muchas variables y elementos específicos.

Dentro del tema elegido, quedó una arista no menor que involucraba uno
de los elementos que desde la década el 70, está presente entre nosotros
asociado directamente a todas las cadenas de supermercados que luego
fueron seguidas por los comercios en general. Un elemento que hace casi
10 años a la fecha transitaban la misma exigencia por parte de los
clientes ante locales comerciales que hoy tiene el servicio de WIFI,
estamos hablando de los MILLONES DE BOLSAS PLÁSTICAS (llamadas CAMISETA)
que los cordobeses ``utilizábamos'' como contención de residuos. Si bien
la arista no fue tratada en el trabajo práctico puntual, la inquietud
gestó en mí, interrogantes que dieron lugar posteriormente a un
proyecto.

~

\textbf{Involucrando a las bolsas}

Al año siguiente (2011), se decide proseguir con el análisis de la
inquietud definiendo que dicha temática debería ser abordada en el
trabajo final (tesis) para obtener mi título o como en un proyecto
personal. Ambas ideas tenían el mismo objeto, involucrar a las bolsas
camiseta, aquel elemento abundante en el ciclo de residuos que escapaba
a una mirada social crítica, y su consumo intentando comprender ¿cuál
era la problemática real que presentaban? aparte de los impactos
visuales y ambientales.

Lo que sigue a continuación son las etapas que dicho proyecto siguió,
desde el 2011 hasta el 2016, donde implementado, redujo el consumo de
bolsas camiseta entregadas por cadenas de supermercados en más de un
86\%.

\textbf{¿Cuál es el problema?}

Al inicio lo importante fue descubrir, luego de mucho analizar, que era
un error pensar que la fabricación o entrega de bolsas camiseta eran
todo el problema\emph{.}

``Cuando la formulación de la problemática se vuelve confusa, se debe
procurar comenzar a involucrar a los actores de la temática\ldots{}''
decía una docente de la carrera que cursé, frase que comenzó a resonar
en un punto donde la hipótesis era ``el problema con las bolsas camiseta
es generado por los supermercados, ya que no controlan la cantidad
entregada''\ldots{} teoría simple, pero no por ello acertada.

Es más, eso tampoco definía un problema, sino que expresaba una opinión
y dejaba más dudas que certezas ¿Las bolsas contaminan? ¿Es mucho el
plástico que se usa para fabricarlas? ¿Cuánto demoran en degradarse?
¿Otras ciudades usan más o menos que nosotros? ¿La normativa que exige
que sean OXIDEGRADABLES, como se controla?

Todos estos interrogantes que se desarrollaron a lo largo de casi 10
meses, entre investigaciones y relevamientos sin resultados, llevaron a
un acto puntual, se decidió contactar en forma directa a quienes
parecían ser los responsables del tema, los supermercados representados
en la Cámara de Supermercados y Autoservicios de Córdoba -- CASAC.

En un primer momento se pensó que el total de las respuestas, sumado a
una obvia evasión de darlas, iban a estar tras esa puerta vidriada de la
oficina de dicha cámara empresaria en pleno centro cordobés, pero no fue
así. Por lo contrario, muchas de las preguntas que hice, muy amablemente
Luciana y Valeria --encargadas de CASAC quienes serían a futuro,
compañeras de trabajo-- pudieron responder y también replantear otras,
lo que generó una nueva percepción del problema y al a vez una reacción
casi adrenalínica, y que en este caso iba a ser el puntapié de una
oportunidad única\ldots{} junté coraje y con voz temblorosa pero segura,
les propuse tomar el problema como trabajo final, a los 15 minutos de la
reunión que nos convocó por primera vez.

Sorprendidas por la propuesta pero elogiando de alguna forma la
proactividad, quedaron en comunicarme la decisión que la presidencia de
CASAC diera, y al cabo de unas semanas de aquella propuesta arrojada
dieron el visto bueno para hacerlo, y varios pasos después, (aclarando
que la temática no podía ser tomada como tesis, pero sí como un proyecto
asociado a una pasantía o práctica profesional), inició el proceso de
acompañamiento mutuo que derivó en mi primer experiencia laboral formal
por más de 4 años. El haber encarado a una organización similar, y el
haber propuesto iniciar un proyecto sumamente ambicioso, creo que fue EL
paso que dio nacimiento al proyecto.

~

\textbf{Objetivo concreto}

Febrero del 2012, inició la pasantía donde se plantearon objetivos
concretos ``Queremos disminuir la entrega de bolsas que la gente nos
exige. Sabemos que contaminan y que somos de alguna forma los
protagonistas del cuento y queremos que nos ayudes a proponer un
proyecto para un consumo responsable'' decían las autoridades de CASAC
en aquel primer día de trabajo. Los nervios, la ansiedad y la sensación
de estar en un desafío que podía superarme, estuvieron conmigo las
primeras semanas hasta que pude vislumbrar el primer obstáculo que
formaba parte principal de la cuestión y que hizo olvidarme por completo
las preocupaciones para poder ocuparme de lleno en resolverlo, la
normativa existente era imposible de aplicar.

Si la ordenanza no podría reglamentarse ni enmarcar los controles, era
peor eso a que no existiera, lo que mostró que era esencial, antes que
cualquier acción, proponer un marco regulatorio sobre el consumo de
bolsas camiseta, tarea que emprendí con dudas pero sin temor.

~

Comencé a trabajar en un proyecto de ordenanza que fue inspirado en el
espíritu de distintas normativas (de nivel provincial y nacional) y con
la ayuda del Ab. Victor Layus (un amigo cercano) quien complementó mi
escasa experiencia en la redacción de documentos legales, se llegó a un
resultado sencillo que apuntaba principalmente a que el uso de bolsa
debería ser restringido por normativa, pensando así que podría
resolverlo todo. Una percepción totalmente errónea ubicada en el
nacimiento y en los primeros pasos del proyecto.

~

\textbf{La fundamentación ambiental}

Pensando que se había logrado el mejor proyecto de la historia, al
momento de presentarlo ante las autoridades municipales, ellos invitaron
a generar cierta fundamentación ambiental del proyecto para intentar
mejorar una propuesta que a simple vista se caía por falta de contenido
y, sobre todo, por cierta ausencia de una línea de base que pudiese
fundamentar cantidades y comparaciones. Es allí donde el segundo
obstáculo (que luego fue aquello que elevó el proyecto hacia una
normativa ejemplar) se hizo presente mediante una pregunta: ¿Cómo
fundamentar el uso regulado de bolsas camiseta, si hoy se desconoce la
cantidad que se usa, y qué tan ``impactante'' es esa cantidad?

En este punto se dio un antes y un después a nivel profesional y sobre
todo, en cuanto al peso e importancia del proyecto, ya que se comenzó a
desarrollar un relevamiento durante todo el 2012 y gran parte del 2013,
en todas las cadenas asociadas a CASAC junto a la investigación de
percepción social sobre consumo general y específico (de bolsas), que
generaron resultados invaluables para éste y muchos otros proyectos,
incluso en países vecinos, ya que en ciudades donde se habían
implementado normativas asociadas al tema, nunca se había logrado un
relevamiento de consumo de bolsas camiseta, estudio de tendencias de
consumo y transporte de mercadería y hábitos sustentables. Sin hablar
del sentido total de pertenencia que uno genera para con el proyecto.

~

\textbf{El consumo}

``Se generó un relevamiento exhaustivo del consumo mensual y bimestral
--dependiendo la cadena- en conjunto con una encuesta que propuse y
realicé a un universo de casi 1000 personas para identificar la
percepción social del consumo de bolsas camiseta''.

Los resultados del relevamiento y de la encuesta fueron tanto alarmantes
como sorprendentes respectivamente:

-Las cadenas asociadas a CASAC, dispensaban 20.000.000 de bolsas
camiseta por mes.

-Las cadenas de supermercados representan el 45\% del consumo total de
bolsas camiseta, el resto son comercios en general. Es decir, en Córdoba
se dispensaba hasta 45.000.000 de bolsas al mes.

-El 95\% de esas bolsas, luego de las compras, se utilizaban
automáticamente para contener residuos. Es decir, no se reutilizaban.

-Los modelos de bolsas variaban sus tamaños según las cadenas, generando
innumerables diferencias de practicidad, contención y percepción de
eficiencia.

-Las personas utilizaban las bolsas de forma ineficiente. Poco contenido
y muchas bolsas.

-Las personas no registraban cifras de su consumo diario o mensual de
bolsas en el hogar. Mostrando un claro desinterés en hacerlo.

-Las personas no reutilizaban la bolsa camiseta para comprar nuevamente,
incluso la bolsa en sí, muchas veces luego de las compras era residuo.

-Las personas consumían muchas bolsas para luego generar stock en sus
casas, para los residuos y ahorrarse comprar bolsas de basura llamadas
bolsas de consorcio.

-Las personas no utilizaban bolsas reutilizables (de tela) para hacer
sus compras, por desinterés, precio e incluso ``por no combinar con su
ropa, el diseño de las mismas''.

\textbf{Seguir el trabajo como profesional}

Luego, en el año 2013, momento donde el proyecto de bolsas no pudo ser
involucrado en la tesis que defendí para obtener el primero de mis
títulos, se reafirmó mi interés por seguir en el trabajo propuesto de
forma no académica como primer ejercicio profesional y con la clara idea
de culminar el proyecto en forma positiva. Esto se tradujo en mi
incorporación oficial como consultor de CASAC, estrenando título,
actividad que hasta la fecha desarrollo.

Este compromiso trajo aparejado que toda la información obtenida y
generada desde años anteriores pudiese ser analizada y volcada en
distintos informes técnicos, creando así un enorme marco técnico y
teórico que aportó a que el proyecto inicial de ordenanza se viera
totalmente modificado evolucionando hacia una propuesta integral
trabajada junto con la municipalidad de Córdoba y sus autoridades de
ambiente durante todo ese año.

~

El proyecto evolucionado presentó la mejora del diseño de bolsas
enmarcado en una NORMA IRAM (Nº 13610) que implementa un modelo
estandarizado, a utilizar en todas las cadenas, de colores distintos
para complementar el sistema de recolección diferenciada de residuos y
la necesidad de una estrategia de comunicación y difusión para disminuir
el consumo de bolsas y educar a los vecinos en hábitos más sustentables.

Propuso la implementación de un plan de reducción de consumo de bolsas
en 4 etapas (aviso -- sustitución de modelo de bolsa a estandarizada en
supermercados -- sustitución en locales comerciales en general --
reducción total), sumó plan de educación, difusión y promoción de la
ordenanza en conjunto con el seguimiento de consumos.

~

\textbf{Hacia una propuesta formal}

La evolución del proyecto y la revisión conjunta con la municipalidad
acerca de las modificaciones, generaron que en el año 2014 se creara un
espacio multisectorial de análisis, discusión y puesta en marcha de un
Proyecto de normativa consensuado entre los actores realmente
involucrados en el tema. Se tomó como base mi proyecto, y se convirtió
la propuesta que presenté años atrás, en un proyecto de ordenanza que
abordaba una problemática que comenzaba a tomar protagonismo en los
medios y en la percepción social.

~

Dicho espacio creado para la discusión de proyectos similares se llamó
``Espacio para la Construcción de Ciudadanía Ambiental'' (ECCA) y fue
aquí donde la humilde propuesta, respaldada por casi 3 años de
relevamientos e investigación, fue sometida a análisis y discusión por
parte de representantes del Instituto Nacional de Tecnología Industrial
INTI, cámaras de productores plásticos de Córdoba y del País,
autoridades municipales en materia ambiental, agrupación de recicladores
provinciales, organizaciones del ámbito civil y por supuesto, el autor
del proyecto representando a CASAC, entre otros.

Durante los dos años que duró el proceso de discusión y consenso (2014 a
2016) jamás dudé del objetivo y espíritu original del proyecto, y aunque
muchas veces comentar a modo técnico y anecdótico toda la información
relevada fue causa de observaciones, objeciones y hasta cierta
descalificación de algunos participantes con intereses puntuales, nunca
se vieron afectadas mis ganas de seguir trabajando.

Fueron dos años arduos a nivel profesional y proyectual, donde TODO lo
que aprendí, aportó de una forma inimaginable a mi rol como futuro
profesional en ambiente, a mi rol de consultor, representante de
empresas, vecino y/o posible promotor de hábitos sustentables
compartiendo exposiciones, modificaciones y justificaciones de un
proyecto que se fue puliendo de a poco, para terminar siendo una
ordenanza.

~

\textbf{La Ordenanza}

Como conclusión del proyecto, a fines del 2015 es presentado ante el
honorable Concejo Deliberante de la ciudad de Córdoba mediante el
espacio ECCA, quien aprueba la Ordenanza N° 12.415, que implementa el
Programa de Reducción y Uso Responsable de Bolsas Camiseta que logró
durante su primer año de vida (Febrero 2016 a Febrero 2017) la reducción
de cerca del 90\% en el consumo de bolsas por parte de los vecinos.

Así, luego de que todo naciera desde un interrogante y la falta de un
marco legal eficiente, en un muy breve resumen, tengo el gusto de contar
cómo este proyecto, acompañado de tenacidad, proactividad y sin miedo a
seguir intentando, logró ser tenido en cuenta por ciudades como Santa
Rosa de La Pampa, Rosario, Salta, Bahía Blanca, países limítrofes como
Chile y Uruguay, utilizándolo como herramienta de acción y gestión
sustentable en una problemática tan especial, como lo es el uso de
plástico asociado al consumo responsable de elementos que puedan generar
impactos ambientales. Originalmente fue pensado como un marco legal que
nos ayude a ser más responsables con nuestros impactos derivados de
nuestro consumo.

No duden, no dejen de proponer ni de pensar cómo, desde sus casa,
barrios, trabajo, espacios de formación o propios, pueden iniciar
proyectos mil veces más innovadores, sustentables y mejores para nuestra
sociedad y planeta.