\chapter{LA SUSTENTABILIDAD ES
REBELDÍA}\label{la-sustentabilidad-es-rebelduxeda}

% \begin{figure}[htbp]
% \centering
% \includegraphics[width=1.76875in,height=1.76250in]{media/image8.jpeg}
% \caption{}
% \end{figure}

\textbf{Fabián Gabriel Fábrega}

MAGÍSTER - Fundador y director de ``El Castillo Fábrega Organizational
Center''- Docente y orador sobre ``Innovación y Desarrollo
Sustentable''. Por su trayectoria la Universidad Jesuita de Nueva York
lo designó Fellow Member del ``Fordham Consortium on the Purpose of
Business'', y ha sido citado como ejemplo de pensamiento innovador en
libros de editoriales nacionales e internacionales.

\emph{¡Una oportunidad imperdible de construir un mundo propio!}

\textbf{La sustentabilidad es rebeldía}

El Castillo Hotel Fábrega Organizational Center

~

``Hacen todo en el momento equivocado, en el lugar equivocado y de la
forma equivocada'' -- nos dijeron en el año 2002.

Porque, durante la peor crisis económica de la historia Argentina, nos
mudamos desde Buenos Aires y Nueva York a la zona más pobre de las
sierras cordobesas, para vivir en un castillo destruido que habíamos
comprado vendiendo nuestras propias casas.

Nosotros --papá, mamá, hermano mayor, hermana menor y yo-- siempre
soñamos crear un entorno que tuviera todos los ingredientes de nuestra
vida rebelde: arte, deporte, educación, historia y emprendimiento. Aquel
castillo sería el soporte físico. El servicio de hospitalidad sería el
medio. ¿Un hotel? Sí\ldots{} ¡pero un hotel muy distinto! Aunque para
ello no teníamos dinero suficiente, ni personal calificado, ni
conocimiento sobre turismo. Teníamos, sin embargo, nuestras profesiones
en medicina, ingeniería, economía, y una vasta experiencia emprendedora.
También teníamos pasión rebelde. ¡Y mucha!

Bueno, no quisiera arruinar la posibilidad de incluir algo de tensión o
suspenso contando el final de esta historia, pero mejor les cuento: la
nuestra es una historia de rebeldía con final feliz.

Junto al castillo, a un grupo de personas locales, a una visión
sustentable y a días y noches de trabajo interminable, nos embarcamos en
una aventura que cambió nuestras vidas. Una aventura que nos llevó a
alcanzar horizontes que nadie podría haber imaginado. Hoy, quince años
más tarde, vemos al castillo convertido en el primer y único hotel cinco
estrellas de Argentina fuera de una ciudad de destino internacional, y
el único que diseña y dicta programas originales de ciencia y arte.
Vemos a nuestra familia extendida --formada por la familia nuclear y el
mismo grupo inicial de personas locales-- afirmando que la rebeldía es
una profunda manera de ser felices. Quince años más tarde, nuestra forma
de ser y hacer está apoyada por Instituciones que nos han otorgado
múltiples premios y honores, incluyendo publicaciones que describen
nuestro emprendimiento como un ejemplo sin igual en el mundo.

Que el final --es decir, cronológicamente, el presente-- de esta
historia sea feliz, no significa que haya estado carente de infortunios,
conflictos, decepciones, tristezas (muy, muy grandes tristezas,
créanme)\ldots{} Pero preferimos enfocarnos únicamente en los aspectos
positivos. ¿Nos acompañan entonces en la historia de esta aventura?
Hagamos un viaje en el tiempo\ldots{} y aprovechemos para reflexionar
sobre las etapas de nuestra vida.

\textbf{Los comienzos}

Cuando rememoramos los comienzos de El Castillo, en 2002, aparecen las
opiniones de la gente. A quienes emprendemos, la gente nos dice qué y
cómo debemos trabajar, siempre guiados por lo que hace la
mayoría\ldots{} ¿Qué cosa, no? Eso: que la gente piense y diga que uno
debe hacer lo que hace la mayoría. Sí, es así. Y no sólo en lo que se
refiere al trabajo. Cuando uno es estudiante universitario, tiene que
equilibrar las horas de estudio con las de encuentros sociales, tal como
la mayoría lo hace. Si no, uno no sabe vivir la universidad. Cuando uno
es adolescente, tiene que salir de juerga, tal como la mayoría lo hace.
Si no, uno no sabe vivir la adolescencia. Cuando uno es niño, tiene que
participar en competencias deportivas, tal como la mayoría lo hace. Si
no, uno no sabe vivir la niñez.

Por suerte nuestros padres nos enseñaron a ser rebeldes. No sabemos si
intencionalmente o no, pero nos parece que siempre hicimos lo que la
mayoría no. ¿No se trata de eso la rebeldía? En la universidad,
estudiábamos como animales (mucho, quiero decir); en la adolescencia,
salíamos tanto como los esquimales (poco, quiero decir, aunque nunca
conocí a un esquimal); y en la niñez, competíamos tanto como una cuchara
(nada\ldots{} es que estoy tomando un té y no se me ocurren símiles
mejores). Ojo, ¡que nos encanta el deporte! Siempre entrenamos en varios
deportes, pero jamás nos gustó competir. Y nos gusta también compartir
momentos con personas, pero sin actitud indiferente. Y cuando
estudiamos, estudiamos, y cuando nos divertimos, nos divertimos.

Para recordar mejor, viajemos más atrás, por un momento, a la década de
los ochenta y de los noventa. Desde los primeros años de la escuela
primaria, con mis hermanos formamos una banda de música. Pero no de la
música que escuchaba la mayoría. No. Tocábamos folklore. Pero no el
folklore que le gustaba a la mayoría. No. Tocábamos y componíamos
folklore con instrumentos y arreglos de jazz y rock. Por la música no
relegábamos horas de escuela ni de entrenamiento deportivo, pero eso no
quiere decir que la música fuera un simple pasatiempo. Todo lo hacíamos
en serio. ¡Muy en serio! Nos creíamos adultos cuando éramos niños y
ancianos cuando adultos\ldots{} o algo así. Con nuestra banda llegamos a
grabar discos junto a artistas del nivel del pianista Juan Carlos
Cirigliano (Astor Piazzolla, Chick Corea\ldots{} por nombrar sólo dos de
sus trabajos) y el saxofonista Hugo Pierre (¡con quién no tocó el gran
maestro Pierre!), y a compartir obras con varios más. Experiencias
similares vivimos con otras disciplinas, como las artes plásticas, la
fotografía y la danza.

\textbf{Nuestros padres}

Antes de seguir, a no olvidarnos: nuestros padres. ¿Vieron que hay
padres que opinan que a los hijos no hay que regalarles tantas cosas,
porque sino no aprenderán a valorarlas, y que los hijos tienen que
aprender a ganar su propio dinero desde chicos, porque sino no sabrán ni
cómo ganarlo ni qué hacer cuando lo tengan, y que los hijos tienen que
estar obligados a continuar la misma profesión que sus progenitores u
obligados a no hacerlo? Pues, nuestros padres opinan lo contrario. Todo,
pero todo lo que se nos ocurrió pedirles, y más, nos lo regalaron (aún
arriesgando su situación económica). Y nos invitaron a trabajar con
ellos, escuchando nuestras ideas, desde que aprendimos a andar en
bicicleta. Y nos ofrecieron elegir la profesión que quisiéramos, en el
lugar del mundo que quisiéramos. Por eso, cuando llegó la hora de elegir
universidad, decidimos quedarnos en la ciudad en la que vivíamos e
inscribirnos en la Universidad Nacional de San Luis. En gran parte
porque no queríamos separarnos de nuestra pequeña familia. Y un poco
también para rebelarnos contra la mayoría de nuestros compañeros que se
iría a Córdoba o Buenos Aires en busca de ``universidades que tengan
nivel''. Nosotros: rebeldes, muy rebeldes. Creemos que siendo
perfeccionistas nos rebelamos contra la resignación. Siendo creativos
nos rebelamos contra estandarización. Siendo trabajadores nos rebelamos
contra la especulación.

Y ahora viajemos al final de los noventa y comienzo de los dos mil,
justo antes de empezar a proyectar El Castillo. Resulta que, siete días
después de rendir la tesis de la carrera de grado, nos mudamos con mi
hermano a los Estados Unidos, becados como Investigadores Académicos de
la División de Medicina Molecular del Departamento de Salud del Estado
de Nueva York (¡qué espectacular el nombre del cargo! ¿no?). Allí,
además, estudiamos, también becados, en las universidades SUNY at Albany
y Rensselaer Politechnic Institute. Esta última, la universidad de
ingeniería más antigua del mundo de habla inglesa, dio inventos que
moldearon los últimos dos siglos del planeta: la TV, la radio, la
computadora personal, la cámara digital, el e-mail, los proyectos
espaciales Gemini y Apollo, entre tantísimos otros. Mencionamos a
Rensselaer en particular, porque, además de admirarla muchísimo,
encontramos allí un refuerzo al concepto de rebeldía que nos inculcaron
nuestros padres. Y encontramos allí personas que valoran a los rebeldes
perfeccionistas, creativos y trabajadores. Era un lugar soñado. Pero
faltaba el resto de la familia. Nos extrañábamos. ¿La excusa para volver
a estar juntos? ¡Un castillo destruido!

\textbf{De regreso}

Ahora sí, de regreso al año 2002. Una crisis económica fenomenal, un
castillo destruido, un lugar empobrecido y desconocido, nosotros cinco
solos\ldots{} ¡Unas ganas tremendas de ser rebeldes y crear y trabajar!
¡Una oportunidad imperdible de construir un mundo propio! ¡Muchísimo
tiempo para estar en familia! ¡Muchísimo espacio para instalar, ya que
estábamos, una sala de ensayos para tocar música, un atelier de arte
para pintar, una cancha de tenis para jugar, aulas y talleres para
enseñar! Bueno, pero\ldots{} ¿y el estudio de factibilidad? Al fin y al
cabo, íbamos a ser una empresa que debía recuperar la inversión ¿o no?
¿Y el personal calificado? Si diseñábamos un hotel de la más alta
calidad, la fuerza laboral debía saber exactamente cómo trabajar ¿o no?
¿Y el estudio de mercado? Si nos metíamos en una de las industrias más
competitivas que existen, debíamos conocer qué hacen los competidores ¿o
no? No. ¡No! Nada de eso, no señor; eso es lo que piensa y hace la
mayoría, ¡y nosotros somos rebeldes!

\textbf{Una empresa sustentable}

Frente a todos los malos pronósticos --por pésima ubicación, marca
inexistente, un edificio antiguo y deteriorado, infraestructura local
escasa, red de ventas nula, desconocimiento de la industria, personal no
especializado-- siempre tuvimos muy claro cuál era el propósito de
nuestro proyecto. Crear una empresa familiar sustentable. Una empresa
familiar, porque queríamos estar en familia. Sustentable, porque ese era
el modo en que gestionaban nuestros padres y abuelos (antes, cuando no
existía el término ``sustentable'', ellos decían ``hacer las cosas
bien''). Y rebelarnos a aquellas preguntas tan mayoritarias. No, no
íbamos a pensar en el futuro. No haríamos estudios de factibilidad ni de
mercado. ¿Es posible que nuestro deseo de gestionar sustentablemente no
esté determinado por los resultados futuros? Porque eso significaría que
no tenemos muchos argumentos para crear y gestionar una empresa, ¿o no?
¡Claro que no! Nosotros podemos gestionar sustentablemente sin estar
motivados por los beneficios potenciales. Y podemos gestionar
sustentablemente sin tener otra respuesta más que el genuino ``deber
ser''.

¿Recorramos velozmente los quince años desde aquel 2002? Sí. ¿Cómo
revivimos el castillo? ¿Cómo formamos un equipo de trabajo? ¿Cómo
diseñamos un servicio hotelero distinto? Ahora les cuento.

Bajo la idea, planificación, organización y dirección de mi hermano
Edgardo, restauramos el castillo con pautas ecológicas muy
vanguardistas. Él estudió la incidencia de la luz natural, las
corrientes de aire, las napas subterráneas, la distribución de los
ambientes, el ciclo de vida de los materiales constructivos originales y
mil cosas más, y, consecuentemente, dispuso nuevos cableados,
dispositivos de acondicionamiento de aire, cañerías de agua, estructuras
y mil cosas más. Edgardo nos enseñó que es imposible pensar por separado
cada una de esas mil cosas, ya que ellas conforman un sistema
energético. Y ese sistema energético garantiza un consumo anual tres
veces menor al de un sistema convencional. ¡Tres veces menor! ¡Es una
barbaridad de ahorro energético! Les aseguro que es apasionante,
realmente apasionante escuchar a mi hermano, y ver cómo le ha devuelto
la vida a un edificio tan antiguo y tan imponente. Casi me olvido: ¡son
ocho mil metros cuadrados cubiertos, y cinco hectáreas de parques!

El equipo de trabajo. Difícil formar un equipo, ¿no? Provoca risa cuando
a mi mamá, felicitándola por mi hermana Adriana, le dicen ``¡oh, qué
suerte que tuviste vos, que te saliera una hija tan buena, tan
estudiosa, tan trabajadora, tan artista\ldots{} es un amor!''. Claro,
como si mi mamá no tuviera nada que ver con que Adriana fuera como es.
¡Qué cosa si a mi mamá le tocaba otra hija! Estoy siendo sarcástico,
para, con esta analogía, decir que no es cuestión de suerte estar
rodeado de personas buenas. Vivir entre buena gente conlleva un esfuerzo
descomunal, una entrega total de amor y de tiempo. Así, con ese
esfuerzo, amor y tiempo, mamá y Adriana formaron a nuestro equipo de
trabajo. Equipo compuesto por personas del lugar, quienes, al momento de
conocernos, estaban desocupadas y sin formación en ningún oficio
relacionado con la cultura del buen servicio. Hoy son parte de lo que
orgullosamente llamamos nuestra familia extendida. Y saben y hacen cada
una de las tareas que llevamos a cabo en El Castillo. Todos saben y
hacen todo con dedicación y perfección.

\textbf{Un hotel distinto}

¿Y el hotel? Distinto, sí. Pero ¿qué quiere decir distinto? ¿Sería como
crear un nuevo mercado? ¿Podemos hacer tal cosa? ¡Sí! Podemos crear
nuevos mercados. ¿Cómo? Bueno, eso es más difícil responder. Ya vimos
que hay quienes hablan de estrategias comerciales, planes de negocios,
segmentación\ldots{} Nosotros hablamos de hacer lo que nos apasiona y
ofrecerlo a quienes quieran disfrutarlo. En El Castillo diseñamos
programas para empresas y universidades que combinan seminarios sobre
eficiencia energética y capital humano,~con clases de música, pintura y
cocina. Diseñamos programas para estudiantes de danza clásica que
combinan clases de danza con momentos de amistad. Diseñamos programas
para familias que combinan talleres artísticos, actividades deportivas y
conversaciones felices. Y nos basamos en el amor familiar y en la
inmensa vocación~que sentimos por servir a las personas. Así de distinto
es nuestro hotel.

\textbf{Frases}

Hace unos días nos invitaron a compartir nuestra experiencia empresarial
en~el \emph{workshop} del IAE Business School: ``Cómo crear nuevos
mercados con estrategias innovadoras''.~Los profesores Luis Dambra y
Patricio Guitart explicaron conceptos de innovación disruptiva,
tendencias globales y~la estrategia de \emph{BlueOcean}, con los que
respaldaron la cultura de innovación de El Castillo. Estas frases suyas
nos resultaron muy inspiradoras, y resumen bastante de nuestra esencia
rebelde:

\emph{``Los que contratan servicios en El Castillo compran~un grupo
humano; el equipo de trabajo del Castillo es el gran salto de valor.''}

\emph{``El Castillo ha creado un mercado que no existía en la hotelería,
pero eso no es el resultado de la estrategia de un área específica, sino
de su modelo integral de gestión sustentable.''}

\emph{``Es fácil analizar nuevos mercados con el diario del lunes; pero
¡¿quién se anima a crear nuevos mercados ofreciendo productos o
servicios que hoy nadie pide?!''}

\emph{``El Castillo no busca crecer en facturación o expandirse
replicando su modelo con otros hoteles. Porque su modelo, tal cual es,
los hace felices.''}

\emph{``El Castillo no piensa demasiado en el futuro\ldots{} no sabe qué
sucederá mañana con su producto ni con su equipo de trabajo. Ellos
piensan en el presente y trabajan hoy. Van creando.''}

\emph{``Hay empresas nuevas, como El Castillo, que miden la rentabilidad
de otra forma:~el crecimiento~humano del equipo de trabajo ocupa el
primer puesto de importancia en todos los procesos de toma de
decisión.''}

\textbf{Hoy, después de todo}

Por eso hoy, después de todo lo que hemos vivido y aprendido, creemos
que no existe aquel mentado momento justo en el lugar justo y de la
forma justa.

Muchos de quienes gestionamos sustentablemente lo hacemos por una sana
rebeldía. Por esa rebeldía que nos da la satisfacción de vivir el
presente como ``se debe'', con amor hacia nuestra familia, hacia nuestro
equipo de trabajo, hacia nuestro medioambiente, creyendo que el
propósito de las organizaciones es crear un entorno en el que vale la
pena VIVIR.