% \hypertarget{de-traductora-de-alemuxe1n-a-hotelera-responsable}{
% https://tex.stackexchange.com/questions/23404/how-can-i-fit-my-table-of-contents-into-a-single-page
\addtocontents{toc}{\protect\enlargethispage{3\baselineskip}}


  \end{fullwidth}

\chapter{DE TRADUCTORA DE ALEMÁN A \\ HOTELERA RESPONSABLE}
  \begin{fullwidth}
% \label{de-traductora-de-alemuxe1n-a-hotelera-responsable}}

% \begin{figure}[htbp]
% \centering
% \includegraphics[width=1.76875in,height=1.76250in]{media/image17.jpeg}
% \caption{}
% \end{figure}

\textbf{Lucy Vilte }

Gerente de Hostal Posta de Purmamarca. Traductora pública de alemán.
Multi premiada por Federaciones y Cámaras por su labor en RSE y
Hotelería Sustentable.

\emph{``Ser feliz con lo que te toca''}

De traductora de alemán a hotelera responsable

Soy jujeña, hija de padre purmamarqueño y madre salteña, ambos muy
arraigados a su tierra y enamorados de ella. Así crecí, en el seno de
esta familia y rodeada de un entorno hermoso. Y cuando uno se cría
inmerso en la maravilla de la naturaleza y siendo plenamente consciente
del valor de la Pachamama, el interés por cuidarla se afila, y se
desarrolla aún más al viajar y conocer otras culturas, al asumir la
responsabilidad como de ser ciudadano del mundo y tomar conciencia de
que somos parte de un todo.

Creo además que los jujeños tenemos una idea acabada y profunda de lo
que significa la Madre Tierra, el espacio vital, la hermandad y el
respeto que debe existir entre los seres vivos. Lo que más lamento hoy
es que lo hemos ido olvidando, lo cual nos está conduciendo al colapso
de valores y principios; a una crisis como sociedad y
ecosistema\ldots{}Entonces debemos ¡ACTUAR!

\textbf{Los comienzos}

Pasé la infancia en mi provincia natal y en la adolescencia tuve la
oportunidad de realizar un intercambio estudiantil con una beca, durante
un año, en Alemania. Aprendí el idioma y tomé la decisión de estudiar
más tarde el Traductorado Público en la ciudad de Córdoba. Emigré y me
instalé allí. Concluí mi carrera y trabajé luego para una empresa
automotriz alemana donde me empapé de muchas iniciativas de cuidado
ambiental y de cómo incidir positivamente en lo social.

Mi historia sufrió un giro importante cuando me hice cargo del Hostal
Posta de Purmamarca casi sin querer y de un día para el otro. Mis padres
habían iniciado el emprendimiento hacía pocos años y yo los ayudaba
durante las vacaciones. Hasta entonces no había planeado estar a la
cabeza del negocio. Sin embargo, al faltarme ellos atravesé un momento
difícil y tuve que evaluar si volvería a Jujuy y a Purmamarca para
dedicarme a la actividad turística. Resolví entonces dejar un tiempo de
lado mi profesión como traductora y apostar a esta herencia familiar,
incorporando todos los conocimientos que había adquirido en el
transcurso de mi carrera, conjugándolos con la impronta cultural que
también era parte de mi formación como persona. Básicamente empecé a ver
cómo podía ser feliz con lo que me había tocado en suerte. Y me quedé.

Tomé como premisa basar la gestión del negocio en políticas de
Responsabilidad Social Empresaria (RSE). Por aquel entonces (2004) era
algo bastante nuevo en Argentina y más aún en mi provincia, sobre todo
para una Pyme. Pero a mí no me resultaba nada ajeno: con distintos
nombres (ética en los negocios, respeto hacia los vecinos, etc.) yo lo
había absorbido de mis padres. Sumado a eso, todo lo aprendido en mi
corta experiencia laboral y en mis viajes me pareció perfectamente
replicable y adaptable a cualquier lugar, ¿por qué no? Para ser
responsable, no hace falta tener una gran empresa ni estar en las
grandes ciudades. A veces, la menor escala es una aliada para
aventurarse en iniciativas que obtengan un verdadero impacto local. Lo
importante es querer hacerlo y EMPEZAR.

Lo que inició como un negocio familiar, se fue transformando en una
empresa que no perdió el carisma y el calor de una familia. Situarnos
bajo el paradigma de la RSE implicó mucho esfuerzo y compromiso de todos
los integrantes del equipo, aunque con el claro objetivo de transformar
las aspiraciones en resultados concretos.

~\textbf{Turismo Sostenible en una Pyme}

En el ámbito mundial, el turismo es uno de los sectores económicos más
importantes y con mayores tasas de crecimiento de los últimos años. Por
ello creo firmemente que puede ser una valiosa fuente de recursos que
ayude al progreso de la comunidad y al cuidado del entorno, si redunda
en beneficio del que llega y del que está. De manera que nuestro
bienestar como habitantes y supervivencia como empresarios dependerá
siempre de los esfuerzos que hagamos por mantener el equilibrio entre lo
que ofrecemos al turista para su disfrute y lo que mantenemos para
nosotros y las próximas generaciones: herencia cultural, forma de vida y
naturaleza.

Con los años comprobé que las políticas de RSE son una herramienta
importantísima para alcanzar no solo la sostenibilidad de la empresa
sino también del destino. Las pymes constituyen el terreno más fértil y
apto para aplicarlas, debido a que la proximidad al entorno natural y a
la comunidad local brinda un inmenso potencial para trabajar en este
sentido. Las perspectivas son aún mejores que en las grandes empresas.
En términos de repercusión social a pesar de disponer de menos recursos,
los resultados son percibidos de inmediato, lo que las hace más
creíbles. Su estrecho vínculo, casi personal con todos los públicos,
hace que cualquiera de sus acciones sea fácilmente comprobable.

Sólo se trata entonces de la vieja frase ``querer es poder''. No vamos a
pelear con la fuerza y el tamaño de Goliat, sino que vamos a actuar
inteligentemente como David: haciendo las cosas paso a paso,
estableciendo prioridades, creando alianzas con todo tipo de
instituciones y personas que nos ayuden a alcanzar objetivos comunes.
Porque la pyme no necesita grandes reportes, sino más bien empezar y
sostener pequeñas acciones cada vez más ambiciosas.

\textbf{Dónde estamos}

El Hostal Posta de Purmamarca es un Ecohotel certificado que promueve
desde hace ya 13 años los principios del Turismo sostenible. Está
ubicado en un lugar privilegiado: al pie del famoso Cerro de Siete
Colores, en el pueblo turístico de Purmamarca, provincia de Jujuy,
Argentina. El hermoso lugar donde trabajamos es fuente de inspiración
constante por la belleza de sus paisajes y su extraordinaria riqueza
cultural; probablemente sea el pueblo más pintoresco y el más visitado
de la Quebrada de Humahuaca, Patrimonio de la Humanidad desde el año
2003.

Toda la región posee gran valor ambiental pero, al mismo tiempo, es
sumamente frágil, lo que nos compromete a actuar de manera responsable
para poder vivir, trabajar y legar este paraíso, así como lo conocemos,
a los que vendrán después.

\textbf{Gestión Sustentable }

Hoy en día es cada vez más urgente e indispensable asumir un compromiso
de cambio y ayuda frente a los problemas de nuestra sociedad y la
naturaleza. El Hostal busca, desde esta óptica, articular y llevar a
cabo acciones conjuntas con las personas, organizaciones e instituciones
con las que se relaciona, promoviendo el desarrollo sostenible y una
sociedad más justa e inclusiva; fomentando la conciencia ambiental y
poniendo en valor la riqueza cultural y natural de la región.

En los años que llevamos adelante esta gestión, no sólo cosechamos
frutos en lo cotidiano, sino que también nos reconfortamos con la
aprobación de otras organizaciones que ven en nuestro trabajo una gran
promesa a futuro.

Este modelo de negocio responsable ha ganado creciente aceptación entre
los clientes y reconocimiento en distintos ámbitos siendo galardonados
por distintas Cámaras empresarias, instituciones públicas y privadas.
También, en reiteradas oportunidades, el hostal ha sido caso de estudio
para la elaboración de tesis de alumnos universitarios y de posgrado. En
2014 fue el primer Ecohotel en obtener la Certificación oficial Hoteles
Más Verdes en la provincia de Jujuy, y el tercero en Argentina.

\textbf{Lo que hacemos}

Somos un equipo que día a día trabaja para que tanto huéspedes como
lugareños disfruten de la naturaleza y se sientan a gusto en este
pintoresco pueblo quebradeño, en una convivencia sana y equitativa entre
todos.

Internamente contamos con un programa de ahorro y uso eficiente de
recursos que apuesta sobre todo a la concientización, la toma de
decisiones inteligentes y la adquisición de productos y servicios que
reduzcan el consumo: cartelería indicativa de sensibilización, jardín de
plantas autóctonas, sistemas de iluminación y provisión de agua
eficientes, cocina y artefactos de iluminación solar, etc. Se gestionan
el 75\% de los residuos, reciclando o reutilizando los mismos y
minimizando la generación de envases y embalajes. Nuestra política de
compras, además, se basa en el comercio justo y el consumo responsable,
que implica integrar el componente socio-ambiental en la toma de
decisiones al momento de adquirir productos o contratar servicios.

La edificación sigue normas de construcción bioclimática, utilizando
materiales del lugar vinculados también a la cultura y tradición de los
habitantes andinos pero sin renunciar en ningún momento al confort. La
decoración es amigable con el entorno: telas artesanales u orgánicas,
materiales nobles y de producción local; muebles y objetos de artistas
locales o reciclados con creatividad y encanto.

Sabemos que el impacto que hemos logrado en estos años desde nuestra
PyME puede inspirar a otros para comenzar a actuar, a comprometerse, a
generar conciencia sobre la importancia del turismo responsable y un
mundo sostenible. Socializar información, comunicar, sensibilizar y
compartir lo aprendido es nuestro objetivo final en cualquier
iniciativa. De nada sirve cuidar la pequeña casa si no podemos llegar a
otros, o ir más allá y contribuir a la causa global: el cuidado del
planeta. Debemos actuar localmente pero pensando globalmente y creer que
cada uno, desde nuestro lugar, puede cambiar el mundo.

\textbf{Con la mirada hacia el futuro}

La consolidación de la sustentabilidad económica, social y ambiental en
la gestión de los negocios implica, sin dudas, un proceso de permanente
cambio y mejora que nosotros asumimos de manera gradual, progresiva e
integral. Por eso, año a año apostamos a proyectos de remodelación y
ampliación de nuestras instalaciones teniendo siempre presente
principios de ecoeficiencia, bioconstrucción y accesibilidad. También
estamos atentos a las necesidades de la comunidad, que tratamos de
detectar mediante encuestas y buscamos satisfacerlas trabajando junto
con instituciones públicas y privadas. Tratamos de focalizarnos siempre
en la mejora continua de toda la gestión hotelera para alcanzar
objetivos más altos, satisfactorios y ambiciosos que nos hagan sentir
felices y satisfechos como personas y dignos representantes de la
hospitalidad sostenible.

Estamos seguros de que lo que podemos cambiar, defender, sanear y
proteger existe hoy y no habrá un mañana para hacerlo, si no damos el
primer paso en el presente. La vida es un milagro y tenemos la
oportunidad y el deber de defenderla, cuidándonos entre nosotros y
protegiendo nuestra Madre Tierra. Para eso habrá que repensar nuestras
acciones y decisiones, cambiar hábitos y costumbres malsanas, vivir de
forma más ética, ecológica y armónica.

Por nuestra parte nos comprometemos a seguir en la misma línea de
trabajo que nos da grandes satisfacciones a todos los que conformamos el
equipo del Hostal Posta de Purmamarca. Esto es lo que nos desafía y nos
apasiona. Y le ponemos todo el corazón en la tarea.