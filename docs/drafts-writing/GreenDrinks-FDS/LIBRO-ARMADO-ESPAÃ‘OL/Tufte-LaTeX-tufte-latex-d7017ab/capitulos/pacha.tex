% \protect\hypertarget{_Toc486429368}{}{}\textbf{
% }
% \chapter{VOLVER A LA PACHA}
\hypertarget{VOLVER A LA PACHA}{\chapter{VOLVER A LA PACHA}\label{VOLVER A LA PACHA}}

% \begin{figure}[htbp]
% \centering
% \includegraphics[width=1.76458in,height=1.73542in]{media/image14.jpeg}
% \caption{}
% \end{figure}

\textbf{Claudia Lamas}

Ing. en computación, Especialista en vinculación y gestión tecnológica.
Desde el 2013 organiza Green Drinks en San Salvador de Jujuy. Comenzó su
``vuelta a La Pachamama'' luego de investigar cómo tesis de su
especialidad el problema de los RAEEs.

Volver a la pacha

Conocí la existencia de los Green Drinks accidentalmente, por medio de
una amiga que en un autobús de Córdoba a Jujuy, le tocó estar sentada al
lado de Elga Velásquez, referente de Green Drinks Córdoba. Allí,~Elga le
sugirió hacer este evento en San Salvador de Jujuy.

Personalmente estaba buscando alguna actividad de voluntariado y
comenzando un cambio interno, personal, que me haga ``volver al origen,
a la Pacha''.

En la próxima reunión con mi amiga, me comenta este encuentro casual y
comienzo a investigar de qué trataba realmente esto de los
``\textbf{\emph{Green Drinks}}''.

Me gustó la idea de encuentros libres, de los temas relacionados con la
sustentabilidad, de buscar referentes locales, ya que tenía, todo esto,
mucho que ver con mi búsqueda personal.

Así, en noviembre 2013, organicé el primer ``\textbf{\emph{Green
Drinks}}'' en San Salvador de Jujuy. A través de conocidos llegué a Lucy
Vilte. Si bien la conocía por los diarios y por la tele, muy bien no
tenía en claro que es lo que hacía, pero sabía que tenía que estar,
todos mis conocidos coincidieron en que ``Tiene que estar Lucy''. Y ¡Sí!
no se equivocaron.

Recuerdo perfectamente el título de su charla ``\textbf{Trabajar en red
para un objetivo común}''. Y fueron más de 40' desde que comenzó su
charla, llena de proyectos, ideas llevadas a cabo, un huracán de HACER,
más que DECIR. Una charla llena de adrenalina. Eran tantas cosas
realizadas por ella, que uno se sentía una hormiga frente a tanta
grandeza, tanta energía destinada a HACER, para que otros Hagan y entre
todos, GENERAR un \textbf{bienestar común.} Una mujer con muchas
cualidades, entre las que destaco su humildad y su gran humanidad.

En este largo viaje de GD, encontré varios HACEDORES en mi provincia,
tantos desconocidos que trabajan y viven en pos del bienestar general.
Así, cuando planteamos el tema de ``Alimentación Sustentable'' para un
próximo encuentro, mis ``contactos'' me dijeron: ``Nacho Mayorga''. Me
pasaron su teléfono, lo llamé y conocí a un tipo admirable, sencillo,
capaz, solvente, un hacedor por naturaleza. Y fue maravilloso tenerlo en
ese encuentro. Tanta pasión en su trabajo, en su profesión, fueron
volcadas con la dosis exacta de conocimiento técnico y amor puro por la
Naturaleza. Fue un encuentro en el que nos tuvieron que sacar del lugar
porque las preguntas y las respuestas y la amabilidad hicieron que el
tiempo pasara sin sentirlo.

Me considero una afortunada de encontrarme en este camino personal con
ellos, de poder también poner mi granito de arena para visibilizar el
gran trabajo que realizan. Eso es para mí Green Drinks, una ventana a
``otro mundo'', en donde el interés personal deja de tener sentido e
importancia, para abrir paso a un sentido fraternal y en armonía con
todo lo que nos rodea.