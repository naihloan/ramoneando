     \end{fullwidth}
\chapter{COMPROMISO ESCOLAR \\ CON EL MEDIO AMBIENTE}\label{compromiso-escolar-con-el-medio-ambiente}
  \begin{fullwidth}

La importancia de inculcar el reciclaje a los más chicos.

% \begin{figure}[htbp]
% \centering
% \includegraphics[width=1.74931in,height=1.76250in]{media/image13.jpeg}
% \caption{}
% \end{figure}

\textbf{Gladis Vera}

Directora de la Escuela Pablo Pizzurno - San Antonio de Litín.

\protect\hypertarget{_b05mbdgbx2f6}{}{}\textbf{``Reciclar y no
contaminar TODOS UNIDOS el planeta debemos cuidar''}

\protect\hypertarget{_ui7ubdln8of2}{}{}

\protect\hypertarget{_mtkd1ps46ywt}{}{}La escuela está inserta en San
Antonio de Litín, un poblado de 1800 habitantes, ubicado en el sudeste
de la provincia de Córdoba.

\protect\hypertarget{_c338x57u0zi8}{}{}

En el año 2016, la escuela participa en el Rally Continental de Escuelas
con Futuro Sostenible, siendo la única en la Provincia de Córdoba
asumiendo este compromiso. Contamos con un proyecto cuyo principal
objetivo es reciclar materiales y presentar diferentes soluciones
ambientales a través de ese proceso, llevado a cabo por el campo de
formación de expresiones artísticas y culturales. Además, a partir de
nuestra institución se promovió y presentó una política de protección y
promoción de árboles a través de la participación y donación de
diferentes fundaciones y entidades de todo el país, con el objetivo de
reforestar gran parte de la localidad y la zona.

\textbf{La problemática}

La contaminación por residuos sólidos es un tema que nos concierne a
todos. En este caso nos referiremos al caso específico dentro de nuestra
comunidad y comenzaremos a abordarlo desde nuestra institución,
\emph{aunque consideramos que esta cultura ambientalista debe comenzar
en los hogares para luego ser fortalecida en las escuelas.}

Los residuos sólidos corresponden al material de desecho resultante de
todas las actividades humanas, por lo tanto, son una realidad que no se
puede evitar. Se entiende por residuos sólidos cualquier basura,
desperdicio, lodo y otros materiales sólidos de desechos, resultantes de
las actividades domiciliarias, industriales y comerciales.

Nuestra comunidad, más allá de pertenecer a un poblado pequeño, no
permanece ajena a este problema; de hecho, detectamos la misma
proporción de desechos, aunque en distinta escala, que poblados o
ciudades mayores, lo cual es un indicador suficiente para proponer
hacernos cargo de ello, otorgarle la prioridad adecuada y comenzar a
tratarlo lo antes posible.

\textbf{El Compromiso de la comunidad }

El proyecto actual consiste en reciclar y reutilizar distintos
materiales y elementos, comprometiendo a toda la comunidad educativa.

Este proceso se lleva a cabo con el objetivo de transformar residuos
sólidos contaminantes en objetos decorativos y/o de uso cotidiano,
utilizando diferentes técnicas y métodos: activos, participativos, y
enfoques constructivistas.

\textbf{La razón de hacer }

La razón que nos llevó a realizar este proyecto es la observación de una
actividad contaminante como hábito dentro de la rutina diaria de nuestra
comunidad e institución. La cual nos llevó a reflexionar grupalmente
sobre la situación para luego comenzar con un proceso de concientización
general respecto del deterioro del medio ambiente.

\textbf{Tenemos objetivos}

Nuestros objetivos se basan en la sensibilización y fortalecimiento de
los valores éticos relacionados a la protección, uso y conservación del
entorno ambiental, principalmente, de los alumnos para luego extenderlos
a la comunidad.

\textbf{La propuesta y su desarrollo}

La propuesta nace a partir de la enseñanza, a todos los alumnos por
igual y desde todos los espacios curriculares posibles, acerca de la
problemática existente con el objetivo de concientizarlos al respecto. A
partir de allí comenzamos a trabajar en la identificación de diferentes
materiales que ellos mismos podrán detectar y reciclar, aprendiendo como
es el tratamiento adecuado para los mismos. El trabajo de reciclado se
lleva a cabo de forma íntegra dentro del establecimiento, conformando
actividades especiales dentro de espacios escolares agradables y
provistos de todas las herramientas e insumos necesarios para su
ejecución.

\textbf{Aprendemos jugando}

Para el desarrollo de la propuesta, la escuela propone reutilizar
diferentes materiales, reciclándolos a través de un proyecto
interdisciplinario denominado Aprendemos Jugando: Producto de la
vinculación de diferentes campos curriculares como Ciencias e
Investigación, Actividades Corporales y Ludomotrices y Expresión
Artística y Cultural, entre otros espacios curriculares convencionales
como Lengua, Ciencias y Matemática.

En nuestra escuela se desarrollan espacios en donde la creatividad, la
curiosidad, la imaginación y la experiencia, ocupan un lugar muy
importante, valorando lo que se hace y procurando un acercamiento
afectivo que desarrolla el respeto y el cuidado del entorno.

\textbf{Materiales empleados}

Prácticamente todo lo que vemos a nuestro alrededor puede ser
recicla-do, es por ello que creemos en el gran potencial que esta
actividad presenta, además de entender los diferentes niveles en los que
se puede colaborar. Entre los materiales más utilizados por nosotros se
destacan: plástico, goma, papel, telas, madera, ladrillo, cartón, metal,
cables, cajas, cajones, botellas, ropa, revistas, troncos.

\protect\hypertarget{_lfjh5v8ovbq5}{}{}La importancia de los juegos EN
EL PROCESO EDUCATIVO

\protect\hypertarget{_crimm4xjx5j5}{}{}

Una excelente forma de generar interés en el niño, en distintos ámbitos,
es a través de juegos didácticos, los cuales nos garantizan
entretenimiento preocupación y dedicación colectiva por un objetivo en
común.

\emph{``El juego no es sólo juego infantil. Jugar, para el niño y para
el adulto\ldots{} es una forma de utilizar la mente e, incluso mejor,
una actitud sobre cómo utilizar la mente. Es un marco en el que poner a
prueba las cosas, un invernadero en el que poder combinar pensamiento,
lenguaje y fantasía.''} JP

\textbf{Prolongar la vida de los recursos}

Los resultados fueron y son absolutamente positivos. Se ha logrado
comprender el proyecto como un proceso de transformación de elementos
con el objetivo de re funcionalizarlos y reutilizarlos, entendiendo la
importancia de estos en la vida cotidiana, como si se tratara de un
reciclaje doméstico. Por otro lado, tanto los alumnos, como la comunidad
han aprendido la importancia de ``prolongar la vida útil de los
recursos'', siendo testigos de los beneficios que actividades simples y
sencillas pueden presentar tanto para nosotros como para el medio
ambiente y la gratificación que esto nos genera; además de entender la
capacidad de participación que estas actividades permiten, conformando
diferentes grupos humanos con el más amplio abanico de edades y
experiencias trabajando por un objetivo en común.

\protect\hypertarget{_3qahj1i8c1w3}{}{\protect\hypertarget{_dje8yrrvew7s}{}{}}Impacto
social

El impacto social fue reflejado por diferentes factores: en principio,
la participación de la comunidad en su totalidad, incluyendo diferentes
medios de comunicación audiovisual de la zona. Además, es importante
destacar la importancia de los espacios áulicos relacionados con
bibliotecas y sectores para lectura, los cuales tuvieron un éxito
rotundo, contribuyendo a la concentración de alumnos y padres para la
comunicación de diferentes mensajes. Por otro lado, el proyecto se
complementó con la participación de nuestra institución Educativa en el
Rally Continental con Futuro Sostenible, dentro del cual se elaboró el
proyecto ``Árboles y Niños'' que constó de promover una política de
protección de árboles de la zona, a través de la reforestación de
diferentes sectores del poblado y los alrededores, la cual fue posible
gracias a la donación de diferentes cantidades y calidades de vegetación
por parte de distintos entes y fundaciones como por ejemplo Fundación
Natura, Green Drinks o la Facultad de Ciencias Agropecuarias. La
mecánica del proyecto simplemente se trató de otorgarle a cada familia
un cierto número de árboles para incluirlos como ``miembros verdes''
dentro de su núcleo familiar con el objetivo de insertarlos en un área
del poblado y luego hacerse cargo de la mantención del mismo, fomentando
la actividad colectiva y comunitaria en su totalidad.

\protect\hypertarget{_e8d930zeu4l6}{}{}El impacto ambiental

A partir de nuestro proyecto se ha logrado el apoyo y acompañamiento de
diferentes instituciones ambientalistas para el logro de los objetivos y
desafíos sustentables propuestos. Además, la escuela le presenta a la
comunidad posibles soluciones a una problemática concreta, generando una
reflexión colectiva para continuar con la búsqueda de respuestas.

\protect\hypertarget{_dld5dcvfjax6}{}{}El impacto económico

El impacto económico está relacionado directamente con los elementos
re-elaborados dentro del instituto, los cuales provienen de materiales
con ningún tipo de valor monetario, para luego pasar a ser elementos de
bajo costo que, si bien pueden ser comercializados como tales, cumplen
funciones muy claras y son muy útiles a la hora de otorgarles un uso,
por lo tanto se puede afirmar que el proyecto es 100\% rentable en este
sentido, ya que a partir de un elemento sin valor, uso ni función, se
genera uno nuevo y totalmente opuesto en esos aspectos.

\protect\hypertarget{_fo3lb8bbxv81}{}{}

Trabajar en comunidad

A partir de esta experiencia, personalmente, pude comprender, una vez
más, la importancia que tiene el trabajo en conjunto y las implicancias
logísticas y organizativas que esto implica porque, si bien estamos
acostumbrados a realizar trabajos en equipo a lo largo de nuestras
vidas, en esta oportunidad tuvimos que aprender a desarrollar un trabajo
directamente como comunidad, en todos sus niveles.

Además, siento que hemos logrado concientizar a los principales
colaboradores y protagonistas de este proyecto, los alumnos, los cuales
han demostrado un alto grado de responsabilidad con respecto al tema a
la hora de realizar cualquier tipo de investigación o trabajo.

Hemos aprendido, también, a trabajar con diferentes herramientas y
materiales que hasta ahora no conocíamos o no habíamos maniobrado, lo
cual nos permitió desarrollar diferentes capacidades y habilidades
aplicables en muchos otros ámbitos.

Humildemente, creo que nuestro trabajo puede servir de ejemplo o ayuda
para cualquier otra persona o institución interesada en este tipo de
proyectos, por lo cual permanecemos atentos y abrimos nuestras puertas a
cualquier tipo de consulta o contacto al respecto.