% \hypertarget{recuperando-el-verde-en-la-ciudad}{
\chapter{RECUPERANDO EL VERDE EN LA CIUDAD}\label{recuperando-el-verde-en-la-ciudad}
% }

% \begin{figure}[htbp]
% \centering
% \includegraphics[width=1.76528in,height=1.75347in]{media/image2.png}
% \caption{}
% \end{figure}

\textbf{Leonardo Peralta}

Coordinador del Proyecto de Forestación Urbana en Córdoba -- Green
Drinks Cba.

\emph{``Si supiera que el mundo se acaba mañana, yo, hoy todavía,
plantaría un árbol''. Martin Luther King}

~

\textbf{\\
}

\textbf{Recuperando el verde en la ciudad.}

\textbf{Proyecto de forestación urbana de la ciudad de Córdoba}

El arbolado urbano tiene una importancia fundamental en el desarrollo de
la vida de las personas. Son muchísimas las ventajas que trae aparejada
la presencia de los árboles en nuestras ciudades, cuando esta presencia
se da en la calidad y cantidad adecuada. Son el primer contacto que el
ciudadano tiene con la naturaleza al salir de sus casas. En muchos
casos, son el único contacto. Silenciosa y desinteresadamente ellos nos
brindan numerosos servicios ambientales, sociales, paisajísticos y
económicos. Sin embargo, y a pesar de todos los beneficios que los
arboles gentilmente nos regalan, las ciudades crecen y crecen sin pausa,
y el verde está cada vez más lejos. Como en la mayoría de los casos,
nuestra amada Córdoba al ir creciendo, sin darnos cuenta (a veces), hizo
a un lado todo el magnífico y exuberante verdor que alguna vez habitó en
estas tierras, para erigir en su lugar un gigante de acero, de cemento y
de cables. Pocos lugares hay dentro de la ciudad que den cuenta del
monte que alguna vez supo reinar.

Por ello, este proyecto se propone hacer de Córdoba una ciudad más verde
y sustentable, a través de un programa de forestación urbana activo y
participativo que, haciendo foco en la educación, contribuya a crear una
sociedad más consciente de la importancia del árbol y del cuidado de
naturaleza. En este proyecto, se hará hincapié en la reintroducción de
especies nativas, la capacitación de los participantes, la educación, y
generación de conciencia en la ciudadanía, sobre todo en los más
pequeños, serán fundamentales. No es posible pensar en una ciudad
sustentable sin un arbolado urbano consolidado y saludable, y un vecino
comprometido con el cuidado del árbol. Hacia allá vamos. Hacia una
ciudad más sustentable.

% \textbf{\\
% }