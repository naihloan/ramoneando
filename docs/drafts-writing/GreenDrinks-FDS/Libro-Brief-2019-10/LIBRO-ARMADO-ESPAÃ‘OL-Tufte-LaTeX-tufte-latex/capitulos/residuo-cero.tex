  \end{fullwidth}
   

\chapter{WEBER SAINT GOBAIN: PROYECTO RESIDUOS CERO}\label{weber-saint-gobain-proyecto-residuos-cero}
  \begin{fullwidth}

% \begin{figure}[htbp]
% \centering
% \includegraphics[width=1.75278in,height=1.75347in]{media/image5.jpeg}
% \caption{}
% \end{figure}

\textbf{Daniel Blanco}

Ingeniero Mecánico Aeronáutico. Magister en Ingeniería en Calidad.
Gerente de planta Weber Saint Gobain Argentina S.A.

Weber Saint Gobain - PROYECTO RESIDUO CERO

¿El problema aparece o ya estaba?

Por más que tratemos de hacer memoria, nunca sabremos definir el momento
exacto en el que vimos, por primera vez, que algo cotidiano como sacar
la basura de nuestra planta industrial, era en realidad un problema
ambiental y social.

Desde su inicio, y sin reparar en que algo no funcionaba, por más de
siete años realizamos nuestra producción, diecisiete horas diarias,
cinco días a la semana, más de 7000 tn mensuales, sin preocuparnos para
nada por aquello que frecuentemente sacamos del predio de la planta con
un destino definido, llamado: Basura.

Habiendo conseguido operar una planta de premezclados cementicios con
los más altos estándares de calidad y conservación del espacio, con un
control de polución, higiene, orden y limpieza envidiables, incluso por
las más prestigiosas firmas de rubros industriales, así y todo,
sacábamos la basura como cualquiera.

Fue mucho tiempo después de comenzar a operar esta moderna instalación,
que implementamos un sistema de gestión integral que certificara a la
planta en el referencial de ISO 14001 y OHSAS 18001. Momento en el que
definitivamente le pusimos nombre a nuestros impactos ambientales y
comenzamos a medir indicadores en los que nunca antes habíamos reparado.

Al principio nos concentramos en cumplir con lo que el referencial nos
pedía y, de ese modo, completar cada uno de los requerimientos, casi
como si se llenara un largo formulario para declarar lo que se hace y
hacer lo que se declara. Nada más, seguir el camino marcado.

Pasó el primer año y empezamos a sentir ese vacío que se siente cuando
uno cumple con lo que le piden pero en realidad no hace lo suficiente.

¿Será que estamos haciendo lo correcto? ¿Será que entendimos bien lo que
se debe hacer? ¿Será que estamos en el camino que queríamos estar?

Empezar a entender que algo no estaba bien

Desde las bases de la gestión ambiental se habla de ``Aspectos e
impactos ambientales''. La reacción inmediata fue ir directo a lo que
consumimos en nuestro proceso e incluso cuestionar todo aquello que
representa un daño para el ambiente.

La lista era muy corta, o al menos, así parecía al principio. ¡Claro!
¡Qué simple! Si era cuestión de echarle la culpa a todo lo que es
recurso no renovable, como el combustible que utilizamos en el proceso,
plásticos, nylons, pinturas, aceites, grasas.

En fin, todo lo contaminante. Pero,~por qué seguíamos sintiendo que no
estábamos viendo el problema, si los indicadores nos mostraban que al
poco tiempo de medir los consumos de estos insumos, ya los habíamos
optimizado, al punto de no usar casi nada más que lo que se precisaba.
Incluso hubo proyectos que reemplazaron a algunos de ellos por otros con
menor impacto y hasta lograron eliminar algunos de ellos del proceso.

Dando un giro al modo de ver las cosas, saliendo de lo habitual y
animándonos a reconocer una falla en nuestra mirada, nos dispusimos a
plantear, ¿cómo realmente estábamos impactando en el ecosistema en el
que la planta estaba inserta?

Cuando finalmente entendimos que el impacto ambiental se basa en la
modificación o deterioro del ecosistema, nos dimos con que se trataba de
un efecto biodinámico. Nunca nos hubiésemos imaginado cómo impacta el
proceso industrial en este ecosistema porque considerábamos al proceso
como algo sin vida, sólo como un mecanismo.

Entendemos, ahora, que toda la actividad industrial no es más que un
cuerpo vivo, lleno de sistemas, redes neuronales, órganos funcionales y
con ciertas reglas o metodologías propias de esa actividad.

Sin caer en una comparación banal, podríamos pensar esta planta
industrial como si fuese un ser humano, y toda su actividad como la que
realizaría un humano en su actividad diaria, es decir procesa alimentos
(materias primas), que generan energía (productos terminados), que se
consumen para el movimiento (recursos de energía) y requiere reposo
(mantenimiento).

Ahora, miremos este organismo vivo, que está impactando en el ecosistema
en el que vive y analicemos sus \emph{Aspectos e Impactos}, tal como lo
propone la ISO 14001, ya no estamos siendo tan ``ingenieros'' como
antes, nos estamos comportando como si fuésemos ``médicos''.

¿Y cómo hacen los médicos para saber lo que no funciona en sus
pacientes? Comienzan a hacer preguntas sobre los síntomas, dolores y
sensaciones y luego encaminan una serie de estudios y análisis de
laboratorio.

Allí es donde descubrimos que un análisis de laboratorio para la planta
industrial, implicaba mirar qué había en la basura. Y a partir de este
punto entendimos el problema.

Ya estábamos en condiciones de responder la primera pregunta: este
problema, sin saber bien qué es, no apareció sino siempre existió y,
probablemente, en gran medida sea congénito. ¡Qué bien nos sentimos en
ese momento por haber reconocido esto!

¡Necesitamos una estrategia!

Si íbamos a trabajar sobre estos Aspectos e Impactos, considerando que
este organismo genera algo en el entorno, no podíamos ir a la deriva,
necesitábamos definir cómo interiorizarnos de los detalles del problema
y atacar con firmeza uno o varios objetivos claros, para reducir estos
impactos.

Lo que nos movilizó a este desafío fueron los objetivos del sistema de
gestión, por lo que resolvimos atacar lo que hasta ahora no había sido
un problema. Ahora la basura es un problema y le vamos a declarar la
guerra: no queremos generar más basura.

Regresando en el tiempo, vemos por qué la basura NO era un problema y
ahora SÍ lo es. Cuando organizamos lo que se tiraba, separamos de
acuerdo a lo indicado por las regulaciones locales, los patógenos por un
lado, los contaminantes a tratamientos especiales y el resto, a
enterramiento sanitario. Cumpliendo esto, la empresa responde como un
modelo a seguir, satisface en su totalidad a la normativa aplicable. Es
decir no evidencia un problema.

Sin embargo, a partir de esta mirada diferente que imprimimos, generamos
algo distinto. Por más que esté bien tratada como patógena, contaminante
o residuo urbano, sabemos que sigue siendo un impacto no deseado.

Entonces entendimos que en el estado más puro de no impactar, el
objetivo sería que la basura, simplemente, no existiera. Pero, ¿será
esto posible? ¿Puede ser que este cuerpo vivo consuma materias primas,
produzca lo esperado y no tenga ningún tipo de excreción? Lo más
probable era que eso no era posible, porque de ser así, no sería
comparable con un organismo natural.

Entonces reformulamos la pregunta: ¿es posible que este desecho sea
menor al actual y si existe, sea posible volver a utilizarlo en lugar de
tirarlo?

Hasta ahora teníamos más preguntas que respuestas, pero ya empezábamos a
encontrar una nueva definición: el proceso ya no generaría basura, a
partir de ahora lo llamaríamos residuo. Y ¿por qué hicimos esto?
Simplemente por reconocer que lo que llamamos basura es algo
definitivamente inservible, mientras lo que llamamos residuo es lo que
un proceso no aprovecha, pero puede ser aprovechado por otro proceso o
en un segundo ciclo del mismo proceso.

A partir de este gran cambio comenzamos a generar una serie de
definiciones nuevas, que en su conjunto serían la estrategia más
adecuada para llamar a nuestro desafío: ``Proyecto de residuo cero''.

Los pasos siguientes fueron casi automáticos. Entendimos muchas de las
herramientas y conceptos generales que ya habíamos estudiado
anteriormente, repitiendo como loros las ventajas de utilizar estos
métodos y siguiendo ciertos procedimientos.

Una de las más generales pero más eficiente de las herramientas fue las
3R, que se convirtieron en una herramienta de identificación, medición y
decisión.

Otro gran concepto fue el mapeo del origen de los residuos y, también,
poder distinguir si es necesario o no acondicionar estos residuos antes
de enviarlos fuera de las instalaciones de la planta. En este punto
también surgen preguntas como si es necesario o no sacar de la planta el
residuo o su destino está puertas adentro.

También surgieron las cuestiones de clasificación. Ya teníamos en la
planta una serie de recipientes de colores con el propósito de separar
según su clasificación general los materiales de plástico, papel, metal,
vidrio y generales, pero a partir de entender este problema, la
clasificación ya no responde a juntar elementos similares, la
clasificación parte de lo que las 3R digan que se hará con estos
materiales.

En secuencia, lo que hicimos fue:

\begin{itemize}
\item
  Reconocer lo que termina como residuo del proceso.
\item
  Identificar en que etapa del proceso se genera, confeccionar un mapa.
\item
  Partiendo del reconocimiento, aplicar la primer R. El sólo hecho de
  analizar y medir ya ayuda a reducir.
\item
  De lo que indefectiblemente no se puede reducir, encontrar en qué
  proceso puede ser útil, ya sea interno o externo.
\item
  Para que sea útil a ese nuevo proceso, si requiere un
  acondicionamiento, determinar cuál será.
\item
  Por último, lo que no se redujo o se reprocesó, será destinado al
  reciclado, proceso por el cual ya dejará de ser lo que era para
  convertirse en algo diferente y también útil a un proceso, que
  posiblemente se pueda aprovechar en nuestra fábrica.
\end{itemize}

Cambiar la cultura o crear cultura

Es muy difícil definir el proceso de la creación de la cultura, mucho
más difícil es modificar o cambiar la cultura, porque representa
desandar un camino y construir otro.

Repasemos los pasos de cómo se crea la cultura, en términos simples.

Los pasos que se deben dar, comienzan por la repetición, en general
conducida por quienes quieren lograr que se adopte un determinado modo
de actuar, pensar o comportarse.

De la repetición permanente, como acto consciente, surge, con el tiempo,
una repetición rutinaria, como acto no consciente, concepto aplicable al
individuo, llamado generalmente ``hábito''. Sin embargo, cuando hablamos
de una población, como es el caso del grupo humano que trabaja en la
empresa, el comportamiento colectivo no consciente, es atemporal y
transmisible generacionalmente. Este comportamiento es lo que podemos
llamar ``cultura''.

Después de haber llegado a estas conclusiones, y saber que después de
muchos años haciendo algo de determinada forma es la cultura existente,
podemos imaginarnos lo difícil que es, en un proyecto de corto o mediano
plazo, cambiar esa cultura.

El plan de trabajo requiere una gran cantidad de definiciones, reglas
claras, información a la vista, gestión por indicadores, entrenamiento
y, sobre todo, la revisión y adaptación permanente.

¿Cómo fue que creamos esta cultura?

Como mencionamos antes, después de cambiar el concepto de ``basura'' a
``residuo'', se lo contamos a todos, lo escribimos y lo identificamos en
un mapa. Luego se propuso que cada etapa clasificara en el puesto de
trabajo los residuos generados. Esto ya ayudó a que se redujera mucho.
Lo que quedaba estaba bien separado e identificado. Además, los residuos
que el proceso era capaz de volver a utilizar se llamaron ``recuperos'',
como un sinónimo de ``reutilizar'', y finalmente lo que serviría para
otro proceso, pero fuera de la planta, fue trabajado en cada caso de un
modo particular.

Nuestro primer cambio cultural fue la recuperación de las tarimas
descartadas por nuestros clientes, y trabajadas por cooperativas para
volver a ser compradas por nosotros e incorporadas al proceso. Este
desafío, generó puestos sociales y, por otro lado, la primera
experiencia en una nueva dimensión: ``la economía circular''. A partir
de allí, nos vimos en la obligación de incorporar este nuevo ángulo al
análisis. De todo aquello que salía de la planta como una materia prima
para el proceso de alguien más, ¿cuánto podíamos volver a comprar?
¿Cuánto de lo que se compraba podría venir de fuentes de recuperación? Y
si no fuera posible para nosotros, ¿seríamos capaces de vincular partes
interesadas externas entre sí?

La nueva cultura llegó de la mano de las empresas sociales,
cooperativas, fundaciones y empresas de reciclado. Este paso fue crucial
en la motivación de crearle conciencia al empleado, al momento de
clasificar o separar. Ya no estábamos simplemente separando residuos,
empezamos a creer en el proyecto de este tercero, que hacía cosas muy
importantes con lo que nosotros le preparábamos. Este aspecto de
contenido emocional, ayudó mucho en la creación de la cultura.

Siguieron a las tarimas, los papeles y los plásticos, los tubos de
cartón y el cartón corrugado, el polvo de barrido y los productos
terminados endurecidos o contaminados. ¡Fueron días de gloria! Cada
nueva incorporación era un mundo nuevo y cada descubrimiento, un nuevo
estímulo. Conforme pasaban los meses, se notaba más y más, cómo se
volvía más simple hacer la separación y el acondicionamiento.

Alcanzando el éxito

Arrancamos este proyecto con un nivel de generación de basura alto, seis
contenedores salían mensualmente de nuestra planta, hacia el
enterramiento sanitario de la ciudad de Córdoba. Al cabo de un año, la
reducción fue completa, logramos reducir a cero la salida de
contenedores a enterramiento.

Actualmente, a poco más de dos años de haber comenzado con este desafío,
seguimos sacando cero contenedores de basura. El hecho de lograr
mantener en cero durante más de un año la salida de basura, se
transformó en un testimonio de que la cultura está arraigada y que,
difícilmente, se caiga en un retorno innecesario de volver al
enterramiento sanitario.

Por otro lado, la reducción de los residuos continuó existiendo,
impulsada fundamentalmente por quienes están clasificando, se ha
generado una especie de requerimiento permanente, aunque no formal, de
que es mejor que no haya nada que clasificar, adecuar o mover.

El resultado no solo ha traído impactos ambientales de alto nivel,
además redundó en beneficios económicos directos, tales como la
desaparición del gasto de traslado de residuos en contenedores de basura
y el costo de su disposición. La disminución de los desperdicios en el
proceso, el menor tiempo de limpieza y traslado, lugares libres en
planta, una mejor y más limpia imagen general, así como también la
disminución de materiales e insumos para embalar estos residuos.

Por sobre todos estos beneficios, lo más gratificante vino por el
impacto ambiental, que fue lo que nos motivó a comenzar a trabajar. Y el
regalo no esperado fue el impacto social que generó fundamentalmente en
las organizaciones beneficiadas con nuestro programa de clasificación y
reaprovechamiento.

En este aspecto, fuimos partícipes de la creación de actividades de
aprovechamiento de residuos, de la expansión de cooperativas y otras
empresas del tipo B, la implementación de una economía circular
sostenible, el beneficio de plantear el precio justo y, por sobre todo,
desmitificar los conceptos de que nuestro proceso, por génesis, generaba
``basura''.

Aún queda mucho por mejorar y pulir, siempre surgen nuevas oportunidades
y herramientas superadoras, pero es posible compartir esta experiencia e
incluso generar nuevos proyectos tan exitosos como este.