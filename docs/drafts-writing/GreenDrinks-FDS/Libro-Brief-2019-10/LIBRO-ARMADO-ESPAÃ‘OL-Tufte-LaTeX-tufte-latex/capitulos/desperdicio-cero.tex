% \protect\hypertarget{_Toc486429358}{}{}\textbf{
\chapter{DESPERDICIO CERO: UNA FORMA DE VIDA}
% }

% \begin{figure}[htbp]
% \centering
% \includegraphics[width=1.77500in,height=1.71875in]{media/image4.jpeg}
% \caption{}
% \end{figure}

\textbf{Natalia Emma Basso}

Licenciada en Nutrición -UBA. Especialista en seguridad y legislación
alimentaria. Forma parte del equipo de nutrición y educación alimentaria
del Ministerio de Agroindustria de la Nación.

\emph{``Valoremos los alimentos para un sistema agroalimentario más
sostenible''.}

Desperdicio cero: una forma de vida.

Programa Nacional de Reducción de Pérdida y Desperdicio de Alimentos

Un poco de historia

En épocas de conflictos bélicos y ante desastres naturales, plagas o
adversidades climáticas la comida casi siempre es un bien escaso. Más
allá de los esfuerzos en materia de seguridad alimentaria, todavía hay
795 millones de personas mal alimentadas en todo el mundo. Por estos
motivos, resulta intolerable que se descarten alimentos que se
encuentran aptos para consumo.

El despilfarro de comida no es un tema nuevo, sin embargo, las primeras
muestras concretas del problema se pusieron en evidencia hace algunos
años.

En 2009, Tristram Stuart publicó \emph{Waste: undercovering the global
food scandal}, un libro que logró mostrar en detalle la paradoja. El
lanzamiento del libro se acompañó de la creación de la Fundación
Feedback, por medio de la cual se implementan campañas como
\emph{Feeding the 5000} y \emph{The Pig Idea}; transformándose en una de
las principales usinas de actividad en la lucha contra el desperdicio de
alimentos.

Paralelamente, en 2011 la Organización de las Naciones Unidas para la
Alimentación y la Agricultura presentó el informe ``\emph{Pérdida y
desperdicios de alimentos. Alcance, causas y prevención}'' donde se
estimó que el 30\% de los alimentos producidos a nivel mundial no llega
a ser consumido por las personas.

Este informe, además, muestra los porcentajes de descarte sobre el total
de la producción, realmente alarmante para los grupos de frutas,
hortalizas, raíces y tubérculos, que arrojan una pérdida de 45\%. Más
preocupante es aún si tenemos en cuenta que estos alimentos resultan
indispensables para la dieta diaria en todas las etapas de la vida.

\textbf{La necesidad de plantear consumo y producción sostenible}

Con el correr de los años, para comprender mejor este fenómeno y
encauzar acciones concretas, los organismos internacionales y también
los estados y las empresas, comenzaron a investigar el campo de acción y
a profundizar estudios de diagnóstico.

En 2015 la Organización de las Naciones Unidas tomó intervención en el
tema y lanzó los Objetivos de Desarrollo Sostenible, cuyo número 12
propone: \emph{``Garantizar modalidades de consumo y producción
sostenibles''.}

En América Latina y el Caribe, la Oficina Regional de la FAO organizó en
2014 una Consulta Regional a Expertos. La reunión convocó a personas de
todos los países que se encontraban desempeñando acciones relacionadas
con pérdidas y desperdicios desde diferentes ámbitos, y nosotros fuimos
seleccionados para participar. A partir de esto, se inició un plan de
acción regional y se conformó la Red Latinoamericana y Caribeña de
Expertos con el compromiso de avanzar en estrategias dentro de los
países y a la vez compartir experiencias entre sí.

Con este antecedente, durante los años siguientes se celebraron el
Primer y Segundo Diálogo Regional sobre Reducción de Pérdidas y
Desperdicios de Alimentos (2015 y 2016) con el propósito de comprometer
a los representantes de alto nivel de los estados a emprender acciones.
Argentina estuvo siempre presente y fue avanzando en diferentes
aspectos.

\textbf{Conceptos claves}

La comunidad internacional habla de dos conceptos que vale la pena
diferenciar para un correcto análisis: el de \textbf{pérdidas} y el de
\textbf{desperdicios}.

\begin{itemize}
\item
  \textbf{Pérdidas de alimentos -\emph{food losses}}-. Alude a la
  disminución de la masa comestible de alimentos específicamente para
  consumo humano, que se produce durante las etapas de producción,
  postcosecha, procesamiento, almacenamiento, transporte y distribución.
  Incluye los alimentos que se pierden por daños mecánicos, derrames,
  degradación, enfermedades de los animales o por descartes debido a los
  elevados estándares estéticos de calidad (tamaño, forma, color,
  textura, etc.).
\item
  \textbf{Desperdicio de alimentos -\emph{food waste}}-. Es la
  denominación aplicada a los descartes en las etapas finales de las
  cadenas durante la comercialización y el consumo en los hogares y los
  servicios de alimentación. Son productos que han alcanzado la calidad
  adecuada, pero que son desechados, incluso antes de estropearse. El
  fenómeno está relacionado con la inadecuada gestión y manipulación, y
  con los malos hábitos de compra y consumo. Las causas son generalmente
  fallas de infraestructura o cadena de frío, falta de planificación,
  estándares estéticos muy exigentes, y compras innecesarias vinculadas
  a ofertas. También a la incorrecta interpretación de las fechas de
  duración, e incluso a la actitud de algunos consumidores cuyo nivel
  económico les permite actuar con desaprensión y descartar por
  cuestiones triviales alimentos ya comprados.
\end{itemize}

Argentina emprende su camino

La revista Alimentos Argentinos, editada por el Ministerio de
Agroindustria publicó en julio de 2013 nuestra primera nota sobre
pérdidas y desperdicios de alimentos donde describimos la situación
general y tomamos algunos datos del orden nacional.

Las consultas no tardaron en llegar y esto fue el puntapié para
investigar más a fondo la información disponible y comprender que
nuestro país, como gran productor y exportador de alimentos, necesitaba
una iniciativa nacional. En primera instancia lo entendimos desde un
enfoque social y ambiental, aunque realmente está ligado al aspecto
económico porque ser eficientes es perder menos para ganar más.

Así fue que durante los años siguientes avanzamos con acciones de
divulgación como presentaciones en congresos nacionales, notas en medios
de comunicación y reuniones con contrapartes para trabajar en conjunto.

En particular se realizó un primer ejercicio de estimación que trabajó
sobre los principales sectores agroalimentarios representativos de la
actividad económica del país, en términos de producción, de
exportaciones y de importancia relativa para las economías regionales.

Más allá que fueron estimaciones preliminares y que requieren ajustes
metodológicos, el trabajo arrojó un volumen total de PDA de 16 millones
de toneladas (T) de alimentos en su equivalente primario, lo que
representa el 12,5\% de la producción agroalimentaria. Allí las
``pérdidas'' explican el 90\% del total (14,5 millones de T), mientras
que el ``desperdicio'' sólo el 10\% (1,5 millones de T).

Además, se observaron sectores como el hortícola o el frutícola con
porcentajes que ascienden al 45\% y 55\%, similares e incluso superiores
al promedio mundial (45\%). Esto coincide con apreciaciones del INTA,
donde se consigna que en Argentina sólo se consume la mitad de las
frutas y hortalizas que se cosechan, y aproximadamente el 80\% de esa
pérdida se registra en las etapas de producción, postcosecha y
procesamiento.

Entendiendo que la cuestión necesitaba una política integral, en junio
de 2015 desde el Ministerio de Agroindustria creamos el Programa
Nacional de Reducción de Pérdida y Desperdicio de Alimentos con el doble
propósito de agregar valor a los alimentos argentinos, y promover una
producción y consumo más eficiente y sostenible.

Tal como se enuncia en la Resolución 392, el objetivo del Programa es
coordinar, proponer e implementar políticas públicas, en consenso y con
la participación de representantes del sector público y privado,
sociedad civil, organismos internacionales, entre otros, que atiendan
las causas y los efectos de la pérdida y el desperdicio de alimentos.
Esto permite priorizar acciones, establecer redes de contactos y
estimular la implementación de mejoras continuas en tecnologías y en
normativa.

Nace la Campaña ``Valoremos Los Alimentos''

Para acompañar el Programa Nacional, armamos la campaña ``Valoremos Los
Alimentos'', con un mensaje positivo sobre el buen aprovechamiento de la
comida y el consumo responsable, para incentivar hábitos que reduzcan el
desperdicio de alimentos.

Desde el Ministerio, reconocemos que la única forma de avanzar en
soluciones sostenibles es trabajar con la multiplicidad de actores
involucrados para lograr estrategias efectivas y a largo plazo. Desde el
principio articulamos acciones con la Organización de las Naciones
Unidas para la Alimentación y la Agricultura, el Instituto Nacional de
Tecnología Agropecuaria y la Red Argentina de Bancos de Alimentos.

Asimismo, a partir de 2016 convocamos a las contrapartes a adherir al
Programa Nacional a través de la firma de una carta de intención. La
suscripción de esta carta implica el desarrollo conjunto y coordinado de
acciones en, al menos, uno de los tres ejes planteados, en función de
las capacidades y posibilidades de cada adherente.

Desde ese momento y hasta la fecha son más de 50 las organizaciones que
adhirieron a la iniciativa y encararon labores dirigidas a la reducción
de pérdidas y desperdicios de alimentos.

Los efectos del despilfarro

El desarrollo sostenible se ha definido como el desarrollo capaz de
satisfacer las necesidades del presente sin comprometer la capacidad de
las futuras generaciones para satisfacer sus propias necesidades.

Este concepto presenta tres elementos básicos: el crecimiento económico,
la inclusión social y la protección del medio ambiente, que están
interrelacionados y resultan esenciales para el bienestar de las
personas y las sociedades.

En tal sentido, cuando evaluamos el sistema agroalimentario actual
entendemos que debe repensarse la producción y el consumo para que sea
sostenible y eficiente.

Desde el enfoque social, tirar alimentos significa una menor cantidad de
comida disponible para aquellos que no tienen asegurado un plato a
diario. Además, las grandes pérdidas de alimentos pueden tener efectos
sobre los precios de estos a escala local y mundial.

En el plano ambiental, todo lo que se produce y no se consume, se
traduce en recursos, bienes y servicios mal utilizados, o mejor dicho
desperdiciados. Agua, suelo, combustibles, fertilizantes, mano de obra y
otros que se invierten para obtener alimentos que nadie consume.

Además, cuanto mayor es el grado de procesamiento en un alimento o
cuanto más compleja es su cadena de comercialización, mayor es el uso de
recursos.

Por ejemplo, el volumen de agua utilizada para producir la comida
desperdiciada equivale a 3 veces el volumen de nuestro tan querido Lago
Nahuel Huapi.

A esto se suma el costo ambiental representado por la emisión de gases
de efecto invernadero -generados durante todo el proceso de la cadena
alimentaria- que de este modo contribuyen inútilmente al calentamiento
global y al cambio climático. En uno de los estudios publicados por la
FAO, se destaca que cada año los alimentos que se descartan son
responsables del 8\% de las emisiones globales de GEI.

Por último, los alimentos que van a parar a la basura aumentan el
volumen de residuos, y sabemos que hoy en día el tratamiento y la
disposición final representa un grave problema, especialmente en las
ciudades.

Finalmente, desde lo económico, es claro que esta problemática implica
la ineficiencia en la producción y comercialización, que se evidencia en
mayores costos y menores rindes.

Una forma de vida

Como tantos otros temas de actualidad, parece que lo lógico y lo ético
fueran la excepción y no la regla. En este caso, cuidar los alimentos y
no despilfarrar debería ser un hábito cotidiano.

Los alimentos son la fuente de nutrientes que nuestro organismo necesita
para crecer y desarrollarse, pero además forman parte de nuestra
cultura, costumbres e historia. En todos los rincones de Argentina nos
reunimos alrededor de una mesa con comida para compartir momentos. Desde
un encuentro de familia o de amigos, hasta una reunión de trabajo, todo
lo acompañamos con alimentos. En algunas situaciones son bizcochos o
bocadillos frugales y en otros, grandes banquetes.

Entonces, si el alimento es protagonista de nuestra vida social y además
es indispensable para nuestro cuerpo y mente ¿Por qué tenemos una
actitud tan indiferente?, ¿Por qué tiramos comida aun cuando todavía se
puede consumir?

Esto no pasa sólo en Argentina. Como vimos es un fenómeno que afecta a
todo el mundo, pero que no requiere ``del mundo'' para modificarse sino
de las acciones concretas de cada uno de nosotros como miembros de la
sociedad.

Más información

Panorama de la Seguridad Alimentaria y Nutricional en América Latina y
el Caribe 2013. Hambre en América Latina y el Caribe: acercándose a los
Objetivos del Milenio. Organización de las Naciones Unidas para la
Alimentación y la Agricultura, 2013.

Gustavsson Christel Cederberg, J.; Van Otterdijk, R.; Meybeck. Global
food losses and food waste. Extent, causes and prevention. Estudio
realizado para el congreso internacional Save Food en Interpack 2011.
Düsseldorf, Alemania. Organización de Naciones Unidas para la
Agricultura y la Alimentación. Roma, 2011.

Definitional Framework of food loss. Working paper. Global Initiative on
Food Loss and Waste Reduction. Food and Agriculture Organization of the
United Nations. Rome, 2014.

Álvarez de Toledo, B.; Blengino, C.; Franco, D.; Rivas, A. Ejercicio de
estimación de las pérdidas y desperdicio de alimentos en Argentina. Área
de Sectores Alimentarios -- Dirección de Agroalimentos, Secretaría de
Agricultura, Ganadería y Pesca. Enero 2015.

TED Talks. Tristram Stuart: The global food waste scandal. 17 de
septiembre de 2012.

Stuart, T. Waste: undercovering the global food scandal. WW Norton \&
Co. 2009.

FEEDBACK. Putting a stop to global food
waste.\href{http://feedbackglobal.org/}{\emph{http://feedbackglobal.org/}}

Sitio web Alimentos Argentinos.\emph{www.alimentosargentinos.gob.ar}