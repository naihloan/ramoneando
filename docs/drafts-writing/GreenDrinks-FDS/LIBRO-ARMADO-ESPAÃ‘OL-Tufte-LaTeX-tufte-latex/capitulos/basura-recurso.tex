   \end{fullwidth}

\chapter{LA BASURA: DE UN PROBLEMA A UN RECURSO}\label{la-basura-de-un-problema-a-un-recurso}
  \begin{fullwidth}

% \begin{figure}[htbp]
% \centering
% \includegraphics[width=1.77014in,height=1.76250in]{media/image7.jpeg}
% \caption{}
% \end{figure}

\textbf{Pablo Martín Capitanelli}

Arquitecto, Fundador de Quinua Arquitectura, especialista en
Bioarquitectura.

La basura: de un problema a un recurso

\textbf{Tecnología Social}

La \emph{Tecnología Social} es un
\href{https://es.wikipedia.org/wiki/Movimiento_social}{\textbf{movimiento
social}} espontáneo que responde a una nueva línea de pensamiento
popular, que une a todas las razas y culturas del mundo tras ideales y
proyectos sociales, que van más allá de las fronteras políticas o
religiosas. El principal eje es la
\href{https://es.wikipedia.org/wiki/Comunicaci\%C3\%B3n}{\textbf{comunicación}},
que surge por la necesidad vital de crear una situación, o un estado de
las cosas, distinto del que se vive en la actualidad. Como fin, pretende
fomentar el desarrollo de hacer algo importante de la tecnología de sus
bienes y aplicación de conocimientos y tecnologías con fines netamente
sociales, absolutamente pacíficos y opuestos a los objetivos comerciales
o militares que determinan el actual avance y desarrollo tecnológico de
la humanidad.

\textbf{Primeras Experiencias}

Uno de nuestros primeros proyectos fue \textbf{\emph{Prensa\_Papel}},
cuyo objetivo era producir, conjuntamente con comunidades de
\emph{cartoneros}, paneles rígidos de papel reciclado para su
utilización en la industria de la construcción. Preparamos una
presentación ante una ONG internacional para conseguir los fondos para
desarrollar el proyecto, no fuimos seleccionados, y el proyecto quedó en
\emph{stand by}. Lejos de desanimarnos por esto, pudimos sacar algunas
conclusiones importantes; los proyectos no son un fin en sí mismos, son
un medio para relacionarse con las personas con la que uno quiere
trabajar, para conjuntamente, tratar de conseguir objetivos comunes.
Para poder llevar adelante cualquier proyecto es necesario contar con
fondos, así que por más excelente que sea la propuesta, hay que aprender
y darle importancia a la \emph{gestión de fondos}, para que se pueda
realizar.

\textbf{Quinua Arquitectura}

En el año 2003, con objetivos similares fundamos Quinua Arquitectura,
conjuntamente con el arquitecto Diego Dragotto. Trabajamos con
\emph{Grupos Productivos} y otras temáticas, hasta que en el año 2007
conseguimos fondos de la ONG AVINA, para desarrollar el proyecto
\emph{Tejido Urbano}, con cooperativas de cartoneros de Córdoba. Las
relaciones con las cooperativas se habían profundizado, nuestro
conocimiento del ``\emph{Mundo de la basura en Córdoba}'' y del
reciclaje también. \emph{Tejido Urbano}, además de cumplir su objetivo
de mejorar los ingresos para los integrantes de las cooperativas, nos
permitió profundizar y afianzar nuestra relación con ellas, fortalecer
las \emph{Comunidades Productivas} dentro de ellas y sobre todo sembrar
la convicción de que la basura podía pasar de ser un problema a ser un
recurso.

\textbf{Crecimiento a través de la experiencia}

En el año 2011, ya no se encontraba el arquitecto Diego Dragotto; había
ingresado la arquitecta María Florencia Orellana y otro grupo de
personas con quienes nos presentamos a una convocatoria del Programa
\emph{Por América}, conformada por una alianza entre el BID y empresas
privadas de distintos países latinoamericanos. El proyecto era \emph{Red
de Productores Artesanales Urbanos para una Mejor Calidad de Vida.}
Nuestra propuesta fue seleccionada y se desarrolló durante tres años con
distintas cooperativas y grupos de cartonero\emph{s} de la ciudad de
Córdoba, con el objetivo de buscar generar \emph{valor agregado} a los
distintos tipos de residuos. Se llevaron a cabo distintos productos, con
los diferentes residuos, pero lo más importante fue que durante tres
años pensamos, junto a todas estas personas que viven de y en la basura,
qué se podía hacer con ella y reafirmamos que realmente podía generarse
trabajo a partir de la misma. Profundizamos las relaciones y los
compromisos humanos, un factor fundamental para llevar adelante
cualquier tipo de proyecto. Y dentro de Quinua Arquitectura pudimos
sacar algunas conclusiones importantísimas: teníamos que tratar de
acercarnos a producciones más industriales, si bien manejábamos cada vez
más los aspectos técnicos de la basura y su reciclado, debíamos buscar
alianzas con personas que conociesen mejor la parte de ``Planificación
de Negocios'' y ``Comercialización'', para que todas estas ideas y
proyectos realmente tuvieran una incidencia significativa en el
mejoramiento de las condiciones de vida de estas comunidades.

\textbf{Nuevos encuentros}

Ya fuera de cualquier proyecto formal, seguimos trabajando con las
comunidades de cartoneros y fue en esos tiempos donde nos comenzamos a
encontrar con otras personas de otras disciplinas, pero con los mismos
objetivos, lo que veníamos buscando.

En ese momento ya estaba formado un equipo capacitado y con experiencia
en el tema de la basura y del trabajo de reciclado buscando valor
agregado con las cooperativas de cartoneros de la ciudad de Córdoba.
Inclusive ya nos reconocían como referentes en el tema a tal punto que
en el comienzo de la gestión del Intendente Mestre se nos convocó para
conformar la Mesa de la Basura, espacio participativo reuniendo a
referentes de los distintos sectores que intervienen en el tema para
debatir las diferentes propuestas y/o soluciones. Paralelamente un
emprendedor cordobés, Lucas Recalde, había experimentado con el uso de
fardos de botellas de plástico como cerramiento de construcciones
arquitectónicas. Estas pruebas lo llevaron a acercarse a la cooperativa
Los Carreros de Va. Urquiza. Por otro lado, un licenciado en economía,
Martín Fogliacco, trabajando para el gobierno de la provincia de
Córdoba, había comenzado a trabajar en la misma cooperativa en la
realización de un plan de negocios que organice y mejore sus actividades
económicas, y por consiguiente sus ingresos. Ese fue el punto de
encuentro.

\textbf{Un nuevo proyecto}

En sucesivas reuniones pudimos llegar a algunos acuerdos de base: había
tres ejes sobre los que basaríamos nuestras propuestas, buscando siempre
el equilibrio entre ellos: la sustentabilidad ecológica, la
sustentabilidad social y la sustentabilidad económica; que la propuesta
se debía acercar a la lógica industrial; que debíamos partir de las
lógicas y saberes previos que poseían la gente de las cooperativas.
Pensamos en una necesidad que lejos de estar cubierta, cada día se va
agrandando: el hábitat, la vivienda. A partir de ahí organizamos un
equipo llamado 3 Construcciones incluyendo a los arquitectos María
Florencia Giraudo y Federico Fogliacco, partes del equipo de Quinua
Arquitectura. La organización, buscando la armonía entre los tres ejes
antes descritos (de allí su nombre), desarrollará un sistema
constructivo, que tuviera buenas condiciones de habitabilidad, procesos
simples de producción y construcción, y un costo muy accesible para
poder salir al mercado a ofrecer una buena propuesta. Partimos de un
material que estuviéramos seguros del que siempre íbamos a tener
disponibilidad, el plástico, y de un saber muy incorporado en todas las
cooperativas, la producción de fardos de residuos.

A partir de allí comenzamos a desarrollar el \emph{Sistema Constructivo
MPR}, pensando todos los aspectos conjuntamente, no solo un desarrollo
tecnológico de una vivienda, sino también cómo era la mejor forma para
producirla en las cooperativas y que generara trabajo en estos sectores
marginales y que se lograra un producto muy atractivo para el mercado a
fin de asegurar la sustentabilidad económica y un buen retorno.

\textbf{Desarrollo del primer Sistema Constructivo MPR}

A partir de este punto, en Quinua Arquitectura, partiendo del
conocimiento previo y la experiencia que teníamos en
\emph{Bioarquitectura}, comenzamos a desarrollar el sistema
constructivo, con estructura de madera independiente y una envolvente
conformada por \emph{molones}, fardos de plásticos que variaron su
tamaño del que en las cooperativas estaban habituados a hacer.
Paralelamente otro grupo fue desarrollando el proceso productivo,
primeramente, adaptando la prensa hidráulica a las nuevas medidas del
\emph{molón}, para luego conjuntamente con los socios de las
cooperativas, ponerse a experimentar en esta nueva producción. Este
proceso fue liderado por el técnico Federico Brunas. Desde un primer
momento pudimos ver aspectos positivos en esta nueva producción. No
hacía falta clasificación, servían todos los tipos de plásticos, aun los
que no tenían precio de venta a granel. Y por otro lado, no hacía falta
lavarlos, muy importante en el ahorro de un recurso como el agua y en la
reducción del trabajo que esto significaba. Otro grupo estaba
desarrollando un plan de negocios basado en un sistema de \emph{micro
franquicias}, dónde cada cooperativa no era solamente proveedora de
\emph{molones}, a pesar de que se le podía duplicar lo que se le pagaba
al \emph{carrero}, (por la recolección y a la cooperativa por la
elaboración, ya que este aumento no tenía incidencia importante en el
costo final de la vivienda), sino que eran socias del emprendimiento,
por lo tanto, aparte del ingreso que ganaran por su trabajo en la
producción de los elementos del sistema constructivo MPR, en el caso de
que se vendiera una vivienda, una parte de la ganancia le corresponde a
las cooperativas; con el compromiso de destinar un porcentaje de ella al
fortalecimiento de su estructura ya sea en salud, educación,
infraestructura, etc.

Rápidamente aparecieron interesados en el producto, en las viviendas, (a
diferencia de otras veces) sin necesidad de salir a captarlos. Todo
parecía óptimo, es por ello que para no arriesgarse a que el proyecto al
poco tiempo se caiga, es conveniente realizar un análisis profundo de lo
que se ha hecho, inclusive si hace falta, como fue en nuestro caso,
consultar a especialistas en los distintos temas.

\textbf{Pausa, reflexión y nuevo desarrollo: Sistema Constructivo MPRA}

Estos no son momentos fáciles, porque hay que empezar otra vez casi de
cero, pero si uno no pierde la tranquilidad y acepta el desafío, se
pueden conseguir grandes avances, porque lo que se hizo no se pierde, a
partir incluso de las cosas que uno vio que no estaban buenas, sirven de
plataforma para conseguir nuevas propuestas desde una nueva mirada. Es
así que a partir de ese punto hubo reflexiones muy importantes. El
verdadero desafío del sistema constructivo MPR, era que las columnas de
\emph{molones} fueran \emph{autoportantes}, para prescindir de la
estructura independiente, de este modo estábamos aprovechando los
recursos al máximo y bajando considerablemente el costo. También que la
realidad de las cooperativas es distinta a la de cualquier otro ámbito
productivo, por eso hay que pensar el proceso desde su lógica propia,
desde cosas tan concretas como que no todos los días se cuenta con
energía eléctrica, o que muchas veces la gente no podía concurrir a la
cooperativa, no por no tener voluntad de trabajar, sino que el entorno
marginal, muchas veces condiciona la voluntad de las personas tratando
de sobrevivir en un medio tan precario. Así fue que se decidió diseñar
las maquinarias y herramientas de cero, para que se puedan adaptar a
estas condiciones tan cambiantes, máquinas que puedan funcionar sin
energía eléctrica, o que, simplemente acoplándole un taladro de mano,
pueda funcionar a motor. Que su peso no sea un condicionante para que
pueda emplazarse en cualquier lado, e incluso trasladarse a la vivienda
de alguien que no pueda concurrir a la cooperativa, pero que puede
producir en su casa. Buscar simplificar todos los procesos, todos los
mecanismos, toda la producción, y el montaje, porque así también se
facilita la \emph{replicabilidad}.

Desarrollamos cada uno de los eslabones nuevamente, pero no era de cero,
la visión crítica que te dan todos los desarrollos anteriores, hace que
el desarrollo sea más profundo. Diseñamos nuevamente el Sistema
Constructivo, pero esta vez con \emph{Módulos de Plásticos Reciclados
Autoportantes (MPRA)}, diseñamos las máquinas y herramientas y el
proceso productivo del taller, partiendo de las realidades concretas de
las cooperativas, rediseñamos cada eslabón del plan de negocio, para que
cada persona que participe, tenga una retribución justa, que se acerque
a un \emph{salario mínimo vital y móvil}, sistematizamos toda la
secuencia constructiva, buscando la mayor eficientización.

Un primer dato alentador fue que el costo de la vivienda de \emph{MPRA}
se había reducido a alrededor de un 60\% con respecto al de una vivienda
tradicional.

\textbf{Un proceso industrial}

En este punto y con un gran avance en comparación al punto en que
habíamos parado en el desarrollo anterior, pensamos en cuáles serían los
pasos más estratégicos a seguir, así fue que nos presentamos al Programa
\emph{Emprende Industria}, del Ministerio de Industria, Comercio,
Minería y Desarrollo Científico Tecnológico, de la provincia de Córdoba.
Le propusimos el proyecto a una fábrica de gaseosas de nuestra ciudad,
para que fuera nuestra \emph{Empresa Madrina} y comenzamos a realizar
gestiones con la Facultad de Ingeniería de la U.N.C. para utilizar su
laboratorio. Al momento de realizar todos los ensayos necesarios para
homologar el sistema \emph{MPRA}, nos reunimos con uno de los expertos
especialistas en estructuras de madera, el ingeniero José Luis Gómez,
para realizar el cálculo estructural que demostrara que el sistema
constructivo MPRA cumplía con todas las normativas requeridas para su
aprobación en cualquier municipio de Córdoba.

Fuimos seleccionados en el Programa \emph{Emprende Industria} que, más
allá de que fue muy importante para conseguir fondos para terminar de
desarrollar el proyecto, para poder calificar, fuimos reconocidos como
un \emph{Proceso Industrial}: una meta conseguida.

\textbf{Conformando redes}

Después de un tiempo de gestión, el Decano de la Facultad de Ingeniería,
nos comunicó que nuestro proyecto era de interés para su Facultad, y
seguimos dándole forma a nuestra relación a través de su Subsecretaría
de Vinculación Social, y en 2017 firmaremos un Convenio de Colaboración;
paralelamente comenzaremos a reunirnos con la Directora del Laboratorio
de la Facultad. Y, por último, el ingeniero José Luis Gómez, terminará
de realizar los cálculos donde muestra que el sistema constructivo
\emph{MPRA} cumple con todas las normativas necesarias para su
aprobación en cualquier municipio. Paralelamente a todo esto, se tomó
una decisión fundamental: teníamos que armar un taller y hacerlo
funcionar además construir el prototipo de una primera vivienda, y
lograr terminar de pulir los últimos aspectos para obtener algunos datos
y comprobar algunas cosas, que solo se pueden hacer de forma empírica.

\textbf{Primer prototipo}

En este momento estábamos en la recta final del proceso, habíamos
corregido, cambiado y reafirmado un montón de aspectos gracias a la
experiencia en el taller, que produjo todos los elementos necesarios
para la realización de la primera vivienda. De igual manera, la
construcción de un \emph{Primer Prototipo}, nos sirvió para ajustar,
corregir, cambiar y reafirmar un montón de detalles del sistema
constructivo. Éste está en su última etapa de construcción y también nos
brindó valiosa información fehaciente para afirmar cuál será el costo de
construcción de una vivienda, que resultó siendo más positivo que los
cómputos previos, ya que pudimos verificar que el costo de una vivienda
sería alrededor de entre un 50 y un 60\% con respecto a una vivienda
tradicional, teniendo la premisa clara de que todas las terminaciones
fueran de primera calidad (pisos de cerámico, aberturas de aluminio,
revoques finos con color, etc.) ya que sabemos que es fundamental
responder al \emph{imaginario colectivo} de la vivienda propia. Esta
construcción trajo aparejado, además, que se despertara interés en los
medios de comunicación, lo que produjo que se realizaran numerosas notas
en medios gráficos, radio y televisión, e inmediatamente se llamó la
atención de muchas personas en este sistema constructivo MPRA para la
construcción de su vivienda.

\textbf{Volver a los grupos productivos comunitarios}

A partir de gestiones realizadas por el licenciado Martín Fogliacco,
hemos sido convocados por el Programa \emph{Creer y Crear} del gobierno
nacional, conjuntamente con el gobierno provincial, como tutores para
llevar adelante proyectos productivos, que en nuestro caso, se tratarían
de la instalación de tres talleres de producción de los componentes del
sistema constructivo \emph{MPRA}, en tres cooperativas distintas de
cartoneros de la ciudad de Córdoba: Los Carreros de Va. Urquiza, La
Esperanza de Va. La Lonja y la Fundación Moviendo Montañas de barrio
Müller. El gobierno de la provincia nos ofrece sumar a esta iniciativa
el plan Vida Digna para que con este sistema constructivo y los insumos
producidos en estos talleres se mejoren las viviendas de los mismos
vecinos de la villa.

\textbf{Nuevos desafíos}

En este punto se ha llegado a un acuerdo dentro del equipo, donde Quinua
Arquitectura conjuntamente con el licenciado Fogliacco se abocarán a
esta rama social del emprendimiento, sumándole la gestión realizada con
algunos sindicatos y cooperativas interesados en realizar planes de
viviendas con el sistema MPRA, que serían futuros clientes de los
talleres antes mencionados, con un carácter enfocado en proyectos
colectivos y dentro de la economía social. Por otro lado, Lucas Recalde
seguirá una rama relacionada al emprendedurismo, siguiendo con la
producción del sistema constructivo y en la búsqueda de nuevos mercados
para aumentar las posibilidades de venta.

Lo importante de este acuerdo es que seguiremos compartiendo un banco de
conocimientos común, tanto en lo técnico como en la gestión, para seguir
perfeccionando y fortaleciendo al sistema en todos sus aspectos.

\textbf{El futuro}

Hemos llegado a este punto, tenemos el desafío de terminar de concretar
todas las metas planteadas, y sobre todo, lograr que el emprendimiento
comience a funcionar económicamente, para que al crecer la demanda, se
pueda ir incorporando la mayor cantidad de cooperativas de
\emph{cartoneros}, y mejorar la situación laboral de este sector tan
relegado; así también de que muchas personas puedan acceder a su
vivienda casi a la mitad del costo con el que lo pueden realizar en la
actualidad, sin resignar calidad constructiva, y que cada vez menos
plástico vaya al enterramiento sanitario y pase de ser un problema a ser
un \emph{recurso}.