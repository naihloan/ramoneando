  \end{fullwidth}
   

\chapter*{¡RESERVA SAN MARTÍN: \mbox{DONDE CÓRDOBA RESPIRA!}}
% \addcontentsline{toc}{subsubsection}{\protect\numberline{\thesubsubsection} ¡RESERVA SAN MARTÍN: DONDE CÓRDOBA RESPIRA!}
\addcontentsline{toc}{chapter}{¡RESERVA SAN MARTÍN: DONDE CÓRDOBA RESPIRA!}


\label{reserva-san-martuxedndonde-cuxf3rdoba-respira}
  \begin{fullwidth}

% \begin{figure}[htbp]
% \centering
% \includegraphics[width=1.76042in,height=1.76250in]{media/image11.jpeg}
% \caption{}
% \end{figure}

\textbf{Rita Stanislavs}

Vicepresidenta de la Asociación Amigos de la Reserva Natural San Martín.

Ing. Química. Especialista en Ingeniería Ambiental.

Docente de la Facultad de Ciencias Exactas~Físicas~y Naturales~(U.N.C) y
de la Universidad~Tecnológica~Nacional.

\emph{``Conservar para educar, educar para conservar''}

% \hypertarget{chapter}{\chapter{~}\label{chapter}}

Asociación Civil Amigos de la Reserva Natural San Martín

\textbf{Y nos juntamos a defender la plaza del barrio}

Todo empezó en la ciudad de Córdoba, en mayo de 2009. Una nota del
diario anunciaba obras en el sector noroeste de la ciudad: un puente
conectaría Valle Escondido en la ribera Sur del Río Suquía con la margen
Norte para llegar a la Av. Recta Martinolli. Ese puente desembocaba en
la única plaza del barrio Villa Argüello en una zona totalmente
residencial.

Mientras los vecinos nos juntábamos en la plaza para ver qué actitudes
tomar, se nos acercó Cacho Salzgeber, guardaparque del ex Parque San
Martín, que estaba ahí nomás, cruzando el río. Él nos dijo que si bien
defender una plaza era importante, más lo era preservar el último
relicto de bosque nativo que quedaba en la ciudad de Córdoba. Muchos de
nosotros ni lo conocíamos a pesar de tenerlo a menos de 500 metros, río
por medio.

Lo primero fue hacer una visita guiada por Cacho, oportunidad en la que
además vimos un zorro\ldots{} Héctor, quien luego llegaría a ser
presidente de la Asociación, estaba tan emocionado que con sus gritos
espantó al curioso animal. A partir de allí todos nos enamoramos del
lugar.

A pesar de estar casi en un estado de abandono, vimos el valor y el
potencial del lugar por lo que nos decidimos luchar. Luego
descubriríamos que varios ``desarrollistas'' también pensaban lo mismo,
pero no con el fin de preservación.

\textbf{Y nos pusimos a trabajar}

Eran 114 hectáreas de bosque nativo en pleno ejido municipal, con casi
200 especies de aves, mamíferos, reptiles y anfibios que uno no
imaginaría en una ciudad, y con un camping municipal formando parte del
parque.

Pero rescatar el lugar del estado de abandono en que se encontraba se
requeriría de mucho esfuerzo: jornadas de limpieza, de forestación y lo
más difícil, convencer a quienes usaban el parque como pista de
entrenamiento para motos, cuadriciclos y vehículos todoterreno fueron
actividades prioritarias. Les explicamos que estaba prohibido. Algunos
lo aceptaron amablemente, otros nos insultaban, y no faltó quien nos
tiró el vehículo encima. Pero a pesar de las dificultades el parque
comenzaba a recuperarse.

Logramos incluso gestionar que Piñón Fijo, el payaso, Chichilo Viale,
humorista, y Diego Osella, basquetbolista, grabaran un spot para radio,
juntamente con una campaña gráfica en las calles defendiendo el parque
como pulmón de la ciudad.

La gestión municipal tenía graves deficiencias y problemas económicos,
al punto que llegamos a adquirir y donar una batería a la Dirección de
Defensa Civil que debía patrullar el Parque procurando evitar incendios.

\textbf{Y había que interactuar con leyes y políticos}

En ese año el parque sufrió siete incendios, todos intencionales.
Evidentemente había intenciones de que ese lugar no se preservara.
Además de las intenciones de hacer obras viales, existían intereses
inmobiliarios. Había que darle forma a la defensa.

Decidimos crear la Asociación Civil Amigos de la Reserva Natural San
Martín, ya que el contar con personería jurídica nos ayudaría. Con mucha
difusión se logró interpelar a funcionarios municipales en el Concejo
Deliberante, que con pobres explicaciones dijeron que en concreto lo de
las obras viales era solo una ``idea''. Pero vimos que el Parque
necesitaba una mayor protección, por eso junto a otras ONG nos pusimos a
redactar un proyecto de ordenanza que jerarquizara el parque llevándolo
a categoría de Reserva, adquiriendo así un mayor nivel de protección.
Con entusiasmo se suplió la falta de experiencia. Y finalmente Graciela
Villata concejal por el Frente Cívico y vecina del barrio, tomó el
proyecto y lo presentó. Como la gran mayoría de los concejales no
conocían el Parque, les alquilamos una Trafic y los llevamos a
recorrerlo.

También llevamos especialistas al Concejo para que informen la
importancia de preservar espacios nativos. Liliana Argüello de la
Universidad Nacional de Córdoba y Carlos Chávez, naturalista, cumplieron
brillantemente la tarea de divulgación. Aprovechando la visita de Chávez
a Córdoba organizamos la presentación de sus libros en la librería el
Ateneo. Lamentablemente un par de años después de creada la Reserva
falleció, pero un algarrobo plantado en su honor lo sigue recordando. No
descuidamos tampoco la comunicación social, así es que para que los
vecinos de Córdoba conocieran la existencia del rico ecosistema nativo,
y el riesgo que corría, llevamos la problemática a encuentros, talleres,
e incluso, con una masiva difusión de afiches en el centro de la ciudad,
costeada con el apoyo de una tarjeta de crédito.

\textbf{Y se creó la Reserva}

Finalmente tras mucho ``peregrinar'' detrás de los concejales, el último
día de sesión el 30 de Noviembre del 2009 a las 21 hs se votó por
unanimidad la creación de la Reserva Natural Urbana San Martín. El
objetivo era y es ``Conservar para Educar y Educar para Conservar'':
destacar los servicios ambientales que brindan los espacios verdes
nativos (ser reservorio genético que favorece la persistencia de la
biodiversidad animal y vegetal nativa, reducir del ruido y la
escorrentía de agua superficial y conservarla humedad, regular del
microclima, brindar beneficios de tipo psicológico, social y cultural,
mejorar de la calidad del aire, brindar un hábitat para polinizadores,
etc.).

\textbf{Y nos premiaron}

Después de meses de mucha actividad, fuimos reconocidos. La Voz del
Interior postuló al entonces presidente de la Asociación como candidato
a ``Cordobés del año''. El Centro de Estudios Alicia Moreau nos
distinguió por la ``destacada labor en la construcción de ciudadanía'' y
el ``Dubai International Award for best practices'' nos seleccionó por
el trabajo realizado. También la Vice Intendencia de la ciudad de
Córdoba nos entregó la distinción ``Agustín Tosco'' por nuestra labor en
defensa de la Reserva San Martín.

\textbf{Y había que seguir trabajando}

Se consiguió mucho a partir de la creación de la Reserva. Aumentó la
cantidad de guardaparques, lo cual permitía más visitas guiadas para
escuelas (en la actualidad totalizan unos 5.000 alumnos por año), se
gestionó un vehículo para que pudieran recorrer y vigilar el predio, se
solicitó la colaboración de la Fundación Arcor para material didáctico:
libros de aves autóctonas para colorear, prismáticos, brújulas, carpas,
etc. También desde el Senado de la Nación se consiguieron fondos para
comprar tanques de agua para estar preparados ante incendios. Donamos un
equipo electrógeno. La empresa Cantesur donó árboles y mesadas de metal
para el vivero de nativas que se fue armando en la Reserva. También se
siguieron realizando las actividades de siempre: jornadas de
forestación, limpieza, germinación de nativas, etc. Y se incorporaron
otras como concursos fotográficos, uno por año desde el año 2010,
participación en congresos, charlas a escuelas, participación en Green
Drinks Córdoba, etc. También todos los años para el último fin de semana
de agosto se organiza una barrileteada donde los propios niños arman sus
barriletes que luego remontan.

Además, como el límite Oeste no estaba cercado, previo a gestionar ante
la autoridad municipal la confirmación del límite catastral, se procedió
a delimitarlo con postes de cemento alambrado de tres hilos y varillas
de madera. Se contrató personal para realizar los pozos. Los postes
estaban instalados en otro sector de la Reserva, donde otrora se pensaba
en hacer un Jardín Botánico, mudos testigos de la inacción municipal,
dado que nunca el proyectado Jardín tuvo cerramiento ni plantas. Eso sí,
no faltó la placa conmemorativa, ni el olvidado mástil. Con colaboración
de guardaparques y vecinos, se extrajeron y trasladaron los postes, y se
realizó el cerramiento de más de 700 metros, habiendo colaborado incluso
grupos de niños Boy Scout.

\textbf{Y se desarrolló el programa de Vigías voluntarios}

En el año 2011 un incendio (intencional, claro) destruyó unas 30
hectáreas de la zona alta de la Reserva y otro en 2013 también
intencional volvió a quemar la zona Oeste. Si bien esta vez se estaba
mejor preparado para prevenirlo, para que esto no se repitiese, en las
épocas de sequía se organiza el programa ``Vigías Voluntarios'' que
consiste en convocar a la ciudadanía a que participe los fines de semana
y feriados, que es cuando hay menos guardaparques, recorriendo la
Reserva y dando el alerta temprano ante cualquier señal de alarma (humo,
vandalismo, actitudes sospechosas de algún visitante). Desde que se
implementó esta iniciativa no ha habido más incendios durante los fines
de semana.

\textbf{Y siguieron los problemas y peligros}

Un día los guardaparques se encuentran con máquinas ajenas a la Reserva
intentando talar la ribera del canal maestro sur, límite de la Reserva.
Correspondían al emprendimiento Santina Norte del grupo Edisur, y
querían hacer un acceso secundario al country, contando ya con la
autorización municipal. Es que lo que en planos era una calle pública,
límite de la Reserva, en realidad era un bosque con árboles nativos de
más de 200 años muy bien preservados. Además Edisur planeaba adueñarse
del Canal poniendo el alambrado que demarcaba el límite del barrio
cerrado del lado de la Reserva. Se organizó una gran movida con
colegios, intervenciones artísticas en carácter de denuncia al grupo
desarrollista que finalmente dejó de lado su proyecto de acceso. La mala
publicidad de talar bosque de una empresa que se jacta de su
``responsabilidad social'' era demasiado alta.

No sólo los desarrollistas amenazaban la Reserva, la propia
Municipalidad quería que la ``calle Miguel Lillo'', que accede a la
Reserva desde el puente Los Carolinos fuese más transitable. Para ello
puso máquinas a trabajar para emparejarla, ampliar su ancho,
consolidarla con ripio, y reducir sus curvas. Era un paso previo al
asfalto. Y eso sería el fin de la Reserva: mucho tráfico implicaba más
animales atropellados al acercarse al río a beber agua, más riesgo de
incendio, más basura, más ruido. La Asociación presentó una acción de
amparo, alegando que la ``calle Lillo'' es en realidad un camino interno
de la Reserva y que solo debería ser usada para acceso a la misma. En 1°
y 2° instancia la Justicia dio la razón a la Asociación ordenando a la
Municipalidad que se abstenga de realizar dichas tareas. Todavía el
proceso está en el Tribunal Superior de Justicia, con motivo de otro
recurso más de la Municipalidad. Con ello se logró, hasta el momento,
que no sea utilizada para traslado en vehículos a motor de un lugar a
otro de la Ciudad.

En predios de la Reserva se encuentran también las ruinas del Molino de
Hormaheche, antiguo casco de estancia de valor histórico y patrimonial.
A pesar de los numerosos reclamos las ruinas siguen estando\ldots{} ¡en
ruinas! Y desmejorando con cada año que pasa.

Otro peligro es el proyecto de continuar la Costanera. Ante cada
elección municipal la Asociación invita a los candidatos a intendentes a
que firmen un compromiso de defensa de la Reserva. Casi todos los
candidatos firmaron menos el actual intendente Mestre, quien tiene el
sueño de continuar la Costanera con la excusa de que es necesaria una
mayor conectividad vial.

En respuesta a esto y juntamente con la Red ciudadana ``Nuestra
Córdoba'' elaboramos un proyecto vial alternativo que implicaría un
alivio para el tránsito sin dañar la Reserva y conectaría la zona Oeste
y la Noroeste. Se presentó el proyecto ante autoridades competentes sin
tener respuesta hasta el momento.

\textbf{Y participamos en ámbitos académicos y de participación
ciudadana.}

Como la reserva es un ámbito educativo por excelencia firmamos con la
Facultad de Ciencias Exactas, Físicas y Naturales de la Universidad
Nacional de Córdoba un convenio marco para realizar diferentes tareas,
en pos de la educación ambiental. Concretamente con el CERNAR se
organizó la ``Diplomatura Ambiente, Territorio y Conservación'', donde
con docentes de la Universidad y la organización administrativa de la
Asociación ya van dos promociones de alumnos formados en esta
problemática ambiental.

También desde la Asociación dictamos cursos para docentes de la
Universidad Libre del Ambiente, de distintos niveles, sobre las Reservas
naturales como espacios educativos.

En el año 2015 organizamos en la Reserva el ``1° Encuentro de Reservas y
Áreas protegidas de la provincia de Córdoba''. Siendo ya anteriormente
parte de la Coordinadora de las Sierras Chicas, en dicho encuentro se
avanzó en la creación del Corredor biológico de las Sierras Chicas, en
el cual se pretende crear una continuidad entre distintas áreas
protegidas desde Ascochinga hasta la ciudad de Córdoba.

En el año 2016 para el ``2° Encuentro de Reservas y Áreas Protegidas''
tuvimos la satisfacción de contar con el apoyo del CONICET a través del
Instituto de Ecología Animal, lo cual significó contar con fondos para
el evento pero por sobre todo el reconocimiento de que organizamos
seriamente los eventos y por eso el aval de instituciones de peso
científico.

También participamos en la organización de la ``Cruzada de las Sierras
Chicas'', que se realiza todos los años a fines de noviembre. Y
actualmente estamos participando activamente en todas las marchas en
defensa del bosque nativo.

\textbf{Y pudimos, se puede y podremos}\ldots{}

Desde 2009 el Parque San Martín es Reserva Natural Urbana. Los incendios
la dañaron pero ya el bosque está renaciendo. Renovales de más de dos
metros de altura crecen donde solo había cenizas. Nuevas generaciones
con más conciencia ambiental seguirán defendiendo este espacio, para que
nuestro bosque siga creciendo y protegiendo plantas, animales y personas
que amen la naturaleza.

\ldots{} En la Reserva San Martín, donde Córdoba respira\ldots{}