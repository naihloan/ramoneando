\documentclass[a4,twoside]{article}

\usepackage{blindtext} % Package to generate dummy text throughout this template 

\usepackage[sc]{mathpazo} % Use the Palatino font
\usepackage[T1]{fontenc} % Use 8-bit encoding that has 256 glyphs
\linespread{1.05} % Line spacing - Palatino needs more space between lines
\usepackage{microtype} % Slightly tweak font spacing for aesthetics

% \usepackage[english]{babel} % Language hyphenation and typographical rules
\usepackage[T1]{fontenc}
\usepackage[utf8]{inputenc}
\usepackage[spanish]{babel}

\usepackage[hmarginratio=1:1,top=32mm,columnsep=20pt]{geometry} % Document margins
\usepackage[hang, small,labelfont=bf,up,textfont=it,up]{caption} % Custom captions under/above floats in tables or figures
\usepackage{booktabs} % Horizontal rules in tables

% \usepackage{lettrine} % The lettrine is the first enlarged letter at the beginning of the text

\usepackage{enumitem} % Customized lists
\setlist[itemize]{noitemsep} % Make itemize lists more compact

% \usepackage{abstract} % Allows abstract customization
% \renewcommand{\abstractnamefont}{\normalfont\bfseries} % Set the "Abstract" text to bold
% \renewcommand{\abstracttextfont}{\normalfont\small\itshape} % Set the abstract itself to small italic text

\usepackage{titlesec} % Allows customization of titles
\renewcommand\thesection{\Roman{section}} % Roman numerals for the sections
\renewcommand\thesubsection{\roman{subsection}} % roman numerals for subsections
\titleformat{\section}[block]{\large\scshape\centering}{\thesection.}{1em}{} % Change the look of the section titles
\titleformat{\subsection}[block]{\large}{\thesubsection.}{1em}{} % Change the look of the section titles

\usepackage{fancyhdr} % Headers and footers
\pagestyle{fancy} % All pages have headers and footers
\fancyhead{} % Blank out the default header
\fancyfoot{} % Blank out the default footer
\fancyhead[C]{Dictamen para trabajo final de la Maestría en Antropología [UNC] $\bullet$ \today }%$\bullet$ Vol. XXI, No. 1} % Custom header text
\fancyfoot[C]%[RO,LE]
{\thepage} % Custom footer text

\usepackage{titling} % Customizing the title section

\usepackage{hyperref} % For hyperlinks in the PDF

%----------------------------------------------------------------------------------------
%	TITLE SECTION
%----------------------------------------------------------------------------------------

\setlength{\droptitle}{-4\baselineskip} % Move the title up

\pretitle{\begin{center}\Huge\bfseries} % Article title formatting
\posttitle{\end{center}} % Article title closing formatting
\title{Dictamen para trabajo final de la \\ Maestría en Antropología [UNC]} % Article title
% \author{%
% \textsc{Benjamín Juárez}\thanks{\today} \\[1ex] % Your name
% % \normalsize University of California \\ % Your institution
% % \normalsize \href{mailto:john@smith.com}{john@smith.com} % Your email address
% %\and % Uncomment if 2 authors are required, duplicate these 4 lines if more
% %\textsc{Jane Smith}\thanks{Corresponding author} \\[1ex] % Second author's name
% %\normalsize University of Utah \\ % Second author's institution
% %\normalsize \href{mailto:jane@smith.com}{jane@smith.com} % Second author's email address
% }
\date{ %\today
} % Leave empty to omit a date
% \renewcommand{\maketitlehookd}{%
% \begin{abstract}
% \noindent \blindtext % Dummy abstract text - replace \blindtext with your abstract text
% \end{abstract}
% }

%----------------------------------------------------------------------------------------

\begin{document}


% Print the title
\maketitle

\subsubsection*{Título de la tesis: 
``¿Cómo convertirse en un loco?''  \\
Una etnografía sobre la carrera de los atletas del Kratos Hard Cross en Santiago del Estero}
\label{tuxedtulo-de-la-tesis-cuxf3mo-convertirse-en-un-loco-una-etnografuxeda-sobre-la-carrera-de-los-atletas-del-kratos-hard-cross-en-santiago-del-estero}

\subsubsection*{Nombre del/a estudiante: Fernando Ezequiel Kalin}\label{nombre-dela-estudiantefernando-ezequiel-kalin}

\paragraph{Director/a: Dra. María Inés Landa}\label{directora-dra.-maruxeda-inuxe9s-landa}

\paragraph{Co Director/a: Mg. Fabiola Heredia}\label{co-directora-mg.-fabiola-heredia}

\paragraph{Fecha de recepción del manuscrito: 2020-03-05}\label{fecha-de-recepciuxf3n-del-manuscrito-2020-03-05}

\subsubsection*{Nombre del miembro del tribunal: Benjamín Juárez}\label{nombre-del-miembro-del-tribunal-benjamuxedn-juuxe1rez}

% \hrule
% 
% \bigskip

Por favor, emita sus opiniones, valoraciones y todos los comentarios que
estime pertinentes sobre los siguientes puntos:
\section*{1. Runners \& The City}

The total number  of runners moving through any space seems to be an independent flow from the rest of the city's circulation. This is true in the sense that green spaces are mainly designed for leisure and as traffic-free zones. On the  other hand,  however, it is not quite true that runners are independent of other flows because car-traffic and other types of non-running traffic cross the runners’ way and, hence, make them stop: breaking runners' momentum%
\footnote{ETTEMA, Dick. "Runnable Cities. How Does the Running Environment Influence Perceived Attractiveness, Restorativeness, and Running Frequency?" \textit{Environment and Behavior}. Pp. 1-21. 2015. P. 17.}.
Runners, just as all others, depend on getting available paths as they go. This has two major implications:

\begin{itemize}
 \item The need for paths to move freely in a city
 \item No two objects/people can be in the same place at the same time 
\end{itemize}

\subsection*{The need for paths to circulate in}

This has a huge dimension in which non-humans get into play. For each space that is used in the city one could follow a science studies method: to determine all the objects and people that come into action to deliver a single object. The generic city as a civilized construction always has a set of layers upon which it has been built: be it an arid,  rocky, or damp or even forest-like, or any other kind of environment there are ways of setting in. Humans have customized spaces for millenniums. Only the past couple of centuries, at the most, have taken into account the use of delimited areas of public space for new purposes such as leisure.

\subsection*{The need to share space}

True as it may be, this last point seems to be overlooked in today's flawed auto-mobility system%
\footnote{SHELLER, Mimi; URRY, John (eds.). "The new mobilities paradigm". In \textit{Environment and Planning}. volume 38, pages 207-226, 2006.
%SHELLER, Mimi; URRY, John (eds.). \textit{Mobile Technologies of the City}. Routledge. 2006.
}:
not only do cars (and drivers) burn fuels, and leave a lasting carbon footprint, but  private vehicles can become  quite impractical with the normalcy and abundance of traffic jams as well. LeCorbusier, in his Athens Letter (1933), settled the four main modern uses of urban space: inhabiting, working, circulating and recreating. Granted that this view has a somewhat non-layering of
functionalities, and an oversimplification of uses; however, it was intended to take into account city livability for human beings, hence prioritizing the housing and green areas on urban planning. Also, transportation was the least considered element, in a period where automobile expansion numbers (an \textit{overpopulation} of non-humans, so to speak) had only just recently begun. In the XXIst  century, this seems to be a much more critical issue, where these old proposed functions have %, at least generally speaking, 
nearly collapsed. How do runners find non-occupied paths in such an overflowed system?

\subsection*{The need to share times of use}

The physical environment is not used at all times in the same way. Social space has areas in which one acts among other people; and others, in which this presentation is left aside: this is what has for long been called the front and back regions of human conduct, also well known as front-stage and backstage%
\footnote{GOFFMAN in HANNERZ, Ulf. “The City as Theater: Tales of Goffmann”. In Exploring the city: inquiries toward an urban anthropology. New York, NY: Columbia University Press, 1980. P. 206.}.
So attention is shifted from one \textit{stage} %situation 
to the other. 

It could be arguable that, of the classic functions presented by LeCorbusier, three of them are to be pursued as part of social and even animal life: working, sleeping and wandering. Transportation, even if exaggerating and stretching the argument a bit too far, as a means to an end has no real function. It seems that all time lost in traffic is time in the backstage with no actual point. However, runners do seek to transport themselves, but with a whole other meaning, closer to leisure in their free time (even \textit{serious leisure}), or even the mental-rest aspect of sleep time.
 % investigación ORIGINAL, inédita y de carácter personal %% NO ES PERSONAL
\clearpage
\section*{2. Auto-ethnography}

The plan of work proposed here sets axis on which to develop future ideas, these axis being: 
affect,
body, 
and materiality.
These \textit{sensitizing concepts} (rather than restrictive prescriptions) shall be guiding points to suggest directions where to look at, as germs of analysis on how and where to collect information. Data finding also relies on the researcher's agenda: "What sorts of patterns one is looking for depends, of course, on research focus and theoretical orientation". Benefits of in-field immersion include not only direct access in general but additionally to non-structured conversations in which "[unusual participant terms] may stress theoretically important or interesting phenomena". In the same vein, concepts may also be, alternatively, "observer-identified"%
\footnote{HAMMERSLEY and ATKINSON. \textit{Ethnography: principles in practice}. 3rd ed. London; New York, NY: Routledge, 2007. P. 164 ("Sensitizing concepts" is Blumer's), 163.}.

The axial concepts are not %be used as fixed tautologies 
to give a taken-for-granted understanding of behaviors. The approach here is first \textit{exploratory}, rather  than explanatory. The deeper understanding of behaviors and use of tools, resources and knowledge %in general/
on the whole, %shall be developed later 
shall come later, during research. The intention is first to gather data, concepts, and a series of insights from in-field work.

Ultra-running has a certain tension in the way it connects participants with people from the outside social worlds.

\begin{itemize}
 \item On one side, it is an ultimately public activity, runners are exposed to permanent contact with other runners (and non-runners as well) in the open, and races depend on a wide number  of actors, both participating and non-race related: in sum, a very wide orchestrated and coordinated social activity.
 \item On the other side, ultra-running entails a certain \textit{Loneliness of the long distance runner}%
 \footnote{Short story by Alan Sillitoe, published in 1959.}. 
 Running ultra distances may well be one of  the most \textit{outdoor} activities or sports. It involves several hours, even days sometimes "out in the  open", amongst  almost untouched nature and wild green spaces afar from a city in cross-country races. And in training season, even in city context: the silent early night-to-dawn moment (from 4 to 6 am) is when nearly no ordinary person is going about, and birds have not even began to chirp. As well as with lone spaces, running involves far many solitary moments in which runners get to collect themselves and revolve in their thoughts, the bareness of the surroundings, and at many flowing times:  
 think of nothing and seize the moment.  
 %not think in anything and be in the moment.
\end{itemize}

The \textit{in situ} work is intended to grasp these two areas (intimate-personal; and social-network-dependent) in ultra-running: the first, during training; and the second, during specific ultra-running events.

\begin{enumerate}
 \item The first aspect, training, is to be dealt with  through auto-ethnography, not as a biographical account, but as a means to grasp the main topics developed. Many of the available material on ultra-running in text and video documentary depict narratives from the sole perspective of runners themselves, in first person, and how they prepare for their practices with different styles of running and post practice cool downs and stretching as well as general nutrition and resting time. The researcher may well take a similar approach without being an outsider of common practice in this social world.
 
 \begin{quote}
  Gertrude Kurath (1960) recommended ethnographers to "learn the movements" and Adrienne Kaeppler (1978) proposed that ethnographers learn certain movements and  receive instructions on  what is done "incorrectly", or "differently" with a methodology that would allow for better understanding.
  %to understand better. 
  [José Bizerril has argued that the practical formation of the researcher has its advantages.] This knowledge allows  access to aspects of the research topic that otherwise would pass unnoticed if only done with a distant approach based on observation and interview. [the experiential dimension makes it possible to gain entry to the experience and] "to the psycho-physical and -why not-, to the spiritual states that that this experience triggers%
  \footnote{ASCHIERI, Patricia. "Hacia una etnografía encarnada: La corporalidad del etnógrafo/a como dato en la investigación". X RAM- Reunión de Antropología del Mercosur. Córdoba, Argentina, 2013. P. 16. My translation.}.
 \end{quote}
 
 Of course,  auto-ethnography may work with a potential source for bias, but at the same time provides both the most inner side view possible, and reveals the speaker's interests, perspectives and preconceptions; to which one can always add contrast with other references to compare and find the most reliable common ground%
 \footnote{HAMMERSLEY and ATKINSON. \textit{Ethnography: principles in practice}. 3rd ed. London; New York, NY: Routledge, 2007. P.%164, %("Sensitizing concepts" is Blumer's), 
 124.}.
 
 \item On the second aspect, on racing events, there is very little material in academic research on events from a qualitative approach. There is scarce material, and when so, only done through surveys or measurement based. Hence, the importance to move forward. Some of the key features of an \textit{ethnographic approach} are taken into account in the present proposal: to prioritize the insider perspective highlighting the experiential, an active immersion in the field during a reasonable amount of time, minimal interference to gather data to be triangulated%
 \footnote{HOLLOWAY, Imma; BROWN, Lorraine; and SHIPWAY, Richard. "Meaning not measurement: Using ethnography to bring a deeper understanding to the participant experience of festivals and events". \textit{International Journal of Event and Festival Management}. Vol. 1 Nº 1, 2010. Pp. 75-76.}.
 And not to focus on measuring variables, but rather on \textit{collecting and constructing new variables} to build up ever more complex concepts: this adds nuance to the understanding of the phenomenon, and provides material to suggest new questions and aspects to be worked on%
 \footnote{BECKER, Howard S. \textit{What About Mozart? What About Murder? Reasoning From Cases}. The University of Chicago Press, Chicago, 2014. Pp. 13-14, 18.}.

\end{enumerate}


 % trabajo preciso y LEGIBLE en cuanto al estilo %% FALTA RITMO
\subsection*{3) ¿Cuál es el propósito de la tesis? ¿Lo cumple? 
(Se solicita evaluar la formulación del problema, y la adecuación de la metodología y de los ejes de análisis)}\label{cuuxe1l-es-el-propuxf3sito-de-la-tesis-lo-cumple-se-solicita-evaluar-la-formulaciuxf3n-del-problema-y-la-adecuaciuxf3n-de-la-metodologuxeda-y-de-los-ejes-de-anuxe1lisis}

La pregunta del título es:
``¿Cómo un iniciado en el Crossfit se convierte en un Loco?''.

Esta pregunta es respondida por una serie clara de pasos en los que se detalla el desarrollo de los atletas.
Sí convendría delimitar la categoría Loco en lo que significa para los atletas y no absolutizar el término cómo único identificador de legitimación.
Esto es, ¿un entrenado puede ser loco antes de iniciarse en la carrera de Crossfit?
¿Puede un atleta, como pasa en muchos deportistas profesionales, ser un representante serio del box, sin ser loco?
Estas preguntas no buscan negar la fuerza de la idea de loco, sino buscar modos de articulación con la vivencia de los atletas, y para hacer más denso el concepto de loco, así como las legitimaciones en la carrera de los atletas en el box, y las reconfiguraciones en sus vidas individuales.

La pregunta ``¿Cuáles son los pasos que caracterizan a la carrera del loco?'' quedó claramente delimitada en los capítulos de la tesis. Lo mismo se puede decir de la pregunta `` ¿Cómo se consagra un loco al interior de Kratos Hard Cross?''.

Más atención se podría apuntar a responder con descripción y análisis más detallado sobre las otras preguntas:
``¿Cómo es la distribución espacial de este box y de qué manera opera sobre los cuerpos de los miembros?'',
``¿Cómo se construyen relaciones de pertenencia al interior de Kratos Hard Cross?''.
Acá también se podría apoyar más en Goffman, como se propone al inicio, en vez de ir hacia Foucault.
En caso de usar Foucault se puede argumentar de qué manera se articula con Goffman. % PROPÓSITO de la tesis? ¿Lo cumple?
% evaluar la formulación del problema, y la adecuación de la metodología y de los ejes de análisis
\subsection*{4) ¿El manejo de la teoría y de la bibliografía es
relevante, actualizada y suficiente? Si no lo fuera, indicar las
dificultades.}\label{el-manejo-de-la-teoruxeda-y-de-la-bibliografuxeda-es-relevante-actualizada-y-suficiente-si-no-lo-fuera-indicar-las-dificultades.}

% TEORIA
% \subsection{Teoría}\label{teoruxeda}

Se empieza el trabajo con un marco teórico basado en Becker, Goffman y el trabajo de Alejandro Rodríguez.
Para el concepto de comunidad, se menciona a Rose
y se lo cita en el texto para confirmar que hay comunidad porque hay esfuerzos conjuntos, valores, integración.
Cuestionar los aspectos en los que sí hay o en los que no hay podría darle mayor cuerpo a la descripción.
Tal vez sea una comunidad en un sentido restringido: ¿son amigos? ¿colegas? ¿compañeros de entrenamiento? 
¿algunos se ignoran fuera del box? 

(* Ver, explicar el caso de la ex pareja de Cacho y otros podría dar 
mayor dimensión al concepto en juego. ¿Qué pasó con las mujeres que se mencionan en el texto (Malena, Fanny, Cris), 
y no aparecen en el anexo?)

El texto toma también consideraciones en el marco de Foucault y  de otras teorías
sociales, incluyendo Wacquant y otras investigaciones sobre Crossfit como los de
Dawson, Merino, Crossley, Beckenstein, Coakley, Rose.
Se podría diferenciar qué artículos se toman como datos, y qué se toma como diálogo fértil
a nivel conceptual y metodológico.
Tal vez no todas las combinaciones sean lo más recomendable. 
Hay autores, conceptos, metodologías que no siempre funcionan bien juntos con otros.
% No queda claro cómo se hace esta combinación o si es deseable.
% Tal vez se pueda buscar un autor que combine esos autores.

En el caso de Becker, el explícitamente se aleja de la teoría de
Bourdieu y su escuela (que incluye a Wacquant), esto lo lo hace explícito repetidas veces, un caso resumido es una entrevista,
\textit{A Dialogue on the Ideas of “World” and “Field” }(2006),
donde habla de las categorías de mundo y de campo, como no combinables: señala que el concepto de
\textit{mundo} permite una mayor flexibilidad empírica, mientras que \textit{campo}
asume, en la visión de Becker, un \textit{a priori} de reproducción social a una identidad repetida, sin variabilidad,
de dominación siempre hegemónica, sin quiebres.

En la entrevista (2006) Becker explicita su visión de los conceptos de mundo y campo,
separación general entre su teoría y metodología y los supuestos de Bourdieu y su escuela (acá Wacquant).
Latour titulará esta disputa de manera amplia como entre 
la escuela de la sociología de las asociaciones y la sociología de lo social (¿Le Breton?).
Latour explaya este argumento en su libro \textit{Reensamblar lo social} (2008), donde hace esta crítica en detalle.
Latour va mas lejos en esta crítica donde dice, en su artículo \textit{How to Talk About the Body?} (2004)
que no sirve decir que A explica A, justamente lo que hay lograr es describir
qué es lo que produce qué, esto es, describir A: lo cual abre mas el campo de investigación.
Wendy Bottero y Nick Crossley discutieron las posibles combinaciones entre estas perspectivas
en \textit{Worlds, Fields and Networks: Becker, Bourdieu and the Structures of Social Relations} (2011).

Para Duneier, que discute a Wacquant en \textit{What Kind of Combat Sport Is Sociology?} (2002),
no hay reconciliación posible entre estas distintas metodologías y posicionamientos teóricos.
O mejor dicho, las reconciliaciones son viables si hay un terreno común de discusión.
En el caso que analiza Duneier se busca entender cuáles son las vivencias de los marginados en un barrio que 
analizó décadas antes Jane Jacobs.

En la presente tesis: ¿Cuál es el fondo de discusión conceptual?
¿Hay una atención al concepto de salud más enfocado a lo físico, o a lo lo mental, del bienestar?
¿Es otro el concepto central? ¿Cuál es el diálogo con la literatura del tema?



% \begin{itemize}
% \tightlist
% \item
%   Ir al grano. Y después dar detalle. El texto redunda en introducciones
%   y no da explicaciones ulteriores. Esto pasa varias veces con el
%   concepto de carrera de Becker. ¿Hay bibliografía secundaria?
% \end{itemize} % BIBILOGRAFÍA relevante, actualizada y suficiente % INSUFICIENTE INCOHERENTE
% \input{partes/teoria} TEORIA
\subsection*{Bibliografía citada en texto y faltante en las referencias finales}

\begin{itemize}
\tightlist
\item
  Crespo, Cecilia. 2013
\item
  Frederic. 2011
\item
  Le Breton. 1995
\item
  Murphy. 2012
\item
  Oseguerra Parra. 2004
\item
  Yamil. 2019
\end{itemize}

\subsection*{5) ¿La metodología está claramente formulada? Si no lo estuviera, indicar las dificultades.}\label{la-metodologuxeda-estuxe1-claramente-formulada-si-no-lo-estuviera-indicar-las-dificultades.}

La metodología que
se usa es observación participante, toma de notas en campo, y entrevistas.

No queda claro si en el campo había siempre un grabador
prendido o no. También habría que aclarar si hubo entrevistas
individualizadas y dónde y cómo se hubieran hecho, o no, estas
entrevistas. En el caso del coach esto sí queda explícito en cuanto
a lugar y tiempo pero no se detalla cómo fue el registro de entrevista.
En el resto de los casos no se puntualiza cómo se grabaron las
conversaciones o si las citas son de memoria.

 % METODOLOGÍA
\subsection*{6) ¿Presenta un análisis de los datos consistente? Si no fuera así, indicar las dificultades.}
\label{presenta-un-anuxe1lisis-de-los-datos-consistente-si-no-fuera-asuxed-indicar-las-dificultades.}

El abordaje de los datos es ordenado y prolijo.
Se delimita qué se va estudiar y se siguen los pasos pautados.
Se podría revisar si son exactamente cada uno de esos pasos los que son necesarios para conformar la tesis del autor.
Por ejemplo, establece que hace falta conocer la historia del box para ser un loco.
¿Es realmente así para los atletas o esto es más bien un parámetro de importancia para el trasfondo de la investigación? 
Si alguien entra al box en 2016 y compite en 2017, y no conoce la historia desde el 2014: ¿no es loco?

La vivencia grupal se enfatiza en diferentes puntos del texto pero no queda claro de qué manera se dan
las conexiones entre las personas por fuera del gym,
¿hay realmente tal conexión? 
¿Cuánto tiempo de permanencia en el gym tienen los locos citados?
Se podría establecer cuál es ese tiempo de permanencia. 

¿Cuánto es lo menos que estuvo activo un iniciado? Auto-etnografía puede ser útil en ese punto.
¿Cuánto es lo máximo de tiempo que estuvo un atleta, un loco?
Estos podrían ser datos tangibles que dimensión el peso de cuán visible es la intensidad de vivencia grupal.
¿Quiénes pasaron por conversión a ser Locos pero después se fueron a otro gimnasio o dejaron la práctica?
¿Se forma una comunidad? 

% \subsection{Personas}\label{personas}
Dado que el trabajo de campo es un buen recurso a disposición, se puede
resaltar el lugar que tiene cada persona en el grupo y destacar un
balance de si se lo puede considerar a cada uno un loco, o no. %
Las personas del estudio son presentadas en anexo. 
Una consideración es si se podría incluir una presentación de cada atleta como parte del texto,
tomando como referencia ese texto de anexo pero ampliándolo con el balance individual, y más adelante, grupal.
Se podría atender a en qué momento aparece cada atleta y su relevancia en el relato del texto completo y del grupo.
%
Tomemos el caso de Chupa, fuera de una mención en agradecimientos y otra mención en un ejercicio se lo menciona en detalle recién en la página 93. Tal vez se puede redimensionar este caso con más aspectos de su perfil en el contexto del escrito.
%
En lo que hace a las distintos personas que aparecen en la tesis, y al sentimiento grupal,
podría ser ocasión para un abordaje conceptual de las vivencias individuales y grupales que podría darle más fuerza a la tesis.

En términos de los datos: se establece que hay un
respeto y seguimiento a la autoridad del líder,
¿pero equivale esto a decir que los atletas están unidos entre sí?
%
Puede que los atletas formen comunidad y que realcen valores.
Se podría hacer más evidente esa conexión si fuera el caso.
Cuando hacen un asado no aparecen todas las partes involucradas.
Cuando viajan en auto se podría señalar de qué manera y con qué criterios se componen los grupos, 
si es una organización espontánea o planificada.

En la conclusión se resalta el valor del trabajo de sí. ¿Cuál es la conexión entre la descripción del texto y esta conclusión?
% este concepto se menciona pocas veces en el texto.
Puede haber un concepto de sí, y también puede haber comunidad, pero ¿funcionan bien ambas juntas? ¿Cómo se articulan? 
% Quizás no haya ni lo uno ni lo otro.

% \subsection{Dieta}\label{dieta}

En el tema de la nutrición está muy logrado la valoración moral de los alimentos. 
Pero ¿hay un criterio unificado de perspectiva y de conducta?
¿Todos siguen invariablemente la dieta paleo? 
Es claro que hay casos de grandes esfuerzos en cambios de conducta dietaria, pero
¿hay desviaciones sistemáticas o de tipo ocasional? 
¿Cómo se desarrollan las carreras no exitosas? (Nueva ocasión para más auto-etnografía)
¿Por qué comen harina y leche cuando van a su segunda competencia? 
¿Por qué una nutricionista no mencionada antes recomienda para el día del torneo? 
¿Esta innovación sobre la marcha es considerada seria?
% ¿Prueban cosas nuevas el día de la competencia?

La investigación resalta lo global del fenómeno del Crossfit,
pero podría ser interesante hacer una comparación de las diferentes regiones por las que pasó la etnografía.
¿Son todas las provincias iguales o se sintieron diferencias,
hermandades, con distinto énfasis en estilos de ejercicios o influencias
de diferentes lugares o todos absorben el Crossfit de una misma casa
central? ¿Y el marketing cómo se maneja?
Si la hay, por rústica que sea, puede ser una vía interesante de exploración.
Las redes sociales suelen
estar presentes en las descripciones del texto. % análisis de los DATOS consistente
\subsection*{7) ¿Hace una contribución a su campo de estudio específico?
¿Cuál es esa contribución?}\label{hace-una-contribuciuxf3n-a-su-campo-de-estudio-especuxedfico-cuuxe1l-es-esa-contribuciuxf3n}

La originalidad %está en 
es ser un trabajo etnográfico sobre Crossfit
% pionera en el universo geográfico de 
en Santiago del Estero.
% que se aborda.
La contribución es la de establecer la serie de pasos que se requiere para
venir a formar parte de la comunidad específica analizada.

Para darle más solidez a la contribución de la tesis se podría enfocar en cuáles son los temas a los que más se busca contribuir:
sea la carrera, las valoraciones morales, la comunidad, el desafío.
También se podría aclarar a cuál disciplina se enfoca más: a la etnografía, a un concepto central, al universo.

Se cita por ejemplo numerosas veces la tesis de Alejandro Rodríguez. 
% ¿en qué facultad está inscripta su investigación: en ciencias sociales, periodismo, psicología, management?
¿Cuál es el concepto y el diálogo que se busca abrir con aquella investigación y las otras investigaciones citadas?
Dejar claro eso podría darle más fuerza a los propios argumentos de esta tesis. % CONTRIBUCIÓN a su campo de estudio 
\subsection*{8) Calidad y pertinencia de figuras, mapas, tablas, gráficos.}\label{calidad-y-pertinencia-de-figuras-mapas-tablas-gruxe1ficos.}

No hay ningún registro visual y las imágenes son algo que los propios investigados valoran.

Si el autor eligiera agregar imágenes, fotos, gráficas del gimnasio,
podría ser un paso fácil y agilizaría el texto y lo haría más rico.
Los atletas sacan muchas fotos y las comparten online.
De hecho el autor sí hace alusión a los comentarios en torno de las imágenes.

¿Se podrían incluir imágenes de los propios entrevistados?
Tal vez esto podría ayudar a contar la historia en imágenes, cosa que Becker sugiere
en su libro \emph{Para hablar de la sociedad: La sociología no basta}
({[}2007{]}2015), sobre todo las partes 10. \emph{Los gráficos: pensar
con dibujos} (pp.~193-214); 11. \emph{La sociología visual, la
fotografía documental y el fotoperiodismo} (pp.~215-234).

 % Calidad y pertinencia de figuras, mapas, tablas, GRÁFICOS
\subsection*{En base a lo expuesto la tesis es:}
\label{en-base-a-lo-expuesto-la-tesis-es}

\begin{itemize}
\tightlist
\item
  \textbf{ACEPTADA CON SUGERENCIAS DE MODIFICACIÓN.} Se recomienda pasar
  a la instancia de defensa oral y pública y tener en cuenta las
  sugerencias del tribunal en esa instancia. (Lxs evaluadorxs formulan
  sugerencias o pedidos de modificaciones explícitas que señalan en el
  dictamen y que deben ser consideradas por el/la tesista en el momento
  de la defensa según el formato que lxs evaluadorxs indiquen o según lo
  que se considere más conveniente).
\end{itemize}


\end{document}


%----------------------------------------------------------------------------------------
%	ARTICLE CONTENTS
%----------------------------------------------------------------------------------------

\section{Introduction}

\lettrine[nindent=0em,lines=3]{L} orem ipsum dolor sit amet, consectetur adipiscing elit.
\blindtext % Dummy text

\blindtext % Dummy text

%------------------------------------------------

\section{Methods}

Maecenas sed ultricies felis. Sed imperdiet dictum arcu a egestas. 
\begin{itemize}
\item Donec dolor arcu, rutrum id molestie in, viverra sed diam
\item Curabitur feugiat
\item turpis sed auctor facilisis
\item arcu eros accumsan lorem, at posuere mi diam sit amet tortor
\item Fusce fermentum, mi sit amet euismod rutrum
\item sem lorem molestie diam, iaculis aliquet sapien tortor non nisi
\item Pellentesque bibendum pretium aliquet
\end{itemize}
\blindtext % Dummy text

Text requiring further explanation\footnote{Example footnote}.

%------------------------------------------------

\section{Results}

\begin{table}
\caption{Example table}
\centering
\begin{tabular}{llr}
\toprule
\multicolumn{2}{c}{Name} \\
\cmidrule(r){1-2}
First name & Last Name & Grade \\
\midrule
John & Doe & $7.5$ \\
Richard & Miles & $2$ \\
\bottomrule
\end{tabular}
\end{table}

\blindtext % Dummy text

\begin{equation}
\label{eq:emc}
e = mc^2
\end{equation}

\blindtext % Dummy text

%------------------------------------------------

\section{Discussion}

\subsection{Subsection One}

A statement requiring citation \cite{Figueredo:2009dg}.
\blindtext % Dummy text

\subsection{Subsection Two}

\blindtext % Dummy text

%----------------------------------------------------------------------------------------
%	REFERENCE LIST
%----------------------------------------------------------------------------------------

\begin{thebibliography}{99} % Bibliography - this is intentionally simple in this template

\bibitem[Figueredo and Wolf, 2009]{Figueredo:2009dg}
Figueredo, A.~J. and Wolf, P. S.~A. (2009).
\newblock Assortative pairing and life history strategy - a cross-cultural
  study.
\newblock {\em Human Nature}, 20:317--330.
 
\end{thebibliography}

%----------------------------------------------------------------------------------------

\end{document}

%%%%%%%%%%%%%%%%%%%%%%%%%%%%%%%%%%%%%%%%%
% Journal Article
% LaTeX Template
% Version 1.4 (15/5/16)
%
% This template has been downloaded from:
% http://www.LaTeXTemplates.com
%
% Original author:
% Frits Wenneker (http://www.howtotex.com) with extensive modifications by
% Vel (vel@LaTeXTemplates.com)
%
% License:
% CC BY-NC-SA 3.0 (http://creativecommons.org/licenses/by-nc-sa/3.0/)
%
%%%%%%%%%%%%%%%%%%%%%%%%%%%%%%%%%%%%%%%%%

%----------------------------------------------------------------------------------------
%	PACKAGES AND OTHER DOCUMENT CONFIGURATIONS
%----------------------------------------------------------------------------------------