\subsection*{8) Calidad y pertinencia de figuras, mapas, tablas, gráficos.}\label{calidad-y-pertinencia-de-figuras-mapas-tablas-gruxe1ficos.}

No hay ningún registro visual y las imágenes son algo que los propios investigados valoran.

Si el autor eligiera agregar imágenes, fotos, gráficas del gimnasio,
podría ser un paso fácil y agilizaría el texto y lo haría más rico.
Los atletas sacan muchas fotos y las comparten online.
De hecho el autor sí hace alusión a los comentarios en torno de las imágenes.

¿Se podrían incluir imágenes de los propios entrevistados?
Tal vez esto podría ayudar a contar la historia en imágenes, cosa que Becker sugiere
en su libro \emph{Para hablar de la sociedad: La sociología no basta}
({[}2007{]}2015), sobre todo las partes 10. \emph{Los gráficos: pensar
con dibujos} (pp.~193-214); 11. \emph{La sociología visual, la
fotografía documental y el fotoperiodismo} (pp.~215-234).

