\section{Aciertos}\label{aciertos}

\subsection{Trabajo de campo}\label{trabajo-de-campo}

Hay una gran cantidad de tiempo invertido en varios frentes: en el
contacto continuo con la gente, en la persistencia de las tomas de notas
de campo, en la profundización de las relaciones, en la idea de seguir
trabajando el tema a futuro.

\subsection{Estructura del trabajo}\label{estructura-del-trabajo}

Hay una claridad de esquema de los pasos a seguir. Se delinean los pasos
para hacer una carrera y se siguen esos pasos, de a uno por vez, para
dar una imagen acabada de lo que significa pertenecer a este universo.

\subsection{Pertinencia del marco teórico y de
referencia}\label{pertinencia-del-marco-teuxf3rico-y-de-referencia}

La combinación de Becker y Goffman promete tanto una vision sobre
carrera, como repertorio moral y un entendimiento metodológico de como
hacer trabajo de campo: se puede volver tanto sobre los aspectos
organizativos internos, como las relaciones con agentes externos, y la
manera en que toda la actividad se desenvuelve en el espacio. También se
toman como referencia los trabajos de otros autores que han trabajado
sobre el Crossfit: tanto publicaciones en revista como trabajo finales
de grado y posgrado. Y se los cita en detalle a lo largo del texto.