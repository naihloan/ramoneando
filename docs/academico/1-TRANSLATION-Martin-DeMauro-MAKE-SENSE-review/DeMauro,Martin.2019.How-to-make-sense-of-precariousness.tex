\documentclass[a4paper,]{scrartcl}
\usepackage{lmodern}
\usepackage{amssymb,amsmath}
\usepackage{ifxetex,ifluatex}
\usepackage{fixltx2e} % provides \textsubscript
\ifnum 0\ifxetex 1\fi\ifluatex 1\fi=0 % if pdftex
  \usepackage[T1]{fontenc}
  \usepackage[utf8]{inputenc}
\else % if luatex or xelatex
  \ifxetex
    \usepackage{mathspec}
  \else
    \usepackage{fontspec}
  \fi
  \defaultfontfeatures{Ligatures=TeX,Scale=MatchLowercase}
\fi
% use upquote if available, for straight quotes in verbatim environments
\IfFileExists{upquote.sty}{\usepackage{upquote}}{}
% use microtype if available
\IfFileExists{microtype.sty}{%
\usepackage{microtype}
\UseMicrotypeSet[protrusion]{basicmath} % disable protrusion for tt fonts
}{}
\usepackage[unicode=true]{hyperref}
\hypersetup{
            pdftitle={How to make sense of precariousness? Bíos-precarious and sensitive life},
            pdfauthor={Martín De Mauro Rucovsky},
            pdfborder={0 0 0},
            breaklinks=true}
\urlstyle{same}  % don't use monospace font for urls
\IfFileExists{parskip.sty}{%
\usepackage{parskip}
}{% else
\setlength{\parindent}{0pt}
\setlength{\parskip}{6pt plus 2pt minus 1pt}
}
\setlength{\emergencystretch}{3em}  % prevent overfull lines
\providecommand{\tightlist}{%
  \setlength{\itemsep}{0pt}\setlength{\parskip}{0pt}}
\setcounter{secnumdepth}{0}
% Redefines (sub)paragraphs to behave more like sections
\ifx\paragraph\undefined\else
\let\oldparagraph\paragraph
\renewcommand{\paragraph}[1]{\oldparagraph{#1}\mbox{}}
\fi
\ifx\subparagraph\undefined\else
\let\oldsubparagraph\subparagraph
\renewcommand{\subparagraph}[1]{\oldsubparagraph{#1}\mbox{}}
\fi
\usepackage[margin=0.97in]{geometry}
\usepackage{multicol}
\newcommand{\hideFromPandoc}[1]{#1}
\hideFromPandoc{ \let\Begin\begin \let\End\end }

\title{How to make sense of precariousness? \emph{Bíos}-precarious and
sensitive life}
\providecommand{\subtitle}[1]{}
\subtitle{¿Cómo hacer sentido en la precariedad? \emph{Bíos}-precario y vida
sensible}
\author{Martín De Mauro Rucovsky}
\date{\today}

\begin{document}
\maketitle

\textbf{Abstract}

The operation that we are interested in emphasizing revolves around
these dimensions that are illuminated in the critical work of reading
with cultural materials and that suppose a conceptual complementarity
between the paths of biopolitics and precariousness. What these analyzes
and their work with materials indicate, in a very clear way, is a set of
dimensions, lines of inquiry and areas of problematization that do not
acquire enough relevance in contemporary debates about biopolitics and
precarity. Hence the need to develop a conceptual framework tool, which
accounts for these markers that encode a precarious life. And precisely,
that is the blind spot in common and at the same time, a space of
conceptual intersection: the paths in biopolitics that have not
considered the processes of precarization of life and in equal measure,
the theorizations about the precarious condition that they have not
thought about in strictly biopolitical terms. We call this conceptual
knot: \emph{bíos}-precarious. The \emph{bíos}-precarious concept spawns
from the conjunction between Judith Butler's toolbox in relation to the
bodily ontology of precariousness; and the crossing in Roberto
Esposito's biopolitical path between impersonal life and affirmative
biopolitics.

\textbf{Keywords:} Bíos-precarious, Biopolitics, Precarity, Judith
Butler, Roberto Esposito

\textbf{Resumen}

La operación que nos interesa subrayar gira en torno a la zona de
problematicidad que ilumina el trabajo crítico con materiales culturales
y que supone una complementariedad conceptual entre los recorridos de la
biopolítica y de la precariedad. Lo que estos análisis y su trabajo con
materiales indican, de un modo muy nítido, es un conjunto de
dimensiones, líneas de indagación y zonas de problematización que no
adquieren la suficiente relevancia en los debates contemporáneos sobre
biopolítica y sobre precariedad. De allí la necesidad de elaborar una
herramienta de entramado conceptual que dé cuenta de los marcadores que
codifican una vida precaria. Y justamente, ese es el punto ciego en
común y a la vez, el espacio de intersección conceptual que nos
interesa: los recorridos en biopolítica que no han considerado los
procesos de precarización de la vida y en igual medida, las
teorizaciones sobre la condición precaria que no han pensado en términos
estrictamente biopolíticos. Denominamos \emph{bíos}-precario a este nudo
conceptual a partir de la conjunción entre la caja de herramientas de
Judith Butler en relación a la ontología corporal de la precariedad y el
cruce en el recorrido biopolítico de Roberto Esposito entre vida
impersonal y biopolítica afirmativa.

\textbf{Palabras Clave:} Bíos-precario, Biopolítica, Precariedad, Judith
Butler, Roberto Esposito

\Begin{multicols}{2}

\subsection{\texorpdfstring{\textbf{How to make sense of precariousness?
\emph{Bíos}-precarious and sensitive
life}}{How to make sense of precariousness? Bíos-precarious and sensitive life}}\label{how-to-make-sense-of-precariousness-buxedos-precarious-and-sensitive-life}

\emph{Mar Del Plata, Buenos Aires, Argentina. July 2017}. The local
Health Secretary shares his view on ``homeless people'', sparked by the
``homeless'' death of Sergio Fernández, highlighting into a case of a
woman that frequently sleeps in the streets. Speaking with radio
journalists on the show ``Not Gone with the Wind'' (``Lo que el viento
no se llevó''), the Secretary, Gustavo Blanco states: ``We've gone 17
times to pick her up. Every time we leave her at the hospital she goes
back. \emph{Just like a dog}, she goes back to the place where she feels
comfortable''. The Health Secretary quoted old normative code, with
early positivist and hygienist reminiscences. He argued about a
sustained effort to ``retire'' this woman to get her into the hospital
but despite sanitary policies and the efforts made, she insists on
``going back to her place because she feels comfortable''.

\emph{Buenos Aires {[}Autonomous City{]}, Argentina, September 2018}.
Journalist writer Carolina Koruk publishes in the \emph{Para Tí}
Magazine an article on ``Emotional salary time: what is this new work
benefit about'' which shows a new trend with substantial repercussions
in Europe: \emph{the emotional salary}. Koruk comments on research by
the \emph{iOpener Institute for People \& Performance}, in England,
about happiness in the daily duties of workers, and the emotional state
the returns in a stronger commitment of employees at their companies. In
critical times, explains Koruk, many companies dive into extra expenses
as a decisive element ``so that employees can be happy, even when you
don't pay them'', and this way, the company avoids workers from
migrating to other companies. Attached to the flexible culture (where
goal completion is above everything, and there is certain freedom
regarding work time) this kind of salary aims towards emotions and well
being, to appreciating each of the people working, hearing to them to
keep them motivated, and work climate.

The ideas at the core of these scenes are biopolitics and
precariousness. The events exposed above are images that allow a thought
upon how public servants and journalists and in general our societies
mark hierarchical distinctions between lives to be protected, taken care
of, or plan a future for, on one side; or insead lives to abandon,
sacrifice, or eliminate. This primary mark, which is at the main center
of biopolitics and precarization processes, entails a series of
incisions, gradients, and thresholds around which a stake is decided
about which individuals and groups can be recognized as humans or
less-than-humans. These images are inscribed on a line of inquiry about
the conditions in which it is possible to learn about a life, or about
the specific power mechanisms through which life is produced, taken
cared of, or valued as differential. These gazes return a sharpened
image, yet clearly self-evident, of a dynamic which aims towards a life,
a living being or the animate over the basis of a series of distinctions
and oppositions -more or less stable-: between life and no-life, alive
and not alive, or between purely biological life (\emph{zoé}) in
comparison to a way of life (\emph{bíos}). Indeed, life and
precariousness name a displacement of meanings, just as Gustavo Blanco,
the Health Secretary, when he refers to a field of concepts and
practices that casts thought beyond the humane because he locates
homeless to a frontier along with the wilderness and the animal world.
In such a way, precariousness stages a reconfiguration of structural
inequality associated with poverty and its inequity markers through the
recurring inquiry about the species limits, about what is human and its
edges.

In this context, a series of critical analysis and cultural studies that
work based on aesthetic materials made in Latin America explore that
life as an expansive field and a set of reading operations that mobilize
meanings upon the visible and the sensitive, which are primarily defined
by the biopolitics logic, but also by livings' precarization processes.
Such materials include works by Fermín Rodríguez (2010), Florencia
Garramuño (2015), Gabriel Giorgi (2014), Ximena Briceño (2017) and many
others.

The critical work of reading with cultural materials (De Mauro Rucovsky,
2016; 2018a; 2018b; 2019a; 2019b) illuminate dimensions that we intend
to highlight. The operation that we want to emphasize upon this life,
supposes a conceptual complementarity between the paths of biopolitics
and precariousness. What these analysis and their work with materials
indicate, is a set of dimensions, lines of inquiry and areas of
problematization that do not acquire enough relevance in contemporary
debates about biopolitics and precarity. Hence the need to develop a
tool for a conceptual framework, which can account for these markers
that encode a precarious life. And even further, such is at the same
time both the blind spot and space of conceptual intersection: the paths
in biopolitics that have not yet considered the processes of
precarization of life; and in the same vein, the theorizations about the
precarious condition that have not been thought about in strictly
biopolitical terms.

We call this conceptual knot: \emph{bíos}-precarious. The
\emph{bíos}-precarious concept arises from the conjunction between
Judith Butler's toolbox in relation to the bodily ontology of
precariousness; and the crossing in Roberto Esposito's biopolitical path
between impersonal life and affirmative biopolitics. There are many
points of divergence between Butler's and Esposito's ponderations
indeed, but perhaps there are even more points of convergence. We want
to highlight two operations with this remark: both of Butler's and
Esposito's corporeal ontologies as well as their affirmative
biopolitics; because they allow a delimitation for this way precarious
way of life and the relationships between \emph{bíos}, culture and
politics surrounding the question about the present time: Up to which
degree is the current time crossed through by precariousness and
biopolitics? Or in other terms: In which degree is the present time
understood in terms of Judith Butler's and Roberto Espósito's conceptual
remapping?

And so, what guides the present inquiry is not an exegetic pursue around
the work and thought of both Butler and Esposito (by their proper names)
which would seek justice to the nominal reputation of these authors, but
rather to aim for another procedure and towards another epistemological
direction. The intent is to unfold the viewpoint -or even better, to
locate this \emph{bíos}-precarious in a superimposed level- following a
topologic and systematic procedure, that is, a fold among the broader
figure it intends to counter: Which are the possible conditions of the
precarized living, of the \emph{bíos}-precarious? How can we account
these vectors and modulations, as well as the epochal dimension or of
the historical present time which codifies a life?

\textbf{Biopolitics is the insurmountable horizon of our time}

\begin{quote}
Muita coisa não posse te contar.

Não vou ser autobiográfica.

Quero ser `bio'.

Escrevo ao correr das palavras

Clarice Lispector - \emph{Água viva} (1973)
\end{quote}

Biopolitics is a heterogeneous research field, it has diffuse limits and
is in constant expansion. Biopolitics involves a broad set of studies
and research lines that would be hard to put together on a single
perspective. In an intuitive sense, the term seems to shed light to an
imprecise constellation that circuits around the concept pair formed by
\emph{bíos} (the nourishing life according to Aristotle, the body, or
the living) and politics (power, government, institutions, laws,
conflicts). According to the word's meaning, biopolitics refers to the
politics that occupies itself with and takes care of life (in greek
\emph{bíos politikós}) but upon the distinction between \emph{bíos} and
\emph{zoé}, biopolitics refers to men's qualified life (\emph{bíos}).
\footnote{The term biopolitics moves away, from the idea of
  \emph{zoo-politics}, which refers to the politics that takes into
  consideration \emph{zoé}, the undifferentiated totality of the living,
  animals, humans, and non-humans. The prevalence of the term
  \emph{bíos} over \emph{zoé} is due to the appearance, at the beginning
  of the XIXth Century of the term ``biology''. The prevalence of
  \emph{bíos} is precisely due to Jean-Baptiste Lamarck's (1744-1829)
  project of a science of living bodies, of the living as living, as
  Edgardo Castro (2011:19) points out. Fabián Ludueña Romandini (2010)
  currently reappropriates the idea of \emph{zoo-politics}, as a way of
  connecting life and politics without relying on the exclusion of
  \emph{zoé}, but rather by its politicization.}

Biopolitics, as a term, has been considered an \emph{oxymoron} (a fusion
of two concepts that contradict each other, because politics in a
classical sense goes beyond the mere creature and the bodily); or even a
simple \emph{tautology} -Isn't politics always a matter that delves
around life?- (Lemke, 2017:13). The term biopolitics implies a
constitutive instability that reflects the terms vitality and the
``peculiar semantic mobility'' that is proper to it (Bazzicalupo,
2017:41). Hence its oscillation, which goes through a hesitation between
the two terms that compound the category: What should we understand as
\emph{bíos}? Would it be possible to elaborate on an exclusive
hypothesis between life and politics?

All the prior drives to a fold, according to Espósito (2006), between
two tonalities and categories: on one side, life as a matter of politics
or life as politics objects, the capability to make live or life
translatable as politics (a politics that is exerted exteriorly
\emph{upon life}); and on another side, the political side of
\emph{bíos}, the interior of politics, a matter of life, life as a
politics matter (a politics that is immanent \emph{in life}). If we
attend to the Greek lexicon, especially attending to the Aristotelic
(Agamben, 2010:9-23), biopolitics remits to a \emph{zoé} dimension: life
in its simple biological maintenance, with no qualification, stripped
from any formal attributes (perhaps we should then refer to this as a
\emph{zoopolitics}?). In its semantic content, the term highlights the
connexion between the meaning of \emph{what is alive} and especially of
\emph{what is human}. In this sense, a biopolitical line of thought can
then allow opening a vast field of problems and questions: What
consequences does this encounter have, the conceptual syntagma or the
reciprocal interpellation between life and power? What is the nature of
this relationship? Are they external dimensions? Or do they reveal an
intrinsical imbrication, of native knotting? (Giorgi \& Rodríguez,
2007:32) In this sense, we want to highlight the set of oppositions and
epistemological demarcations that seem to function as conditions of
possibility of the fixation of a sense to the idea of biopolitics: the
difference between life (as an exceptionally humane matter) and no life
(animal, mechanical, vegetal, spectral), the limit between life and
death (which in Foucault are interplayed amongst the \emph{making die}
and the reverse, \emph{making live}), the living entities against the
non-living (Haraway, 2016) and the purely biological life compared to a
way of life, a formed life or a qualified one (Biset, 2016).

Over the broad routes paved around biopolitics, there is a focus that
places to the center of the constitutively political dimension of life
(at the individual, and population levels) and the managerial ways of
that life, the making live and its counterpart, the letting die. These
distinctions are the axis of the canonic theorizations by Michel
Foucault, along with the Italian reading by Giorgio Agamben, Antonio
Negri, and Roberto Esposito; up to the considerations by Nikolas Rose,
Peter Miller, and Paul Rabinow. In the discussions that delve around the
precarious condition, with Judith Butler at the center but also with
Richard Gilman-Opalsy, Guillaume Le Blanc, Guy Standing, Athena
Athanasiou, Lauren Berlant, Isabell Lorey, and including the Spaniard
Remedios Zafra, the emphasis is placed on the type of exposed corporeal
life and dependant upon others, defined mainly by its physical
vulnerability and its potential of being damaged, and its existentially
finite and contingent condition. But specific paths undertaken by the
cultural critique in the past decades bring to surface an epochal
dimension, not only a political logic, or a governmental rationality
(what Foucault described as the neo-liberal governmentality), a
corporeal life signaled by the mutual exposition and vulnerability, or
the processes of dispossession and expropriation that damages such
condition (Athanasiou \& Butler, 2013); but rather a precarious life, a
\emph{bíos}-precarious which opens a politicization threshold and which
can be, at the same time, a field of experimentations across several
levels: conceptual, formal, aesthetic, and political.

From this angle, there seems to be a point that would be useful to
clarify in advance: Why \emph{bíos}, and not \emph{nuda vita} or
\emph{zoé}? Why \emph{bíos}, and why not a way-of-life? We can locate
\emph{bíos}-precarious at this intersection, in between of Roberto
Esposito and Judith Butler and separated from the thinking of Giorgio
Agamben (2010), who identifies in biopolitics -and the sovereign regime-
a field of incisions and partitioning practices between the recognizable
lives and the lives to abandon, or the conversion of the \emph{bíos} (a
specific and qualified way of life) into \emph{zoé} (the stripped life).
According to Agamben, the governmental occidental machine is the one
that articulates a theological-political paradigm, a
theological-economic one, and the third one of glory and spectacle, and
which operates as an exception state; that is, as a State that includes
upon itself the anomic element that establishes it and whose mission is
to capture and produce the stripped life, the \emph{nuda vita}.
Biopolitics, in Agamben's path, is characterized by producing the
supposition of mere life, and in pretending so, in the way of a vicious
cycle, produces it (Moyano, 2019: 294). But biopolitics is also
characterized in the conceptual development of his thought, in a wager
in favor of minor biopolitics (Agamben, 2003) which directs attention to
a life that is inseparable from its forms, or a life that cannot be
isolated as a stripped life. In other terms, every life is already a
way-of-life that is about, primarily, individual ways, living acts, and
processes that are life and imagination possibilities, of common
potencies (\emph{General Intellect}).

\emph{Bíos}-precarious is a syntagma that appears away from Agamben's
proposal. \emph{Bíos}-precarious can superimpose and juxtapose at the
same time Esposito's affirmative biopolitics and impersonal philosophy;
with Butler's bodily social ontology. In Esposito's (2005; 2006; 2007;
2009a; 2009b) line of work \emph{bíos} names the singularity of life
processes that are recognized at the interior of the immunization
mechanism (\emph{katékhon, phármakon} figure), which operates
dialectically by increment, protection, and development of life and
reaching an aporetic point, it impedes any further development, or
destruction and annihilation. At the heart of the immunity functioning,
which as we know is included as the third term in between of the
sovereign and bipolitics, Esposito identifies a line of flight to the
theological-juridical-biomedic capture of the deploying immunity, now
not \emph{upon} life, but rather upon life's immanent normativity.
\emph{Bíos} signals something else: it is not the negation or privation
of what is shared in common (the \emph{proprium}), nor is it the
mechanics of enclosing the body over itself and inside itself.
\emph{Bíos} rather signals a projectual horizon of the subject outside
of himself, a shared reciprocal relationship that exposes the subject to
contact and even to contagion with the other, or with the \emph{sôma},
which is the constitutive part of \emph{the flesh of the world}. A vital
and compositive potential, which is a capability of modifying ourselves,
\emph{bíos} is a transplant, prosthetic incorporation, and graft because
it shatters the frontiers of personal property, the dimensions of what
is inside and outside, natural and artificial.

\emph{Bíos} has yet another defining characteristic, and that is by
contrast to the norm that makes incisions into life. Esposito's remark
is the following: at the opposing reverse of life's normativization,
\emph{bíos} inflects as an attempt of vitalizing the norm, or as pure
vital facticity in its absolute singularity. It is all about a living
being that is always beyond its self, going over the individual sphere,
its pre-established forms and figures, in variation and mobilization of
bodies. \emph{Bíos} is any form of existence that has equal legitimacy
to live in a complex relationship with the surroundings, and in a
framework of relations in which it is necessarily inserted in. Such is
the situation in which \emph{bíos} moves upon, as a living being that
depends on connexions and encounters with other intensities that, as an
immanent rule of life, is a result of a process of successive
individuations and reproduction, but as a process of
de-individualization or of de-subjectivation as well, because nobody
remains for a long time in the same state, self-identical. As is
noticeable to this point, Esposito follows the Deleuzean and Spinozist
legacy, which configures a line of thought about the virtual in
relationship to a life that permanently oscillates between the actual
and the virtual, which exceeds any possible actualization and precisely
because of that, becomes along with others, producing relationships,
affirming its singular style and rhythm.

Esposito identifies, in a farther meaning, at the interior of the
person's device mechanism an area that overflows and surpasses its
mechanism. The person, as a category, functions as a biopolitical
dominance operator, because it exposes living hierarchies, unequal
distributions and corporeal reifications. The «theological-political
machine» captures from the person a duplicity threshold in between of
the person (as juridical-artificially abled) and the production of its
negative opposite, the thing (a biological element with no value, part
beast and animal, inert matter or utterly non-human). This structural
unfolding element, or even of excluding assimilation, settles over a
logic that seems to articulate unity and division, in two asymmetric
parts, the spheres of \emph{bíos} and \emph{zoé} (the one submitted to
the other). The person, as a concept, also entails a tanato-politics
derive, which functioning consists of leaving or violently dismissing
all of that which is not considered a person in the human being; and as
a consequence of this, it can be destined to die. It is precisely there
where the Italian thinker locates an impersonal \emph{bíos} as a field
of responses, in the alteration and contamination of its prevalent
meaning, which gives potency to the opening to other life possibilities.
In between of the extreme positions of the person and the thing, between
human and natural; one of the genealogical analysis's focus is located
in the emptying of the humanist overtone, in a long-standing tradition
that defines men as distant and different from the animal genre, or in
contraposition to an area of beastly humanity. The upsetting order on
what is human and animal allows to open up to change and metamorphosis,
multiplicity or plurality, all of which can begin to take into account
the infinite difference between each singular life and at the same time
that which is the pre-individual and post-individual in each one of
living beings. The immanent potency of the \emph{impersonal bíos}, that
is recognizable in Maurice Blanchot's neutral (\emph{ne-uter}) and in E.
Benveniste's appraisal of the sphere of the (oneself) possessive «se»,
which constitutes an interrogation plane about the forms, bodies and
their orientation models, and is associated with a mobile margin of
vicinity and interchange amongst livings.

\textbf{Precariousness is the insurmountable horizon of our time}

Judith Butler (2006; 2010; 2011; 2017), on her account, names as
precarious a social ontology of bodies that proposes as an alternative
epistemology to the liberal and neo-liberal matrix of the proprietary
subject. Life's precarity, in this vein, opposes the discrete and walled
ontology of possessive individualism. Life's precarity, the vulnerable
condition of being-with, leads us to ask about the ways in which our
societies and our structural dependency to social recognition norms
build up what we define as life; and precisely because of that, the
social and economic conditions so that it all remains as such.

Butler opposes the neo-liberal ratio that underlies in possessive
individualism's ontology, and identifies the second level of juxtaposed
and convergent processes of precarization: on the one side,
\emph{precariousness} and \emph{dispossession} name a bodily
ontological-existential condition, a constitutive openness that always
being out of the self, the fact of being made of bonds and relationship
to each other. This condition implies a precise recognition of the
\emph{relational characteristic of our existence} with people and with a
surrounding as well as with norms and normative framing: every existence
is inserted in a framework of power relations, and there is no life that
is capable of exceeding the normative framework, but rather
reiterations-iterations that are internal to it, slippages or normative
re-significations \emph{in situ}.

Our existence, according to Butler, has a relational character, which
aims towards the linkage with interdependent (social, economic,
biological, ecological) nets, and allows all together survival,
protection; as well as violence, physical disappearance, femicide and
aggression. Likewise, the point of departure of this constitutive
relatedness in linking networks assumes that all human life is basically
bodily and because it assumes death, finitude, physical and
physiological needs, its condition is to be a constitutively vulnerable
being, exposed to contact with others. But on another hand, in
convergence and juxtaposition, this shared condition of being precarious
is what makes us different: some bodies are more exposed and protected
than others. The situation is such that what irremediably happens is
that precarity is assigned differentially, or becoming disposed appears
as privative form, a category that exposes the maximization of precisely
the vulnerability that constitutes who we are (a fragile but necessary
dimension of our interdependence) but which is left to differential
distributions, which is to say: this means specific historical forms
that deal with social and economic relationships, about the presence or
absence of infra-structures and institutions that organize the
protection of bodily needs.

\textbf{Blind areas and common problems}

\emph{Bíos}-precarious. The issue, then, comes back again. We need to
argue about why we use this conceptual formula. Why
\emph{bíos}-precarious? Both terms are in mutual tension to point out
something that cannot be named in any other way. What we can find at the
point of convergence is where we find the double valency of the
syntagma: On the one side, \emph{the question about the living (bíos)}
which is at the core of the biopolitical thought, and which Esposito
addresses into the terms of a type of impersonal-neutral-anonymous life
that goes outside of the person's silhouette, out of the auto-immune
body's shape, and out of the object-thing regime. But Butler points out:
the question regarding the living lives in the inside of the mechanism,
or at the interior of the normative framework as in inner displacement.
Under the same current of meaning, the question regarding the living
points towards the possibility conditions (social, economic, political)
so that life can continue going on as-is: This is the line of inquiry
that Butler reclaims about our structural dependency to social
recognition norms, and how our societies formulate definitions about
what life is. And it precisely under these definitions where \emph{the
possibility conditions so that life con be livable} and sustainable
appear. What is life? And which are the normative, social, economic,
eco-political conditions that make it sustainable and livable? From this
angle, the question about the living and its possibility conditions
overlap, because they aim to the same transversal axis, along with
precariousness.

On another hand, against an underground legacy that Esposito identifies
with the Roman-Christian tradition, and which Butler refers to as a
liberal heritage, in both paths, we find an unquestioned assumption that
cuts across and continuously seals off an understanding about what it
means to ``be-with'', or what are the interdependence relationships with
others. That is, all of this constitutes the theological-political
person's device, the theological-biomedical immunity's semantic, and the
late liberal matrix of possessive individualism. From this perspective,
both Butler and Esposito propose relational ontologies for the
ex-tactical subject, as for the being-with, but at different levels. In
Esposito, the subject's projection outside of the self supposes a type
of reciprocal relationship that exposes the subject beyond the frontiers
of personal property, and even to the other's contagion, or to the
\emph{sôma}, which is a constitutive part of \emph{the world's flesh}.
In Butler, we can find an exposition that defines interdependence with
others and in terms of social norms that constitute us. The common
precariousness is an ontological condition that assumes life's
interdependence (to other living beings, but also norms and power
relations) and the ecstatic character of vulnerable bodies.

At this point appears an outline of a propositive aspect of
\emph{bíos}-precarious, which -at least from this angle-, seems to
assign the ecstatic and open, cut across by exterior agents; in contrast
to the way-of-life that refers mainly to the individual modes, the acts,
and processes of living. Unlike Agamben, the being-with and the
interdependence networks are not limited to the individual forms, and
the acts of a life inseparable from its forms, but instead point out to
an expansive process of precariousness, that inflects knowledge,
experiences, and collective areas at heterogeneous levels.

\emph{Bíos}-precarious summons Esposito and Butler, it uses a toolbox
that can use one line of inquiry that the other doesn't account for: a
life's ontology (Esposito's impersonal \emph{bíos} is defined over the
life norm, departing from the exposition, vital opening and bodily
contagion) that is configured as precarious (existentially vulnerable,
ecstatic, and exposed to others) precisely due to an epochal diagnosis,
the times of a new neo-liberal intensity (Butler, 2004 y 2010) and its
corresponding process that normalizes precariousness (Lorey,
2016).\footnote{Our proposal is unlike Janell Watson's (2012), who finds
  in Butler and Esposito a shared conceptual logic, maintaining a
  relationship ``linked to the biopolitical limits of a liberal
  discourse'' in the valencies of the pairs: bíos/immunitas and
  precarity/precariousness. We propose, and this is our position, to
  read a complementarity in a shared ontology that overflows the
  (neo)liberal framing, in which one conceptual toolbox is embedded with
  the other, but only after achieving a critical diagnosis and
  resistance to the present time.} And precisely because of that, we
should notice that even if both Butler and Esposito identify an
unquestioned common core, it is with Butler that this being-with
(impersonal \emph{bíos}) can then be called \emph{bíos}-precarious. And
as it goes with an interplay of inverted mirrors, it is with Esposito
that we can call precarious life as an impersonal \emph{bíos}, or even,
understand such a precarious life in explicitly biopolitical
terms.\footnote{In Butler's work the explicit mentions to biopolitics
  are at least scarce. To take an example: the identification with the
  life sciences, vitalism, and State racism in \emph{Frames of War}.
  Butler herself recognizes her debt with this vast field of research
  (Soley-Beltran \& Preciado, 2007). But despite this, it is possible to
  trace a reading code or a biopolitical procedure in her interest in
  thinking the adjustment, life's limits and even the question about the
  social and economic conditions that sustain life. Eduardo Mattio's
  work (2017) ``Governmentality and resistant agency. Biopolitical
  considerations in Judith Butler's recent work'' turns out to be a key
  piece in this line of inquiry.} \emph{Bíos}-precarious appears over
the contrasts of affirmative biopolitics and a bodily social ontology
that is drafted over linkage of contact and contagion (a \emph{unicum})
between \emph{bíos} and \emph{zoé}, form and force, modality and
substance, the dispossessed-precarious life but in terms of gifting
relationships and interdependence with human livings; and not humans,
norms, and normative framing.

In this way, considering the articulation between both tool-boxes, it is
crucial to notice which legacies and traditions does each prioritize. In
general terms, Butler attends to Derrida and Foucault; and Esposito
builds up through Deleuze. Just in the way that Esposito points out,
\emph{bíos} is a line of flight from the person's dispositive, as is of
the theological-biomedical semantic on immunity. In Butler, it is a
ontological condition with regards to interdependence with norms and
other lives: in such a way, no life exceeds this normative frame but
there are rather internal displacements. And this is just one side that
does not acquire enough relevance in Esposito's affirmative biopolitics:
every life is saturated, in a greater or lesser extent, by power. The
excess of life, the capability for variation and empowerment does not
presuppose as much of a life's overflow in contrast to a norm that tries
to break through it, or even a norm that tends to subdue life's
innovative potency; but instead, the assumption is that, in Butler's
Derridean interpretation, the displacements and subversive
re-significations happen at the inside of the norm. It is then, in its
intrinsically iterative nature where the deviances and excesses undergo.
Life is from the beginning inserted in mechanisms: of immunity,
normativization and personifing, and in their own reiterative
reproduction they reach a shape towards a displacement towards vital
facticity, or that enables the norm's subversive vitalization.

\textbf{A seismograph of the present time}

This conceptual assembling and connexion space allows us to notice the
triangulation in which \emph{bíos}-precarious is configured, coming out
of the toolboxes provided by Esposito, Butler, and the dimensions that
are not present in them. In other terms, three great vectors compose the
\emph{bíos}-precarious: Esposito's affirmative biopolitics, Butler's
corporeal ontology, and a glimpse at an epochal dimension, all of which
together, settle the conditions to think precariousness as ontology.

The syntagma \emph{bíos}-precarious operates like a coagulator for
imaginaries, figurations, languages, and images, as a cultural mechanism
to condense meanings, but also as a conceptual and systematic tool with
which it is possible to trace a seismograph of the present time. We are
referring here to the analytical treatment with the works and cultural
material that think about the present, the responsive and reformulating
ways that culture brings forward (De Mauro Rucovsky, 2018; 2019). The
matter is about what that present is: What is this present about in
which we all, in one way or another, belong? What does it exactly mean
when we talk about the present time, «today», now? What is the
difference that today brings in comparison to yesterday? What
characterizes it in its analytical description and diagnostic test, but
in its contradictions as well as confrontations?

It is about a relationship with the present time that, in the wake of a
Foucaultian reading on Kant (1983-1984), means a shift in the way we
look at ourselves. One question points out to the other: What is my
actuality? And what impact does it have that I speak with it? What is
the current range of our experiences? And what is the current range of
possible experiences? The attitude and will to assign to oneself the own
present time as a duty, is what Michel Foucault calls \emph{current
ontology}, following the illuminist inspiration of Kant's texts. This
expression labels a way of relating ontologically with and against the
current time, a duty and a type of analytical attitude (an \emph{ethos},
or continual critique) of the singular time, upon that historical way of
being in which writing happens and the reason to do so. It is about a
reflexive relationship with the present time, that aims not only to the
vertiginous movement of what happens (a transitory time, of the fugitive
and contingent) nor to the tight forces that cross it through; but
mainly instead it refers to the permanent critique of one's own story,
about the choice about what we are, and which in its latent potency can
reveal and liberate that which we could be. In this sense,
\emph{bíos}-precarious emerges in the reading operations with cultural
materials that demarcate a present time defined greatly by
neo-liberalism, by the fall of the modernization and progress' dreams,
indeterminacy and fluctuation, the lack of guarantees, or projections,
and the deflection of teleological temporalities.

Our time is when precariousness becomes perceptible, and we make sense
among precarity: Anna Tsing (2015: 20) writes that ``\emph{our time is
ripe for sensing precarity}''. Or in other terms, precariousness is not
the exception among how things work in a well-balanced world but rather
the \emph{ontological condition of our time} (Anna Tsing, 2015:20).
There is a part of this epochal sense in \emph{bíos}-precarious as a
category: this is what grounds the conditions to think about
\emph{precariousness as an ontology of the current time}. Indeed, this
is what happens when critically analyzing cultural materials (De Mauro
Rucovsky, 2018; 2019) or as exposed above in the first two scenes
(referring to the Health Secretary and the emotional salary): there is
an opening up to a field of formal experimentations, but also to
alternative epistemologies when apprehending the present as such, in its
struggle fields and tension lines, in how it is possible to transform,
transgress and imagine potentialities. Culture and aesthetics have,
then, a capability to condense and capture. And that capability is
measured here around the meanings of what this formula entails:
\emph{bíos}-precarious, a ground in which precarious life becomes a
disputed threshold, where ways of agency spawn around politization and
critical essay.

An expansive field of formal experimentations opens up through
\emph{bíos}-precarious: about what it means to ``make live'' and its
counterpart, the ways of managing the ``to make die'', how to understand
a life and how to make it recognizable, what are the conditions under
which a life can be sustainable (\emph{livability}), how to make life or
a life livable, \footnote{The oscillation between \emph{a} life and
  \emph{life} («life as such») marks a cleavage point in Butler's
  biopolitical interpretation. Beyond the explicit references that the
  author uses, the matter is not so much about the life's ontological
  specificity that Butler (in \emph{Frames of War}, for instance)
  identifies with the question about animal's \emph{bíos} compared with
  humans (animal rights), or the living being compared to what is not (a
  fetus, an embryo, or interruption rights); but rather about the
  category's instability and mobility. In this sense, the question about
  \emph{a life}, about the conditions under which a life can be livable
  and worthy of being cried for, its capability for being recognized as
  precarious goes hand in hand with a relational and modal understanding
  of life. That is a wager towards affirmative biopolitics, an
  impersonal and neutral life norm, a life in its singularity and
  difference.} what are the (human and non-human) networks to which
lives are given, and where they are sustained, what pre-individual and
impersonal living forces have place, and what is the variation and
excess potencies that inhabit life, or which are the thresholds:
unthought of, irrepresentable, and of what is possible from corporeal
vulnerability.

The \emph{bíos}-precarious matter and its place in culture implies a
rethinking of how culture, philosophy, and cultural critique -but also
the knowledge that is produced immanently with aesthetic materials-,
``thinks about and responds to a historical horizon defined greatly by
biopolitics'' (Giorgi, 2014:17) and precarity (Butler, 2004; Standing,
2011; Lorey, 2012). And to tackle such figure, tautological by
definition -life is from the start: precarious, finite, contingent and
vulnerable-, we will consider the inquiry about the matter of
\emph{bíos} and precariousness that acquired increasing relevance in
philosophical and cultural critique. But there is also trouble in the
lexicon itself, the conceptual syntagma \emph{bíos}-precarious in the
horizons of biopolitics and precarity: what are the possible conditions
for the precarized livings? What is this transversal condition that
illuminates the living's general dimensions and that we call
\emph{bíos}-precarious? In which way are these two ideas related, these
two critical diagnoses, and what is there relevance to think about the
present, the time of what is present? Is there a mutual implication
relationship? And to be more exact: what is the specificity about
precariousness as a concept, compared to biopolitics' logic in a present
time marked by the openly consolidating neo-liberal program?

This is why it is convenient to ask if it is a sole concept working with
two coordinated terms where both expressions may appear as synonyms, or
if rather it has a different semantic value. And if this is the case,
what does this difference consist of, and which is the strategic sense
of its conjunction. Rather than redundancy, \emph{bíos}-precarious
situates itself in a triangulation that unites and relates both
analytical apparatus, Butler and Esposito, but also into a knowledge
that is produced from different critical analysis, aesthetic and
cultural materials. We are referring to a toolbox that is built over the
immanent complementarity of their conceptual practices, but
\emph{bíos}-precarious is also the path to think about a shared blind
spot, at certain aspects and levels, such as temporality and the epochal
matter, the moody and affective regime (De Mauro Rucovsky, 2019a;
2019b), the non-human and the contamination and devastation eco-systemic
contexts (De Mauro Rucovsky, 2018a; 2018b), the working with aesthetic
materials, and the formal operations that are not considered enough, but
also regarding work, poverty, and class indicators (De Mauro Rucovsky,
2016).

Finally, the conceptual tie in \emph{bíos}-precarious inhabits in a
generalized estranging state that seems to function as a privileged area
for inquiries in culture. Effectively: what do literature and culture
know about new ways of subjectivation and ways of life for which work
and poverty as places where identities and projects stopped being a
measure and substance of the social? (Laera \& Rodríguez, 2019:33). What
culture and literature are declaring refers to the decomposition of the
Fordist type of work universe and the associated cultural grammar
related to poverty but whose contents do not attain symbolization. In
this same sense, \emph{bíos}-precarious has not been the other side of
poverty and work, but rather that, departing from the neo-liberal
inflection, precarity works as a sensor of the incipient displacement:
work loses track as a social grammar because, precisely, to have a job
does not place the person into a given social level anymore, rather
having a job can be compatible with living in poverty. Here, the figure
of the \emph{working poor} is the sign that brings new senses to the
surface: it is different from the prevailing stability of the industrial
proletarian (at salaries level, but also as a social classifier) and it
refers to the ``structural and organizational fragmentation of the
formally employed class'' (Pacheco, 2019:169), but it also points
towards the adaptations: of expectation, and vital will in terms of a
permanent rotation, the lack of foreseeable futures, or social progress
narratives, and even more, pointing towards increasing volatility and
labor instability (Standing, 2014:8).\footnote{In the Fordist-industrial
  imagery work represents a figure characterized by stability and
  permanence, all of which allows the social promotion and which acts as
  a biopolitical reverse to poverty. As a pious vision upon the working
  class, work runs as the last rank of human dignity and citizenship, a
  mark for identity and social protection; and work appears as a
  possibility for redemption against poverty. Poverty (\emph{paupertas})
  is then a sign of dispossession and abandonment, a state of continual
  need and resignation -which Agamben (2013) identifies with
  Franciscanism- but at the same time places focus onto conducts,
  gestures, physiognomies and bodily features of the racialized
  otherness, the sub-human, the in-human, the beast-like and the
  zoological. Nonetheless, what happens in literature and contemporary
  art when work and poverty become unrecognizable because the ways in
  which reality and meaning are produced have been transformed under the
  neo-liberal landscape? Unlike the violence that inscribes poverty to
  radical distance, precarity illuminates physical proximity of the
  contagious and adjacent that begins to filter and permeate with new
  ways into the social landscape. In other words, if ``the poor is
  always the other; the precarious is, in change, the messenger of new
  insecurity from which I am not, nor will I ever be, well enough
  protected'' (Giorgi, 2019: 70). On this point, Gabriel Giorgi's work
  (2019) on Macabea, in Clarice Lispector is a crucial piece. Also, see
  mi previous inquiry in De Mauro Rucovsky (2016).}

\emph{Bíos}-precarious is not even reducible, in this sense, to new
subjectivities, to a new class or a social indicator such as the
precariat (Standing, 2011), the cognitariat, or the povertariat as Pablo
Semán (2017) suggests. Precariat, cognitariat or povertariat are
categories around a political dispute, malleable conceptualizations and
with imprecise borders, rather than sociological taxonomies, or
demographic indexes. As precarious we could name a number of figures:
online marketers, company's interns and freelance workers, itinerant
salespeople and popular economy sellers, cognitive workers and in the
cultural industry, the app delivery bike dealers and transport services
(Rappi, Pedidos Ya, Globo, Uber, etc.), home workers in domestic care,
laundry and cleaning services, housekeepers and babysitters, supermarket
cashiers, carton gatherers, and recyclers, security guards, temporary
and/or seasonal jobs in textile or industry assemblies. Ambivalent
categories and with permeable edges, they point out to a problematic
unresolved area with a long going conceptuality: who are those that fit,
or who do the precariat, cognitariat, povertariat name? What novelty
signs do they carry through, and what other things to they mobilize?
What is their epistemological reach and ontological malleability?

\End{multicols}

\textbf{Bibliography}

AAVV (2012) ``Precarity Talk. A virtual Roundtable with Laurent Berlant,
Judith Butler, Bojana Cvejić, Isabell Lorey, Jasbir Puar, and Ana
Vujanović''. En \emph{TDR: The Drama Review}, Volume 56, Number 4,
Winter 2012 (T216) pp.~163-177. Versión on line:
\href{https://muse.jhu.edu/article/491900}{\emph{https://muse.jhu.edu/article/491900}}
\href{https://muse.jhu.edu/journal/193}{\emph{(consultado 1/6/18)}}

\emph{\href{https://muse.jhu.edu/journal/193}{Agamben, Giorgio (2017)}
El uso de los cuerpos}. Buenos Aires: Adriana Hidalgo.

Agamben, Giorgio (2013) \emph{Altísima pobreza. Reglas monásticas y
formas de vida}. Bs.As.: Adriana Hidalgo.

Agamben, Giorgio (2010) \emph{Homo sacer. El poder soberano y la nuda
vida}. Valencia: Pre-textos.

Agamben, Giorgio (2006) \emph{Lo abierto El hombre y el animal}. Buenos
Aires: Adriana Hidalgo.

Agamben, Giorgio (2003) ``Biopolitica minore''. Entrevista por
P.Perticari. Roma: Edizione Manifestolibri

Athanasiou, Athena \& Butler, Judith (2013) \emph{Dispossession : the
performative in the political}. Cambridge: Polity Press.

Bazzicalupo, Laura (2017) \emph{Biopolítica. Un mapa conceptual}.
España: Melusina

Biset, Emmanuel (2016) ``Deconstrucción de la biopolítica''. En Pléyade
Nº17 / enero-junio (2016) / online Issn 0719-3696 / iSSn 0718-655X / PP.
205-222 / 219

Biset, Emmanuel (2017) ``Biopolítica, Soberanía y deconstrucción''. En
\emph{Thémata}. Revista de Filosofía Nº 56, julio-diciembre (2017) pp.:
285-303. ISSN: 0212-8365

Butler, Judith (2004). \emph{Undoing gender}. New York/London:
Routledge. Versión castellana: \emph{Deshacer el género}. Buenos Aires:
Paidós, 2006.

Butler, Judith (2004). \emph{Precarious life: the powers of mourning and
violence}. London New York: Verso. Versión castellana: \emph{Vidas
Precarias. El poder del duelo y la violencia}. Buenos Aires: Paidós,
2006.

Butler, Judith (2009). \emph{Frames of war: when is life grievable?}.
London New York: Verso. Versión castellana: \emph{Marcos de guerra. Las
vidas lloradas}. Buenos Aires: Paidós, 2010.

Butler, Judith (2011) ``For and Against Precarity''. En \emph{TIDAL:
Occupy Theory} Vol 1. Versión on line disponible:
\url{https://occupyduniya.files.wordpress.com/2011/12/tidal/_occupytheory.pdf}
(consultado 1/6/18)

Butler, Judith; Athanasiou, Athena (2013). \emph{Dispossession: the
performative in the political}. Cambridge, UK Malden, Massachusetts:
Polity Press. Versión castellana: \emph{Desposesión: lo performativo en
lo político}. Buenos Aires: Eterna Cadencia, 2017.

Butler, Judith (2015). \emph{Notes toward a performative theory of
assembly}. Cambridge, Massachusetts: Harvard University Press. Versión
castellana: \emph{Cuerpos aliados y lucha política. Hacia una teoría
performativa de la asamblea}. Buenos Aires: Paidós, 2017.

Briceño, Ximena (2017) ``Vidas secas or Canine Melancholia: Reflections
on Living Capital''. En Latin American Cultural Studies, 26:2, 299-319.
Routledge

Castro, Edgardo (2011a) \emph{Lecturas foucaulteanas. Una historia
conceptual de la biopolítica}. Buenos Aires: Unipe

Castro, Edgardo (2011b) ``Biopolítica: orígenes y derivas de un
concepto''. En \emph{Cuadernos de trabajo \#1 Biopolítica,
Gubernamentalidad, educación, seguridad}. Buenos Aires: Unipe

Castro, Edgardo (2013) ``Una cartografía de la biopolítica''. En
\emph{Cuadernos de pensamiento biopolítico latinoamericano \# 1}. Buenos
Aires: Unipe

Castro, Edgardo (2014) \emph{Introducción a Foucault}. Buenos Aires:
S.XXI

Canseco, Alberto beto (2017) \emph{Eroticidades Precarias. La Ontología
Corporal de Judith Butler}. Córdoba: Asentamiento Fernseh \&
Sexualidades Doctas Edit.

Chambers, Samuel y Carver, Terrel (2008) \emph{Judith Butler \&
Politics. Troubling Politics}. New York and London: Routledge.

Deleuze, Gilles (2008) \emph{Foucault}. Buenos Aires: Paidós

Deleuze,Gilles (1991) ``Posdata sobre las sociedades de control''. En
Christian Ferrer (Comp.) \emph{El lenguaje libertario}, Tº 2,
Montevideo: Ed. Nordan

De Mauro Rucovsky, Martín (2016) ``El fin del trabajo y la emergencia de
lo precario''. En Revista \emph{Nombres}, Revista de Filosofía, Nº~
30.~Dossier: Crítica de la economía política. Córdoba: Alción.

De Mauro Rucovsky, Martín (2018a) ``Tanta vida mutua. Mujeres y
precariedad animal'' en \emph{Alea: Estudos Neolatinos}. Revista do
Programa de Pos-Graduação em Letras Neolatinas, Faculdade de Letras
--UFRJ. Alea vol.20 no.2 Rio de Janeiro mayo/ago. 2018. Versión on line:
\href{http://www.scielo.br/pdf/alea/v20n2/1807-0299-alea-20-02-17.pdf}{\emph{http://www.scielo.br/pdf/alea/v20n2/1807-0299-alea-20-02-17.pdf}}
(consultado 15/8/19)

De Mauro Rucovsky, Martín (2018b) ``La vaca que nos mira: vida precaria
y ficción'' en \emph{Revista Chilena de Literatura}, Universidad de
Chile. Versión on line:
\href{https://revistaliteratura.uchile.cl/index.php/RCL/article/view/49094/51597}{\emph{https://revistaliteratura.uchile.cl/index.php/RCL/article/view/49094/51597}}
(consultado 15/8/19)

De Mauro Rucovsky, Martín (2019a) ``Rotar en la precariedad o sobre el
trabajo de los jóvenes''. En \emph{AContracorriente. Una revista de
estudios latinoamericanos}.~ NC State University, Vol. 16, Num. 3
(Spring 2019): 139-160. Versión on line:
\href{https://acontracorriente.chass.ncsu.edu/index.php/acontracorriente/article/view/1912/3277}{\emph{https://acontracorriente.chass.ncsu.edu/index.php/acontracorriente/article/view/1912/3277}}(consultado
15/8/19)

De Mauro Rucovsky, Martín (2019b) ``Taedium Vitae: Precarity and affects
in porteña night'' en \emph{E-Scrita} Revista do Curso de Letras da
UNIABEU Nilópolis, v.10, Número 1, janeiro-abril, 2019. Versión on line:
\href{https://revista.uniabeu.edu.br/index.php/RE/article/view/3554/pdf}{\emph{https://revista.uniabeu.edu.br/index.php/RE/article/view/3554/pdf}}
(consultado 15/8/19)

Esposito, Roberto (2005) \emph{Inmmunitas. Protección y negación de la
vida}. Buenos Aires: Amorrortu

Esposito, Roberto (2007) \emph{Communitas. Origen y destino de la
comunidad}. Buenos Aires: Amorrortu

Esposito, Roberto (2006) \emph{Bíos}. \emph{Biopolítica y filosofía}
Buenos Aires: Amorrortu

Esposito, Roberto (2009a) \emph{Tercera persona. Política de la vida y
filosofía de lo impersonal}. Buenos Aires: Amorrortu

Esposito, Roberto (2009b) \emph{Comunidad, inmunidad y biopolítica}.
España: Herder

Esposito, Roberto (2009c) ``Biopolítica y Filosofía: (Entrevistado por
Vanessa

Lemm y Miguel Vatter)'' en \emph{Revista de Ciencia Política} / VoLumen
29/ nº 1 / 2009 / 133 -- 141.

Esposito, Roberto (2011) \emph{El dispositivo de la persona}. Buenos
Aires: Amorrortu

Esposito, Roberto (2015) \emph{Dos. La máquina de la teología política y
el lugar del pensamiento.} Buenos Aires: Amorrortu

Esposito, Roberto (2016) \emph{Las personas y las cosas}. Buenos Aires:
Eudeba \& Katz

Foucault, Michel (1984) ``¿Qué es la Ilustración? {[}Qu'est-ce que les
Lumières?{]}'' Traducción de Jorge Dávila. En \emph{Actual}, No. 28,
1994.

Foucault, Michel (2003a) \emph{Historia de la sexualidad I. La voluntad
de saber}. Buenos Aires: Siglo XXI

Foucault, Michel (2004) \emph{Vigilar y castigar. Nacimiento de la
prisión}. Buenos Aires: SXXI

Foucault, Michel (2006) \emph{Seguridad, territorio y población}.
\emph{Curso en el Collège de France (1977-1978)} Buenos Aires: FCC

Foucault, Michel (2007) \emph{Nacimiento de la biopolítica} Curso en el
Collége de France \emph{(1978-1979)}. Buenos Aires: FCC

Foucault, Michel (2009) \emph{El gobierno de sí y de los otros}. Curso
en el Collége de France (1982-1983). Buenos Aires: FCE.

Foucault, Michel (2010) \emph{Defender la sociedad. Curso en el collège
de France 1975-1976}. Buenos Aires: FCE

Foucault, Michel (2016a) \emph{La sociedad punitiva}. \emph{Curso en el
Collège de France (1972-1973).} México: FCE

Gago, Verónica (2014) \emph{La razón neoliberal. Economías barrocas y
pragmática popular}. Buenos Aires: Tinta y Limón.

Garramuño, Florencia (2015) Mundos en común. Ensayos sobre la
inespecificidad del arte. Buenos Aires: FCE.

Giorgi, Gabriel \& Rodríguez, Fermín (2007) \emph{Ensayos sobre
biopolítica. Excesos de vida}. Buenos Aires: Paidós

Giorgi, Gabriel (2017). ``¿De qué está hecha macabea? Lispector y lo
precario''. En \emph{¿Por qué Brasil, que Brasil?}. V. María: Eduvim.

Giorgi, Gabriel (2016a) La noche de los cuerpos. En Kilómetro 11, 13 de
Julio de 2016. Versión on line:
\href{http://kilometro111cine.com.ar/la-noche-de-los-cuerpos/}{\emph{http://kilometro111cine.com.ar/la-noche-de-los-cuerpos/}}

Giorgi, Gabriel (2016b) Precariedad animal. En Boca de Sapo, año XVII,
N°21

Giorgi, Gabriel (2019) ``La incompetente. Precariedad, trabajo,
literatura''. En \emph{AContracorriente.} Vol. 16, Num. 3.

Giorgi, Gabriel (2014) \emph{Formas comunes. Animalidad, cultura,
biopolítica}. Buenos Aires: Eterna Cadencia

Huffer, Lynne (2015) ``Foucault's Fossils: Life Itself and the Return to
Nature in Feminist Philosophy''. En Foucault Studies, No.20, pp.122-141.
Versión on line:
\href{https://rauli.cbs.dk/index.php/foucault-studies/article/view/4933}{\emph{https://rauli.cbs.dk/index.php/foucault-studies/article/view/4933}}
(consultado 1/6/18)

Koruk, Carolina (2018) ``Tiempo de salario emocional: de qué se trata
este nuevo beneficio laboral''.~ Nota aparecida en ParaTi , Diario
Infobae, 26 de Septiembre de 2018. Versión on
line:\href{https://www.infobae.com/parati/estar-mejor/2018/09/26/tiempo-de-salario-emocional-de-que-se-trata-este-nuevo-beneficio-laboral/}{\emph{https://www.infobae.com/parati/estar-mejor/2018/09/26/tiempo-de-salario-emocional-
de-que-se-trata-este-nuevo-beneficio-laboral/}} (consultado 16/8/19)

Haraway, Donna (2016) \emph{Staying with the Trouble: Making Kin in the
Chthulucen}. Durham: Duke

Laera, Alejandra \& Rodriguez, Fermín (2019) ``El cuerpo del trabajo''.
En \emph{AContracorriente.} Vol. 16, Num. 3 (Spring 2019):31-38.

Lemke, Thomas (2017) \emph{Introducción a la biopolítica}. México:FCE

Lorey, Isabell (2006) ``Gubernamentalidad y precarización de sí Sobre la
normalización de los productores y las productoras culturales''. En
\emph{Transversal}, Nº11, 2006. Versión on line:
\href{http://eipcp.net/transversal/1106/lorey/es}{\emph{http://eipcp.net/transversal/1106/lorey/es}}
(consultado 1/6/18)

Lorey, Isabell (2010) ``Devenir común: la precarización como
constitución política''. En \emph{e-flux journal}, No 17, Julio de 2010.
Versión on line:
\href{https://privadotextos.wordpress.com/2011/01/28/devenir-comun-la-precarizacion-como-constitucion-politica/}{\emph{https://privadotextos.wordpress.com/2011/01/28/devenir-comun-la-precarizacion-como-constitucion-politica/}}
(consultado 1/6/18)

Lorey, Isabell (2016) \emph{Estado de inseguridad. Gobernar la
precariedad}. Madrid: Traficantes de sueños

Lorey, Isabell (2018) ``Preservar la condición precaria, queerizar la
deuda''. En \emph{Los feminismos ante el neoliberalismo}. Malena
Nijensohn (comp). Buenos Aires: Latfem \& La cebra

Mattio, Eduardo (2017) ʺGubernamentalidad y agencia resistente.
Consideraciones biopolíticas en la obra reciente de Judith Butlerʺ, en
\emph{¿Qué hacemos con las normas que nos hacen?}, editado por María
Victoria Dahbar, Alberto Canseco y Emma Song. Córdoba: Sexualidades
doctas.

Mills, Catherine (2016) ``Biopolitics and the Concept of Life''. En V.
W. Cisney, \& N. Morar (Eds.), \emph{Biopower: Foucault and Beyond}
(pp.~82-101). Chicago IL USA: The University of Chicago Press.

Moyano, Manuel Ignacio (2019) \emph{Giorgio Agamben}. \emph{El uso de
las imágenes}. Buenos Aires: La Cebra y Programa de Estudios en Teoría
Política.

Nancy, Jean Luc (1999) ``Conloquim''. En \emph{Communitas. Origen y
destino de la comunidad}. Esposito, Roberto (2007). Buenos Aires:
Amorrortu

\href{https://issuu.com/revista_pleyade/docs/pleyade7}{\emph{https://issuu.com/revista\_pleyade/docs/pleyade7}}
(consultado 28/5/17)

Neilson, Brett y Rossiter, Ned (2005) ``From precarity to Precariousness
and Back Again: Labour, Life and Unstable Networks''. En \emph{The
Fibrecultural Journal,} Nº5, December 2, 2005. Disponible en:
http://five.fibreculturejournal.org/fcj-022-from-precarity-to-precariousness-and-back-again-labour-life-and-unstable-networks/(consultado
28/5/17)

Pacheco, Mariano (2019) \emph{Desde abajo y a la izquierda. Movimientos
sociales, autonomía y militancias populares}. Buenos Aires: Cuarenta
Ríos.

Reid, Julian. 2011. ``The Vulnerable Subject of Liberal War.'' En
\emph{South Atlantic Quarterly} 110.3: 770-779. Versión on line
disponible:
\href{https://read.dukeupress.edu/south-atlantic-quarterly/article-abstract/110/3/770/3548/The-Vulnerable-Subject-of-Liberal-War}{\emph{https://read.dukeupress.edu/south-atlantic-quarterly/article-abstract/110/3/770/3548/The-Vulnerable-Subject-of-Liberal-War}}
(consultado 1/6/18)

Revel, Judith (2009) ``Identity, Nature, Life. Three Biopolitical
Deconstructions''. En \emph{SAGE, Theory, Culture \& Society}, Vol.
26(6): 45-54. Versión on line:

\href{http://journals.sagepub.com/doi/10.1177/0263276409348854}{\emph{http://journals.sagepub.com/doi/10.1177/0263276409348854}}
(consultado 1/6/18)

Rodriguez, Fermin (2010) Un desierto para la nación: la escritura del
vacío. Buenos Aires: Eterna Cadencia.

Romandini, Fabián Ludueña (2010) \emph{La comunidad de los espectros. I
Antropotecnia}. Buenos Aires: Miño y Dávila.

Semán, Pablo (2017) ``La grieta opositora''. En \emph{Le Monde
Diplomatique}, Edición Nº 217, Julio 2017.

Soley-Beltran, Patrícia y Beatriz Preciado (2007), ``Abrir
posibilidades. Una conversación con Judith Butler''. En \emph{Lectora},
13: 217-239. ISSN: 1136-5781 D.L. 395-1995.

Standing, Guy (2014) ``Por qué el precariado no es un «concepto
espurio»''. En \emph{Sociología del trabajo}, Nº82, Otoño de 2014,
Dossier ¿qué es el precariado?. Madrid: SXXI.

Standing, Guy (2011) \emph{El precariado. Una nueva clase social}.
Madrid: Pasado y Presente

Tsing, Anne Lowenhaupt (2015) \emph{The Mushroom at the End of the
World. On the Possibility of Life in Capitalist Ruins}. Princeton and
Oxford: Princeton University Press.

Watson, Janell (2012) ``Butler's Biopolitics: Precarious Community''. En
\emph{Theory \& Event} , Vol. 15 , Iss. 2; 2012 Baltimore: Johns Hopkins
University Press.

\end{document}
