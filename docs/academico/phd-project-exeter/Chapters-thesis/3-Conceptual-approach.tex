\pagebreak
\subsection*{Conceptual approach} %Research questions
The general problems of mass production and consumption %and their consequences 
have been noted even from the dawn of the industrial ages. 
The topic of massivity has run from the \textit{marvelous rise} of the car industrialization in the early XX\textsuperscript{th} century going back to the actual need to decongest traffic and search for new patterns of mobility and a sense of participating in the environment instead of driving over all ecosystems. This does not happen in a neutral and clean socio-political situation.
%
These conducts are part of a broad systematic pattern.
%More than 35 years ago 
Deleuze \& Guattari (2010: 527)
%a couple of french philosophers had 
signaled that
things as different as monopoly and the specialization of most of the medical knowledge,
the complication of the automobile motor, the gigantism of machines, do not correspond to any technological need,
but rather to economic and political imperatives. %(Deleuze \& Guattari, 2010: 527). 
%that aim to concentrate potency and control in the hands of a dominant class.
%\begin{quote} % Es evidente que cosas tan diferentes como el monopolio o la especialización de la mayor parte de los conocimientos médicos, la complicación del motor del automóvil, el gigantismo de las máquinas, no corresponden a ninguna necesidad tecnológica, sino solamente a imperativos económicos y políticos que se proponen concentrar potencia y control en las manos de una clase dominante (Deleuze y Guattari, 2010: 527).  %\end{quote}

Certain objects and conducts of today's societies have built and shaped urban landscapes in an ever growing manner.
Many of them blocking and constraining transit of people, of resources, and even being a blockage for ideas and customs.
It becomes increasingly widespread and evident the way in which vehicles stagnate in traffic during 
long inner-week-hours and amounts of cars lost in traffic through the world. 
Billions of people also follow, or intend to do so with tidy regularity, a standarized daily work schedule from 9 to 5.
Roads can function as boundaries when they striate space into fixed compartiments of places of circulation,
but roads can also be connectors of smooth space that open to the world and to infinite paths (Brighenti, 2009: 64).

%On the other hand, 
While several flows tend to become a little more at stop,
there are also countermovements that move against the said stagnation:
all together many different currents of flow could be taken into account, but mostly two different types coexist:
those which favor movement, and those that tend to collapse. They could be called rythms and anti-rythms.
We can percieve %experience 
%all 
variations in possibilities of movement by passing through spaces and becoming part with the surroundings.

% INGOLD page 167
\begin{quote}
%I remarked above that 
we experience the contours of the landscape by moving through it,
%so that it enters - as Bachelard would say - into our `muscular consciousness'. 
%Reliving the experience in our imagination, we are inclined to recall the road we took as 'climbing' the hill, or as 'descending' into the valley, as though 'the road itself had muscles, or rather, counter-muscles' (Bachelard 1964: 11). And this, too, is probably how you recall the paths and tracks that are visible to you now: after all, you must have travelled along at least some of them to reach the spot where you are currently standing. Nearest at hand, a path has been cut through the wheat-field, allowing sheaves to be carried down, and water and provisions to be carried up. Further off, a cart-track runs along the valley bottom, and another winds up the hill behind. In the distance, paths criss-cross the village green. Taken together, these paths and tracks 'impose a habitual pattern on the movement of people' (Jackson 1989: 146). And yet they also arise out of that movement, for 
[...]
every path or track shows up as the accumulated imprint of countless journeys that people have made 
% - with or without their vehicles or domestic animals - 
as they have gone about their everyday business. Thus the same movement is embodied, on the side of the people, in their `muscular consciousness', and on the side of the landscape, in its network of paths and tracks. In this network is sedimented the activity of an entire community, over many generations. It is the taskscape made visible. (Ingold, 1993: 167)
\end{quote}

Some people prefer to walk through a forest; others enjoy driving a car.
%
Many urban persona relate to the city in varying degrees. Some authors, such as Goffman and Von Uexküll, have 
seen how %much and the type of relation to 
the surroundings interact with animals and humans to %so as to 
create an environment, which they call the surroundings, or more technicaly: the \textit{Umwelt}, 
the involving space from which signs of alarm are expected. 
As it happens in ethology, it also applies in humans: ``the size of Umwelten
varies considerably according to the species''. Of interest here is that
this area can move, and can expand and contract according to whom, and how, is at the
center of this phenomenom. These perceptions, shared and negotiated, make
up to a pluralism of views and reactions. 

Additionally, every social world, in Becker’s terms, attracts a number
of resources, knowledge as well as consensus and resistance. Some social worlds
more than others depend on, and have an influence in
productive systems. Here it is claimed that the ultrarunners environment impacts 
on ways of living, and are on the tip of a certain curve of behaviour, that
is: of discipline, deprivation, potential mental disturbance and generally extreme
experiences, all of which affect the way cities are lived in.



