% https://latex.org/forum/viewtopic.php?t=31329 CHANGING TITLES IN SECTIONS // ABOUT ME ... ETC.
\documentclass[a4paper,hidelinks]{twentysecondcv} % a4paper for A4
\usepackage[spanish]{babel}
\selectlanguage{spanish}
\usepackage[utf8]{inputenc}

%----------------------------------------------------------------------------------------
%	 PERSONAL INFORMATION
%----------------------------------------------------------------------------------------

% If you don't need one or more of the below, just remove the content leaving the command, e.g. \cvnumberphone{}

\profilepic{avatar.png} % Profile picture

\cvname{Benjamín Juárez} % Your name
\cvjobtitle{Sociólogo} % Job title/career

\cvdate{\today} % Date of birth
\cvaddress{Córdoba, Argentina} % Short address/location, use \newline if more than 1 line is required
\cvnumberphone{(+54 9 351) 153 104043} % Phone number
\cvsite{http://ramoneando.com/} % Personal website
\cvmail{benjaminjuarezarlt@gmail.com} % Email address

%----------------------------------------------------------------------------------------

\begin{document}

%----------------------------------------------------------------------------------------
%	 ABOUT ME
%----------------------------------------------------------------------------------------

\aboutme{%
Hice Sociología a nivel de Licenciatura y Maestría. Recientemente trabajé fuera de la universidad. Armé proyecto propio: un bar de jugos de fruta natural, \textit{Juice Bar}. También conocí el mundo empresarial trabajando en \textit{Pearson} entre 2017 y 2018.
% En los últimos seis meses trabajé en la editorial Pearson como Asesor de Servicios de aprendizaje: me reunía con docentes, coordinadores y directivos de institutos y colegios para hacer seguimiento y recomendación de los materiales de trabajo.\\
% En los últimos cinco años trabajé académicamente: supervisando proyectos de investigación, dando clases de sociología y cursos de conversación en inglés. \\
% En todos los casos me interesan las áreas educativa, literaria, y el desarrollo del público amplio en todos los niveles de aprendizaje.
% Over the past 5 years, I have assisted students with graduate level projects. I read their texts
% and make observations on methodological and empirical topics, pointing out contents and
% readings, following each student’s progress.
% The last 2 years I have been giving Sociology classes at Centro de Espiritualidad Santa María.
% I see it as a possibility to put research questions into play and to probe the general feeling that
% sociological research has for a wider public.
% Alice is a sensible prepubescent girl from a wealthy English family who finds herself in a strange world ruled by imagination and fantasy. Alice feels comfortable with her identity and has a strong sense that her environment is comprised of clear, logical, and consistent rules and features. Alice's familiarity with the world has led one critic to describe her as a "disembodied intellect". Alice displays great curiosity and attempts to fit her diverse experiences into a clear understanding of the world.
} % To have no About Me section, just remove all the text and leave \aboutme{}

%----------------------------------------------------------------------------------------
%	 SKILLS
%----------------------------------------------------------------------------------------

% Skill bar section, each skill must have a value between 0 an 6 (float)
\skills{%
{programación: javascript php/3.7},
{web: html css md/4.2},
{edición-tecnología: \LaTeX+vim+ranger/4.4},
{portugués (fluído)/4.7},
{inglés (nativo)/5.5},
{esfuerzo/5.9},
{atención al detalle/5.7},
{buenos modales/5.3}%
}

%------------------------------------------------

% Skill text section, each skill must have a value between 0 an 6
\skillstext{%
{usuario linux/desde 2005 [hoy i3wm]},
{teclado dvorak/desde 2004},
{capacidad de aprendizaje/9,2},
{relajado/6,5},
}

%----------------------------------------------------------------------------------------

\makeprofile % Print the sidebar

\vfill 

%----------------------------------------------------------------------------------------
%	 INTERESTS
%----------------------------------------------------------------------------------------

% \section{Intereses}
% 
% The heroine and the dreamer of Wonderland; Alice is the principal character.


\section{Perfil General}

Tengo buena capacidad de adaptación, interés en áreas nuevas y la constancia para empujar de manera continua. Me crié en Córdoba desde los 7 años. Antes viví en EEUU. Desarrollé desde esa temprana edad gusto por el idioma y estudié inglés hasta un nivel nativo C1 [CEFR] y lo uso a diario para leer y escribir. También aprendí a leer, escuchar y hablar portugués de manera fluida, teniendo una estancia en Brazil durante más de dos años. Programación: autodidacta.\\

\section{Trabajo reciente}

\begin{twentyshort} % Environment for a short list with no descriptions
	\twentyitemshort{2017-2018}{Pearson [Córdoba, Argentina]} 
	\twentyitemshort{2017-2018}{Juice Bar [Córdoba, Argentina]}
	\twentyitemshort{2012-2014}{Beca de investigación \textsc{CNPq} en Sociología [S\~{a}o Paulo, Brasil]}
	\twentyitemshort{desde 2015}{Cursos de conversación en inglés [Córdoba, Argentina]}
	\twentyitemshort{desde 2012}{Clases de Sociología [Argentina-Brasil]}
	\twentyitemshort{2009--2010}{Librería Borders [Atlanta, EEUU]}
	%\twentyitemshort{<dates>}{<title/description>}
\end{twentyshort}


%----------------------------------------------------------------------------------------
%	 EDUCATION
%----------------------------------------------------------------------------------------

\section{Formación}

\begin{twenty} % Environment for a list with descriptions
 	\twentyitem{desde 2017}{Candidato doctoral en Sociología [a distancia]}{University of Exeter, UK}{El æfecto de los atletas de distancia en el ritmo urbano: cómo corren los ultramaratonistas en ciudades automatizadas?}
 	\twentyitem{desde 2013}{Lenguajes de programación-edición}{aprendiendo al hacer}{Autopubliqué libro [\textit{exhalaciones}] y sitio web [\texttt{ramoneando.com}]}
	\twentyitem{2012-2014}{Investigación de Maestría en Sociología}{UNICAMP, S\~{a}o Paulo. Brazil}{Arte Urbano [graffiti y pixaç\~{a}o]}
	\twentyitem{2002-2009}{Licenciado en Sociología}{Universidad de Buenos Aires, Argentina}%
	{%Estudios de graduación
	}
	%\twentyitem{<dates>}{<title>}{<location>}{<description>}
\end{twenty}

\section{Publicaciones/Presentaciones}

\begin{twenty} % Environment for a list with descriptions
	\twentyitemshort{2017}{being ralphy wiggum}
	\twentyitemshort{2016}{The visual and social indeterminacy of pixação: the inextricable moods of São Paulo’s inscriptions}
% 	\twentyitemshort{2016}{exhalaciones}
	\twentyitemshort{2014}{Espacios abiertos: la calle como hábitat}
	\twentyitemshort{2013}{Arte urbano paulistano: degradación urbana y paisajismo turístico}
	\twentyitemshort{2012}{Urban Geopolitics: Recongurations in Art, Activism, and Research}
	\twentyitemshort{2011}{(No) es tan fácil ser grafitero}
	\twentyitemshort{2008}{El legado de Durkheim en Schutz: hacia un horizonte en diálogo}
	\twentyitemshort{2008}{Observaciones sobre la opinión pública a partir de El ciudadano bien informado}
	%\twentyitem{<dates>}{<title>}{<location>}{<description>}
\end{twenty}


\section{Traducciones}

\begin{twenty} % Environment for a list with descriptions
	\twentyitemshort{2018... }{23 ideas sobre la juventud [Howard Becker]}
	\twentyitemshort{2011}{La influencia de las cosmovisiones en el pasado y en el presente según Max Weber [Stephen Kalberg]}
	\twentyitemshort{2005}{Los tipos de racionalidad de Max Weber: piedras angulares para el análisis del proceso de racionalización de la historia [S. Kalberg]}
	%\twentyitem{<dates>}{<title>}{<location>}{<description>}
\end{twenty}

%----------------------------------------------------------------------------------------
%	 OTHER INFORMATION
%----------------------------------------------------------------------------------------

\section{Información general}

% \subsection{Review}

En los últimos seis meses trabajé en la editorial \textit{Pearson} como Asesor de Servicios de aprendizaje: me reunía con docentes, coordinadores y directivos de institutos y colegios para hacer seguimiento y recomendación de materiales de trabajo. 
Antes aprendí en sociedad comercial a gestionar \textit{Juice Bar}.
\\
% Tengo buena capacidad de adaptación, interés en áreas nuevas y la constancia para empujar de manera continua. Me crié en Córdoba desde los 7 años. Antes viví en EEUU. Desarrollé desde edad gusto por el idioma y estudié inglés hasta un nivel C1 [CEFR] y lo uso a diario para leer y escribir. También aprendí a leer, escuchar y hablar portugués de manera fluida, teniendo una estancia en Brazil durante más de dos años.\\
En los últimos cinco años además trabajé académicamente: supervisando proyectos de investigación, dando clases de sociología y cursos de conversación en inglés. En todos los casos me interesa trabajar con un ambiente relacionado al mundo editorial y de servicio.

% materiales de texto que sirvan no solamente en los ambientes especializados de sociología y de inglés, sino para el aprendizaje del público general.
% Alice approaches Wonderland as an anthropologist, but maintains a strong sense of noblesse oblige that comes with her class status. She has confidence in her social position, education, and the Victorian virtue of good manners. Alice has a feeling of entitlement, particularly when comparing herself to Mabel, whom she declares has a ``poky little house," and no toys. Additionally, she flaunts her limited information base with anyone who will listen and becomes increasingly obsessed with the importance of good manners as she deals with the rude creatures of Wonderland. Alice maintains a superior attitude and behaves with solicitous indulgence toward those she believes are less privileged.

\section{Otros proyectos}

\begin{twenty} % Environment for a list with descriptions
	\twentyitemshort{2016-hoy}{\textit{ediciones inextricables} [publicación independiente]}
	\twentyitemshort{2014-hoy}{Sitio web personal: \texttt{ramoneando.com}}
	\twentyitemshort{2013-hoy}{herramientas de programación: html [+ css javascript md vim]}
	\twentyitemshort{2009-hoy}{herramientas de edición de texto y diseño: \LaTeX}
% 	\twentyitemshort{2005-hoy}{Sistemas operativos: microsoft, apple, gnu/linux}
	%\twentyitem{<dates>}{<title>}{<location>}{<description>}
\end{twenty}

%----------------------------------------------------------------------------------------
%	 SECOND PAGE EXAMPLE
%----------------------------------------------------------------------------------------

%\newpage % Start a new page

%\makeprofile % Print the sidebar

%\section{Other information}

%\subsection{Review}

%Alice approaches Wonderland as an anthropologist, but maintains a strong sense of noblesse oblige that comes with her class status. She has confidence in her social position, education, and the Victorian virtue of good manners. Alice has a feeling of entitlement, particularly when comparing herself to Mabel, whom she declares has a ``poky little house," and no toys. Additionally, she flaunts her limited information base with anyone who will listen and becomes increasingly obsessed with the importance of good manners as she deals with the rude creatures of Wonderland. Alice maintains a superior attitude and behaves with solicitous indulgence toward those she believes are less privileged.

%\section{Other information}

%\subsection{Review}

%Alice approaches Wonderland as an anthropologist, but maintains a strong sense of noblesse oblige that comes with her class status. She has confidence in her social position, education, and the Victorian virtue of good manners. Alice has a feeling of entitlement, particularly when comparing herself to Mabel, whom she declares has a ``poky little house," and no toys. Additionally, she flaunts her limited information base with anyone who will listen and becomes increasingly obsessed with the importance of good manners as she deals with the rude creatures of Wonderland. Alice maintains a superior attitude and behaves with solicitous indulgence toward those she believes are less privileged.

%----------------------------------------------------------------------------------------

\vfill 
\end{document} 

%%%%%%%%%%%%%%%%%%%%%%%%%%%%%%%%%%%%%%%%%
% Twenty Seconds Resume/CV
% LaTeX Template
% Version 1.1 (8/1/17)
%
% This template has been downloaded from:
% http://www.LaTeXTemplates.com
%
% Original author:
% Carmine Spagnuolo (cspagnuolo@unisa.it) with major modifications by 
% Vel (vel@LaTeXTemplates.com)
%
% License:
% The MIT License (see included LICENSE file)
%
%%%%%%%%%%%%%%%%%%%%%%%%%%%%%%%%%%%%%%%%%

%----------------------------------------------------------------------------------------
%	PACKAGES AND OTHER DOCUMENT CONFIGURATIONS
%----------------------------------------------------------------------------------------
