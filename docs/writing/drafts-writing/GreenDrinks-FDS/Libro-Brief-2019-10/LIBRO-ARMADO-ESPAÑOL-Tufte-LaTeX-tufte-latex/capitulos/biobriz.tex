  \end{fullwidth}
   

\chapter*{BIOBRIZ: AGUA MÁS LOMBRICES \\ 
IGUAL A TRANSFORMACIÓN}
\addcontentsline{toc}{chapter}{BIOBRIZ: AGUA MÁS LOMBRICES IGUAL A TRANSFORMACIÓN}

\label{biobriz-agua-muxe1s-lombrices-igual-a-transformaciuxf3n}
  \begin{fullwidth}

% \begin{figure}[htbp]
% \centering
% \includegraphics[width=1.75000in,height=1.75903in]{media/image6.jpeg}
% \caption{}
% \end{figure}

\textbf{Daniel Horacio Blando}

Ingeniero Agrónomo, Fundador de Biobriz, tecnología limpia para
efluentes de agua.

\emph{``Tú bienestar es mi bienestar''}

Agua + Lombrices = transformación

Biobriz es el nombre que utilizo en mi actividad: diseñar filtros
biológicos para purificar aguas contaminadas, con la gran curiosidad que
los actores principales en estos filtros son las lombrices.

~

La primera idea

En esta ocasión la idea era tener una actividad paralela a la que venía
realizando como asesor agropecuario, que tenga que ver con las
cuestiones productivas y de la naturaleza, que son los temas que me
gustan y conozco. La producción de enmiendas orgánicas por medio de
lombrices siempre dieron vuelta por mi cabeza, por lo que comencé a
imaginar cómo hacerlo, cómo ponerlo en marcha y para que funcionara con
rentabilidad tenía que ser a una escala significativa.

~

Recursos, madurando la idea

Sabía que no disponía de tiempo, ni terreno, ni de suficiente
dinero\ldots{} ¡Eso sí, ganas no faltaban! Fue cuando en mi imaginario
pensé que podía armar un sistema en el que al alimento de las lombrices
(estiércol de caballo o vaca en este caso) podría repartirlo utilizando
agua como vehículo en vez de carretillas. Pero no quería consumir agua
en exceso, por lo tanto, las camas de lombrices debían funcionar como
lecho filtrante para recuperar esa agua y volverla a utilizar,
repitiendo el ciclo para la alimentación.

~

El primer paso

Decidí comprar mis primeros núcleos de lombrices. Para solucionar la
falta de tiempo, acordé con el vendedor de las mismas que él las
mantuviera y para cubrir ese costo de mantenimiento, se quedara con el
humus producido\ldots{} ¡El vendedor estaba feliz! Me vendió las
lombrices, no me entregó nada y se quedaba con la producción. Yo me fui
feliz también, sabiendo que tenía mi capital lombrices ya trabajando.

~

Puliendo la idea, sorpresa

Mientras tanto, mi cabeza seguía ``maquinando'' e investigando. Encontré
en internet un artículo que hablaba del tratamiento de aguas efluentes
con filtros biológicos de lombrices y luego una tesis acerca del mismo
tema aunque sumaba el control de enfermedades en el agua. Mi cabeza y
corazón a esa altura iban muy rápido y muy entusiasmados. Logré darme
cuenta, con la poca información abierta que había, que el diseño de
estos filtros biológicos era prácticamente igual a mi diseño para
producir el lombricompuesto y recuperar el agua.

~

Investigando

Continúe con mi ``negocio redondo'' de comprar lombrices y no tenerlas.
Y además conseguí un socio temporal que compró algunas más y me aportó
la estructura legal para presentar un proyecto de investigación y
desarrollo para el tratamiento de aguas efluentes. El proyecto se aprobó
y es así que iniciamos la fase de investigación en la que participaba la
Cooperativa de Aguas de Alta Gracia, que nos daba el espacio físico y el
efluente. Los análisis de control de calidad del agua se realizaban en
el laboratorio de la Universidad Católica de Córdoba, que ya, previo al
proyecto, en una prueba que hice, se habían sorprendido por los
resultados de mejora del agua.

~

Aparece el trabajo

Los resultados del sistema seguían mostrando su excelente eficiencia.
Por esas cuestiones que aparecen en la vida, en las que yo digo que me
acompaña la Fe y a la que mi hijo denomina con una parte del cuerpo, un
importante medio radial se enteró de lo que estaba haciendo con las
lombrices y me entrevistaron por semejante curiosidad. Pasaron unos
meses y recibí llamados telefónicos de dos fábricas alimenticias, a
pesar de que no había dejado mis datos de contacto en la entrevista. Me
preguntaron qué podía hacer con su problemática, respuesta que no tenía,
``sé solo de materia fecal\ldots{} debemos hacer ensayos'', les dije.
Aceptaron y los resultados nuevamente fueron excelentes. Se vino la
etapa de presupuestar el primer trabajo para la fábrica de mermeladas.

~

¿Y cómo hago?

Necesitaba para este proyecto mil doscientos núcleos de lombrices y solo
tenía para entregar treinta\ldots{} ¿Y ahora? Le propuse a la fábrica
construir el filtro por etapas, entregar las lombrices de a poco y que
puedan hacer la inversión de construcción en un período de un año. La
otra propuesta fue que las lombrices eran mías hasta que las terminaran
de pagar y les di un plazo largo, fue una especie de leasing de
lombrices. Todo esto me dio tiempo.

~

Solución, impacto, criadero

Fui en búsqueda de alimento para multiplicar mis lombrices, golpeé las
puertas en un frigorífico y les dije: ``¡Necesito estiércol!''. -
``Tenemos mucho'', me dijeron, y me invitaron a quedarme, a armar el
criadero ahí, me dieron el terreno, el agua y horas de máquina para
mover el estiércol y compostarlo. Fue el primer gran impacto de este
hermoso trabajo: los dos salimos beneficiados, el frigorífico solucionó
cuestiones ambientales como olores, contaminación de suelo, disposición
final y bajar los costos; ¡Y a mí, la posibilidad del criadero! En
definitiva el filtro se armó y hoy está remediando 300.000 litros de
agua por día, dejándola en condiciones de uso para riego, instancia en
la que es mejorada aún más, siendo éste el segundo gran impacto positivo
en el ambiente, y generando puestos de trabajo.

La transformación: Las alquimistas de los desechos: ¿Por qué sucede?
¿Cómo es posible?

El agua efluente, sólo por pasar un instante por el lecho filtrante de
lombrices mejora su condición en un 90\%, incluso aguas con extrema
acidez son neutralizadas. 600 a 700 millones de años haciendo lo mismo
dan una buena experiencia; es el tiempo que están las lombrices en el
planeta. Uno de los secretos de su supervivencia es la transformación,
no la de ellas, pues su estructura es prácticamente la misma desde su
origen, sino la del medio en el que viven.

¡Los desechos son su especialidad! Y en la naturaleza los desechos son:
hojas caídas, plumas, pelos, insectos muertos, bacterias y enormes
cantidades de estiércol o materia fecal, entre otros. Todo lo orgánico
es su alimento, todo lo digieren con la ayuda de bacterias y hongos, ya
que no tienen dientes para comer directamente, sino que succionan su
alimento. Hasta recién estaba hablando de desechos y ahora hablo de
alimentos, es que en la naturaleza siempre hay alguien que aprovecha los
desechos de otros.

~

Una curiosidad asombrosa de las lombrices

Sus propios desechos son su mejor medio de vida, ahí están seguras,
protegidas, el famoso lombricompuesto es nada más y nada menos que
estiércol de lombriz. Con gran cantidad de sustancias húmicas, formadas
éstas por macromoléculas con propiedades fantásticas, como la de ser un
poderoso ``buffer'', un agua muy ácida es neutralizada rápidamente.
Dicha agua mataría a las lombrices pero el medio que ella generó las
protege. También tiene otras propiedades, como la de atrapar elementos
que pueden ser contaminantes, por ejemplo metales pesados, y
estabilizarlos en el medio. También tienen la propiedad de facilitar la
movilidad de los nutrientes y dejarlos en forma fácilmente disponibles
para las plantas. En este medio las lombrices se movilizan cavando
galerías que facilitan el ingreso del aire al suelo, el ingreso de agua
de lluvia, la movilidad de otros insectos y un lugar por donde penetran
y se desarrollan con mayor facilidad las raíces de las plantas.

A esta altura ya puedo decir que las lombrices no sólo generan un
ambiente sustentable para ellas, sino también para los demás.

Caminando

Anécdotas no faltan pero quiero hacer hincapié en los aprendizajes y en
los ``darme cuenta''. Los nuevos trabajos fueron apareciendo
prácticamente solos como si estuvieran esperando mi evolución. Cada uno
tenía una problemática, un efluente y un entorno diferente. En cada uno
de los trabajos realizaba ensayos de eficiencia del sistema y aparecían
escollos o problemas a solucionar, sin embargo siempre en forma
generosa, el sistema nos aportaba una solución. Me di cuenta que por una
cuestión de necesidad se incorporaron en mi cotidianeidad términos como
\emph{investigación}, \emph{desarrollo} e \emph{innovación}, ¡eso estaba
haciendo! También me enteré que estaba haciendo Responsabilidad Social
Empresarial (RSE) el día en que una empresa me pidió que complete una
planilla que hacía referencia a esto.

~

Perseverancia y alientos

Este ``darme cuenta'' de lo que estaba haciendo me aportaba caricias de
aliento que ayudaron a que la perseverancia siempre tenga luces de
buenos logros. Ni hablar de la familia y las personas que se
entusiasmaban cuando contaba lo que estaba haciendo con esta poderosa
herramienta lombricera.

Los escollos más grandes que encontré en el camino no están en lo
técnico, los obstáculos están en las burocracias de los sistemas legales
que siempre están detrás de las necesidades. Ni que hablar en términos
de sustentabilidad, que impiden y atan de manos incluso a funcionarios
con buenas intenciones y con ganas de buen progreso. Varias amarguras
viví con esto\ldots{}

~

¿Dónde estoy parado?

Fue cuando encarné el término de ``pionero'', no por una cuestión de
egos, si no porque esto me potenció las fuerzas al saber que el camino
era más lento, que debía tener más paciencia y que un pionero siembra
para el presente, pero sobre todo para las futuras generaciones. Y los
trabajos del presente no son menores. Ya hay muchos filtros de lombrices
purificando agua, millones de litros al año. Muchos de ellos en
industrias, hoteles, un dispensario, cabañas, un edificio de
departamentos, incluso en un trabajo de remediación de 20 hectáreas de
suelo, aportando grandes volúmenes de materia orgánica en forma de humus
de lombriz.

Cambié mi percepción del concepto de \emph{éxito}. Cada vez que tenía un
gran proyecto en frente y lograba realizarlo pensaba que llegaba al
éxito del mismo. Pero esto no me gusta así, pues es efímero, se escurre
entre los dedos, y la realidad te dice que hay que seguir y seguir
trabajando. Por eso, más que nunca disfruto de mi éxito cotidiano que
es, ni más ni menos, que poner alegría, felicidad y entusiasmo todos los
días a lo que toque vivir.

También me gustaría mencionar que en estos tiempos, donde todo es
urgente y los resultados deben ser ya, aparecieron en este camino frutos
de siembra que hice durante toda mi vida. Es increíble la cantidad de
hechos pasados que fueron necesarios aparecer para que este proyecto sea
hoy una realidad.

Biobriz es un buen negocio, con buen presente y con muchas oportunidades
de crecer, el ambiente lo pide con urgencia. Las lombrices son una
poderosa herramienta. El combustible que moviliza esto es la misión como
empresa y como persona, que cada vez se impregna más de la misión de las
propias lombrices: transformar lo malo en bueno, generar ambientes
sustentables para ellas y para los demás, el bienestar es fruto de ello.
Disfruto mucho de esta hermosa ``misión''.