% \hypertarget{parques-argentinos}{
\chapter{PARQUES ARGENTINOS}\label{parques-argentinos}
% }

% \begin{figure}[htbp]
% \centering
% \includegraphics[width=1.78542in,height=1.74444in]{media/image3.jpeg}
% \caption{}
% \end{figure}

\textbf{Emmanuel Baghin}

Fundador de ``Parques Argentinos''.

\emph{``El secreto no es correr tras un Colibrí\ldots{} es cuidar tu
jardín para que él no dude en visitarte''.}

\textbf{\\
}

Parques Argentinos\footnote{\textbf{\emph{Web:
  \href{http://www.parquesargentinos.com}{\emph{www.parquesargentinos.com}}}}}

Empresa experta en Biosoluciones y Consultoría medio ambiental,
Paisajismo Inteligente y Sustentable.

~

Cuando creamos Parques Argentinos\ldots{}

Lo hicimos pensando en la importancia que tenían los espacios verdes en
nuestras vidas. Pasaron los años y descubrimos que vincularnos con la
naturaleza desde el Diseño, podía mejorar muchos aspectos de lo
cotidiano. ``En Parques Argentinos abordamos el desarrollo de los
Espacios Verdes como un ser vivo, integral y dinámico. Conocemos
profundamente las leyes que rigen la vida de este organismo; para esto
nos especializamos en el arte de combinar herramientas tecnológicas,
insumos y saberes, con el propósito de lograr el mayor potencial de
calidad ambiental que la naturaleza pueda ofrecernos.''

~

Innovamos permanentemente, en el uso racional de productos orgánicos, en
sustratos de alto rendimiento biológico y, en estrategias de manejo en
vivero para cada caso particular.

Nos proponemos alcanzar el equilibrio dinámico de los Espacios Verdes,
con la máxima eficiencia en el uso de recursos, pero con el mínimo
consumo energético posible.

De esta manera, planificamos y restauramos espacios que están ávidos de
colores, aromas, texturas y vida silvestre.

Planteamos una alternativa sustentable en Diseño Ambiental y en Bio
Paisajismo para lograr la armonía deseada, de ese lugar tan importante
para tu vida.

~

~

Trazamos un camino\ldots{}

~

Con estas bases filosóficas, trazamos un camino a lo largo de estos 15
años, donde nuestra misión, se centra en la reconexión con nuestros
principios más arcaicos, volver a convivir y conectar con la naturaleza,
y esto además de hacernos más inteligentes como seres humanos, nos
enseñó a descubrir que los espacios verdes disminuyen el estrés de las
personas que los habitan en el corto plazo y generan esperanza y vida
para los que vendrán.

~

Al considerar un jardín como un ecosistema biológico y dinámico, y
literalmente pasa a ser un ambiente más de nuestros hogares, empresas o
incluso hasta oficinas, no podíamos olvidar que el aspecto estético debe
ser seductor y placentero, ya que personas convivirán durante muchos
años en él.

~

Bueno una vez dicho esto, a modo introductorio, vamos a bajar a tierra y
expandir nuestra propuesta de valor, y a compartir con usted cuáles son
los puntos que nos hicieron fuertes, los errores y fracasos que nos
permitieron crecer, la metodología de trabajo, y muchos aspectos claves
que situaron a nuestra empresa como un ícono en el interior del país.

~

En los comienzos, y en principio en nuestra Córdoba tan querida, había
una enorme brecha entre la jardinería y el paisajismo, con esto quiero
decir, que los espacios verdes, simplemente eran un área minúscula y
poco importante dentro de la proyección arquitectónica, cuando digo
minúscula no me refiero a metros cuadrados, sino a poco considerada.

Esto reducía el ambiente exterior a un proyecto de mínimos
requerimientos: una pradera verde de césped, y no más que alguna que
otra planta dispersa por ahí. Con esta cultura fuimos educados por
muchos años, y sin duda alguna marcó nuestra consciencia racional y
colectiva.

~

Otras culturas del Verde\ldots{}

~

En ésta era de la información, dónde los medios, el internet y la
comunicación nos dan acceso a otras culturas del verde, poco a poco la
gente se fue interiorizando más en el tema y empezaron a germinar
proyectos maravillosos, desafiantes y más personales. La gente comenzó a
viajar más, a explorar y descubrir que habitamos muchas horas de nuestra
vida en el jardín, con lo cual pasó a ser un protagonista más en los
proyectos inmobiliarios; Además, se empezó a considerar la huella
ecológica, los productos ecológicos, el impacto ambiental, las
calificaciones LEED, el green marketing, el consumo de recursos como
agua, electricidad entre otros, y otro aspecto peculiar que es el nivel
de ansiedad y estrés que van adquiriendo las personas en este mundo
donde todo es tan veloz y tan competitivo.

~

Aquí nace nuestro nicho, justo en estas condiciones. Dimos con la gran
promesa personal, se dieron dos constantes simultáneas, por un lado, la
jardinería y el paisajismo se comenzaban a dar la mano, y por el otro
los avances tecnológicos aplicables al mundo verde. Ya se distinguía
notoriamente un ``corta césped, de alguien que podía contemplar un
parque o jardín como un ser integral, sin desmerecer el oficio de las
personas que cortan césped, pero sí contrastando la cantidad de
conocimientos que hacen a un experto en paisajismo. Esta distinción nos
interesa hacerla ya que la oferta de capacitación sobre temas de
cultivos más ornamentales hoy es más nutrida y profesional, pero hace
unos diez o doce años atrás era muy pobre. La biología toca muy de lejos
el diseño, pero profundiza de manera excelente sobre la vida vegetal y
animal y sus necesidades. La agronomía, apunta a la producción y muy
poco a los aspectos estéticos y de manejo del espacio, pero tiene
herramientas valiosísimas en cuanto a esquemas de riego, estructuras de
suelos y sustratos, labores mecánicas, etc\ldots{} Y por último la
arquitectura, donde sí se intensifican aspectos de volumen, manejo del
espacio, diseño, materiales y proyección, pero se les escapa el reino
vegetal, y las necesidades de los diversos cultivos, para lograr
jardines prósperos y armónicos.

~

¡La lamparita!

~

Aquí se prendió la lamparita, logrando una comunión entre los puntos
fuertes de cada profesión, sumando lecturas e intereses personales y un
equipo ganador de profesionales, copiando ideas también, y así nació lo
que hoy es Parques Argentinos, un híbrido con valores
interdisciplinarios, dotado de herramientas y recursos obtenidos en
diferentes áreas mediante cursos y formaciones con docentes
especializados de Argentina, Santiago de Chile, Barcelona, Madrid,
Tel-Aviv entre otros, combinando la tecnología más avanzada con el arte
del paisaje y cultivos de primera calidad.

Hace ya diez años dejamos atrás la idea de ser un vivero más, y nos
enfocamos en dar el máximo valor a nuestros clientes. Como en todo
camino, monetizar la propuesta no fue fácil, éramos una empresa joven,
con mucha voluntad y disciplina, pero ganar confianza y fidelizar
clientes es un objeto que abarca mucho más que conocimiento y ganas.

~

La máxima puesta a prueba\ldots{}

~

Fue un merito escalón a escalón, ampliando conocimientos en marketing,
ventas, diseño, financiación, validando ideas, prototipando nuevos
productos, alianzas estratégicas, y hasta incluso conseguir tecnología y
materiales poco comunes en nuestro país, para ser honestos en el camino
muchas veces vivimos grandes desilusiones, ganas de abandonar todo y
decirnos este negocio no es rentable, estamos dejando la vida y esto no
funciona, nadie valora lo que hacemos y cosas por el estilo,
considerando hasta incluso cerrar y cambiar de actividad, y la máxima
puesta a prueba fue en el año 2012 que de repente un socio fundador, de
manera inesperada y repentina sufre un infarto masivo en plena jornada
de trabajo ante varios clientes. En ese momento, el mundo se derrumbó
para todos los que quedamos. Teníamos dos viveros minoristas
funcionando, un vivero mayorista con alta tecnología produciendo, y
cuatro áreas de servicios caminando; Ofrecíamos Servicios como Diseño y
ejecución Paisajismo Integral, con todo lo que esto significa, desde un
buen sistema de riego hasta el mantenimiento y cuidado a medida que
evolucionaban los jardines.

~

Podríamos decir que este fue uno de los momentos clave que nos hicieron
madurar por la fuerza, a nivel emocional y profesional, una pérdida
valiosa; en ese momento, el único responsable de cada decisión que se
tomara era solo yo. Me encontraba con 27 años, una familia compuesta por
mi mujer, una hija de 2 años y un varón en camino. Lo primero que
considere fue verme fuerte, pero era evidente que por dentro estaba
totalmente vulnerable y débil, a los pocos días, se acercaron algunos
clientes, personas muy estimadas, a traer sus pésames, y fue ahí que me
rendí ante ellos pidiendo consejos y enseñanzas, eran personas
empresarias, de éxito y muy maduras en sus vidas, me brindaron todo su
apoyo, sabios consejos, y a los diez días de este evento desafortunado
tome una decisión trascendental para el futuro de la empresa.

~

Lo primero que hice fue cerrar el vivero mayorista de atención al
público, y decidí mejorar la imagen de lo que pudiera controlar,
quedando así dos viveros de venta al público minorista en zona norte de
la ciudad, reduje el cultivo, solo producía el mínimo que después
consumiría en mis locales propios, y me desafíe a iniciar una nueva
etapa donde el plan predominante era compartir de manera gratuita todos
mis conocimientos, arranqué en televisión, en radio, y en charlas
abiertas a gente aficionada o profesionales del área. Esto nos posicionó
como un referente en el mundo del paisaje, y paralelamente comenzamos a
acelerar nuevos productos que, hasta ese momento, eran solo ideas.

~

Nuestros Productos y Servicios\ldots{}

~

A finales del 2012, Parques Argentinos como marca paraguas, logró
profundizar y brindar nuevos productos y servicios, dejando atrás la
metodología de trabajo convencional que nos caracterizaba y reformulando
un plan y una visión basada en los recursos y fortalezas que consideraba
oportuno y llamativo para los años venideros. De alguna manera dejé de
lado la vocecita de mi cabeza que hasta ese entonces jugaba para no
perder, y viré a una mentalidad de jugar para ganar, y pude darme cuenta
que, si bien a veces se pierde, al final de cada balance el sube-baja
siempre me daba positivo.

~

En la actualidad tenemos varias unidades de negocios trabajando. Hoy
puedo decir que soy un emprendedor afortunado y que me esfuerzo día a
día por ser un experto en mi área, que empecé con muy poco, solo tenía
ganas, y mucho amor por las plantas, abandoné la carrera de agronomía a
mitad de camino, leía cuanta cosa se me cruzaba relacionada a mis
intereses, participo de cuanto curso me pueda sumar, y agradezco
profundamente a mis raíces y en particular, a mis abuelos que me dieron
muchas razones para amar el campo, las plantas y la naturaleza en toda
su integridad.

Hoy la gente está llena de excusas, se victimizan y justifican de mil
maneras porque no lo logran, pero nuestro humilde aporte social es que,
si amás lo que hacés y soñás a lo grande, los recursos, personas e
incluso el dinero aparecen en tu vida. Por eso digo que, si realmente el
camino tiene corazón, eso que hacés finalmente te amará mil veces más.

~

Nuestra propuesta de valor actual\ldots{}

~

Se compone de viveros de venta al público y una plataforma virtual,
donde encontrarás profesionales capacitados, para instruirte sobre la
solución a tu problema.

Brindamos consultoría, diseño, dirección técnica y ejecución de
cualquier tipo de desarrollo ligado al mundo verde, con propuestas y
tecnología de primera calidad, sobre Techos Vivos, Jardines Verticales,
Piscinas Naturales, Bio Paisajismo, Huertos urbanos y ecológicos,
Sistemas de Riego, y por supuesto todo tipo de plantas, sustratos y
herramientas para obtener resultados óptimos en los ecosistemas
naturales, que llamamos Jardín.

Este año debido a la cantidad de personas que nos piden capacitación,
diseñamos un modelo llamado ``Aula de Formación'' donde hay propuestas
increíbles sobre cómo hacemos lo que hacemos, dispuestos a compartir
todos nuestros conocimientos y experiencias para que nuestros alumnos
puedan generar espacios llenos de vida e incluso hacerse del bagaje
técnico para salir a trabajar con herramientas y sistemas probados.

~

Siempre sentimos que diseñar un parque, planificar áreas verdes de
recreación, forestar una plaza, instalar un riego, alimentar familias de
huertas construidas por nosotros, tapizar con césped un campo de
deporte, o simplemente preparar un suelo para cultivar un cantero de
hierbas, iba a impactar de~\emph{forma positiva en las personas}.Fue
nuestra actitud insaciable hacia la mejora la que nos llevó a lograr
todo lo que nos propusimos.

~