\hypertarget{mikuna}{\chapter{MIKUNA. CAMBIO AGROALIMENTARIO}\label{mikuna}}

% \begin{figure}[htbp]
% \centering
% \includegraphics{media/image15.jpeg}
% \caption{}
% \end{figure}

\textbf{Ignacio y Agustín Mayorga}

Los hermanos Mayorga son los fundadores de Mikuna. Ignacio como
Ingeniero Agrónomo promueve los sistemas productivos ecológicos, y
Agustín Gerencia la empresa.

Mikuna

~

\emph{Mikuna, una pequeña gran empresa trabajando para un cambio de paradigma en el sistema agroalimentario.}

~\textbf{Orígenes}

Mi nombre es Ignacio Mayorga y desde muy pequeño tuve un profundo
interés y amor por la naturaleza. Antes de ingresar a la carrera de
Ingeniería Agronómica ya había estado de voluntario en ONG, había creado
un grupo de ecología, había querido ser guardaparque e incluso había
estado un tiempo como estudiante en Veterinaria y en Biología. Al poco
tiempo de graduarme fui invitado a sumarme a un grupo que arrendó un
campo en Buenos Aires para hacer soja con el paquete convencional. Como
loco del grupo fui el impulsor de investigar si los maíces andinos se
podían cultivar bien en Buenos Aires. Hicimos dos viajes, colectamos
semillas en Jujuy y sembramos dos hectáreas. La idea era venderlos para
consumo o decoración. Un laburo súper interesante, pero complejo.

Todo esto lo hacía en simultáneo a una serie de posgrados y
capacitaciones cuya motivación principal, incluso no muy consciente, era
encontrar desde el saber académico-científico, una solución a los
problemas resultantes de los paradigmas imperantes. No respondía al
perfil estereotipado del ingeniero agrónomo, al punto que mi abuela
materna solía preguntarle a mi mamá qué era lo que yo hacía.

\textbf{China y el llamado a recorrer otro camino}

En 2010 estaba haciendo un doctorado en economía rural en China. Esa
experiencia en Nanjing fue un momento bisagra en mi vida en la que
decidí retornar a Argentina, dejar la vida académica y trabajar ``desde
la trinchera''. El 4 de enero de 2011, luego de declinar una oferta de
empleo en la Capital Federal,~llegaba junto a mi compañera Natalia a la
provincia de Jujuy. Venía a instalarme para trabajar en un proyecto
independiente al que me habían convocado y que, si bien nunca llegó a
comenzar, fue el inicio de una nueva vida.

Al tiempo, viviendo ya en Tilcara, comencé a trabajar en un proyecto
privado que se propuso producir en forma orgánica y participé
activamente en espacios vinculados a los alimentos saludables. A fines
de 2013 en una reunión en Salta con referentes nacionales en la temática
se da la oportunidad de contar la idea de armar una red de
emprendimientos productivos de tipo orgánico. La propuesta era la de
agregar valor a través de iniciar con un local comercial que coordine la
comercialización, les de visibilidad y promueva el turismo rural en
ellos. Allí, un par de colegas me animaron a armar el proyecto para
conseguir financiamiento.

\textbf{Mi hermano Agustín}

Mientras tanto, durante todos esos años, mi hermano Agustín se venía
dedicando al mundo de los eventos corporativos. Se había iniciado en ese
medio desde muy chico y con los años logró generar y posicionar
exitosamente eventos en numerosos países de Latinoamérica. El día a día
de Agustín transcurría en pleno microcentro porteño, muy lejos de donde
yo quería estar. Sin embargo, recuerdo muy bien una charla telefónica
que tuve con él ni bien llegué a Jujuy en la que salió un tema que venía
de antes: algo adentro de él le reclamaba un cambio. Sin embargo, pasaba
el momento o período de angustia aguda y como le gustaba su trabajo, él
seguía con esa forma de vida.

\textbf{Primeros pasos}

Estaba viviendo en el campo, tenía un sueldo en blanco, mi primer hijo
ya había nacido y la idea de esta red tuvo la oportunidad en 2014 de
participar en el concurso del Centro Jujeño de Apoyo a Emprendedores
(CeJAE). Allí conocí, por ejemplo, la plantilla de negocios CANVAS, que
en tantos años de academia no había visto nunca, como tantas otras
cosas. Además, conocí a gente muy interesante, inquieta y que le
aportaron mucho al proyecto. En simultáneo, gestionaba financiamiento
vía un canal institucional en Salta que para los últimos meses del año
objetó la propuesta. Esto motivó que impulsará junto a otras personas y
referentes, la creación de una vía similar en el seno de un ministerio
en Jujuy.

Para fines de 2014 la propuesta de la red había sido premiada por el
CeJAE y esto significó que yo viajara a una capacitación en el Centro de
Intercambio de Conocimiento (CIC) de Costa Rica. También había logrado
el apoyo institucional desde un ministerio provincial para solicitar
financiación a nivel nacional y un primer local comercial que alquilamos
conjuntamente a un colega-socio que se sumó al proyecto. Mi deseo era
alinear acciones e integrar lo que estaba haciendo en relación de
dependencia junto con este proyecto de la red. Le propuse y ofrecí esto
al entonces jefe mío, era un ganar-ganar desde mi perspectiva, pero la
respuesta fue no positiva.

\textbf{Casualidades y la llegada de Agustín}

La situación convergió casualmente con tres sucesos muy sincrónicos: 1)
el 14 de marzo de 2015 cerca del mediodía me avisan que el proyecto de
la red había sido aprobado en una instancia institucional nacional e iba
camino a ser financiado 2) por la tarde concreto con ayuda de uno de mis
cuñados un stand para participar con la red en una feria en Buenos Aires
y 3) a eso de las siete de la tarde del mismo día recibo un llamado de
Agustín que al día siguiente tendría su cumpleaños número 33.

Como lo había hecho 4 años antes, mi hermano me llamo nuevamente
angustiado por sentirse vacío y desmotivado. Se había ido hacía un año
de una empresa -que dirigía como propia- y en el último año no lograba
encontrase con algo que lo motivara. Si bien él había venido a visitarme
a Jujuy un año antes y conoció el ``tomate con gusto a tomate'' que se
logra sólo cuando se lo cultiva de forma agroecológica, no conocía mucho
de la propuesta de la red. Esa noche hablamos un buen rato por teléfono
y lo invité a que venga a Jujuy para que renueve los aires y vea lo que
estaba impulsando, sobre todo con las buenas nuevas de ese día. Antes de
fin de mes, no solamente ya estaba en Jujuy, sino que había decidido
venirse a vivir e iniciar el proyecto de MIKUNA, que significa alimento
en voz quechua\ldots{} ¡y la novia se enteraría de esto a su regreso a
Buenos Aires cuando fue a preparar su venida definitiva!

\textbf{Situación difícil, viaje y salto al vacío}

Para abril me encontraba en una situación muy difícil: 1) mi jefe no
había apoyado la idea de integrar el proyecto de la red desde su empresa
2) yo no estaba dispuesto a resignar ese proyecto 3) pero mi principal
fuente de ingreso era el empleo en la empresa de mi jefe 4) y si bien el
proyecto de la red estaba aprobado, yo no sabía cuándo tendría
efectivamente el financiamiento del que iba a cobrar honorarios en
calidad de consultor y 5) como si fuera poco, mi hermano Agustín ya
había tomado la decisión de sumarse al mismo haciendo un cambio total de
vida.

A fines de abril viajo para participar en un stand en una feria en San
Isidro, mi pago de origen: la Expo Bio Argentina Sustentable. Mi rol en
la empresa en la que trabajaba era la de responsable de producción y
comercialización y tenía por parte de la empresa la autorización y el
requerimiento de ser eficaz y eficiente en mis tareas. A fines de marzo
informé a mi jefe que iba a ir a Buenos Aires a vender productos, pero
una vez allá no quise recordárselo. Él vivía en Buenos Aires y yo sentía
que con seguridad iba a cuestionarme por no hacer este tipo de cosas de
la manera que él hubiera querido, es decir, con un enfoque de empresa
individual y no como una red temática o sectorial. Por mi parte entendía
que sobre cumplía con la empresa que me pagaba mi sueldo ya que le
conseguía participar en Buenos Aires en un stand muy bonito sin tener
que pagarlo, además de todas las ventas concretadas como así otros
beneficios.

La cuestión es que al día siguiente de terminar la feria mi jefe supo
que yo estaba en Buenos Aires y que no se lo había recordado. No había
coincidencias entre él y yo respecto de la estrategia y las tácticas más
convenientes para impulsar su empresa. Yo deseaba integrar acciones y
tenía la certeza de que desde esa empresa podía ``hacer crecer la
torta'' para el sector. Mi jefe no lo veía igual que yo, no lo avaló y
eso significó que al poco tiempo yo le presentara mi renuncia\ldots{}
tiempo impregnado de angustias, incertidumbres y la sensación de estar
saltando al vacío.

\textbf{El baile emprendedor}

El 27 de mayo de 2015, luego de recibir alguna ayuda y con mucho
esfuerzo, inaugurábamos el primer local de Mikuna en el barrio Los
Perales, en la periferia de la ciudad de San Salvador de Jujuy. Éramos 3
socios y teníamos tan sólo 3 consultorías como ingresos aparte de lo que
iba a generar el local. El mismo ponía a la venta productos elaborados
orgánicos y naturales de almacén, más la venta de frutas y verduras
frescas de productores de la red de distintas partes de Jujuy y Salta.
Cumplir con esto último implicó mucha inversión en tiempo, combustible,
dinero, y costo de oportunidad. Aun así,~era claro que este punto de
venta estaba en el corazón de la propuesta de valor y había que hacerlo
funcionar. Por diferentes motivos no fue nada fácil hacerlo y el
contexto político-económico pre-electoral y luego, post-electoral fue
complicado por todo lo vivido en Jujuy y en el país.

A pesar de todo, bailamos y bailamos. Se consiguieron nuevas
consultorías, un microcrédito, una serie de premios y distinciones,
entre las que merece ser resaltado un primer premio a nivel nacional
otorgado por la Fundación ArgenINTA. En marzo de 2016 las ventas del
local habían caído al suelo y en lugar de cerrar, la apuesta fue mudarlo
a una muy buena ubicación en pleno centro de la ciudad de San Salvador.
En mayo estábamos inaugurando el nuevo local con Agustín al frente del
mismo y los otros socios trabajando en temas más agronómicos. Mientras
tanto seguimos armando e impulsando proyectos y participando activamente
en numerosos espacios para dar a conocer nuestras propuestas. Mucha
energía pusimos en esto, cuyo costo se sentía día a día y cuyos
beneficios sabíamos que iban a llegar en algún momento, pero sin saber
cuándo.

\textbf{Más avances}

En ese tiempo fuimos becados por el PROCAL para capacitarnos, avanzamos
con los registros de nuestra marca, aprendimos sobre la demanda en el
nuevo punto de venta, participamos en la Escuela Inclusiva de Negocios
MINKA y fuimos finalistas del concurso que organizaba, nos reunimos con
mil u un actores privados y públicos, dimos una buena cantidad de notas
en diversos medios locales y nacionales, fuimos invitados a participar
de un grupo de trabajo de Naciones Unidas -¡Muy inesperadamente
ingresando al mismo junto a YPF y el Banco Provincia de Buenos Aires!-,
participamos en concursos como el Jóvenes Empresarios de CAME a la par
de llevar adelante la difícil tarea de asistir técnica y comercialmente
a pequeños productores, entre tantas cosas más que muchos emprendedores
se podrán ya imaginar.

Todo esto generó mil y una situaciones que no son nada sencillas de
atravesar y produjo inevitablemente un desgaste y una serie de
diferencias entre los socios. Para fines de 2016, de los tres socios
iniciales, sólo seguíamos los hermanos Mayorga y la situación para
Mikuna estaba bien complicada. Las notas de color más lindas de esos
tiempos fueron que la novia de mi hermano decidió también venir a Jujuy,
consiguiendo a los pocos meses una muy buena oportunidad laboral y con
mi mujer iniciamos la \emph{dulce espera} de una niña.

\textbf{Los más recientes pequeños grandes logros}

Tenemos total certeza que producir alimentos genuinamente saludables con
triple impacto simultáneo social, ambiental y económico es el presente y
es el futuro al que la humanidad debe retornar. En 2016 aprendimos que
en nuestro nuevo punto de venta lo que se demandaba eran los alimentos
listos para ser consumidos ahí mismo o en domicilio. Entonces, ante la
situación de crecer o cerrar, decidimos crecer. Al local lo
transformamos en un Bar temático franquiciable (MIKUNA BAR) para el
encuentro con y de la sociedad. Hace unos días, el 28 de abril de 2017,
no sabemos bien cómo lo logramos, pero lo re inauguramos. Las ventas son
buenas y en todo momento recibimos las felicitaciones y el apoyo tanto
de ``viejos'' como de nuevos clientes y amigos.

Luego de más de un año de mucho trabajo y de varias idas y venidas,
estamos trabajando en el armado de una planta industrial de elaboración
de alimentos saludables. Con ella nos proponemos incrementar
progresivamente las oportunidades para los productores que desde el
medio rural, trabajan con la tierra, y no en su contra. Ojalá que a todo
el esfuerzo que ponemos, se nos sume una buena dosis de suerte, y
celebremos pronto también la concreción de dicha planta. Gracias, tanto
a Green Drinks por la oportunidad de compartir estas líneas como al
lector por interesarse en ellas.
