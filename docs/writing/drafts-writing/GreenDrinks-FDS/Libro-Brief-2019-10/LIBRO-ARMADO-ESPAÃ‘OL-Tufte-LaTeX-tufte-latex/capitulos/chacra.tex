\chapter{CHACRA DE LUNA: INTELIGENCIA NATURAL}{CHACRA DE LUNA: INTELIGENCIA NATURAL}}\label{chacra-de-luna-inteligencia-natural}
%\texorpdfstring{\protect\includegraphics[width=1.72500in,height=1.76250in]{media/image12.jpeg}

\textbf{Federico Uanino}

Gerente General de CARPAL y Fundador de Chacra de Luna.

``\emph{Si todo te da igual, estás haciendo mal las cuentas.'' A.
Einstein}

\textbf{\\
}

Chacra de Luna: inteligencia natural

La historia de Chacra de Luna, ocho hectáreas en las que producimos de
forma responsable y que dejamos abierta al público, para que todos
puedan encontrarse y descubrir la inteligencia natural.

¿Por qué debemos resignarnos a que el paso del tiempo arrase con
lugares, aromas, sabores, canciones, saberes? ¿Cómo podemos contribuir
para que el paso de los días nos permita mejorar nuestra existencia y la
de los que vendrán? ¿Cómo hacemos para que los momentos felices que
pasaron podamos postergarlos en el tiempo, como hacer para que el día a
día nos permita rescatar lo bueno de la vida y mejorar lo que pudo ser
mejor?

En una tierra que fue forjada por familias que dejaron su tierra natal
en busca de una esperanza de progreso. En una tierra que era virgen,
donde el hombre no había impreso su sello, llegaban hombres y mujeres,
niños y ancianos, sólo con sus baúles cargados de sabiduría y amor por
el trabajo. Con muchas ganas de hacer florecer de la tierra fértil de
Argentina, sus sueños y experiencias de su tierra italiana. Cómo a pesar
de tantas dificultades supieron germinan una ciudad.

\textbf{Luego de cinco generaciones}

Hoy, cuando pasaron más de 130 años y cinco generaciones, con todas las
herramientas y conocimientos al alcance de nuestras manos, pareciera que
nos tenemos que resignar a que los acontecimientos sucedan, que debemos
aceptarlos, quedándonos como espectadores de una vida que se escapa
delante de nuestros ojos, sólo alimentándonos de lo artificial, porque
es lo que nos toca.

El desarrollo nos fue cambiando los hábitos, los juegos, las comidas,
los sabores, los sonidos, los valores. Hoy ya nos parece natural lo que
hacemos o consumimos, hasta que probamos lo auténtico y nos damos
cuenta. Nos damos cuenta que la naturaleza es inteligente y aunque los
hombres usen la propia o la artificial, la inteligencia natural es
sabia. Conocer esa sabiduría nos hace valorarla y cuidarla, pero para
poder conocerla debemos estar formados, y esto es lo que hemos perdido.
Lo natural hoy nos parece extraño, los sabores auténticos nos llaman la
atención, los aromas y texturas nos parecen extraños.

Chacra de Luna, en Colonia Caroya, provincia de Córdoba, es un espacio
que fue pensado para que podamos encontrar esa inteligencia natural.

\textbf{Los antecedentes}

En 1892 se instaló en el predio que ocupa actualmente este
emprendimiento, una familia que provenía del norte de Italia. Como
tantas otras, escapaban de la falta de oportunidades pero con ilusión
por la iniciativa del entonces Presidente Argentino Nicolás Avellaneda,
quien impulsaba una ley de promoción territorial en este país.

Llegaron a Colonia Caroya, donde encontraron un monte denso y un suelo
fértil, y lo enfrentaron con su cultura, educación, fe y voluntad de
progreso. Fueron transformando el monte agreste en parcelas productivas,
de donde extraían lo suficiente para vivir. Supieron cultivar la tierra,
plantar árboles frutales, viñedos, criar animales y hasta fabricar sus
propias herramientas y medios de transporte, además de construir sus
propias casas con los materiales que tenían disponibles a su alrededor.

Con los frutos de sus crianzas y cosechas obtuvieron alimentos de
guarda, para poder administrarlos a lo largo del año. Así se forjó
nuestro pueblo, hoy ciudad.

Las generaciones pasadas supieron convivir con la naturaleza, porque se
sentían parte, porque la conocían. Sabían cómo administrar el agua, cómo
aprovechar la influencia de la luna y los astros en las labores diarias,
cómo manejar las malezas y las plagas, cómo valerse del entorno hasta
para mejorar su salud. También cómo divertirse y cómo ser felices.

\textbf{La historia y la cultura}

En 2000, mi nono me comentó que había decidido demoler una de las
primeras industrias dedicadas a la fabricación de herramientas para la
labranza de la tierra y medios de movilidad de tracción a sangre, porque
corría peligro de derrumbe. Por ella pasaron cuatro generaciones de
carpinteros y herreros, además de muchos aprendices, funcionaba también
como taller escuela. En ese momento pensé que no podía dejar que el
tiempo siguiera arrasando con la historia y con la cultura.

Allí surgió la necesidad de ser actor de la película que todos los días
se está rodando. Imaginé un espacio rural donde las nuevas generaciones
pudieran interactuar con la historia y la naturaleza, volver a producir
la tierra con respeto, aprovechando las nuevas tecnologías para hacer un
uso más racional de los recursos. Convencido que somos sólo un instante
de la historia y que es nuestro deber ser responsables con el futuro que
dejamos.

\textbf{Queríamos producir algo}

Así comencé, sin saber muy bien cuál era el fin, ni cuando lo lograría,
pero convencido que quería producir frutales, hortalizas, vides; quería
frutos sanos, ricos y sabrosos; quería producir alimentos de una forma
responsable y eficiente; no quería producir dinero y utilizar la
producción como medio, porque allí las ecuaciones cambian totalmente.

Quería también criar animales en forma responsable y saludable, para
sentir los sabores naturales: el olor a la leche recién ordeñada o un
huevo con gusto y color a huevo.

Esos productos tenían que llegar a todas las casas, donde la gente
pudiera compartirlos con sus seres queridos. O que la chacra pudiera ser
visitada por las familias, para mostrarles cómo se trabaja, cómo se
producen los alimentos, cómo la naturaleza puede darnos lo mejor. Esto
nos parecía de suma importancia, porque estábamos seguros de que al
saber cómo se elabora lo que sus hijos consumen, sus padres lo valorarán
y exigirán que sea lo mejor.

También soñaba con la Chacra como una gran aula interactiva, donde los
docentes pudieran tener un espacio para enseñar a sus alumnos,
interactuar con los diferentes actores y espacios, producir y probar los
frutos que la tierra nos da.

Si queremos formar personas respetuosas y responsables del medio donde
se desarrollan, es fundamental que sepan dónde se origina lo que nos da
vida, cuáles son las cosas que no podemos prescindir. No podremos nunca,
si queremos ser felices, dejar de alimentarnos, tener salud y vivir en
paz con nuestro entorno.

\textbf{Actualmente}

En el predio se encuentra restaurada completamente la vieja fábrica
familiar. En ella se puede ver cómo el ingenio y la cultura del trabajo
supieron levantar una industria con lo que había alrededor. Es un
espacio que se usa para eventos culturales, como muestras de pinturas,
fotografías, conciertos, para reuniones sociales y empresariales.

Contiguo a este espacio se encuentra el sótano, donde se conservan los
vinos y embutidos que elaboramos y que los visitantes pueden comprar
para degustar en el lugar o llevarlo a sus casas.

Donde se hacía pintura y fileteado de sulkys y carretas, funciona la
cantina, con productos de la Chacra.

Toda producción es realizada de forma agroecológica, prescindiendo del
uso de agroquímicos y utilizando biopreparados. Con los desperdicios de
la granja se produce compost que es utilizado para fertilizar, y con
labores particulares, como rotación de cultivos, cultivos de cobertura,
corredores biológicos, entre los principales, logramos reducir el
impacto de malezas e insectos.

Los visitantes también pueden sentarse bajo los árboles y disfrutar la
calma de un espacio sin prisa, sentir los aromas y los sonidos de la
Chacra, el viento y el canto de las aves. Si vienen con niños pueden
verlos disfrutar del espacio: un patio que se convierte en un parque de
diversión, donde tirarse en el césped, subirse a un árbol o jugar en la
tierra son los juegos principales; y del tiempo, que pareciera pasar más
lento.

Hoy, luego de nueve años de trabajo de remodelación y puesta en valor de
edificios y cultivos, y a dos años desde que abrimos al público, Chacra
de Luna cuenta con ocho hectáreas en las que producimos de forma
responsable y está abierta al público, para que todos puedan encontrarse
y descubrir la inteligencia natural.