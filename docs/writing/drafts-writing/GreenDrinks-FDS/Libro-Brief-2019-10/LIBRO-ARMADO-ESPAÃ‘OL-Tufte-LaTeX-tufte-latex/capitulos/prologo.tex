\protect\hypertarget{_Toc486429354}{}{}

\chapter{Prólogo}\label{pruxf3logo}

Un grupo de voluntarios fabrica ecoladrillos reutilizando el plástico de
los envases desechados en una ciudad cordobesa. En otra, una ONG
recolecta por su cuenta siete toneladas por mes de papel que destina al
reciclado, haciendo que en vez de basura sea un nuevo recurso. En un
pueblo, más al sur, un empresario se empecina --aunque aún no le cierren
los costos-- en fabricar bolsas realmente biodegradables, sin ningún
insumo plástico. En varios parajes serranos, surgen emprendimientos
turísticos ecoamigables, que priorizan reducir el impacto sobre el
entorno y el paisaje. Más arriba, varios grupos dan pelea para
reforestar con especies nativas los cerros que las fueron perdiendo.
Hacia el este, un pueblo se propuso ser el que más basura separa y ya el
75 por ciento de sus vecinos clasifica en sus hogares los residuos que
generan, mientras cientos de intendentes repiten, sin siquiera
intentarlo, que eso es muy difícil. La lista sigue y crece.

Es probable que Córdoba, como el país y el planeta, enfrente hoy los
mayores dilemas ambientales de su historia. Paradójicamente, es seguro
que nunca hubo tanta gente (aunque aún no la suficiente) generando
acciones y proyectos en defensa del ecosistema.

Lo que falta es que esas improntas individuales o grupales se
transformen en mayor medida en fenómenos sociales y, sobre todo, que se
impongan en la agenda política y económica que domina al mundo.

Sin embargo, y a la vez, no habrá modo de que los grandes temas
ambientales se transformen en políticas de Estado si no parten de una
concientización social que nace, también, de esos esfuerzos de personas,
grupos y comunidades comprometidas.

De eso hablan estas historias que Green Drinks Córdoba seleccionó para
este libro como muestra de los aportes y búsquedas de gente inquieta que
ha generado variadas experiencias sustentables que transforman su
entorno.

``El tema más importante sobre el medio ambiente es uno que rara vez se
menciona: la falta de ética de nuestra cultura'', planteó alguna vez
Gaylord Nelson.

La cita parece destinada a encajar en el relato de unos de los 16
testimonios que integran este volumen. Un hotelero que encaró en Punilla
un proyecto turístico amigable con el ambiente cuenta: ``Antes, cuando
no existía el término `sustentable', nuestros padres y abuelos hablaban
de `hacer las cosas bien''.

De algún modo, ese ``deber ser'' opera como la más sencilla explicación
de la relevancia de que como humanidad avancemos, pero de modo
sustentable, por aquello de que la tierra nos ha sido ``prestada'' por
nuestros hijos, por la generación que viene.

El debate sobre lo sustentable, para crecer en términos de
concientización social y tornarse más masivo, requiere a esta altura que
el discurso sobre la defensa del ambiente no se concentre sólo en la
visión romántica, verde, de amor a la naturaleza, conservacionista y
hasta idílico. Para que crezca y pase a ser más determinante en la
agenda pública, el enfoque más inteligente debería apuntar a mostrar el
impacto que en la vida cotidiana de la gente, aquí y ahora, representa
la desaprensión por el ecosistema. Sobran ejemplos para mostrar en ese
sentido.

Las experiencias que aquí se cuentan --algunas entre muchas posibles--
sirven como disparadores, como modelos que inspiran y motivan, como
muestras de lo que cada uno puede hacer (y no sólo declamar) para
apostar por el desarrollo sustentable. Agrupadas y relacionadas
adquieren otro valor.

El aporte de estos textos, como de otras acciones de divulgación, es que
impulsan los debates de estas temáticas para regenerar ideas, abrir
miradas y perspectivas, repensar soluciones y alternativas, compartir
inquietudes, contagiar entusiasmos.

Fernando Colautti

Periodista

\textbf{\\
}