\subsection*{2) ¿Es el trabajo preciso y legible en cuanto al estilo? Si no lo fuera, indique las dificultades.}
\label{es-el-trabajo-preciso-y-legible-en-cuanto-al-estilo-si-no-lo-fuera-indique-las-dificultades.}

El trabajo es claro en cuanto a los niveles de descripción y es legible
en el sentido de que no abre ambigüedades en lo empírico. Sí hay elementos que se
podrían mejorar, para hacer el texto más ágil y directo, y completo.

\begin{itemize}
\tightlist
\item
  ¿Qué es lo importante en cada sección? Resaltar desde el inicio lo que
  es significativo a la hora de presentar los temas. Un ejemplo, entre otros posibles: en la
  página 91 presenta un Torneo que se indica importante para el grupo
  y para el deporte es una referencia en todo el país, Torneo La Rioja,
  que se presenta con nombre y fecha ``Sábado 10 de septiembre''. A esta
  sección se le podría dar un nombre que sea representativo del peso que
  tiene el evento, algo como: El evento principal, o el reto mayor, o el
  desafío histórico en Argentina, o alguna otra indicación que señale
  la importancia y características del evento.
  Este enfoque, si se replica en otros casos, podría dar al lector una idea más global y definida
  de la progresión de temáticas al ver el índice del texto.
\item
  Evitar redundancias. Otro ejemplo posible, entre varios: en la página 4 se aclara el título de
  tesis. En caso de publicación se dejará: ``Locos por el Crossfit: una
  etnografía sobre atletas de esta práctica en un box de Santiago del
  Estero''. Se puede eliminar ``de esta práctica'' y el sentido se
  mantiene, con mayor claridad y simpleza, sin redundancia.
\item
  Evitar información innecesaria. El texto está estructurado
  cronológicamente y el orden desde el índice sigue esa estructura. Esto
  de por sí es perfectamente válido como orden interno del trabajo, pero
%   para los títulos no hace falta y de hecho 
  puede desayudar dar la fecha
  para diferentes acciones, por ejemplo cuándo se hizo una práctica, 
  porque se pierde la relevancia de lo que es central
  al campo. La fecha sí se puede usar de referencia para algunas citas
  pero igualmente se podría sintetizar la información en un campo
  numérico simple de ocho dígitos con año, mes, día: YYYY.MM.DD.
  También revisar si es relevante qué día de la semana es, si lo es
  entonces incluirlo y explicarlo, de lo contrario, se podría quitar.
\item
  Evitar repeticiones, por ejemplo ``como hablé antes con x'', ``como
  escribí anteriormente'', cuando dieron acceso al campo, etc. 
  El texto amerita una edición cuidadosa en la
  que se pueda eliminar información que no hacen avanzar al lector en su
  entendimiento y en la inmersión.
% \item
  Hay cuestiones de contenido donde se diluye lo que se dice al demorar el texto con validaciones y otros detalles repetidos en diferentes secciones.
  Escribir es, en la vía de Becker (\textit{Manual de escritura para científicos sociales} [1986]2011), re-escribir:
  es un trabajo continuo de re-edición sobre lo trabajado para depurar las ideas y las palabras.
%   De la misma manera con el estilo es que se puede evitar poner
%   las fechas con letras usando simplemente con 8 números (YYYY.MM.DD). %   (siglas en inglés): 
\item
  Usar corrector. Página 7: ``Consideraciones Inciales''. Falta una letra.
\item 
  Se podría estructurar el texto por temática, y no por fecha.
  La cronología no es relevante ni interesante para el
  lector que quiere entender el tema.
  Sí es una referencia para el
  investigador y es bueno tenerlo claro en las notas de campo.
  Pero para el propósito de cada relato se podrían usar las fechas solamente si es relevante al caso,
  por ejemplo cuando se hace un asado un domingo a la noche, o en los
  preparativos para el viaje hacia un torneo.
\item Conceptualmente se puede explorar más en núcleos teóricos para aportar mayor claridad.
\item 
  ¿Qué lleva adelante el texto a nivel de estilo narrativo? 
  ¿Se busca una intensidad, un ritmo?
\end{itemize}