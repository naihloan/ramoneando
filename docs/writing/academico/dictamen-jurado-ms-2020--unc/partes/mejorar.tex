
\section{A mejorar}\label{a-mejorar}



\subsection{Puede que haya compromiso adentro del gimnasio pero no
afuera?}\label{puede-que-haya-compromiso-adentro-del-gimnasio-pero-no-afuera}

También se podría lograr el mismo resultado de lograr una descripción
etnográfica sin llegar a la conclusion de que hay una comunidad de
atletas o de amigos. Es posible que los practicantes respeten al líder,
y que trabajen en conjunto cuando ejercitan y compiten. Pero como son
las sociabilidades que se tejen fuera del gimnasio? Hay una continuidad
de relación por fuera del gimnasio? Si alguno deja de practicar, se
siguen viendo porque se hicieron amigos? Es posible que si, porque hay
gente que consigue pareja dentro del gimnasio, seria bueno hacerlo mas
visible.

Si las relaciones dentro del gimnasio son intensas: hay alguien que haya
seguido en contacto con la ex pareja de Cacho? Por que si o no, como?

\subsection{¿Es relevante la historia del gimnasio para los
practicantes?}\label{es-relevante-la-historia-del-gimnasio-para-los-practicantes}

Etnográficamente tiene su valor conocer la historia del lugar, pero no
queda claro de que manera juega un papel en la carrera del loco. Es
posible que alguien aprenda a conocer los ejercicios sin conocer la
historia del gimnasio? Hay clases teóricas o panfletos que expliquen la
historia del local? Difícilmente, entonces por que se insiste en que
hace falta saber la historia?

