\subsection*{6) ¿Presenta un análisis de los datos consistente? Si no fuera así, indicar las dificultades.}
\label{presenta-un-anuxe1lisis-de-los-datos-consistente-si-no-fuera-asuxed-indicar-las-dificultades.}

El abordaje de los datos es ordenado y prolijo.
Se delimita qué se va estudiar y se siguen los pasos pautados.
Se podría revisar si son exactamente cada uno de esos pasos los que son necesarios para conformar la tesis del autor.
Por ejemplo, establece que hace falta conocer la historia del box para ser un loco.
¿Es realmente así para los atletas o esto es más bien un parámetro de importancia para el trasfondo de la investigación? 
Si alguien entra al box en 2016 y compite en 2017, y no conoce la historia desde el 2014: ¿no es loco?

La vivencia grupal se enfatiza en diferentes puntos del texto pero no queda claro de qué manera se dan
las conexiones entre las personas por fuera del gym,
¿hay realmente tal conexión? 
¿Cuánto tiempo de permanencia en el gym tienen los locos citados?
Se podría establecer cuál es ese tiempo de permanencia. 

¿Cuánto es lo menos que estuvo activo un iniciado? Auto-etnografía puede ser útil en ese punto.
¿Cuánto es lo máximo de tiempo que estuvo un atleta, un loco?
Estos podrían ser datos tangibles que dimensión el peso de cuán visible es la intensidad de vivencia grupal.
¿Quiénes pasaron por conversión a ser Locos pero después se fueron a otro gimnasio o dejaron la práctica?
¿Se forma una comunidad? 

% \subsection{Personas}\label{personas}
Dado que el trabajo de campo es un buen recurso a disposición, se puede
resaltar el lugar que tiene cada persona en el grupo y destacar un
balance de si se lo puede considerar a cada uno un loco, o no. %
Las personas del estudio son presentadas en anexo. 
Una consideración es si se podría incluir una presentación de cada atleta como parte del texto,
tomando como referencia ese texto de anexo pero ampliándolo con el balance individual, y más adelante, grupal.
Se podría atender a en qué momento aparece cada atleta y su relevancia en el relato del texto completo y del grupo.
%
Tomemos el caso de Chupa, fuera de una mención en agradecimientos y otra mención en un ejercicio se lo menciona en detalle recién en la página 93. Tal vez se puede redimensionar este caso con más aspectos de su perfil en el contexto del escrito.
%
En lo que hace a las distintos personas que aparecen en la tesis, y al sentimiento grupal,
podría ser ocasión para un abordaje conceptual de las vivencias individuales y grupales que podría darle más fuerza a la tesis.

En términos de los datos: se establece que hay un
respeto y seguimiento a la autoridad del líder,
¿pero equivale esto a decir que los atletas están unidos entre sí?
%
Puede que los atletas formen comunidad y que realcen valores.
Se podría hacer más evidente esa conexión si fuera el caso.
Cuando hacen un asado no aparecen todas las partes involucradas.
Cuando viajan en auto se podría señalar de qué manera y con qué criterios se componen los grupos, 
si es una organización espontánea o planificada.

En la conclusión se resalta el valor del trabajo de sí. ¿Cuál es la conexión entre la descripción del texto y esta conclusión?
% este concepto se menciona pocas veces en el texto.
Puede haber un concepto de sí, y también puede haber comunidad, pero ¿funcionan bien ambas juntas? ¿Cómo se articulan? 
% Quizás no haya ni lo uno ni lo otro.

% \subsection{Dieta}\label{dieta}

En el tema de la nutrición está muy logrado la valoración moral de los alimentos. 
Pero ¿hay un criterio unificado de perspectiva y de conducta?
¿Todos siguen invariablemente la dieta paleo? 
Es claro que hay casos de grandes esfuerzos en cambios de conducta dietaria, pero
¿hay desviaciones sistemáticas o de tipo ocasional? 
¿Cómo se desarrollan las carreras no exitosas? (Nueva ocasión para más auto-etnografía)
¿Por qué comen harina y leche cuando van a su segunda competencia? 
¿Por qué una nutricionista no mencionada antes recomienda para el día del torneo? 
¿Esta innovación sobre la marcha es considerada seria?
% ¿Prueban cosas nuevas el día de la competencia?

La investigación resalta lo global del fenómeno del Crossfit,
pero podría ser interesante hacer una comparación de las diferentes regiones por las que pasó la etnografía.
¿Son todas las provincias iguales o se sintieron diferencias,
hermandades, con distinto énfasis en estilos de ejercicios o influencias
de diferentes lugares o todos absorben el Crossfit de una misma casa
central? ¿Y el marketing cómo se maneja?
Si la hay, por rústica que sea, puede ser una vía interesante de exploración.
Las redes sociales suelen
estar presentes en las descripciones del texto.