\chapter*{Context} %Contexto de situación

Millions of persons run nowadays in different urban scenarios: they just put
their clothes on and leave. It is crucial to feel the body when running. And this
does not imply that this attention and perception is always a given. In July of
2015, an athlete in Frankfurt finished an Ironman (a triathlon that takes 11
hours on average to complete). The participant died after being convalescent
due to over hydration/hyponathremia. The issue raised here is that certain
sport practices demand a more thorough type of health care, a  
kind of learning that becomes vital: that is, of life or death.
%Running, however, is seldom a high risk activity.

%Millones de personas corren hoy en cualquier escenario urbano: se ponen la ropa deportiva y salen. Para correr es crucial sentir el cuerpo. Y eso no siempre implica una atención, una percepción obvias. En julio de 2015 un atleta en Frankfurt termina una competencia de Ironman (triatlón de 11 hs en promedio). Muere después de estar convaleciente por sobrehidratación/hiponatremia. Se plantea acá que algunas prácticas deportivas, como esta, involucran una dosis de cuidados que van más allá del cotidiano. El problema es que este aprendizaje se vuelve vital: esto es, de vida o muerte. 

Running, however, generally contributes positive elements to fight against
obesity, depression, to mention but a few. It is also used as a lucrative activity
by an industry that produces sport supplies; thus generating an apparatus
of organization that hosts a variety of events: short races, olympic marathon
distance (42.195 k) and ultramarathons that go from 50 k up to the 246 k
Spartathlon –and even races that can last 48 consecutive hours. While running
is seldom a high risk activity, the latter challenges do pose the question of
public physical (and mental!) health.%
%Correr raramente es una actividad de alto riesgo. Aporta en general elementos para una lucha contra la obesidad, la depresión, y una larga lista. Pero también es aprovechada como actividad lucrativa por la industria que produce insumos deportivos y gestores que organizan eventos de carreras cortas, maratones de distancia olímpica (42,195 k) y ultramaratones que llegan hasta lo que se siente como el infinito, como el Spartatlón de 246 k, incluso hay carreras que duran 48 hs de corrido. 
%

The use of urban and wild spaces require that they be managed in an agile,
free, and articulate way. This also detonates in a exploitation of natural
and tourist resources that oscilates between environmental care and decay.
UNESCO for example, looks to take care of Mont Blanc, the place of the
emblematic ultramarathon Ultra-Trail du Mont-Blanc, as a World Heritage
Site.

%Las carreras más largas, además de plantear el tema de la salud pública a nivel físico (¡y mental!), necesitan usar los espacios urbanos y silvestres de una manera ágil, libre y articulada. Esto también detona en una explotación de recursos naturales y turísticos que tambalea entre el cuidado ambiental y el deterioro. La UNESCO por ejemplo busca proteger el Mont Blanc, sitio de la emblemática carrera de ultramaratón Ultra-Trail du Mont-Blanc, %(UTMB), 
%como Patrimonio de la Humanidad. 

Runners experiment the activity in many different manners: as meditation
in motion, to listen to music, to eliminate some colesterol from blood, to
experience the vitality of their body/mind, to clear their head and/or gaze at
the green landscape anywhere in the city. “Just buy a good pair of shoes and
you’re ready” say those who promote a less sedentary and quite cheap activity:
one might be tempted to say (almost) for free.
%Gente que corre vivencia la actividad de maneras muy diferentes: hacen meditación en movimiento, escuchan música, %hacen un sprint, 
%se quitan un poco de colesterol y grasa de la sangre, experimentan la vitalidad de su cuerpo/espíritu, se despejan la cabeza y/o miran el paisaje verde de una parte de la ciudad. "%Solamente 
%Comprá un buen calzado y listo" dicen en común todos los que promueven una actividad menos sedentaria y bastante barato: (casi) gratis se diría. 

