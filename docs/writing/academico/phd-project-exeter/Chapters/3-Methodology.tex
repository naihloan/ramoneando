\chapter*{Materials and methods}
%\chapter{Materiales y métodos de investigación}

This work looks to a wider spectrum of escape mechanisms from the inertia of
the productive system, a production of machinic uselessness, where cars and
public transportation \textit{circulate, overpopulate and congest}, all of which control
movement in favor of an economic and political order and power.

%\begin{quote} (Deleuze y Guattari, 2010: 527). %\end{quote}

Ultrarunners are the case study. However, the implications of the topic
go beyond this social world. A good number of ultrarunners are so aware of
the need of natural food that they are more inclined to eating more fruits and
vegetables than average people and many of them become long term vegetarians.
This is related, but differently, to massive consumption patterns: of avoiding
the dangers of canned foods, with preservatives, refined sugars, processed flour;
and a political view that requires that the land  be distributed and
cultivated to feed people and not animals.
%(Arrieta, 2013). 

The proposal is of qualitative research. On one side, it will nurture from
secondary material in texts and videos made by and about the participants of
ultramarathon races. On the other side, first hand material will be collected
 from fieldwork. As study material, the general training method shall be
reviewed and at least, the ethnography of one specific competition. These
elements pursue to give new life to the ideas of how people move, beyond
a mere transportation function, and how urban spaces can be circulated,
expanding their uses.

%\clearpage
\section*{2. Auto-ethnography}

The plan of work proposed here sets axis on which to develop future ideas, these axis being: 
affect,
body, 
and materiality.
These \textit{sensitizing concepts} (rather than restrictive prescriptions) shall be guiding points to suggest directions where to look at, as germs of analysis on how and where to collect information. Data finding also relies on the researcher's agenda: "What sorts of patterns one is looking for depends, of course, on research focus and theoretical orientation". Benefits of in-field immersion include not only direct access in general but additionally to non-structured conversations in which "[unusual participant terms] may stress theoretically important or interesting phenomena". In the same vein, concepts may also be, alternatively, "observer-identified"%
\footnote{HAMMERSLEY and ATKINSON. \textit{Ethnography: principles in practice}. 3rd ed. London; New York, NY: Routledge, 2007. P. 164 ("Sensitizing concepts" is Blumer's), 163.}.

The axial concepts are not %be used as fixed tautologies 
to give a taken-for-granted understanding of behaviors. The approach here is first \textit{exploratory}, rather  than explanatory. The deeper understanding of behaviors and use of tools, resources and knowledge %in general/
on the whole, %shall be developed later 
shall come later, during research. The intention is first to gather data, concepts, and a series of insights from in-field work.

Ultra-running has a certain tension in the way it connects participants with people from the outside social worlds.

\begin{itemize}
 \item On one side, it is an ultimately public activity, runners are exposed to permanent contact with other runners (and non-runners as well) in the open, and races depend on a wide number  of actors, both participating and non-race related: in sum, a very wide orchestrated and coordinated social activity.
 \item On the other side, ultra-running entails a certain \textit{Loneliness of the long distance runner}%
 \footnote{Short story by Alan Sillitoe, published in 1959.}. 
 Running ultra distances may well be one of  the most \textit{outdoor} activities or sports. It involves several hours, even days sometimes "out in the  open", amongst  almost untouched nature and wild green spaces afar from a city in cross-country races. And in training season, even in city context: the silent early night-to-dawn moment (from 4 to 6 am) is when nearly no ordinary person is going about, and birds have not even began to chirp. As well as with lone spaces, running involves far many solitary moments in which runners get to collect themselves and revolve in their thoughts, the bareness of the surroundings, and at many flowing times:  
 think of nothing and seize the moment.  
 %not think in anything and be in the moment.
\end{itemize}

The \textit{in situ} work is intended to grasp these two areas (intimate-personal; and social-network-dependent) in ultra-running: the first, during training; and the second, during specific ultra-running events.

\begin{enumerate}
 \item The first aspect, training, is to be dealt with  through auto-ethnography, not as a biographical account, but as a means to grasp the main topics developed. Many of the available material on ultra-running in text and video documentary depict narratives from the sole perspective of runners themselves, in first person, and how they prepare for their practices with different styles of running and post practice cool downs and stretching as well as general nutrition and resting time. The researcher may well take a similar approach without being an outsider of common practice in this social world.
 
 \begin{quote}
  Gertrude Kurath (1960) recommended ethnographers to "learn the movements" and Adrienne Kaeppler (1978) proposed that ethnographers learn certain movements and  receive instructions on  what is done "incorrectly", or "differently" with a methodology that would allow for better understanding.
  %to understand better. 
  [José Bizerril has argued that the practical formation of the researcher has its advantages.] This knowledge allows  access to aspects of the research topic that otherwise would pass unnoticed if only done with a distant approach based on observation and interview. [the experiential dimension makes it possible to gain entry to the experience and] "to the psycho-physical and -why not-, to the spiritual states that that this experience triggers%
  \footnote{ASCHIERI, Patricia. "Hacia una etnografía encarnada: La corporalidad del etnógrafo/a como dato en la investigación". X RAM- Reunión de Antropología del Mercosur. Córdoba, Argentina, 2013. P. 16. My translation.}.
 \end{quote}
 
 Of course,  auto-ethnography may work with a potential source for bias, but at the same time provides both the most inner side view possible, and reveals the speaker's interests, perspectives and preconceptions; to which one can always add contrast with other references to compare and find the most reliable common ground%
 \footnote{HAMMERSLEY and ATKINSON. \textit{Ethnography: principles in practice}. 3rd ed. London; New York, NY: Routledge, 2007. P.%164, %("Sensitizing concepts" is Blumer's), 
 124.}.
 
 \item On the second aspect, on racing events, there is very little material in academic research on events from a qualitative approach. There is scarce material, and when so, only done through surveys or measurement based. Hence, the importance to move forward. Some of the key features of an \textit{ethnographic approach} are taken into account in the present proposal: to prioritize the insider perspective highlighting the experiential, an active immersion in the field during a reasonable amount of time, minimal interference to gather data to be triangulated%
 \footnote{HOLLOWAY, Imma; BROWN, Lorraine; and SHIPWAY, Richard. "Meaning not measurement: Using ethnography to bring a deeper understanding to the participant experience of festivals and events". \textit{International Journal of Event and Festival Management}. Vol. 1 Nº 1, 2010. Pp. 75-76.}.
 And not to focus on measuring variables, but rather on \textit{collecting and constructing new variables} to build up ever more complex concepts: this adds nuance to the understanding of the phenomenon, and provides material to suggest new questions and aspects to be worked on%
 \footnote{BECKER, Howard S. \textit{What About Mozart? What About Murder? Reasoning From Cases}. The University of Chicago Press, Chicago, 2014. Pp. 13-14, 18.}.

\end{enumerate}



\clearpage
\section*{2. Auto-ethnography}

The plan of work proposed here sets axis on which to develop future ideas, these axis being: 
affect,
body, 
and materiality.
These \textit{sensitizing concepts} (rather than restrictive prescriptions) shall be guiding points to suggest directions where to look at, as germs of analysis on how and where to collect information. Data finding also relies on the researcher's agenda: "What sorts of patterns one is looking for depends, of course, on research focus and theoretical orientation". Benefits of in-field immersion include not only direct access in general but additionally to non-structured conversations in which "[unusual participant terms] may stress theoretically important or interesting phenomena". In the same vein, concepts may also be, alternatively, "observer-identified"%
\footnote{HAMMERSLEY and ATKINSON. \textit{Ethnography: principles in practice}. 3rd ed. London; New York, NY: Routledge, 2007. P. 164 ("Sensitizing concepts" is Blumer's), 163.}.

The axial concepts are not %be used as fixed tautologies 
to give a taken-for-granted understanding of behaviors. The approach here is first \textit{exploratory}, rather  than explanatory. The deeper understanding of behaviors and use of tools, resources and knowledge %in general/
on the whole, %shall be developed later 
shall come later, during research. The intention is first to gather data, concepts, and a series of insights from in-field work.

Ultra-running has a certain tension in the way it connects participants with people from the outside social worlds.

\begin{itemize}
 \item On one side, it is an ultimately public activity, runners are exposed to permanent contact with other runners (and non-runners as well) in the open, and races depend on a wide number  of actors, both participating and non-race related: in sum, a very wide orchestrated and coordinated social activity.
 \item On the other side, ultra-running entails a certain \textit{Loneliness of the long distance runner}%
 \footnote{Short story by Alan Sillitoe, published in 1959.}. 
 Running ultra distances may well be one of  the most \textit{outdoor} activities or sports. It involves several hours, even days sometimes "out in the  open", amongst  almost untouched nature and wild green spaces afar from a city in cross-country races. And in training season, even in city context: the silent early night-to-dawn moment (from 4 to 6 am) is when nearly no ordinary person is going about, and birds have not even began to chirp. As well as with lone spaces, running involves far many solitary moments in which runners get to collect themselves and revolve in their thoughts, the bareness of the surroundings, and at many flowing times:  
 think of nothing and seize the moment.  
 %not think in anything and be in the moment.
\end{itemize}

The \textit{in situ} work is intended to grasp these two areas (intimate-personal; and social-network-dependent) in ultra-running: the first, during training; and the second, during specific ultra-running events.

\begin{enumerate}
 \item The first aspect, training, is to be dealt with  through auto-ethnography, not as a biographical account, but as a means to grasp the main topics developed. Many of the available material on ultra-running in text and video documentary depict narratives from the sole perspective of runners themselves, in first person, and how they prepare for their practices with different styles of running and post practice cool downs and stretching as well as general nutrition and resting time. The researcher may well take a similar approach without being an outsider of common practice in this social world.
 
 \begin{quote}
  Gertrude Kurath (1960) recommended ethnographers to "learn the movements" and Adrienne Kaeppler (1978) proposed that ethnographers learn certain movements and  receive instructions on  what is done "incorrectly", or "differently" with a methodology that would allow for better understanding.
  %to understand better. 
  [José Bizerril has argued that the practical formation of the researcher has its advantages.] This knowledge allows  access to aspects of the research topic that otherwise would pass unnoticed if only done with a distant approach based on observation and interview. [the experiential dimension makes it possible to gain entry to the experience and] "to the psycho-physical and -why not-, to the spiritual states that that this experience triggers%
  \footnote{ASCHIERI, Patricia. "Hacia una etnografía encarnada: La corporalidad del etnógrafo/a como dato en la investigación". X RAM- Reunión de Antropología del Mercosur. Córdoba, Argentina, 2013. P. 16. My translation.}.
 \end{quote}
 
 Of course,  auto-ethnography may work with a potential source for bias, but at the same time provides both the most inner side view possible, and reveals the speaker's interests, perspectives and preconceptions; to which one can always add contrast with other references to compare and find the most reliable common ground%
 \footnote{HAMMERSLEY and ATKINSON. \textit{Ethnography: principles in practice}. 3rd ed. London; New York, NY: Routledge, 2007. P.%164, %("Sensitizing concepts" is Blumer's), 
 124.}.
 
 \item On the second aspect, on racing events, there is very little material in academic research on events from a qualitative approach. There is scarce material, and when so, only done through surveys or measurement based. Hence, the importance to move forward. Some of the key features of an \textit{ethnographic approach} are taken into account in the present proposal: to prioritize the insider perspective highlighting the experiential, an active immersion in the field during a reasonable amount of time, minimal interference to gather data to be triangulated%
 \footnote{HOLLOWAY, Imma; BROWN, Lorraine; and SHIPWAY, Richard. "Meaning not measurement: Using ethnography to bring a deeper understanding to the participant experience of festivals and events". \textit{International Journal of Event and Festival Management}. Vol. 1 Nº 1, 2010. Pp. 75-76.}.
 And not to focus on measuring variables, but rather on \textit{collecting and constructing new variables} to build up ever more complex concepts: this adds nuance to the understanding of the phenomenon, and provides material to suggest new questions and aspects to be worked on%
 \footnote{BECKER, Howard S. \textit{What About Mozart? What About Murder? Reasoning From Cases}. The University of Chicago Press, Chicago, 2014. Pp. 13-14, 18.}.

\end{enumerate}



%\clearpage
\section*{2. Auto-ethnography}

The plan of work proposed here sets axis on which to develop future ideas, these axis being: 
affect,
body, 
and materiality.
These \textit{sensitizing concepts} (rather than restrictive prescriptions) shall be guiding points to suggest directions where to look at, as germs of analysis on how and where to collect information. Data finding also relies on the researcher's agenda: "What sorts of patterns one is looking for depends, of course, on research focus and theoretical orientation". Benefits of in-field immersion include not only direct access in general but additionally to non-structured conversations in which "[unusual participant terms] may stress theoretically important or interesting phenomena". In the same vein, concepts may also be, alternatively, "observer-identified"%
\footnote{HAMMERSLEY and ATKINSON. \textit{Ethnography: principles in practice}. 3rd ed. London; New York, NY: Routledge, 2007. P. 164 ("Sensitizing concepts" is Blumer's), 163.}.

The axial concepts are not %be used as fixed tautologies 
to give a taken-for-granted understanding of behaviors. The approach here is first \textit{exploratory}, rather  than explanatory. The deeper understanding of behaviors and use of tools, resources and knowledge %in general/
on the whole, %shall be developed later 
shall come later, during research. The intention is first to gather data, concepts, and a series of insights from in-field work.

Ultra-running has a certain tension in the way it connects participants with people from the outside social worlds.

\begin{itemize}
 \item On one side, it is an ultimately public activity, runners are exposed to permanent contact with other runners (and non-runners as well) in the open, and races depend on a wide number  of actors, both participating and non-race related: in sum, a very wide orchestrated and coordinated social activity.
 \item On the other side, ultra-running entails a certain \textit{Loneliness of the long distance runner}%
 \footnote{Short story by Alan Sillitoe, published in 1959.}. 
 Running ultra distances may well be one of  the most \textit{outdoor} activities or sports. It involves several hours, even days sometimes "out in the  open", amongst  almost untouched nature and wild green spaces afar from a city in cross-country races. And in training season, even in city context: the silent early night-to-dawn moment (from 4 to 6 am) is when nearly no ordinary person is going about, and birds have not even began to chirp. As well as with lone spaces, running involves far many solitary moments in which runners get to collect themselves and revolve in their thoughts, the bareness of the surroundings, and at many flowing times:  
 think of nothing and seize the moment.  
 %not think in anything and be in the moment.
\end{itemize}

The \textit{in situ} work is intended to grasp these two areas (intimate-personal; and social-network-dependent) in ultra-running: the first, during training; and the second, during specific ultra-running events.

\begin{enumerate}
 \item The first aspect, training, is to be dealt with  through auto-ethnography, not as a biographical account, but as a means to grasp the main topics developed. Many of the available material on ultra-running in text and video documentary depict narratives from the sole perspective of runners themselves, in first person, and how they prepare for their practices with different styles of running and post practice cool downs and stretching as well as general nutrition and resting time. The researcher may well take a similar approach without being an outsider of common practice in this social world.
 
 \begin{quote}
  Gertrude Kurath (1960) recommended ethnographers to "learn the movements" and Adrienne Kaeppler (1978) proposed that ethnographers learn certain movements and  receive instructions on  what is done "incorrectly", or "differently" with a methodology that would allow for better understanding.
  %to understand better. 
  [José Bizerril has argued that the practical formation of the researcher has its advantages.] This knowledge allows  access to aspects of the research topic that otherwise would pass unnoticed if only done with a distant approach based on observation and interview. [the experiential dimension makes it possible to gain entry to the experience and] "to the psycho-physical and -why not-, to the spiritual states that that this experience triggers%
  \footnote{ASCHIERI, Patricia. "Hacia una etnografía encarnada: La corporalidad del etnógrafo/a como dato en la investigación". X RAM- Reunión de Antropología del Mercosur. Córdoba, Argentina, 2013. P. 16. My translation.}.
 \end{quote}
 
 Of course,  auto-ethnography may work with a potential source for bias, but at the same time provides both the most inner side view possible, and reveals the speaker's interests, perspectives and preconceptions; to which one can always add contrast with other references to compare and find the most reliable common ground%
 \footnote{HAMMERSLEY and ATKINSON. \textit{Ethnography: principles in practice}. 3rd ed. London; New York, NY: Routledge, 2007. P.%164, %("Sensitizing concepts" is Blumer's), 
 124.}.
 
 \item On the second aspect, on racing events, there is very little material in academic research on events from a qualitative approach. There is scarce material, and when so, only done through surveys or measurement based. Hence, the importance to move forward. Some of the key features of an \textit{ethnographic approach} are taken into account in the present proposal: to prioritize the insider perspective highlighting the experiential, an active immersion in the field during a reasonable amount of time, minimal interference to gather data to be triangulated%
 \footnote{HOLLOWAY, Imma; BROWN, Lorraine; and SHIPWAY, Richard. "Meaning not measurement: Using ethnography to bring a deeper understanding to the participant experience of festivals and events". \textit{International Journal of Event and Festival Management}. Vol. 1 Nº 1, 2010. Pp. 75-76.}.
 And not to focus on measuring variables, but rather on \textit{collecting and constructing new variables} to build up ever more complex concepts: this adds nuance to the understanding of the phenomenon, and provides material to suggest new questions and aspects to be worked on%
 \footnote{BECKER, Howard S. \textit{What About Mozart? What About Murder? Reasoning From Cases}. The University of Chicago Press, Chicago, 2014. Pp. 13-14, 18.}.

\end{enumerate}




%%%%%%%%%%%%%%%%%%%%%%%%%%%%%%%%%

% benjaminjuarez.com/ARCHIVO/2016.02.29.OutlineProjectPhD-extended.html 

%Dear Benjamin,  I very much enjoyed reading your draftPhD proposal. I think it is an exciting project. 

%I wonder if you could expand a bit more on Section 3 Materials and Methods. This would also need that you specify a couple of research questions which affect your methodological account, your methods etc (ho will you research the lifeworld of runners?)...

%You may also say a bit more about your conceptual approach? 

%Kind regards
%Michael. 
%\section*{Research questions }


\pagebreak
\subsection*{Conceptual approach} %Research questions
The general problems of mass production and consumption %and their consequences 
have been noted even from the dawn of the industrial ages. 
The topic of massivity has run from the \textit{marvelous rise} of the car industrialization in the early XX\textsuperscript{th} century going back to the actual need to decongest traffic and search for new patterns of mobility and a sense of participating in the environment instead of driving over all ecosystems. This does not happen in a neutral and clean socio-political situation.
%
These conducts are part of a broad systematic pattern.
%More than 35 years ago 
Deleuze \& Guattari (2010: 527)
%a couple of french philosophers had 
signaled that
things as different as monopoly and the specialization of most of the medical knowledge,
the complication of the automobile motor, the gigantism of machines, do not correspond to any technological need,
but rather to economic and political imperatives. %(Deleuze \& Guattari, 2010: 527). 
%that aim to concentrate potency and control in the hands of a dominant class.
%\begin{quote} % Es evidente que cosas tan diferentes como el monopolio o la especialización de la mayor parte de los conocimientos médicos, la complicación del motor del automóvil, el gigantismo de las máquinas, no corresponden a ninguna necesidad tecnológica, sino solamente a imperativos económicos y políticos que se proponen concentrar potencia y control en las manos de una clase dominante (Deleuze y Guattari, 2010: 527).  %\end{quote}

Certain objects and conducts of today's societies have built and shaped urban landscapes in an ever growing manner.
Many of them blocking and constraining transit of people, of resources, and even being a blockage for ideas and customs.
It becomes increasingly widespread and evident the way in which vehicles stagnate in traffic during 
long inner-week-hours and amounts of cars lost in traffic through the world. 
Billions of people also follow, or intend to do so with tidy regularity, a standarized daily work schedule from 9 to 5.
Roads can function as boundaries when they striate space into fixed compartiments of places of circulation,
but roads can also be connectors of smooth space that open to the world and to infinite paths (Brighenti, 2009: 64).

%On the other hand, 
While several flows tend to become a little more at stop,
there are also countermovements that move against the said stagnation:
all together many different currents of flow could be taken into account, but mostly two different types coexist:
those which favor movement, and those that tend to collapse. They could be called rythms and anti-rythms.
We can percieve %experience 
%all 
variations in possibilities of movement by passing through spaces and becoming part with the surroundings.

% INGOLD page 167
\begin{quote}
%I remarked above that 
we experience the contours of the landscape by moving through it,
%so that it enters - as Bachelard would say - into our `muscular consciousness'. 
%Reliving the experience in our imagination, we are inclined to recall the road we took as 'climbing' the hill, or as 'descending' into the valley, as though 'the road itself had muscles, or rather, counter-muscles' (Bachelard 1964: 11). And this, too, is probably how you recall the paths and tracks that are visible to you now: after all, you must have travelled along at least some of them to reach the spot where you are currently standing. Nearest at hand, a path has been cut through the wheat-field, allowing sheaves to be carried down, and water and provisions to be carried up. Further off, a cart-track runs along the valley bottom, and another winds up the hill behind. In the distance, paths criss-cross the village green. Taken together, these paths and tracks 'impose a habitual pattern on the movement of people' (Jackson 1989: 146). And yet they also arise out of that movement, for 
[...]
every path or track shows up as the accumulated imprint of countless journeys that people have made 
% - with or without their vehicles or domestic animals - 
as they have gone about their everyday business. Thus the same movement is embodied, on the side of the people, in their `muscular consciousness', and on the side of the landscape, in its network of paths and tracks. In this network is sedimented the activity of an entire community, over many generations. It is the taskscape made visible. (Ingold, 1993: 167)
\end{quote}

Some people prefer to walk through a forest; others enjoy driving a car.
%
Many urban persona relate to the city in varying degrees. Some authors, such as Goffman and Von Uexküll, have 
seen how %much and the type of relation to 
the surroundings interact with animals and humans to %so as to 
create an environment, which they call the surroundings, or more technicaly: the \textit{Umwelt}, 
the involving space from which signs of alarm are expected. 
As it happens in ethology, it also applies in humans: ``the size of Umwelten
varies considerably according to the species''. Of interest here is that
this area can move, and can expand and contract according to whom, and how, is at the
center of this phenomenom. These perceptions, shared and negotiated, make
up to a pluralism of views and reactions. 

Additionally, every social world, in Becker’s terms, attracts a number
of resources, knowledge as well as consensus and resistance. Some social worlds
more than others depend on, and have an influence in
productive systems. Here it is claimed that the ultrarunners environment impacts 
on ways of living, and are on the tip of a certain curve of behaviour, that
is: of discipline, deprivation, potential mental disturbance and generally extreme
experiences, all of which affect the way cities are lived in.

%\pagebreak
\subsection*{Research questions and workplan} 

The above leaves the way closer to attempt a series of questions.
Would it be possible then to see freeways and cities as something else than a mere containment of controlled flows?
Or said in a more positive way and onto a slightly other physical direction: \medskip
%

\textit{Can we consider roads, paths in general, and other technological artifacts as enablers that shape 
human experience and social relationships?} 
%\textsc{Can we consider roads, paths in general, and other technological artifacts as enablers that shape human experience and social relationships?} 
And this can even be considered as a double track proposal:
%
\textit{%could we consider certain 
%Are 
%In which ways 
How do human experiences and social relationships %as well
%shapers of roads, paths in general, 
shape pathways, views, resources and technologies?
%and other technological artifacts?
}
Even if two seemingly separate sides merge to form a socio-technical assemblage, they in fact hibridize: 
hence the importance to rescue all agents, humans and non humans, involed with simetrical weight. 
The double sided view separates analitically what actually forms a network of dependencies.
\medskip
%%%%%%%%%%%%%%%%%%

There are many available resources to study the ultrarunning scene even from a distance.
Secondary material involves both texts%
\footnote{FIXX, James F.; JUREK, Scott; KOSTRUBALA, Thaddeus; LEONARD, George; McDOUGALL, Christopher; MURAKAMI, Haruki; NUNES, Valmir; ROHÉ, Fred. ROLL, Rich.}  
%En segundo lugar, las fuentes secundarios de campo, que tiene que ver con los materiales que los propios ultracorredores y fue citada a lo largo de este proyecto y en la bibliografía. Adicionalmente hay una serie de autores y textos disponibles vía web, habría que tener presente una multiplicidad de actores: los estudios de vibram en zapatillas minimalistas, las revisiones y comentarios del médico Mark Cucuzella, textos varios de Peter Larson, Jason Robillard y en general el acceso infinito que da la red rizomática del sitio web run100s. Los corredores más experimentados tienen en cuenta un abanico de variadades en lo que hace a la técnica. 
written by participants, and journalists; as well as videos\footnote{BENNA J.B; COEMAN, Tom; DUNHAM, Jon; EHRLICH, Judd; FRANKEL, Davey Frankel \& LAKEW, Rasselas; HEISENBERG, Benjamin. RICHARDSON, Tony; ROTHWELL, Jerry; STEWART, Rob; STUART, Mel.}.
Part of this material has already been read and viewed. 
%The following step is to connect main themes with the Theoretical Framework and to a ethnographic/qualitative axis.

Field work allows for direct contact with the ultra world and for day to day updates on normal practices and non-structured interviews. The first-hand material is expected to be a strength of the proposal since the candidate is a long-time runner, with more than two decades of experience in several distances. Having already completed the marathon distance the candidate is highly likely to fulfill races of at least 50 kms. Longer distances (80, 100, 150) could and are expected to be attemped later on. 
Regardless of the kind of participation, be it by running or simply attending to events as observer, the contacts have already began: I already gained information on specific yearly ultra-distance races with different attractives:
the german Rennsteiglauf with an average of 15000 participants, the chilean Rapa Nui trophy at the exotic Easter Island,
and the important NGO that prepare races for awareness to fight aganst human trafficking: Muskathlon, both in South Africa and also crossing the border from Bulgaria into Greece.
%http://www.a21.org/campaigns/content/muskathlon/glarjd?permcode=glarjd

The relevance of what appears in the environment to ultrarunners is not always obvious in a third party written description, or even in conversation. The possibility to participate in the same training and competitions is to be part of the “same capsule of events” as other ultrarunners. What changes is not only the events but rather their at-handedness, which allow for a closer possibility of involvement.

Two possible outcomes of the study involve: on the one hand, the chance to get in-depth insight on the technological analysis of these practices and events. On this matter, time for research at the Sciences Po would be a gain. 
The direction of the project would benefit from the perspective of considering the mechanical-chemical aspects of ultra in relation to scientific humanities, specialty of Bruno Latour’s team, with whom the candidate has taken an online MOOC course, early 2014. On the other hand, the second possibility should aim at spending time together with specific communities of ultrarunners.

