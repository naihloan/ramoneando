\documentclass[english,a4paper,nologo]{europasscv}
\usepackage[english]{babel}
% \usepackage{url}
\usepackage{hyperref}

\ecvname{Benjamin Juarez}
\ecvaddress{Crisol 305, Nueva Córdoba, Córdoba, Argentina}
% \ecvmobile{+54 9 351 310-4043}
\ecvemail{benjij\_1980@yahoo.com}

% \ecvhomepage{http://ramoneando.com/}
\ecvdateofbirth{16 July 1980}
\ecvnationality{%United States of America
American
}
\ecvgender{Male}

\ecvfootnote{* Master degree full text link: 
% \textit{Urban Art -- Expressive uses of public space} %, % [in Spanish]: 
\href{http://ramoneando.com/docs/Juarez,Benjamin_M.pdf}{\texttt{ramoneando.com/docs/Juarez,Benjamin\_M.pdf}}
}

\newcommand*{\mybox}[1]{\framebox{#1}}


% \ecvpicture[width=3.8cm]{picture.jpg} % \ecvpictureright

\begin{document}
  \begin{europasscv}
  \ecvpersonalinfo
%   \ecvbigitem{PhD Project Title}{\textit{Distance athlete’s æffect on urban rhythm:}} \vspace*{-3mm}
%   \ecvbigitem{}{\textit{How do ultrarunners run in automatized cities?}}
% %   \ecvbigitem{Position applied for}{PhD Sociology}
%   \ecvbigitem{PhD Supervisor}{Professor Michael Schillmeier [University of Exeter]}
%   \ecvbigitem{PhD Co-Supervisor}{Professor Mike Michael [University of Exeter]}
  
  
  
% % % % % % % % % % % % % % % % % % % % % % % % % % % % % % % % % % % % % % % % 
    
			\ecvsection{Education and training}

  \ecvtitlelevel{2017--Present}{Phd Candidate [Distance Based]. Project Title:
  \textit{Distance athlete’s \mbox{æffect} on urban rhythm: How do ultrarunners run in automatized cities?} }{ } \vspace*{-4mm}
  \ecvitem{}{University of Exeter, Exeter, England}

%     \ecvtitlelevel{2017--Present}{Phd Candidate [Distance Based]. Project Title:
%     \textit{Distance athlete’s \mbox{æffect} on urban rhythm: How do ultrarunners run in automatized cities?}
%     }{ } 
%     \vspace*{-12mm}
%   \ecvtitlelevel{}{\textit{%Distance athlete’s æffect on urban rhythm: How do ultrarunners run in automatized cities?
%   }}%
%    {\href{http://ramoneando.com/docs/juarez-benjamin-2017-01-23-phd-project-exeter-7.pdf}{\texttt{*online pdf link}}   
%   }   \vspace*{4mm}
% %   \ecvtitlelevel{}{Empirical research on graffiti/pixação}{}
% %   \ecvitem{}{Empirical research on graffiti/pixação. São Paulo, São Paulo. Brazil}
%   \ecvitem{}{University of Exeter, Exeter, England}
% %   \ecvitem{}{Universidade Estadual de Campinas [UNICAMP]. Campinas, São Paulo. Brazil}

  \ecvtitlelevel{2012--2014}{Master in Sociology. Dissertation Title:}{ } \vspace*{-4mm}
  \ecvtitlelevel{}{\textit{Urban Art. Expressive uses of public space}}%
  {\href{http://ramoneando.com/docs/Juarez,Benjamin_M.pdf}{\texttt{*online pdf link}}   }   %\vspace*{-4mm}
%   \ecvtitlelevel{}{Empirical research on graffiti/pixação}{}
%   \ecvitem{}{Empirical research on graffiti/pixação. São Paulo, São Paulo. Brazil}
  \ecvitem{}{Universidade Estadual de Campinas [UNICAMP]. Campinas, São Paulo. Brazil}
%   \ecvitem{}{
%       \begin{ecvitemize}
% 	\item Urban Sociology \& Science and Technology Studies (STS)
% 	\item Research Scholarship given by CNPq 
% 	\item Supervisor: Dr. Pedro Peixoto Ferreira
%       \end{ecvitemize}
%   }

  \ecvtitle{2002--2009}{Bachelor of Science in Sociology}
  \ecvitem{}{Universidad de Buenos Aires [UBA], Buenos Aires Argentina}
  
% % % % % % % % % % % % % % % % % % % % % % % % % % % % % % % % % %   

% \ecvsection{\mybox{Book Store Experience}}
%   \ecvitem{\mybox{2009--2010}}%
%   {\noindent\fbox{\parbox{0.7\textwidth}{%
%   I worked for 5 months at \textit{Borders}, a book store located in Suwanee, Georgia, US.
%   I got to know the workflow of a convenience store, enjoyed the benefits of borrowing books as part of the employer's policy and learned how to orient customers upon book requests.
%   }} }
  
% % % % % % % % % % % % % % % % % % % % % % % % % % % % % % % % % %   

\ecvsection{\mybox{Teaching Positions}}


  
  \ecvitem{\mybox{ACADEMIC}}{%
  \noindent\fbox{%
    \parbox{0.7\textwidth}{%
  Over the past 5 years, I have assisted students with graduate level projects. I read their texts and make observations on methodological and empirical topics, pointing out contents and readings, following each student's progress.}}
  }
  %   Feedback was normally weekly, and sometimes monthly, depending on the frequency of writing by part of the students.
  
%   \clearpage
  
  \ecvtitle{March 2016 – Present}{Appointed assistant} 
  \ecvitem{}{Faculty of Sociology. Universidad Nacional de Villa María -- UNVM [Córdoba. Argentina]}
  \ecvitem{}{\textit{Workshop in Methods and Techniques in Social Research} -- with Professor Angelita Alessio}
  
  \ecvtitle{March 2013 – July 2013}{Teaching assistant} 
  \ecvitem{}{Faculty of Sociology. Universidade Estadual de Campinas -- UNICAMP [Campinas, Brazil]}
  \ecvitem{}{\textit{Introduction to the study of the city}  – with Professor Silvana Rubino}
   
  \ecvtitle{August 2012 – December 2012}{Teaching assistant} 
  \ecvitem{}{Faculty of Sociology. Universidade Estadual de Campinas -- UNICAMP [Campinas, Brazil]}
  \ecvitem{}{\textit{Research Methodology}  – with Professor Roberto Di Carmo}
  
%   \ecvtitle{March 2010 – December 2011}{Appointed assistant} 
%   \ecvitem{}{Faculty of Law. Universidad Nacional de Córdoba -- UNC [Córdoba. Argentina]}
%   \ecvitem{}{\textit{Juridical Sociology} -- with Professor Carlos Lista}

  \ecvsection{%
%   \fbox{
  Teaching Positions
%   }
  }

\ecvitem{\mybox{EXTRA ACADEMIC}}%
  {\noindent\fbox{\parbox{0.7\textwidth}{%
  The last 2 years I have been giving Sociology classes %at an institution for a %the 
  at %general \mbox{public}: 
  \textit{Centro de Espiritualidad Santa María}.
  The responses of the students and staff have been very positive and the experience of \mbox{teaching} has been very gratifying personally. I see it as a possibility to put research questions into play and to probe the general feeling that sociological research has for a wider public.
  }} }
%   \ecvitem{}%
%   {\noindent\fbox{\parbox{0.7\textwidth}{Over the past 4 years I have given a number of Workshops on Running. I instruct on the ChiRunning Technique, and have traveled through the country assisting runners of various levels, from beginners to advanced. Which favors my research purpose, on ultrarunning. }} }
  \ecvitem{}{
  \noindent\fbox{%
    \parbox{0.7\textwidth}{%
  Additionally, English is my native tongue: I was born in USA and am an American living in Argentina since age 7. I have given conversation classes in English, for %scholars 
%   and the general public, 
%   with %a range of academic 
  students 
  coming with academic background ranging from philosophy, to tourism and biochemistry. }}
  }
  
% % % % % % % % % % % % % % % % % % % % % % % % % % % % % % % % % %   
  


% % % % % % % % % % % % % % % % % % % % % % % % % % % % % % % % % %   



  \pagebreak
  \ecvsection{Research Positions}
  
  \ecvtitle{March 2016 – Present}{Research team -- with María Inés Landa} 
  \ecvitem{}{CIECS at CONICET Research Center [Córdoba. Argentina]}
  \ecvitem{}{%Prácticas de gestión corporal y procesos de subjetificación en la cultura contemporánea
  \textit{Corporeal management practices and processes of subjectification in the contemporary culture} }
  
  
  \ecvtitle{March 2011 – December 2011}{Research team -- with Gustavo Blázquez} 
  \ecvitem{}{SECYT / Universidad Nacional de Córdoba -- UNC [Córdoba. Argentina]}
  \ecvitem{}{\textit{Contemporary subjectivities} }
   
  \ecvtitle{October 2004 – April 2007}{Undergraduate researcher -- with Carlos Belvedere} 
  \ecvitem{}{Instituto de Investigación Gino Germani -- IIGG  [Buenos Aires. Argentina]}
  \ecvitem{}{\textit{UBACyT Project: Action, structure and the problem of social order}}
            
%   \pagebreak


			\ecvsection{Publications}
    
%   \ecvblueitem{2017}{``The social worlds of NIKE's CEO Phil Knight. An entrepreneur's effect on trends of massive behaviour, and the reshape of production'', [draft stage]. }
  
  \ecvblueitem{2016}{``The visual and social indeterminacy of pixação: the inextricable moods of São Paulo’s inscriptions'', 
  In \textit{Street Art \& Urban Creativity Scientific Journal} – Lisboa. Center Periphery: Practice | Vol. 2 Nº 1. %November.
  }

  \ecvblueitem{2008}{``The legacy of Durkheim in Schutz: towards a horizon in dialog'', In peer-reviewed [Latindex] \textit{Enfoques} – Entre Ríos, Argentina. 
  Vol. 20, Núm. 1-2.}
    %: uap.edu.ar/es/enfoques/ –
  
  \ecvblueitem{2008}{``Observations on public opinion as from The Well Informed Citizen'', In peer-reviewed [Latindex] \textit{Question} – La Plata, Argentina. 
  Vol. 1, Núm. 18. }
    
    
			\ecvsection{Presentations}
  
  \ecvblueitem{2017}{\textit{%Calzado deportivo como tecnología: trayectorias socio-técnicas del talón elevado/acolchonado
  Sport shoes as technology: socio-technical trajectories of the elevated/cushioned support}. Abstract accepted at \textsc{XXXI Congreso de la Asociación Latinoamericana de Sociología}. ALAS 2017 – Montevideo. December 3 – 8. }

%   \ecvblueitem{2017}{\textit{Urban inscription is not only about crisis. The case of Yellow Stars as a legalized protest against deadly traffic accidents.}
%   In preparation for \textsc{XII Reunión de Antropología del Mercosur}  RAM 2017 – Posadas. December 4 – 7. }

  \ecvblueitem{2014}{\textit{Open spaces: the street as habitat.%   %Espacios abiertos: la calle como hábitat.
  }
  \textsc{I Congreso de Epistemologías Críticas en el campo del Habitat}. Facultad de Arquitectura y Urbanismo [FAU] – Córdoba, Argentina.}
 
  \ecvblueitem{2013}{\textit{1978-2013 Paulistano urban art: between urban degradation and turistic landscaping}. At \textsc{X Reunión de Antropología del Mercosur}. RAM 2013 – Córdoba, Argentina.}

  \ecvblueitem{2012}{Workshop: \textit{Urban Geopolitics: Recongurations from Art, Activism, and Research. Hemispheric Institute of Performance \& Politics / Convergence. The geo/body politics of emancipation}. Duke University – Durham, USA.}

  \ecvblueitem{2011}{``Is there a juridical order? Looking back at the terrain, foundations of the socio-legal''. At \textit{1er Encuentro-Debate en Estudios de Posgrado en Sociología Jurídica}. Córdoba, Argentina. }
  
  \ecvblueitem{2011}{\textit{It is (not) so easy to be a graffiti artist}. At \textsc{III Jornadas de Antropología Social del Centro}. Campus Olavarría – Provincia de Buenos Aires, Argentina.}
  
% % % % % % % % % % % % % %   
\clearpage
			\ecvsection{Collaborations at Events}
  
  \ecvblueitem{2013}{\textit{Visual Anthropology: Posters Committee}.  
Reunión de Antropología del Mercosur. [X RAM] – Córdoba, Argentina. July.}
  
  \ecvblueitem{2011}{%
  Organizer \& Moderator. 
  \textit{1er Encuentro-Debate en Estudios de Posgrado en Sociología Juridica}. Córdoba, Argentina. November. }
  
  \ecvblueitem{2005}{%
  General Visual Theme of Event and presentations. 
  \textit{International Max Weber Reunion Universidad de Buenos Aires and Göethe-Institut}. 
  Buenos Aires, Argentina. }
    
    
			  \ecvsection{Divulgation}

\ecvblueitem{2017}{\textit{Have you seen your city walls lately? The changing life of street art}, In Graphic Design website: \textsc{FiveStone}, Atlanta, Georgia. USA. }
\ecvblueitem{2016}{\textit{Exhalations}} \vspace*{-5mm}
\ecvblueitem{Book}{Independent Publication \textsc{Inexplicable Publishers}.} \vspace*{-5mm}
\ecvblueitem{}{Córdoba, Argentina.}


  
		  \ecvsection{Translations (to Spanish)}
 
%   \ecvblueitem{2017}{Becker, Howard. ``23 ideas sobre la juventud'' Sent to \textit{Apuntes de Investigación del CECYP}, Buenos Aires. [Original version: 2008, ``Twenty Three Thoughts About Youth.'' \textit{La marque jeune}, edited by Marc-Olivier Gonseth, Yann Laville and Grégoire Mayor (Neuchâtel: Musée d'ethnographieNeuchâtel).}
 
 \ecvblueitem{2011}{Kalberg, Stephen. ``La influencia de las cosmovisiones en el pasado y en el presente según Max Weber.'' In \textit{Sociológica}, Mexico D.F., pp. 207-246. [Original version: 2004, ``The Past and Present Influence of World Views: Max Weber on a Neglected Sociological Concept.'' Journal of Classical Sociology 4:139-64.] }
 
 \ecvblueitem{2005}{Kalberg, Stephen. ``Los tipos de racionalidad de Max Weber: piedras angulares para el análisis del proceso de racionalización de la historia'', in Aronson y Weisz (comp.) ́``Sociedad y religión: un siglo de controversias en torno a la noción weberiana de racionalización''. Prometeo, Bs. As., Pp. 73-116. [Original version: 1980, ``Max Weber’s Types of Rationality: Cornerstones for the Analysis of Rationalization Processes in History.'' American Journal of Sociology 85:1145-79.]}

%  \pagebreak
 
  
  \ecvsection{Personal skills}
  \ecvmothertongue{Spanish-English}
  \ecvlanguageheader
  \ecvlanguage{Portuguese}{B2}{C2}{C1}{C1}{B2}
  \ecvlanguage{French}{B1}{B2}{A2}{A2}{A2}
  \ecvlanguage{Italian}{B1}{B1}{A2}{A1}{A2}
  \ecvlastlanguage{German}{A1}{B1}{A1}{A1}{A1}
%   \ecvlanguagecertificate{Diplôme d'études en langue française (DELF) B1}
  \ecvlanguagefooter
   
%   \ecvblueitem{Communication skills}{
%   \begin{ecvitemize}
%     \item team work: I have worked in various types of teams from research teams to national league hockey. For 2 years I coached my university hockey team
%     \item mediating skills: I work on the borders between young people, youth trainers, youth policy and researchers, for example running a 3 day workshop at CoE Symposium ``Youth Actor of Social Change'', and my continued work on youth training programmes 
%     \item intercultural skills: I am experienced at working in a European dimension such as being a rapporteur at the CoE Budapest ``youth against violence seminar'' and working with refugees.
%   \end{ecvitemize}
%   }
%   
%   \ecvblueitem{Organisational / managerial skills}{
%   \begin{ecvitemize}
%     \item whilst working for a Brussels based refugee NGO ``Convivial'' I organized a ``Civil Dialogue'' between refugees and civil servants at the European Commission 20th June 2002
%     \item during my PhD I organised a seminar series on research methods
%   \end{ecvitemize}
%   }
% 
%   \ecvdigitalcompetence{\ecvBasic}{\ecvIndependent}{\ecvProficient}{\ecvIndependent}{\ecvBasic}
  
  \ecvblueitem{Computer skills}{
  \begin{ecvitemize}
    \item \LaTeX $ $ [professional word processor] |
%     \item 
    \texttt{markdown} [markup language] |
    \texttt{pandoc} [transformation of code language] | 
    \texttt{html, css} [construction and use of web content] |
    \texttt{vim} [text editor]
    \item Proficient user of several coding languages, used on a daily basis 
    \item  \url{http://ramoneando.com/} | I run my personal site with %sociological observations, 
    general information and fieldnotes 
    \item     \texttt{Gimp, Inkscape} | Independent user of graphic programs [linux based]
    \item \texttt{javascript, python} | rudiments
  \end{ecvitemize}
  }
  
  
%   \ecvblueitem{Other skills}{I enjoy several sports particularly ultrarunning, soccer. I also practice tai chi. Another related hobbie is cooking. Traveling and experiencing different cultures is also a plus.}

  
  
  \end{europasscv}

\end{document}

% !TEX encoding = UTF-8
% !TEX program = pdflatex
% !TEX spellcheck = en_GB
