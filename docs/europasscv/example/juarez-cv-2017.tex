\documentclass[english,a4paper,nologo]{europasscv}
\usepackage[english]{babel}

\ecvname{Benjamin Juarez}
\ecvaddress{Crisol 305, Nueva Córdoba, Córdoba, Argentina}
\ecvmobile{+54 9 351 310-4043}
\ecvemail{benjij\_1980@yahoo.com}

\ecvdateofbirth{16 July 1980}
\ecvnationality{United States of America}
\ecvgender{Male}

% \ecvpicture[width=3.8cm]{picture.jpg} % \ecvpictureright

\begin{document}
  \begin{europasscv}
  \ecvpersonalinfo
  \ecvbigitem{Position applied for}{PhD Sociology}
    
			\ecvsection{Education and training}
  
  \ecvtitlelevel{2012--2014}{Master - Disertation Title: 'Urban Art. Expressive uses of public space’, Empirical research on graffiti/pixação}{%ISCED~6
  }
  \ecvitem{}{Universidade Estadual de Campinas [UNICAMP], São Paulo Brazil}
  \ecvitem{}{
      \begin{ecvitemize}
	\item Urban Sociology \&
% 	\item anthropology
% 	\item sociology of scientific knowledge / information society
% 	\item 
	Science and Technology Studies (STS)
	\item Research Scholarship given by CNPq 
	\item Supervisor: Dr. Pedro Peixoto Ferreira

      \end{ecvitemize}
  }

  \ecvtitle{2002--2009}{Bachelor of Science in Sociology}
  \ecvitem{}{Universidad de Buenos Aires [UBA], Buenos Aires Argentina}

			\ecvsection{Publications}
    
  \ecvblueitem{2017}{``The social worlds of NIKE's CEO Phil Knight. An entrepreneur's effect on trends of massive behaviour, and the reshape of production'', [draft stage]. }
  
  \ecvblueitem{2016}{``The visual and social indeterminacy of pixação: the inextricable moods of São Paulo’s inscriptions'', 
  In \textit{Street Art \& Urban Creativity Scientific Journal} – Lisboa. Center Periphery: Practice | Vol. 2 Nº 1. %November.
  }

  \ecvblueitem{2008}{``The legacy of Durkheim in Schutz: towards a horizon in dialog'', In peer-reviewed [Latindex] \textit{Enfoques} – Entre Ríos, Argentina.}
    %: uap.edu.ar/es/enfoques/ –
  
  \ecvblueitem{2008}{``Observations on public opinion as from The Well Informed Citizen'', In peer-reviewed [Latindex] \textit{Question} – La Plata, Argentina.}
    
    
			\ecvsection{Presentations}
  
  \ecvblueitem{2017}{\textit{%Calzado deportivo como tecnología: trayectorias socio-técnicas del talón elevado/acolchonado
  Sport shoes as technology: socio-technical trajectories of the elevated/cushioned support}. Abstract accepted at \textsc{XXXI Congreso de la Asociación Latinoamericana de Sociología}. ALAS 2017 – Montevideo. December 3 – 8. }

  \ecvblueitem{2017}{\textit{Urban inscription is not only about crisis. The case of Yellow Stars as a legalized protest against deadly traffic accidents.}
  In preparation for \textsc{XII Reunión de Antropología del Mercosur}  RAM 2017 – Posadas. December 4 – 7. }

  \ecvblueitem{2014}{\textit{Open spaces: the street as habitat.%
  %Espacios abiertos: la calle como hábitat.
  }
  \textsc{I Congreso de Epistemologías Críticas en el campo del Habitat}. Facultad de Arquitectura y Urbanismo [FAU] – Córdoba, Argentina.}
 
  \ecvblueitem{2013}{\textit{1978-2013 Paulistano urban art: between urban degradation and turistic landscaping}. At \textsc{X Reunión de Antropología del Mercosur}. RAM 2013 – Córdoba, Argentina.}

  \ecvblueitem{2012}{Workshop: \textit{Urban Geopolitics: Recongurations from Art, Activism, and Research. Hemispheric Institute of Performance \& Politics / Convergence. The geo/body politics of emancipation}. Duke University – Durham USA.}

  \ecvblueitem{2011}{``Is there a juridical order? Looking back at the terrain, foundations of the socio-legal''. At \textit{1er Encuentro-Debate en Estudios de Posgrado en Sociología Jurídica}. Córdoba.%, November
  }
  
  \ecvblueitem{2011}{\textit{It is (not) so easy to be a graffiti artist}. At \textsc{III Jornadas de Antropología Social del Centro}. Campus Olavarría – Provincia de Buenos Aires.}
  
% % % % % % % % % % % % % %   
			\ecvsection{Collaborations at Events}
  
  \ecvblueitem{2013}{\textit{Visual Anthropology: Posters Committee}.  
Reunión de Antropología del Mercosur. [X RAM] – Córdoba, July.}
  
  \ecvblueitem{2011}{%
  Organizer \& Moderator. 
  \textit{1er Encuentro-Debate en Estudios de Posgrado en Sociología Juridica}. Córdoba, November. }
  
  \ecvblueitem{2005}{%
  General Visual Theme of Event and presentations. 
  \textit{International Max Weber Reunion Universidad de Buenos Aires and Göethe-Institut}. 
  Buenos Aires, Argentina. }
    
    
			  \ecvsection{Divulgation}

\ecvblueitem{2017}{\textit{Have you seen your city walls lately? The changing life of street art}, In Graphic Design website: \textsc{FiveStone}, Atlanta, Georgia. USA. }
  
		  \ecvsection{Translations (to Spanish)}
 
  \ecvblueitem{2017}{Becker, Howard. ``23 ideas sobre la juventud''
  In preparation. %\textit{Sociológica}, Mexico D.F., pp. 207-246. 
  [Original version: 2008, ``Twenty Three Thoughts About Youth.'' 
  \textit{La marque jeune}, edited by Marc-Olivier Gonseth, Yann Laville and Grégoire Mayor (Neuchâtel: Musée d'ethnographieNeuchâtel).}
 
 \ecvblueitem{2011}{Kalberg, Stephen. ``La influencia de las cosmovisiones en el pasado y en el presente según Max Weber.'' In \textit{Sociológica}, Mexico D.F., pp. 207-246. [Original version: 2004, ``The Past and Present Influence of World Views: Max Weber on a Neglected Sociological Concept.'' Journal of Classical Sociology 4:139-64.] }
 
 \ecvblueitem{2005}{Kalberg, Stephen. ``Los tipos de racionalidad de Max Weber: piedras angulares para el análisis del proceso de racionalización de la historia'', in Aronson y Weisz (comp.) ́``Sociedad y religión: un siglo de controversias en torno a la noción weberiana de racionalización''. Prometeo, Bs. As., Pp. 73-116. [Original version: 1980, ``Max Weber’s Types of Rationality: Cornerstones for the Analysis of Rationalization Processes in History.'' American Journal of Sociology 85:1145-79.]}

 \pagebreak
 
  \ecvsection{Work experience}
  
  \ecvtitle{August 2002 -- Present}{Independent consultant}
  \ecvitem{}{British Council\newline 123, Bd Ney, 75023 Paris (France)}
  \ecvitem{}{Evaluation of European Commission youth training support measures for youth national agencies and young people}
   
  \ecvtitle{March 2002 -- July 2002}{Internship}
  \ecvitem{}{European Commission, Youth Unit, DG Education and Culture \newline 200, Rue de la Loi, 1049 Brussels (Belgium)}
  \ecvitem{}{
  \begin{ecvitemize}
      \item evaluating youth training programmes for SALTO UK and the partnership between the Council of Europe and European Commission
      \item organizing and running a 2 day workshop on non-formal education for Action 5 large scale projects focusing on quality, assessment and recognition
      \item contributing to the steering sroup on training and developing action plans on training for the next 3 years. Working on the Users Guide for training and the support measures
  \end{ecvitemize}
  }
  \ecvitem{}{\ecvhighlight{Business or sector}\quad European institution}
  
  \ecvtitle{Oct 2001 -- Feb 2002}{Researcher / Independent Consultant}
  \ecvitem{}{Council of Europe, Budapest (Hungary)}
  \ecvitem{}{Working in a research team carrying out in-depth qualitative evaluation of the 2 year Advanced Training of Trainers in Europe using participant observations, in-depth interviews and focus groups. Work carried out in training courses in Strasbourg, Slovenia and Budapest.}
  
  
  
% \end{europasscv}
% \end{document}  
  
  \pagebreak
  
  \ecvsection{Personal skills}
  \ecvmothertongue{Spanish-English}
  \ecvlanguageheader
  \ecvlanguage{Portuguese}{C1}{C2}{B2}{C1}{C2}
  \ecvlanguage{French}{C1}{C2}{B2}{C1}{C2}
  \ecvlanguage{Italian}{C1}{C2}{B2}{C1}{C2}
  \ecvlastlanguage{German}{A2}{A2}{A2}{A2}{A2}
%   \ecvlanguagecertificate{Diplôme d'études en langue française (DELF) B1}
  \ecvlanguagefooter
   
  \ecvblueitem{Communication skills}{
  \begin{ecvitemize}
    \item team work: I have worked in various types of teams from research teams to national league hockey. For 2 years I coached my university hockey team
    \item mediating skills: I work on the borders between young people, youth trainers, youth policy and researchers, for example running a 3 day workshop at CoE Symposium ``Youth Actor of Social Change'', and my continued work on youth training programmes 
    \item intercultural skills: I am experienced at working in a European dimension such as being a rapporteur at the CoE Budapest ``youth against violence seminar'' and working with refugees.
  \end{ecvitemize}
  }
  
  \ecvblueitem{Organisational / managerial skills}{
  \begin{ecvitemize}
    \item whilst working for a Brussels based refugee NGO ``Convivial'' I organized a ``Civil Dialogue'' between refugees and civil servants at the European Commission 20th June 2002
    \item during my PhD I organised a seminar series on research methods
  \end{ecvitemize}
  }

  \ecvdigitalcompetence{\ecvBasic}{\ecvIndependent}{\ecvProficient}{\ecvIndependent}{\ecvBasic}
  
  \ecvblueitem{Computer skills}{
  \begin{ecvitemize}
    \item competent with most Microsoft Office programmes
    \item experience with HTML
  \end{ecvitemize}
  }
  
  
  \ecvblueitem{Other skills}{Creating pieces of Art and visiting Modern Art galleries. Enjoy all sports particularly hockey, football and running. Love to travel and experience different cultures.}

  
  
  \end{europasscv}

\end{document}

% !TEX encoding = UTF-8
% !TEX program = pdflatex
% !TEX spellcheck = en_GB
