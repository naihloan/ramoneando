%------------------------------------------------------------
%	 WORK
%------------------------------------------------------------
\en{
  \section{Recent work}
}

\en{
\begin{twentyshort} 
	\twentyitem{2020--2020}{Customer Dev Backoffice}{[Cintelink]} 
        {IoT sql python postman} %php, nodejs, vue, docker. 
	\twentyitem{2019--2020}{Back End Web Developer --nodejs}{[Ross Outside the Box]}
        {%Back End | Full Stack [nodejs express mariaDB %| (react angular)
        %] \\ 
%         This was my first experience in a tech/agile environment. 
%         I used source code versioning and 
%         We deliver results in small increments: 
%         We work in an agile environment. 
%         I use source code versioning and Kanban methodology [jira] on a daily basis.
%         We deliver results in small increments: my first solo project used the Knime platform to manage big data sets with SQL-type queries.} % docker
        nodejs/ %+ express / % docker   
        authentication. On my 1st solo project I used Knime with
        big data sets with SQL-type queries.
        testing: e2e + performance.
        } 
	
	\twentyitemshort{2018--2019}{Research Assistant -- Project \textit{In the Name of Wild} | Phillip Vannini}
	\twentyitemshort{2018--2019}{Translations and Reviews: Spanish to English}
	\twentyitemshort{2015--2018}{English Conversation Classes} 
	\twentyitemshort{2017-2018}{Pearson}
\end{twentyshort}
}

\de{
  \section{Trabajo reciente}
}

\de{
\begin{twentyshort} 
	\twentyitem{2020--2020}{Desarrollador Backoffice}{[Cintelink]} 
        {webApp ssh IoT sql python postman} %php, nodejs, vue, docker. 
	\twentyitem{2019--2020}{Back End Web Developer --nodejs}{[Ross Outside the Box]}
        {
        nodejs/ authenticación. 
        En el 1er proyecto a mi cargo usé Knime con big data y consultas tipo SQL.
%         On my 1st solo project I used Knime with
%         big data sets with SQL-type queries.
%         
        testing: e2e + performance.
        } 
	
	\twentyitemshort{2018--2019}{Productor -- Proyecto \textit{En Nombre de Lo Salvaaje} | Phillip Vannini}
	\twentyitemshort{2018--2019}{Traducciones y revisiones: de español a inglés}
	\twentyitemshort{2015--2018}{Clases de conversación de inglés} 
	\twentyitemshort{2017-2018}{Editorial Pearson}
\end{twentyshort}
}