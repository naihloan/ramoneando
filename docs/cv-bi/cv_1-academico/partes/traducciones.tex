\section{\centerline{OTROS %| TRADUCCIONES (AL ESPAÑOL)
}}

\section{\centerline{DOCENCIA %(NO UNIVERSITARIA)
}} 

    {\sl Profesor} \hfill 2016 | Presente \\
Centro de Espiritualidad Santa María \hfill [Córdoba. Argentina] \\
\textsc{Materia: Sociología}

%  \item 2016 | Presente. Profesor de instituto [Centro de Espiritualidad Santa María]

\section{\centerline{TRADUCCIONES (AL ESPAÑOL)}} 

\vspace{15pt} % Gap between title and text

\begin{itemize} \itemsep -2pt % Reduce space between items
% \item ``Job Analysis - A True Picture," Journal of Headhunters, Vol. 5, Number 3. Fall 1990 
% \item ``The Fine Tuning of Interpersonal Skills," Journal of Recruiting/Hiring, Vol. 16, Number 7, August 1990
% \end{itemize}

%   
%   \item 2019. En preparación. ``¿Qué le pasa a las muñecas de Haruki Murakami? Un ensayo en tono hipotético-irrealis sobre los cruces entre el uso de teclados convencionales y el entrenamiento de ultramaratón''.
%   
%   \item 2017. [Seudónimo: Rumi Ando] \textit{Being Ralphy Wiggum [About being \textit{special}]}. En \textsc{Sociological Fiction Fanzine}: \textsc{So Fi \# 2}. Sydney, Australia. \\
%   \url{https://sofizine.com/latest-edition/edition-2/}
%   
%   \item 2017. \textit{Have you seen your city walls lately? The changing life of street art}, En sitio web de diseño gráfico: \textsc{FiveStone}. Atlanta, Georgia. USA. \\
%   \url{http://www.fivestone.com/}
%   
%   \item 2016. ``The visual and social indeterminacy of pixação: the inextricable moods of São Paulo’s inscriptions'', 
%   En \textit{Street Art \& Urban Creativity Scientific Journal} – Lisboa. Center Periphery: Practice | Vol. 2 Nº 1. November. \\
%   \url{http://sauc.website/index.php/sauc/article/view/41}
% 
%   \item 2008. ``El legado de Durkheim en Schutz Hacia un horizonte en común''. En [Latindex] \textit{Enfoques} – Entre Ríos, Argentina. Vol. 20, Núm. 1-2.\\
%       \url{http://publicaciones.uap.edu.ar/index.php/revistaenfoques/article/view/245}
%     %: uap.edu.ar/es/enfoques/ –
%   
%   \item 2008. ``Observaciones sobre la opinión pública a partir de el ciudadano bien informado''. En [latindex] \textit{Question} – La Plata, Argentina. Vol. 1, Núm. 18. \\
%       \url{https://perio.unlp.edu.ar/ojs/index.php/question/article/view/537}
% \end{itemize}  
%   
		  
  \item 2019. Becker, Howard. ``23 ideas sobre la juventud'' En preparación.
%   \textit{Apuntes de Investigación del CECYP}, Buenos Aires. 
  [Original version: 2008, ``Twenty Three Thoughts About Youth.'' \textit{La marque jeune}, edited by Marc-Olivier Gonseth, Yann Laville and Grégoire Mayor (Neuchâtel: Musée d'ethnographieNeuchâtel).
 
 \item 2011. Kalberg, Stephen. ``La influencia de las cosmovisiones en el pasado y en el presente según Max Weber.'' In \textit{Sociológica}, Mexico D.F., pp. 207-246. [Original version: 2004, ``The Past and Present Influence of World Views: Max Weber on a Neglected Sociological Concept.'' Journal of Classical Sociology 4:139-64.] 
 
 \item 2005. Kalberg, Stephen. ``Los tipos de racionalidad de Max Weber: piedras angulares para el análisis del proceso de racionalización de la historia'', in Aronson y Weisz (comp.) ́``Sociedad y religión: un siglo de controversias en torno a la noción weberiana de racionalización''. Prometeo, Bs. As., Pp. 73-116. [Original version: 1980, ``Max Weber’s Types of Rationality: Cornerstones for the Analysis of Rationalization Processes in History.'' American Journal of Sociology 85:1145-79.]

%  \begin{itemize} 
%   \item 
 \end{itemize}

 
%  \pagebreak
\vspace{0.2in} % Some whitespace between sections

\section{\centerline{TRADUCCIONES (AL INGLÉS)}} 

\vspace{15pt} % Gap between title and text

\begin{itemize} 

\itemsep -2pt % Reduce space between items

 

 \item 2019. De Mauro, Martín. ``How to make sense of precariousness? \textit{Bíos}-precarious and sensitive life'', in 
 Bakhtiniana: Revista de Estudos do Discurso - ISSN 2176-4573.

 \item 2019. Various authors. \textsc{Libro}: \textit{Sustainable experiences that transform communities. Get inspired and impulse your idea!}
  [Versión Original Digital, 2017: Amazon (Por Green Drinks)].
  
  \end{itemize}
  
  
