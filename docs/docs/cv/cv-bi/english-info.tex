
%------------------------------------------------------------
%	 INTERESTS
%------------------------------------------------------------

\section{Short Bio} % \section{Intereses}

I am a sociologist by studies and research + %\& %in process of becoming a 
an aspiring Developer.
% programmer. % at heart.

% 2020: second year student as Systems Analyst.
% 
% I am adaptable, curious and tenacious. 
% I was born in Chicago, USA; and raised from age 7 in Córdoba, Argentina.
% I always kept studying English, raising my level to a C1 [CEFR] standard.
% Also, I use English on a daily basis by reading and writing. 
% I have also learned to read, listen and speak Portuguese in a fluent manner, living in Brazil through nearly 2 years. 

I’ve worked in academic environments: researching, supervising research projects, giving sociology classes, and English conversation. I feel at ease in contexts relating to publishing, film production, programming, and non-profit.

%------------------------------------------------------------
%	 WORK
%------------------------------------------------------------

\section{Recent work}

\begin{twentyshort} 
	\twentyitem{2020--2020}{Customer Dev Backoffice}{[Cintelink]} 
% 	Fuel Business Intelligence
% 	]} % Córdoba, Argentina]}
        {IoT sql python postman} %php, nodejs, vue, docker. 
%         We work in an agile environment. 
%         I use source code versioning and Kanban methodology [jira] on a daily basis.
%         We deliver results in small increments: my first solo project used the Knime platform to manage big data sets with SQL-type queries.} % docker
	\twentyitem{2019--2020}{Back End Web Developer --nodejs}{[Ross Outside the Box]}
        {%Back End | Full Stack [nodejs express mariaDB %| (react angular)
        %] \\ 
%         This was my first experience in a tech/agile environment. 
%         I used source code versioning and 
        %We deliver results in small increments: 
        nodejs/%+ express / 
        authentication. On my 1st solo project I used Knime with %platform to manage 
        big data sets with SQL-type queries.
        testing: e2e + performance.
%         Kanban methodology with jira. %on a daily basis.
        } % docker   
	
	\twentyitemshort{2018--2019}{Research Assistant -- Project \textit{In the Name of Wild} | Phillip Vannini}
\twentyitemshort{2018--2019}{Translations and Reviews: Spanish to English}
	% 	\twentyitemshort{2019}{Translation: How to make sense of precariousness? \\
% 	\textit{Bíos}-precarious and sensitive life [Martín De Mauro]}
% 	\twentyitemshort{2020}{Translation: 23 ideas sobre la juventud [Howard Becker]}
	\twentyitemshort{2015--2018}{English Conversation Classes} %[Córdoba, Argentina]
	\twentyitemshort{2017-2018}{Pearson}  %[Córdoba, Argentina]
\end{twentyshort}

%------------------------------------------------------------
%	 EDUCATION
%------------------------------------------------------------

\section{Education}

\begin{twenty} 
        \twentyitem{2019-2021}{Systems Analyst}
        {[ESCMB, Córdoba, Argentina]}{
        Second year student [First year Completed]}
        
        \twentyitem{2018--2020}{Courses} % Several Courses [Some Complete / Others Ongoing]
        {[Udemy, online]}{bash git sql javascript jquery angular php java node} %laravel symfony
 	
 	\twentyitem{2018--2019}{Web Development Course}{[UTN, Córdoba, Argentina]}
 	{js %Javascript 
 	Jquery Angular ApiRest Mongodb % Node.js Express.js 
 	node express  	%\\ 
 	% HTML5 CSS3 
 	Bootstrap}
 	\twentyitemshort{2012--2014}{Master in Sociology [UNICAMP, S\~{a}o Paulo. Brazil]}
	\twentyitemshort{2002--2009}{Graduate in Sociology [Universidad de Buenos Aires, Argentina]}{}%
 	\twentyitemshort{since $  $ 2009}{Markup-editing languages [Hands-on]}
\end{twenty}


%------------------------------------------------------------
%	 OTHER INFORMATION
%------------------------------------------------------------


\section{Other Projects}

\begin{twenty} % Environment for a list with descriptions
	\twentyitemshort{2018--today}{Writer + Editor: \texttt{greendrinkscba.org}}
	\twentyitemshort{2014--today}{Personal web site: \texttt{ramoneando.com}}
	\twentyitemshort{2016--2019}{\textit{inextricable publisher} [independent book publishing]}
% 	\twentyitemshort{2012--2014}{Master in Sociology [UNICAMP, S\~{a}o Paulo. Brazil]}
% 	\twentyitemshort{2002--2009}{Graduate in Sociology [Universidad de Buenos Aires, Argentina]}{}%
	\twentyitemshort{2009--today}{text and design editing tool: \LaTeX}

% 	\twentyitem%
% 	{since 2017}
% 	{PhD [distance] Sociology}
% 	{[University of Exeter, UK]}
% 	{Distance athletes in urban rhythm | How do ultra-runners run through and out of cities?}
 	%Distance athlete’s æffect on urban rhythm | How do ultra-runners run in automatized cities?
%  	{Self-published book [\textit{exhalations}] and web site [\texttt{ramoneando.com}]}
% 	\twentyitem%
% 	{2014}
% 	{Master in Sociology [UNICAMP, S\~{a}o Paulo. Brazil]}{}
% 	{I researched urban art: graffiti \& pixaç\~{a}o}
\end{twenty}

\section{General Profile} % \section{Intereses}

% I am a sociologist by studies and research + 
% Also, I am in process of becoming a curious programmer at heart.

2020: second-year student as Systems Analyst.

I am adaptable, curious, and tenacious. 
I was born in Chicago, USA, and raised from age 7 in Córdoba, Argentina. 
I always kept studying English, raising my level to a C1 [CEFR] standard.
Also, I use English daily by reading and writing. 
I have also learned to read, listen, and fluently speak Portuguese, living in Brazil for nearly two years.

% I've worked in academic environments: supervising research projects, giving sociology classes and English conversation.
% I feel at ease in environments relating to publishing, film production, programming and service.

\bigskip

% \epigraph{ “Nada de lo que hacemos o decimos se pierde en el vacío: el aire está lleno del pensamiento de todos.” }

\twentyitemshort{ }{> “None of what we do or say is lost into a void: \\ 
the air is full of everyone's thoughts.” \\ 
–  Pedro Bonifacio Palacios}

\twentyitemshort{ }{> “Nada de lo que hacemos o decimos se pierde en el vacío: \\ 
el aire está lleno del pensamiento de todos.” \\ 
– ALMAFUERTE}

% \begin{minipage}[r]{0.8\textwidth}
% %  × lalalal
% \begin{quote}
%  “Nada de lo que hacemos o decimos se pierde en el vacío: el aire está lleno del pensamiento de todos.” 
% – ALMAFUERTE
% \end{quote}
% \end{minipage}

\vfill 

% \section{Translations}
% 
% \begin{twenty} % Environment for a list with descriptions
% % 	\twentyitemshort{2020?}{Off the Grid -- Re-Assembling Domestic Life [Phillip Vannini]}
% 	\twentyitemshort{2019}{Research Assistant -- Project \textit{In the Name of Wild} [Phillip Vannini]}
% 	\twentyitemshort{2019}{23 ideas sobre la juventud [Howard Becker]}
% 	\twentyitemshort{2019}{%
% 	How to make sense of precariousness? \\
% 	\textit{Bíos}-precarious and sensitive life
% % 	Taedium Vitae: Precariety and Affects in Porteña Night 
% 	[Martín De Mauro]}
% % 	\twentyitemshort{2011}{La influencia de las cosmovisiones en el pasado y en el presente según Max Weber [Stephen Kalberg]}
% % 	\twentyitemshort{2005}{Los tipos de racionalidad de Max Weber: piedras angulares para el análisis del proceso de racionalización de la historia [S. Kalberg]}
% 	%\twentyitem{<dates>}{<title>}{<location>}{<description>}
% \end{twenty}
