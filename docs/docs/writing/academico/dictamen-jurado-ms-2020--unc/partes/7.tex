\subsection*{7) ¿Hace una contribución a su campo de estudio específico?
¿Cuál es esa contribución?}\label{hace-una-contribuciuxf3n-a-su-campo-de-estudio-especuxedfico-cuuxe1l-es-esa-contribuciuxf3n}

La originalidad %está en 
es ser un trabajo etnográfico sobre Crossfit
% pionera en el universo geográfico de 
en Santiago del Estero.
% que se aborda.
La contribución es la de establecer la serie de pasos que se requiere para
venir a formar parte de la comunidad específica analizada.

Para darle más solidez a la contribución de la tesis se podría enfocar en cuáles son los temas a los que más se busca contribuir:
sea la carrera, las valoraciones morales, la comunidad, el desafío.
También se podría aclarar a cuál disciplina se enfoca más: a la etnografía, a un concepto central, al universo.

Se cita por ejemplo numerosas veces la tesis de Alejandro Rodríguez. 
% ¿en qué facultad está inscripta su investigación: en ciencias sociales, periodismo, psicología, management?
¿Cuál es el concepto y el diálogo que se busca abrir con aquella investigación y las otras investigaciones citadas?
Dejar claro eso podría darle más fuerza a los propios argumentos de esta tesis.