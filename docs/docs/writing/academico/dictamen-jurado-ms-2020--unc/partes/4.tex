\subsection*{4) ¿El manejo de la teoría y de la bibliografía es
relevante, actualizada y suficiente? Si no lo fuera, indicar las
dificultades.}\label{el-manejo-de-la-teoruxeda-y-de-la-bibliografuxeda-es-relevante-actualizada-y-suficiente-si-no-lo-fuera-indicar-las-dificultades.}

% TEORIA
% \subsection{Teoría}\label{teoruxeda}

Se empieza el trabajo con un marco teórico basado en Becker, Goffman y el trabajo de Alejandro Rodríguez.
Para el concepto de comunidad, se menciona a Rose
y se lo cita en el texto para confirmar que hay comunidad porque hay esfuerzos conjuntos, valores, integración.
Cuestionar los aspectos en los que sí hay o en los que no hay podría darle mayor cuerpo a la descripción.
Tal vez sea una comunidad en un sentido restringido: ¿son amigos? ¿colegas? ¿compañeros de entrenamiento? 
¿algunos se ignoran fuera del box? 

(* Ver, explicar el caso de la ex pareja de Cacho y otros podría dar 
mayor dimensión al concepto en juego. ¿Qué pasó con las mujeres que se mencionan en el texto (Malena, Fanny, Cris), 
y no aparecen en el anexo?)

El texto toma también consideraciones en el marco de Foucault y  de otras teorías
sociales, incluyendo Wacquant y otras investigaciones sobre Crossfit como los de
Dawson, Merino, Crossley, Beckenstein, Coakley, Rose.
Se podría diferenciar qué artículos se toman como datos, y qué se toma como diálogo fértil
a nivel conceptual y metodológico.
Tal vez no todas las combinaciones sean lo más recomendable. 
Hay autores, conceptos, metodologías que no siempre funcionan bien juntos con otros.
% No queda claro cómo se hace esta combinación o si es deseable.
% Tal vez se pueda buscar un autor que combine esos autores.

En el caso de Becker, el explícitamente se aleja de la teoría de
Bourdieu y su escuela (que incluye a Wacquant), esto lo lo hace explícito repetidas veces, un caso resumido es una entrevista,
\textit{A Dialogue on the Ideas of “World” and “Field” }(2006),
donde habla de las categorías de mundo y de campo, como no combinables: señala que el concepto de
\textit{mundo} permite una mayor flexibilidad empírica, mientras que \textit{campo}
asume, en la visión de Becker, un \textit{a priori} de reproducción social a una identidad repetida, sin variabilidad,
de dominación siempre hegemónica, sin quiebres.

En la entrevista (2006) Becker explicita su visión de los conceptos de mundo y campo,
separación general entre su teoría y metodología y los supuestos de Bourdieu y su escuela (acá Wacquant).
Latour titulará esta disputa de manera amplia como entre 
la escuela de la sociología de las asociaciones y la sociología de lo social (¿Le Breton?).
Latour explaya este argumento en su libro \textit{Reensamblar lo social} (2008), donde hace esta crítica en detalle.
Latour va mas lejos en esta crítica donde dice, en su artículo \textit{How to Talk About the Body?} (2004)
que no sirve decir que A explica A, justamente lo que hay lograr es describir
qué es lo que produce qué, esto es, describir A: lo cual abre mas el campo de investigación.
Wendy Bottero y Nick Crossley discutieron las posibles combinaciones entre estas perspectivas
en \textit{Worlds, Fields and Networks: Becker, Bourdieu and the Structures of Social Relations} (2011).

Para Duneier, que discute a Wacquant en \textit{What Kind of Combat Sport Is Sociology?} (2002),
no hay reconciliación posible entre estas distintas metodologías y posicionamientos teóricos.
O mejor dicho, las reconciliaciones son viables si hay un terreno común de discusión.
En el caso que analiza Duneier se busca entender cuáles son las vivencias de los marginados en un barrio que 
analizó décadas antes Jane Jacobs.

En la presente tesis: ¿Cuál es el fondo de discusión conceptual?
¿Hay una atención al concepto de salud más enfocado a lo físico, o a lo lo mental, del bienestar?
¿Es otro el concepto central? ¿Cuál es el diálogo con la literatura del tema?



% \begin{itemize}
% \tightlist
% \item
%   Ir al grano. Y después dar detalle. El texto redunda en introducciones
%   y no da explicaciones ulteriores. Esto pasa varias veces con el
%   concepto de carrera de Becker. ¿Hay bibliografía secundaria?
% \end{itemize}