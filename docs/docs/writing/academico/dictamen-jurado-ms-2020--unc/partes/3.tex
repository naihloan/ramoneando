\subsection*{3) ¿Cuál es el propósito de la tesis? ¿Lo cumple? 
(Se solicita evaluar la formulación del problema, y la adecuación de la metodología y de los ejes de análisis)}\label{cuuxe1l-es-el-propuxf3sito-de-la-tesis-lo-cumple-se-solicita-evaluar-la-formulaciuxf3n-del-problema-y-la-adecuaciuxf3n-de-la-metodologuxeda-y-de-los-ejes-de-anuxe1lisis}

La pregunta del título es:
``¿Cómo un iniciado en el Crossfit se convierte en un Loco?''.

Esta pregunta es respondida por una serie clara de pasos en los que se detalla el desarrollo de los atletas.
Sí convendría delimitar la categoría Loco en lo que significa para los atletas y no absolutizar el término cómo único identificador de legitimación.
Esto es, ¿un entrenado puede ser loco antes de iniciarse en la carrera de Crossfit?
¿Puede un atleta, como pasa en muchos deportistas profesionales, ser un representante serio del box, sin ser loco?
Estas preguntas no buscan negar la fuerza de la idea de loco, sino buscar modos de articulación con la vivencia de los atletas, y para hacer más denso el concepto de loco, así como las legitimaciones en la carrera de los atletas en el box, y las reconfiguraciones en sus vidas individuales.

La pregunta ``¿Cuáles son los pasos que caracterizan a la carrera del loco?'' quedó claramente delimitada en los capítulos de la tesis. Lo mismo se puede decir de la pregunta `` ¿Cómo se consagra un loco al interior de Kratos Hard Cross?''.

Más atención se podría apuntar a responder con descripción y análisis más detallado sobre las otras preguntas:
``¿Cómo es la distribución espacial de este box y de qué manera opera sobre los cuerpos de los miembros?'',
``¿Cómo se construyen relaciones de pertenencia al interior de Kratos Hard Cross?''.
Acá también se podría apoyar más en Goffman, como se propone al inicio, en vez de ir hacia Foucault.
En caso de usar Foucault se puede argumentar de qué manera se articula con Goffman.