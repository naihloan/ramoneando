\end{fullwidth}
\chapter{CONECTADO NUEVAS FORMAS DE HACER Y CREAR}
\label{conectado-nuevas-formas-de-hacer-y-crear}
\begin{fullwidth}

% \begin{figure}[htbp]
% \centering
% \includegraphics[width=1.82500in,height=1.78819in]{media/image1.jpeg}
% \caption{}
% \end{figure}

\textbf{Elga Ruth Velásquez}

Lic. en Administración, Coordinadora de Green Drinks Cba. Fundadora y
Directora de la Consultora Apoidea Soluciones Sustentables.
Representante de Especializaciones - Fondo Verde Internacional para LAC.
Columnista.

\emph{``Sé el cambio que quieres ver en el mundo'' Gandhi}

\textbf{Conectando nuevas formas de hacer y crear}

\textbf{Green Drinks Córdoba}

``Cada mes en diferentes ciudades del planeta se reúnen personas que
trabajan en el campo del desarrollo sostenible, en encuentros
informales, con una mezcla de académicos, ONG, empresarios, gobiernos,
emprendedores, ciudadanos vinculados entre sí, formando una red animada
diversa y auto organizada.

En estos eventos mucha gente encuentra amigos, empleo, nuevas ideas para
mejorar su localidad.''Cuando leí esto en la web internacional de Green
Drinks me apasionó la idea de crear una red donde las perspectivas de
personas que provienen de diferentes ámbitos generen una fuente de
oportunidades para promover y empujar el desarrollo sostenible de la
ciudad de Córdoba. Hoy 4 años más tarde siento con ánimo que Green
Drinks no solo se ha transformado en un espacio de diálogo que brinda
redes para la generación de propuestas a los desafíos que enfrentamos en
el planeta; sino también un canal para difundir lo que se está haciendo
bien a través de este nuevo soporte.Este primer libro es el fruto de más
de 150 disertaciones de los eventos de Green Drinks Córdoba y Jujuy, en
esta ocasión elegimos 16 historias y experiencias contadas por los
propios autores, que esperamos te inspiren a que es posible lograr
cambios que impacten positivamente en las comunidades de las que
formamos parte, tomar decisiones respetuosas con el ambiente, generar
nuevos mecanismos de desarrollo productivo, hacer frente al Cambio
Climático con acciones que involucra a los diferentes grupos de interés
(Académicos, Gobiernos, Empresas, Organizaciones, sociedad civil). Pero
ahora pasaré a contarles en detalle cómo surgió Green Drinks, este
espacio donde vinculamos nuevas formas de hacer y crear soluciones a los
desafíos que enfrentamos.\textbf{Inicios}Antes de que conformara Green
Drinks Cba en 2013, transitaba una etapa de apertura en el desarrollo
profesional, había culminado mis estudios de Administración en la
Facultad de Ciencias Económicas de Córdoba, cómo lo hace la mayoría de
los estudiantes que venimos del interior del país a la Docta, en mi caso
de La Quiaca, Jujuy, la primera ciudad del extremo noroeste del país,
ubicada en las zonas más áridas del norte de Argentina, es una provincia
muy pintoresca, por su rica historia y tradiciones, muchas de las cuales
se remontan a los tiempos precolombinos donde convive el esplendor de un
desierto que se conjuga con el cielo transparente y aire puro que se
puede respirar (si aguantas la altura), se encuentra a 3442 metros sobre
el nivel del mar, desde aquí concebimos el mundo desde una óptica muy
profunda porque están dadas las condiciones de contexto y naturaleza
para escucharse en el silencio del horizonte, entre cardos y desolación,
se agradecen con fiestas los milagros que provee la naturaleza debido a
la escasez.Creo que este lugar en el mundo me ayudó a concebir mi
interés por los problemas sociales, ambientales y económicos que
atravesamos. \textbf{Una decisión} Cuando decidí estudiar Ciencias
Económicas, mis padres se asombraron mucho porque desde niña andaba
coleccionando hojas de plantas y leyendo sobre los problemas
ambientales.Por supuesto suponían que estudiaría Biología, cuando les
comuniqué mi decisión sobre seguir ciencias económicas, les sorprendió
bastante la contraposición, pero me apoyaron y apostaron a darnos a mí y
a mis hermanos una educación universitaria. Bueno y se preguntarán por
qué decidí estudiar administración, gestión de organizaciones, empresas.
Lo que pasa es que quería saber ¿cómo hace el sistema productivo para
decidir sobre la gestión de sus recursos, entre ellos los que provee la
naturaleza? y ¿cómo son los mecanismos para determinar qué y cuánto
producir?, ¿quién decide cómo consumir?, ¿de dónde surgen los precios?,
¿cómo es que, si son personas las que toman las decisiones, no pueden
dimensionar que los recursos son limitados?, ¿por qué todo lo que usamos
va a parar a los vertederos/basurales?, ¿por qué si existen los bancos
aún existe la pobreza?, ¿cómo hacen las sociedades para crear dinero?,
¿qué determina que un país sea emergente y desarrollado? Bueno podría
continuar con las preguntas pero aún, después de finalizar la carrera,
continúan. Creo que lo que aprendí es a darme cuenta que el sistema
económico hizo mal muchas cosas y aun hoy hay empresarios que no
encuentran otra variable como el rédito económico a toda costa para
tomar decisiones empresariales, que en muchos casos impactarán a
millones de personas y al ambiente donde se localizan.Frente a todo,
creo que el sistema productivo y los ejecutivos dentro de las empresas,
ahora tienen la oportunidad de usar todas sus debilidades a favor de los
problemas que provocaron, si son capaces de reconocer su miopía. Hoy
estamos atravesando esa transición en diferentes puntos del planeta, con
una nueva agenda 2030 de Naciones Unidas donde se plantea los 17
objetivos de Desarrollo Sostenible que tendrán que accionar desde los
diferentes sectores. \textbf{Humanizarnos es parte de la
sustentabilidad}Los que apostamos a la sustentabilidad de los entornos
seguimos creyendo en la humanidad, en los que habitamos este planeta y
que aún podemos mirarnos con empatía, los unos a los otros y dialogar
para ponernos de acuerdo en las prioridades que van de la mano de
nuestros valores éticos. Creo que a nadie le gustaría vivir en un
basural, tomar agua contaminada, respirar el aire con la polución que te
cause cáncer, desde el lado social dejar a niños en las calles a la
deriva y que sean tus hijos, trabajar en infra condiciones solo por
encontrarte en un país subdesarrollado y no tener estudios o por no
haber nacido en una familia que te brinde soporte y amor para
desarrollarte como persona y profesional. Podría describir cientos de
situaciones que naturalizamos, naturalizamos lo que nos toca y lo que no
nos toca y nos paralizamos frente a los problemas que atravesamos como
sociedad y no nos involucramos por miedo o por no saber cómo, la otra
opción es pensar que el que ejerce el poder (sea político o económico)
es el único que tiene y sabe dar respuestas. \textbf{Es necesario
hacernos cargo}Luego de un año de haber creado junto a dos socios una
Start Up, una plataforma on line donde un grupo de profesionales de
América Latina escribíamos contenido relacionado con conceptos de
sustentabilidad en general, aunque nuestro enfoque se centraba en dar a
conocer productos y servicios de emprendedores o empresas que producían
teniendo en cuenta el triple impacto (ambiental, económico, social).
Poco a poco las prioridades económicas y de ansiedad para que todo
resulte según lo planificado hicieron que mis socios aborten el
emprendimiento. Fiel al equipo virtual que habíamos formado continué
manteniendo la plataforma durante un año, pese a mi inexperiencia los
vínculos que había generado con un grupo de profesionales que brindaban
sus notas en un lenguaje amigable para los usuarios, me hicieron
aprender más de lo que ocurría con los problemas y las soluciones que
emprendían diferentes países. Coseché muchos contactos a nivel
internacional pero la plataforma no traccionaba y decidí abandonar la
virtualidad para conocer lo que estaba sucediendo a nivel local, en
Córdoba se hablaba poco y nada de Desarrollo Sustentable y no conocía a
ningún grupo que hiciera algo por generar acciones concretas de toma de
conciencia o de participación para influir en la toma de
decisiones.\textbf{Red y sinergia}

Green Drinks es una red global de reuniones informales para hablar de
sustentabilidad en espacios distendidos como bares, estas reuniones se
realizan mensualmente en más de 750 ciudades que se fueron sumando a la
iniciativa desde que se inició allá en 1989, en Londres, contando que en
Argentina ya se realizan los Green Drinks desde el 2007 en Buenos Aires.

Me dispuse a escribirle a Edwin Datschefski uno de los fundadores de la
idea, para luego poder conectarnos y comunicar todos los meses que se
está realizando el evento. Lo que más me agradó es que cada lugar podía
tomar su forma de auto organizarse creando un espacio de diálogo neutral
donde todos son bienvenidos o de que es organizado por voluntarios, sin
necesidad de tener tantos recursos para empezar.

Ya siendo oficialmente la Coordinadora de Green Drinks Córdoba, convoqué
a amigos, conocidos por mail para saber a quién le gustaría sumarse al
equipo para la organización del primer evento.Fortuitamente Leonardo
Peralta en ese momento un conocido, me comentaba que le encantaría
formar parte de esta iniciativa, mi colega y amiga Celia Santamaría con
años de experiencia en el mundo académico quería dar apertura a una
nueva vida en Córdoba y esta iniciativa concordaba con sus valores
humanísticos.

Luego se sumó mi hermana Eliana, que con su arte teatral y creatividad
dio vida a los disparadores artísticos de los proyectos por venir.En el
primer evento esperábamos sin mucha expectativa 20 personas, pero
nuestro asombro fue que, gracias a los medios, vino mucha gente, sobre
pasaron las 100 personas, muchos tuvieron que retirarse. El evento nos
demostró el interés por el tema en Córdoba y de que, pese a que haya
sido el primer Green Drinks sobre la temática de Sustentabilidad y
muchos lo hayan tomado como novedad, un espacio de diálogo serviría para
aunar criterios, vincularnos y generar propuestas que hagan una Córdoba
Sustentable.En estos largos 4 años hemos vivido, hemos visto cómo los
problemas de Ambiente de Córdoba y de Argentina en general crecieron mes
a mes, pero también vimos cómo muchos emprendedores, influyentes,
académicos, políticos, líderes sociales han empezado a hacerse oír y
generando cambios que tarde o temprano regenerará el sistema en forma
integral.\textbf{Nuevos proyectos}Con el pasar de los años este equipo
diverso, con muchas ideas, ha ido creciendo, actualmente somos 30
integrantes, convencidos de que es posible generar cambios desde el
lugar que nos toca y aportar a un mundo más próspero.Así fue que a fines
de 2016 decidimos invitar a referentes que pasaron por los eventos a
escribir dando a conocer su historia de vida y su experiencia sobre el
proyecto que llevan adelante. En febrero de 2017 damos inicio a este
libro donde participan 27 personas entre autores y equipo de soporte,
una construcción colectiva con diferentes percepciones acerca de las
temáticas.Hemos hecho malabares para coordinar toda esta hermosa
experiencia, nos sentimos felices, el esfuerzo ha sido grande, pero
llegamos, nuestra intención es inspirarlos a lograr propuestas,
iniciativas, estemos donde estemos, sea cual fuera la condición, si algo
nos molesta y está mal, no nos queda otra que ser proactivos y emprender
nuevos modelos, ser el ejemplo, con pequeñas acciones diarias que harán
que culturalmente vayamos evolucionando.Paralelamente decidimos dar
inicio a nuestro primer programa de acción, el proyecto de Forestación
urbana que será financiado con este libro. Nos encontramos en una
situación en la que a Córdoba solo le queda el 3\% de su bosque nativo,
esto demuestra el poco interés que tenemos, tanto la sociedad así como
también los tomadores de decisiones, sobre nuestra naturaleza. Y en la
ciudad gozamos de una calidad de aire deprimente y los árboles que puede
contrarrestar la ola de calor y contaminación por la polución se han
visto minados de dejadez por parte del frentista (ciudadano dueño de la
propiedad) y de los marcos legales que poco ayudan. Es tiempo de
ponernos las pilas como sociedad e instituciones para lograr una mejor
calidad de vida en el sentido real de la palabra. Avanzamos también en
una prueba piloto sobre la Gestión de Residuos Domiciliarios ya que la
tasa de recupero y reciclaje es del 1,8\% en la ciudad, decidimos tomar
como eje para el plan la participación ciudadana, la educación y la toma
de conciencia para que se realice la separación diferenciada de
materiales reciclables que se introducirán al circuito productivo
impactando positivamente en el ahorro de recursos, el ambiente y también
en las fuentes laborales de la industria del reciclado. A la vez se
trabajará sobre la importancia de generar compostaje o tratamiento en el
hogar de los residuos orgánicos con potencial de transformarse en tierra
fértil para nuestras plantas del jardín de hogares que habitan en
barrios donde pueden hacerlo, disminuyendo así la generación de sus
residuos.Con los equipos interdisciplinarios que se han conformado para
llevar adelante diferentes proyectos, la premisa es vincularnos
sinergizando, para lograr mejores resultados.

Creemos que las estrategias para regenerar y repensar el sistema están
en manos de cada uno de nosotros, así podremos ser protagonistas de
lograr que la sustentabilidad, la ética y la responsabilidad sean la
regla y no la excepción. ¡Allá vamos!

% \textbf{\\
% }