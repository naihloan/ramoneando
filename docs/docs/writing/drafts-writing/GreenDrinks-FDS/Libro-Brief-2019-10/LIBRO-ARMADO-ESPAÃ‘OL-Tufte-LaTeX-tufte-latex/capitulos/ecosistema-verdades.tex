  \end{fullwidth}
   

\chapter{ECOSISTEMAS Y SOCIEDAD: LA DIVERSIDAD DE LAS ``VERDADES''} \label{ecosistemas-y-sociedad-la-diversidad-de-las-verdades}
  \begin{fullwidth}

% \begin{figure}[htbp]
% \centering
% \includegraphics[width=1.74306in,height=1.72639in]{media/image9.jpeg}
% \caption{}
% \end{figure}

\textbf{Marcos S. Karlin}

Ingeniero Agrónomo y Doctor en Ciencias Agropecuarias. Docente en la
Facultad de Ciencias Agropecuarias (UNC). Miembro co-fundador de la
Asociación Civil El Cuenco. Actualmente trabaja en proyectos de manejo y
conservación de cuencas en la Reserva Natural de la Defensa La Calera y
en Chancaní.

\emph{``\ldots{}la ``verdad'' constituida es completamente subjetiva y
{[}\ldots{}{]} tiene que enmarcarse en el contexto de aplicación.''}

Ecosistemas y sociedad: la diversidad de las ``verdades''

\textbf{Introducción}

El debate sobre el uso y conservación de los ecosistemas se ha instalado
para quedarse. Cada vez es mayor el interés sobre el destino de las
áreas naturales, pero también son cada vez más extremas las opiniones.
¿Uso \emph{o} conservación? ¿Uso \emph{y} conservación? ¿Es posible que
estos polos opuestos puedan acercarse para el consenso?

El futuro de nuestras áreas naturales depende de acertadas políticas que
consideren el amplio abanico de información existente sobre su
funcionamiento, herramientas de conservación y posibilidades de uso
sustentable. Lamentablemente las políticas suelen estar teñidas de
intereses económicos y egos personales.

El conocimiento sobre el funcionamiento de los ecosistemas que se
utiliza para definir las políticas se basa casi exclusivamente en el
conocimiento científico formal, el cual se toma como una ``verdad''
establecida e irrefutable. Aunque a veces la ``verdad'' se construye con
información incompleta, o bien se simplifica la enorme complejidad de
los ecosistemas.

En este capítulo, mi intención es discutir acerca de esta ``verdad'',
quien la construye, cómo se interpreta en la sociedad. Además, quisiera
plantear alternativas en su construcción para contar con herramientas
más justas y democráticas para la definición de políticas destinadas a
favorecer el uso y conservación de los ecosistemas y sus recursos
naturales.

Es momento de evaluar la diversidad y la complejidad, no sólo biológica,
sino también social y cultural en relación a nuestro ambiente. Debemos
saber que cuando hablamos de bosques nativos o cualquier otro
ecosistema, hay por detrás comunidades, familias e individuos que viven
de ellos, cada uno con su impronta y realidad particular.

Vayan algunas reflexiones. Algunas apuntan a proponer soluciones; otras
simplemente a ampliar el panorama sobre la problemática.

\textbf{¿Podemos confiar en la verdad?}

Cuando uno estudia una carrera, al alumno se le hace creer que todos los
principios, técnicas y metodologías aprendidas en la academia
constituyen una verdad inmaculada que, hasta tanto la ciencia no genere
un conocimiento superador, no existen motivos por los cuales se deba
dudar de los presentes conocimientos académicos.

Gracias a este precepto he tenido la oportunidad de meter la pata
innumerable cantidad de veces a lo largo de mi vida profesional. El
primer recuerdo del que cuento, inmediatamente después de recibido de
ingeniero agrónomo, es el de haber participado en el dictado de talleres
de capacitación en manejo y producción múltiple de bosques nativos en la
Provincia del Chaco, tanto para pequeños productores como para
comunidades aborígenes.

En esta serie de capacitaciones, de las cuales participé como
colaborador en un proyecto de desarrollo rural coordinado por la Red
Agroforestal Chaco, conocí por primera vez la cruda realidad, diferente
a la que nos pintan cuando estudiamos agroecosistemas ideales. Me
propusieron en ese entonces dictar parte de un taller sobre monitoreo de
calidad de suelos, puesto que yo era ayudante alumno de la Cátedra de
Edafología (disciplina que estudia la ciencia del suelo).

Con todas las ínfulas del recién recibido explicaba a un grupo bastante
numeroso de una comunidad Wichi cómo se procedía para hacer un muestreo
de suelos. Tomé un instrumento usado para extraer un volumen exacto de
suelo y en cuanto me apresté a efectuar la tarea, el cacique me hizo una
pregunta. ``\emph{Y con lo que usted va a hacer, ¿no está sacando la
energía de la tierra?}''. Recuerdo que en el primer momento no entendí
la pregunta y, muy seguro de mí mismo contesté algo así como que
``\ldots{}\emph{no, usted se refiere a la erosión del suelo. Pero para
que haya erosión hace falta mucha agua o viento para que se lleve gran
parte de los horizontes superficiales del suelo}\ldots{}''. Cuando vi al
cacique, el asintió no muy convencido, y comprendí que lo que le dije
estaba completamente fuera de contexto, no sólo porque le respondí algo
que no contestaba lo que me había preguntado, sino que además le había
contestado con términos técnicos que posiblemente jamás había escuchado.
Mis tutores, que estaban al fondo, habían empalidecido ante semejante
situación.

Afortunadamente (¿?), el taller prosiguió, y eventualmente los
participantes (Wichis y técnicos) comprendimos que las interacciones,
recíprocas, por cierto, servían para que todos tuviéramos noción de la
``otredad'' cultural que se nos presentaba ante nuestras narices.

Lo cierto es que este tipo de situaciones nos ponen a prueba para
evaluar en algún momento de nuestras vidas (a veces antes, a veces
después) que la ``verdad'' constituida es completamente subjetiva y que
tiene que enmarcarse en el contexto de aplicación.

\textbf{Es ganar plata, estúpido}

Provengo de una disciplina, la agronomía, que se enfoca generalmente en
lo productivista.

Recuerdo que, en las clases de Producción de Carne, en los últimos años
de la carrera, nos preguntaban *``¿cuál es el objetivo del productor
ganadero?*''. Todos respondíamos ``\emph{obtener un ternero por vaca por
año}''. ``\emph{Noooo. ¡Es ganar plata!}''. Y de esta forma, alumnos que
militaban en movimientos de izquierda ponían el grito en el cielo.

Discutíamos sutilezas tales como que hacer si dentro de un corral de
\emph{feed lot} teníamos un árbol que ``nos ocupaba lugar''. Nuestros
profesores analizaban la situación mediante un simple cálculo económico:
dejar el árbol ofrecería a algunos animales sombra, reduciría algo de
\emph{stress} y el animal podría ganar algunos gramos extra de peso,
pero si sacamos el árbol podemos meter un animal más y la ganancia
monetaria es bastante mayor. Además, eventualmente, semejante cantidad
de animales produciría hectolitros de orina que serían descargados al
piso del corral y seguramente terminarían secando el árbol. De nuevo,
algunos de mis compañeros enfurecidos, trataban de justificar que los
profesores estaban completamente equivocados.

Suelo comparar esta anécdota con una versión re-adaptada de un cuento de
Luis Landriscina, titulado ``¿Para qué?''.

¿Cuál es el objetivo de un pequeño productor, de un campesino, de un
aborigen, de un productor familiar o de un empresario del agro? Cada
persona, familia, comunidad o empresa tendrá su objetivo. Lo importante
es saber identificar dichos objetivos si se va a intervenir en procesos
de seguimiento técnico, capacitación o asesoramiento.

Si su actividad económica (de capitalización o de subsistencia) está
vinculada al aprovechamiento de recursos naturales, digamos, por
ejemplo, de un bosque nativo, distintas serán las miradas ahora y a
futuro sobre dicho ecosistema.

\textbf{¿Dónde ubicamos la naturaleza?}

Distintas cosmovisiones en nuestro diverso mundo ponen a la naturaleza
en diferentes posiciones dentro de la subjetividad del hombre.

Hay culturas entendidas como ``primitivas'' que sitúan la vida social
dentro del contexto de la naturaleza, a la par de una población de
plantas o de animales. Estas culturas ``{[}\ldots{}{]} \emph{jamás
pensaron en que las fronteras de la humanidad se detuvieran a las
puertas de la especie humana} {[}\ldots{}{]}''\footnote{Descola, P.
  2012. Más allá de naturaleza y cultura. Amorrortu Editores, Buenos
  Aires (p.~21).}.

Movimientos modernos tales como los de la Ecología Profunda extraen
fragmentos de estas culturas y otras filosofías orientales, considerando
la humanidad como parte del entorno de la naturaleza y propone profundos
cambios políticos, sociales y económicos.

Hay, pues, otros enfoques que enfrentan el paradigma actual y dominante
del capitalismo como doctrina ineludible. Sin embargo, considero que
nuestra sociedad actual está infinitamente lejos de igualarse con el
resto de la naturaleza, le falta madurez. Ni siquiera logra reconocer la
igualdad entre sus pares.

Es justamente el capitalismo, el que con el iluminismo sitúa a la
naturaleza a un costado del hombre, y es este quien tiene la autoridad
moral de dominarla a su antojo.

Entonces, bajo el sistema dominante actual, ¿qué responsabilidad
tenemos, como seres ``moralmente superiores'', sobre el uso de los
recursos naturales? El capitalismo, como doctrina dominante de
occidente, ¿maneja la verdad absoluta sobre el conocimiento de los
valores y la técnica?

Difícil es contestar estas preguntas, contaminados por un sistema que ha
formado nuestros valores y nos ha enseñado a través de la educación
formal que el hombre tiene la potestad de dominar la naturaleza a su
antojo aplicando una técnica avalada científica y socialmente (o al
menos por sectores del ámbito científico y de la sociedad).

\textbf{Blanco y negro}

Un ejemplo de dónde ubicamos la naturaleza, es la forma de aplicación e
interpretación de la Ley Nacional Nº 26.331, ``Presupuestos Mínimos de
Protección Ambiental de los Bosques Nativos''.

Las provincias argentinas deben renovar, según la Ley Nacional 26.331,
sus ordenamientos territoriales para los bosques nativos cada cinco
años, a fin de poder contar con el beneficio de la asignación de los
fondos del presupuesto mínimo destinado al manejo y uso de sus recursos
naturales.

Prácticamente dos banderas flamean en relación a opiniones y críticas a
los ordenamientos territoriales, la conservación, manejo y/o uso de los
bosques nativos. Sin querer ser reduccionista, estas dos banderas son la
productivista y la ambientalista.

Una, la primera, manifiesta su deseo de que áreas de bosque de alto
valor de conservación puedan ser aprovechadas desde el punto de vista
productivo. Otra, la segunda, aboga por aumentar la proporción de áreas
intangibles de forma de reducir el impacto del hombre sobre estos
sistemas.

La primera postura mantiene la idea de que existe un mundo hambriento de
recursos, de que la población mundial aumenta y, por lo tanto, debe
liberarse espacio ``improductivo'' para producir y para desarrollar
áreas de habitación que son demandadas por la creciente población. La
segunda posición manifiesta su preocupación por el avance en los cambios
de uso del suelo, perdiendo estos la capacidad reguladora y reduciendo
su capacidad de ofrecer servicios ambientales.

Por supuesto, estas dos posturas corresponden a grupos integrados por
personas que detentan diversos capitales, el productivista con capital
económico y político, el ambientalista con capital científico y social,
haciendo posiblemente una generalización muy burda (pero los seres
humanos tendemos a generalizar y simplificar las cosas para poner
énfasis a algo y para que nos entiendan\ldots{}). En teoría, aquel grupo
cuya sumatoria de capitales sea mayor, ganará.

¿Pero qué ocurre con aquellos grupos o individuos que no pertenecen ni a
uno ni a otro grupo?

\textbf{Rojo}

Gran parte de nuestras áreas boscosas se encuentran bajo el régimen de
protección de la ley nacional de Protección de Bosques Nativos, Nº
26.331 y las leyes provinciales derivadas de ella.

Las Sierras de Córdoba, pintadas casi en su totalidad en rojo, son áreas
de alto valor de conservación, según esta categoría. Estos sectores,
según la ley (Artículo 9) son áreas de ``\emph{muy alto valor de
conservación que no deben transformarse. Incluirá áreas que, por sus
ubicaciones relativas a reservas, su valor de conectividad, la presencia
de valores biológicos sobresalientes y/o la protección de cuencas que
ejercen, ameritan su persistencia como bosque a perpetuidad, aunque
estos sectores puedan ser hábitat de comunidades indígenas y ser objeto
de investigación científica}''.

Según su reglamentación\footnote{Decreto Reglamentario Ley Nº26.331,
  91/2009.} ``\emph{en la Categoría I} (pintado de rojo)\emph{, que dado
su valor de conservación no puede estar sujeta a aprovechamiento
forestal, podrán realizarse actividades de protección, mantenimiento,
recolección y otras que no alteren los atributos intrínsecos, incluyendo
la apreciación turística respetuosa, las cuales deberán desarrollarse a
través de Planes de Conservación. También podrá ser objeto de programas
de restauración ecológica ante alteraciones y/o disturbios antrópicos o
naturales}''.

Leyendo atentamente estos pasajes de la ley su reglamentación, y
observando el mapa de Ordenamiento Territorial de Bosques Nativos (OTBN)
en Córdoba, es comprensible escuchar con preocupación a pequeños
productores preguntando: ``\emph{¿Y de qué vamos a vivir?}''. ¿Por qué
menciono esto? El vigente mapa de OTBN parece la camiseta de River
Plate: blanco con una franja roja en el sector serrano, oeste y norte de
la provincia. Prácticamente no hay amarillo o verde que les permita a
los productores presentar planes de manejo sostenible sobre áreas
superiores a 10 has.

¿Y la ganadería bajo monte en zona roja, es posible? Algunos podrán
decir que el pastoreo no altera al bosque\ldots{} Pero sí lo hace si
este no se maneja con sumo cuidado. El ganado altera el sotobosque, la
renovabilidad y el suelo del bosque, poniendo en riesgo las condiciones
mínimas detalladas en el Artículo 16 de la ley 26.331: persistencia,
producción sostenida y mantenimiento de los servicios ambientales que
dichos bosques nativos prestan a la sociedad. En Córdoba, hasta el
momento, las autoridades de aplicación de la ley toleran actividades
ganaderas de subsistencia. Pero, ¿qué sucederá en el futuro?

\textbf{Los grises}

\protect\hypertarget{_gjdgxs}{}{}Es innumerable la cantidad de personas
y familias que desean tanto producir como conservar los recursos
naturales en la medida de sus posibilidades, para poder reproducirse
socialmente\footnote{Reproducción social: conjunto de procesos
  biológicos, sociales, económicos, políticos, etc. que permiten la
  persistencia de un grupo social (familia, comunidad, población, etc.)
  en el tiempo.} y poder dejar un capital natural lo más productivo y
conservado posible para sus hijos y nietos.

En otras provincias donde he tenido la oportunidad de trabajar con
comunidades de pequeños productores y campesinos, como en Salinas
Grandes, sur de Catamarca, otros eran los problemas en relación a las
leyes de protección ambiental.

Es muy común que estas comunidades carezcan de títulos de sus tierras, y
esto, en nuestro país que tiene sumo respeto por los títulos que
acreditan propiedad privada, es un grave problema.

Familias que han heredado sus tierras de antepasados que han trabajado
el campo, cortado postes para hacer pozos balde que todavía existen, que
han sido enterrados hace décadas en sus propias tierras; tienen
amenazado su futuro en esas tierras, no sólo porque extranjeros han
adquirido dichas tierras mediante la adquisición de títulos apócrifos,
sino también porque las autoridades provinciales no les permiten hacer
uso de los recursos presentes en sus tierras. Y ni que hablar de
presentar planes de manejo de dichos recursos cuando no pueden
justificar con un título el dominio de sus tierras.

Recuerdo las acciones legales que pesaban sobre los productores por
haber extraído postes mediante poda de árboles (sabían que la extracción
total estaba prohibida) para reforzar un alambrado que servía como
evidencia de posesión de sus tierras, a recomendación de los abogados
que seguían el caso.

Ni que hablar de la situación en que quedaron aquellas familias que
vivían del recurso forestal para la producción de carbón como actividad
de subsistencia, que al momento de la promulgación de la ley de bosques
debieron aprender otras actividades o migrar a la ciudad en busca de
trabajos o subsidios. Todo esto porque las leyes generalizan.

\textbf{¿Cómo encarar proyectos de desarrollo rural? }

Es muy común escuchar con mucho entusiasmo a jóvenes (y no tan jóvenes)
que la mejor forma de comprender las vicisitudes del campesino es
acompañarlos en los procesos de desarrollo rural como uno más de la
comunidad. Esta visión un tanto romántica me ha traído algunos problemas
a la hora de discutir con técnicos que uno suponía compartían objetivos
comunes en pos de la mejora de las condiciones estructurales de la gente
de campo de menores recursos.

La crítica a esta posición pasa por comparar los diferentes capitales,
social, económico, cultural, etc., de ellos (campesinos, productores
rurales, pueblos originarios o cualquiera de las variantes de los
actores del sector rural) y nosotros (técnicos, científicos, urbanitas,
etc.).

La historia y el bagaje cultural de una y otra parte nos ubican en
posiciones muy diferentes dentro del campo social\footnote{Uno puede
  imaginar un campo social como una fiesta de fin de año de una empresa,
  donde cada persona tiende a juntarse a charlar más afínmente con sus
  compañeros más cercanos; por ejemplo, si uno fuera operario de
  maquinarias es más probable que encuentre más tópicos de conversación
  con sus pares que con los miembros de la junta directiva, aunque esto
  no evita poder tener una charla semi-formal con el gerente.}. Muchos
campesinos y pueblos originarios han co-evolucionado con su ambiente y
lo comprenden de maneras muy particulares e interesantes; es decir,
manejan su propia ``verdad''. Este es un conocimiento íntimo y debe ser
respetado.

La experiencia adquirida por mis maestros, que en algún momento yo
también hice propia, apunta a que la mejor manera de mejorar las
condiciones de las comunidades que hacen uso de diversos ecosistemas,
parte por hacer el mejor diagnóstico de situación posible en determinado
momento de irrupción en dicha comunidad.

Sin conocer de dónde se parte o sin saber efectivamente la o las
problemáticas más acuciantes, es muy difícil poder accionar. Por
supuesto, ese diagnóstico debe basarse no sólo en las manifestaciones de
los actores locales, sino también en la observación y en la
triangulación de información.

En este sentido recuerdo haber llegado al campo de una familia y lo
primero que nos dijeron a unos colegas y a mí, una vez que nos
identificaron como los asistencialistas de turno, fue que necesitaban
una casa nueva. Sabíamos de organizaciones gubernamentales que estaban
repartiendo casas prefabricadas por la zona\ldots{} y una de ellas se
encontraba precisamente en ese lugar. Luego de explicarles que nosotros
sólo ofrecíamos capacitación y acompañamiento técnico, nos invitaron a
retirarnos muy amablemente.

Aprendimos de esta forma que hay una máxima que se repite donde vayamos:
la probabilidad de éxito de un proyecto de desarrollo rural es
inversamente proporcional al número de organizaciones interviniendo en
una comunidad.

Los diagnósticos participativos suelen ser herramientas muy interesantes
para el armado de proyectos. Nadie mejor que los usuarios de los
recursos para evaluar el sistema y sus usos. Esta participación empodera
al futuro beneficiario en los procesos y permite que adquiera confianza
a la hora de seguirlos una vez que los técnicos, que deben facilitar,
acompañar, se retiren del proyecto y del terreno. La retirada de los
técnicos siempre genera angustia, pero es necesario concientizarse y
concientizarlos de que el acompañamiento no puede darse eternamente.

Otro punto muy importante a tener en cuenta en los procesos de
desarrollo rural es la posibilidad de realizar investigación situada
adaptativa y participativa.

\textbf{Conocimiento situado}

Hablábamos algunas páginas atrás sobre la subjetividad de la ``verdad''.
¿Cómo se construye mejor la verdad? ¿Mediante un experimento de
laboratorio, a 400 km de distancia del sistema de estudio, simplificado
a uno o dos componentes; mediante un experimento \emph{in situ},
teniendo en cuenta la amplia variabilidad que ofrece el sistema por la
interacción de todos sus componentes, aunque posiblemente con resultados
tremendamente variables? ¿O mediante experiencias efectuadas por
productores, transmitidas de generación en generación por procesos de
prueba y error entre pobladores rurales? Probablemente la última se
acerque bastante, aunque la ``verdad'' se construye mejor si se acoplan
los resultados de todas las fuentes de verificación. La experimentación
participativa permite al poblador local ver con sus propios ojos los
resultados.

Si se va a considerar el conocimiento tradicional como fuente importante
de información, se debe considerar que quizá dicha información debe
codificarse de manera correcta. Además, la información que quiere
extraerse se debe hacer a través de las preguntas adecuadas y de la
manera correcta.

En zonas áridas, donde más tiempo he destinado mis esfuerzos de
investigación y extensión, los aspectos climatológicos son sumamente
importantes. Muchos miden las condiciones climatológicas de forma
indirecta, y lo miden en el contexto en el que viven. Generalmente,
cuando preguntamos si llovió mucho en una temporada, la respuesta suele
ser dada con la cantidad de cabritos producidos. Si hubo muchos
cabritos, esto se debió a que la majada se mantuvo sana, los porcentajes
de parición fueron elevados. ¿Y a qué se debe que la majada esté sana?
No hay mejor medicina que la disponibilidad de forraje en cantidad y
calidad en el monte. Y esta disponibilidad de forraje depende de la
cantidad de agua efectiva caída. Y hago hincapié en ``efectiva'', porque
esta situación no indica que haya llovido más, sino que el monte se
encuentra en mejores condiciones que otros años y esto puede deberse a
una mejora en el manejo del bosque, o a una seguidilla de años con
mejores precipitaciones (no sólo uno), o mejora en la cobertura y
contenido de materia orgánica del suelo. Es destacable, entonces, cómo
una simple pregunta puede pintar la situación de todo un sistema
productivo y natural.

La ``verdad'', en definitiva, es mejor construirla con la
``subjetividad'' local, aprovechando la experiencia situada de la
población; sumando la ``objetividad'' científica obtenida mediante los
métodos validados y aplicados por la sociedad ``moderna'',
``occidental''. Sin embargo, esta ``verdad'' construida, debe ser aún
validada en el contexto de aplicación.

Este conocimiento híbrido, que se obtiene de forma más lenta, pero que
tiene a mi criterio mayor validez epistemológica, puede constituir el
insumo fundamental para una discusión más centrada y democrática de una
nueva ley de bosques en Córdoba y, por qué no, en el resto de las
provincias.

\textbf{El futuro de los ecosistemas}

Muy difícil es hacer futurología acerca de las áreas naturales y su
gente. Pero lamentablemente el panorama no es muy alentador.

Si bien, cuando uno conduce por los caminos de nuestro oeste cordobés
puede ver (y esto es simplemente una impresión personal) algunos signos
de recuperación de áreas boscosas (más pasto, árboles más vigorosos,
menor área descubierta del suelo, etc.). Probablemente esto se deba a
dos causas dominantes: un ciclo climático más lluvioso\footnote{Puede
  observarse las tendencias en Karlin, M. S. 2012. Cambios temporales
  del clima en la subregión del Chaco Árido. Multequina 21(1): 3-16
  (\href{http://www.scielo.org.ar/pdf/multeq/v21n1/v21n1a01.pdf}{\emph{http://www.scielo.org.ar/pdf/multeq/v21n1/v21n1a01.pdf}});
  en Karlin, M. S. 2013. Cambio climático en zonas semiáridas: El caso
  Chaco Árido. Editorial Académica Española
  (\href{https://www.researchgate.net/publication/259389853_Cambio_climatico_en_zonas_semiaridas}{\emph{https://www.researchgate.net/publication/259389853\_Cambio\_climatico\_en\_zonas\_semiaridas}});
  o en Karlin, M. S. 2013. Desiertos y climas: Historias de civilización
  y barbarie.
  (\href{http://books.googleusercontent.com/books/content?req=AKW5QafV9c3ichUs8R9_EA7dOvxkmUWsyacvgfB0JTjatnsjjtTRAjLtrT0eqARSSKt3EuBPNyEIkxip5Gw2DL1aqNsyhDNMT2V9TdP-f1GasABzoTRn-TZy6zk2TBseCTwtjRnoITNPsnMd8Q39JoRMQee2MGP0D17-Hw8WdsfNP0bvI3OLSYCCvL0gWI8hUlktj-G4oC75JEmXNXBDeNiAEuppRJn2DPXuU1qpSg42cG4ytUJmcDNlZr9OhvaszR7tUpvJ6XTG7WI28TOxsG4joYjRq8o3M3wYu2XOBtDMnuWOLTyaTPs}{\emph{http://books.googleusercontent.com/books/content?req=AKW5QafV9c3ichUs8R9\_EA7dOvxkmUWsyacvgfB0JTjatnsjjtTRAjLtrT0eqARSSKt3EuBPNyEIkxip5Gw2DL1aqNsyhDNMT2V9TdP-f1GasABzoTRn-TZy6zk2TBseCTwtjRnoITNPsnMd8Q39JoRMQee2MGP0D17-Hw8WdsfNP0bvI3OLSYCCvL0gWI8hUlktj-G4oC75JEmXNXBDeNiAEuppRJn2DPXuU1qpSg42cG4ytUJmcDNlZr9OhvaszR7tUpvJ6XTG7WI28TOxsG4joYjRq8o3M3wYu2XOBtDMnuWOLTyaTPs}}).}
y el abandono de los campos por migración a centros urbanizados.

La migración se debió y se debe a la falta de interés de los jóvenes por
el campo, en gran medida motivado por la aplicación desmedida y sin
acompañamiento de subsidios. Los adultos cada vez tienen menos
posibilidades de trabajar el campo y terminan generalmente viviendo de
sus magras jubilaciones y vendiendo ocasionalmente algún animal
remanente en su campo ya casi sin uso ni manejo.

Esta es la puerta de entrada a grandes capitales que pueden llegar a
tener acceso a estos campos para cosecha de vacas o diferimientos
impositivos sobre amplias extensiones.

Mientras tanto muchos campos, ya sea en las Sierras o en los Llanos,
deben reducir sus cargas animales con motivo de las restricciones
legales y la falta de acompañamiento técnico. El aumento de
precipitaciones aumenta la cantidad de biomasa acumulada. Esto es
peligroso, ya que aumenta drásticamente el riesgo de incendios.

Es potestad de nuestros gobiernos realizar entonces diagnósticos de
situación para redefinir las políticas para un mejor manejo y
conservación de nuestros recursos y para restablecer derechos y
obligaciones de nuestra sociedad.

Debe fomentarse y recuperarse el capital cultural de los pueblos
campesinos y pueblos originarios, además de una cultura de trabajo
perdida en las últimas décadas. Y posiblemente una buena estrategia sea
comenzar con los niños y adolescentes, revalorizando las costumbres
propias de cada región, pero asimismo, sin aislarlos en un mundo
completamente globalizado.

En definitiva, construyendo una ``verdad'' que aproveche mucha mayor
diversidad de visiones.

% \textbf{\\
% }