\chapter*{Materials and methods}
%\chapter{Materiales y métodos de investigación}

This work looks to a wider spectrum of escape mechanisms from the inertia of
the productive system, a production of machinic uselessness, where cars and
public transportation \textit{circulate, overpopulate and congest}, all of which control
movement in favor of an economic and political order and power.

%\begin{quote} (Deleuze y Guattari, 2010: 527). %\end{quote}

Ultrarunners are the case study. However, the implications of the topic
go beyond this social world. A good number of ultrarunners are so aware of
the need of natural food that they are more inclined to eating more fruits and
vegetables than average people and many of them become long term vegetarians.
This is related, but differently, to massive consumption patterns: of avoiding
the dangers of canned foods, with preservatives, refined sugars, processed flour;
and a political view that requires that the land  be distributed and
cultivated to feed people and not animals.
%(Arrieta, 2013). 

The proposal is of qualitative research. On one side, it will nurture from
secondary material in texts and videos made by and about the participants of
ultramarathon races. On the other side, first hand material will be collected
 from fieldwork. As study material, the general training method shall be
reviewed and at least, the ethnography of one specific competition. These
elements pursue to give new life to the ideas of how people move, beyond
a mere transportation function, and how urban spaces can be circulated,
expanding their uses.

%\clearpage
\section*{2. Auto-ethnography}

The plan of work proposed here sets axis on which to develop future ideas, these axis being: 
affect,
body, 
and materiality.
These \textit{sensitizing concepts} (rather than restrictive prescriptions) shall be guiding points to suggest directions where to look at, as germs of analysis on how and where to collect information. Data finding also relies on the researcher's agenda: "What sorts of patterns one is looking for depends, of course, on research focus and theoretical orientation". Benefits of in-field immersion include not only direct access in general but additionally to non-structured conversations in which "[unusual participant terms] may stress theoretically important or interesting phenomena". In the same vein, concepts may also be, alternatively, "observer-identified"%
\footnote{HAMMERSLEY and ATKINSON. \textit{Ethnography: principles in practice}. 3rd ed. London; New York, NY: Routledge, 2007. P. 164 ("Sensitizing concepts" is Blumer's), 163.}.

The axial concepts are not %be used as fixed tautologies 
to give a taken-for-granted understanding of behaviors. The approach here is first \textit{exploratory}, rather  than explanatory. The deeper understanding of behaviors and use of tools, resources and knowledge %in general/
on the whole, %shall be developed later 
shall come later, during research. The intention is first to gather data, concepts, and a series of insights from in-field work.

Ultra-running has a certain tension in the way it connects participants with people from the outside social worlds.

\begin{itemize}
 \item On one side, it is an ultimately public activity, runners are exposed to permanent contact with other runners (and non-runners as well) in the open, and races depend on a wide number  of actors, both participating and non-race related: in sum, a very wide orchestrated and coordinated social activity.
 \item On the other side, ultra-running entails a certain \textit{Loneliness of the long distance runner}%
 \footnote{Short story by Alan Sillitoe, published in 1959.}. 
 Running ultra distances may well be one of  the most \textit{outdoor} activities or sports. It involves several hours, even days sometimes "out in the  open", amongst  almost untouched nature and wild green spaces afar from a city in cross-country races. And in training season, even in city context: the silent early night-to-dawn moment (from 4 to 6 am) is when nearly no ordinary person is going about, and birds have not even began to chirp. As well as with lone spaces, running involves far many solitary moments in which runners get to collect themselves and revolve in their thoughts, the bareness of the surroundings, and at many flowing times:  
 think of nothing and seize the moment.  
 %not think in anything and be in the moment.
\end{itemize}

The \textit{in situ} work is intended to grasp these two areas (intimate-personal; and social-network-dependent) in ultra-running: the first, during training; and the second, during specific ultra-running events.

\begin{enumerate}
 \item The first aspect, training, is to be dealt with  through auto-ethnography, not as a biographical account, but as a means to grasp the main topics developed. Many of the available material on ultra-running in text and video documentary depict narratives from the sole perspective of runners themselves, in first person, and how they prepare for their practices with different styles of running and post practice cool downs and stretching as well as general nutrition and resting time. The researcher may well take a similar approach without being an outsider of common practice in this social world.
 
 \begin{quote}
  Gertrude Kurath (1960) recommended ethnographers to "learn the movements" and Adrienne Kaeppler (1978) proposed that ethnographers learn certain movements and  receive instructions on  what is done "incorrectly", or "differently" with a methodology that would allow for better understanding.
  %to understand better. 
  [José Bizerril has argued that the practical formation of the researcher has its advantages.] This knowledge allows  access to aspects of the research topic that otherwise would pass unnoticed if only done with a distant approach based on observation and interview. [the experiential dimension makes it possible to gain entry to the experience and] "to the psycho-physical and -why not-, to the spiritual states that that this experience triggers%
  \footnote{ASCHIERI, Patricia. "Hacia una etnografía encarnada: La corporalidad del etnógrafo/a como dato en la investigación". X RAM- Reunión de Antropología del Mercosur. Córdoba, Argentina, 2013. P. 16. My translation.}.
 \end{quote}
 
 Of course,  auto-ethnography may work with a potential source for bias, but at the same time provides both the most inner side view possible, and reveals the speaker's interests, perspectives and preconceptions; to which one can always add contrast with other references to compare and find the most reliable common ground%
 \footnote{HAMMERSLEY and ATKINSON. \textit{Ethnography: principles in practice}. 3rd ed. London; New York, NY: Routledge, 2007. P.%164, %("Sensitizing concepts" is Blumer's), 
 124.}.
 
 \item On the second aspect, on racing events, there is very little material in academic research on events from a qualitative approach. There is scarce material, and when so, only done through surveys or measurement based. Hence, the importance to move forward. Some of the key features of an \textit{ethnographic approach} are taken into account in the present proposal: to prioritize the insider perspective highlighting the experiential, an active immersion in the field during a reasonable amount of time, minimal interference to gather data to be triangulated%
 \footnote{HOLLOWAY, Imma; BROWN, Lorraine; and SHIPWAY, Richard. "Meaning not measurement: Using ethnography to bring a deeper understanding to the participant experience of festivals and events". \textit{International Journal of Event and Festival Management}. Vol. 1 Nº 1, 2010. Pp. 75-76.}.
 And not to focus on measuring variables, but rather on \textit{collecting and constructing new variables} to build up ever more complex concepts: this adds nuance to the understanding of the phenomenon, and provides material to suggest new questions and aspects to be worked on%
 \footnote{BECKER, Howard S. \textit{What About Mozart? What About Murder? Reasoning From Cases}. The University of Chicago Press, Chicago, 2014. Pp. 13-14, 18.}.

\end{enumerate}



\clearpage
\section*{2. Auto-ethnography}

The plan of work proposed here sets axis on which to develop future ideas, these axis being: 
affect,
body, 
and materiality.
These \textit{sensitizing concepts} (rather than restrictive prescriptions) shall be guiding points to suggest directions where to look at, as germs of analysis on how and where to collect information. Data finding also relies on the researcher's agenda: "What sorts of patterns one is looking for depends, of course, on research focus and theoretical orientation". Benefits of in-field immersion include not only direct access in general but additionally to non-structured conversations in which "[unusual participant terms] may stress theoretically important or interesting phenomena". In the same vein, concepts may also be, alternatively, "observer-identified"%
\footnote{HAMMERSLEY and ATKINSON. \textit{Ethnography: principles in practice}. 3rd ed. London; New York, NY: Routledge, 2007. P. 164 ("Sensitizing concepts" is Blumer's), 163.}.

The axial concepts are not %be used as fixed tautologies 
to give a taken-for-granted understanding of behaviors. The approach here is first \textit{exploratory}, rather  than explanatory. The deeper understanding of behaviors and use of tools, resources and knowledge %in general/
on the whole, %shall be developed later 
shall come later, during research. The intention is first to gather data, concepts, and a series of insights from in-field work.

Ultra-running has a certain tension in the way it connects participants with people from the outside social worlds.

\begin{itemize}
 \item On one side, it is an ultimately public activity, runners are exposed to permanent contact with other runners (and non-runners as well) in the open, and races depend on a wide number  of actors, both participating and non-race related: in sum, a very wide orchestrated and coordinated social activity.
 \item On the other side, ultra-running entails a certain \textit{Loneliness of the long distance runner}%
 \footnote{Short story by Alan Sillitoe, published in 1959.}. 
 Running ultra distances may well be one of  the most \textit{outdoor} activities or sports. It involves several hours, even days sometimes "out in the  open", amongst  almost untouched nature and wild green spaces afar from a city in cross-country races. And in training season, even in city context: the silent early night-to-dawn moment (from 4 to 6 am) is when nearly no ordinary person is going about, and birds have not even began to chirp. As well as with lone spaces, running involves far many solitary moments in which runners get to collect themselves and revolve in their thoughts, the bareness of the surroundings, and at many flowing times:  
 think of nothing and seize the moment.  
 %not think in anything and be in the moment.
\end{itemize}

The \textit{in situ} work is intended to grasp these two areas (intimate-personal; and social-network-dependent) in ultra-running: the first, during training; and the second, during specific ultra-running events.

\begin{enumerate}
 \item The first aspect, training, is to be dealt with  through auto-ethnography, not as a biographical account, but as a means to grasp the main topics developed. Many of the available material on ultra-running in text and video documentary depict narratives from the sole perspective of runners themselves, in first person, and how they prepare for their practices with different styles of running and post practice cool downs and stretching as well as general nutrition and resting time. The researcher may well take a similar approach without being an outsider of common practice in this social world.
 
 \begin{quote}
  Gertrude Kurath (1960) recommended ethnographers to "learn the movements" and Adrienne Kaeppler (1978) proposed that ethnographers learn certain movements and  receive instructions on  what is done "incorrectly", or "differently" with a methodology that would allow for better understanding.
  %to understand better. 
  [José Bizerril has argued that the practical formation of the researcher has its advantages.] This knowledge allows  access to aspects of the research topic that otherwise would pass unnoticed if only done with a distant approach based on observation and interview. [the experiential dimension makes it possible to gain entry to the experience and] "to the psycho-physical and -why not-, to the spiritual states that that this experience triggers%
  \footnote{ASCHIERI, Patricia. "Hacia una etnografía encarnada: La corporalidad del etnógrafo/a como dato en la investigación". X RAM- Reunión de Antropología del Mercosur. Córdoba, Argentina, 2013. P. 16. My translation.}.
 \end{quote}
 
 Of course,  auto-ethnography may work with a potential source for bias, but at the same time provides both the most inner side view possible, and reveals the speaker's interests, perspectives and preconceptions; to which one can always add contrast with other references to compare and find the most reliable common ground%
 \footnote{HAMMERSLEY and ATKINSON. \textit{Ethnography: principles in practice}. 3rd ed. London; New York, NY: Routledge, 2007. P.%164, %("Sensitizing concepts" is Blumer's), 
 124.}.
 
 \item On the second aspect, on racing events, there is very little material in academic research on events from a qualitative approach. There is scarce material, and when so, only done through surveys or measurement based. Hence, the importance to move forward. Some of the key features of an \textit{ethnographic approach} are taken into account in the present proposal: to prioritize the insider perspective highlighting the experiential, an active immersion in the field during a reasonable amount of time, minimal interference to gather data to be triangulated%
 \footnote{HOLLOWAY, Imma; BROWN, Lorraine; and SHIPWAY, Richard. "Meaning not measurement: Using ethnography to bring a deeper understanding to the participant experience of festivals and events". \textit{International Journal of Event and Festival Management}. Vol. 1 Nº 1, 2010. Pp. 75-76.}.
 And not to focus on measuring variables, but rather on \textit{collecting and constructing new variables} to build up ever more complex concepts: this adds nuance to the understanding of the phenomenon, and provides material to suggest new questions and aspects to be worked on%
 \footnote{BECKER, Howard S. \textit{What About Mozart? What About Murder? Reasoning From Cases}. The University of Chicago Press, Chicago, 2014. Pp. 13-14, 18.}.

\end{enumerate}



%\clearpage
\section*{2. Auto-ethnography}

The plan of work proposed here sets axis on which to develop future ideas, these axis being: 
affect,
body, 
and materiality.
These \textit{sensitizing concepts} (rather than restrictive prescriptions) shall be guiding points to suggest directions where to look at, as germs of analysis on how and where to collect information. Data finding also relies on the researcher's agenda: "What sorts of patterns one is looking for depends, of course, on research focus and theoretical orientation". Benefits of in-field immersion include not only direct access in general but additionally to non-structured conversations in which "[unusual participant terms] may stress theoretically important or interesting phenomena". In the same vein, concepts may also be, alternatively, "observer-identified"%
\footnote{HAMMERSLEY and ATKINSON. \textit{Ethnography: principles in practice}. 3rd ed. London; New York, NY: Routledge, 2007. P. 164 ("Sensitizing concepts" is Blumer's), 163.}.

The axial concepts are not %be used as fixed tautologies 
to give a taken-for-granted understanding of behaviors. The approach here is first \textit{exploratory}, rather  than explanatory. The deeper understanding of behaviors and use of tools, resources and knowledge %in general/
on the whole, %shall be developed later 
shall come later, during research. The intention is first to gather data, concepts, and a series of insights from in-field work.

Ultra-running has a certain tension in the way it connects participants with people from the outside social worlds.

\begin{itemize}
 \item On one side, it is an ultimately public activity, runners are exposed to permanent contact with other runners (and non-runners as well) in the open, and races depend on a wide number  of actors, both participating and non-race related: in sum, a very wide orchestrated and coordinated social activity.
 \item On the other side, ultra-running entails a certain \textit{Loneliness of the long distance runner}%
 \footnote{Short story by Alan Sillitoe, published in 1959.}. 
 Running ultra distances may well be one of  the most \textit{outdoor} activities or sports. It involves several hours, even days sometimes "out in the  open", amongst  almost untouched nature and wild green spaces afar from a city in cross-country races. And in training season, even in city context: the silent early night-to-dawn moment (from 4 to 6 am) is when nearly no ordinary person is going about, and birds have not even began to chirp. As well as with lone spaces, running involves far many solitary moments in which runners get to collect themselves and revolve in their thoughts, the bareness of the surroundings, and at many flowing times:  
 think of nothing and seize the moment.  
 %not think in anything and be in the moment.
\end{itemize}

The \textit{in situ} work is intended to grasp these two areas (intimate-personal; and social-network-dependent) in ultra-running: the first, during training; and the second, during specific ultra-running events.

\begin{enumerate}
 \item The first aspect, training, is to be dealt with  through auto-ethnography, not as a biographical account, but as a means to grasp the main topics developed. Many of the available material on ultra-running in text and video documentary depict narratives from the sole perspective of runners themselves, in first person, and how they prepare for their practices with different styles of running and post practice cool downs and stretching as well as general nutrition and resting time. The researcher may well take a similar approach without being an outsider of common practice in this social world.
 
 \begin{quote}
  Gertrude Kurath (1960) recommended ethnographers to "learn the movements" and Adrienne Kaeppler (1978) proposed that ethnographers learn certain movements and  receive instructions on  what is done "incorrectly", or "differently" with a methodology that would allow for better understanding.
  %to understand better. 
  [José Bizerril has argued that the practical formation of the researcher has its advantages.] This knowledge allows  access to aspects of the research topic that otherwise would pass unnoticed if only done with a distant approach based on observation and interview. [the experiential dimension makes it possible to gain entry to the experience and] "to the psycho-physical and -why not-, to the spiritual states that that this experience triggers%
  \footnote{ASCHIERI, Patricia. "Hacia una etnografía encarnada: La corporalidad del etnógrafo/a como dato en la investigación". X RAM- Reunión de Antropología del Mercosur. Córdoba, Argentina, 2013. P. 16. My translation.}.
 \end{quote}
 
 Of course,  auto-ethnography may work with a potential source for bias, but at the same time provides both the most inner side view possible, and reveals the speaker's interests, perspectives and preconceptions; to which one can always add contrast with other references to compare and find the most reliable common ground%
 \footnote{HAMMERSLEY and ATKINSON. \textit{Ethnography: principles in practice}. 3rd ed. London; New York, NY: Routledge, 2007. P.%164, %("Sensitizing concepts" is Blumer's), 
 124.}.
 
 \item On the second aspect, on racing events, there is very little material in academic research on events from a qualitative approach. There is scarce material, and when so, only done through surveys or measurement based. Hence, the importance to move forward. Some of the key features of an \textit{ethnographic approach} are taken into account in the present proposal: to prioritize the insider perspective highlighting the experiential, an active immersion in the field during a reasonable amount of time, minimal interference to gather data to be triangulated%
 \footnote{HOLLOWAY, Imma; BROWN, Lorraine; and SHIPWAY, Richard. "Meaning not measurement: Using ethnography to bring a deeper understanding to the participant experience of festivals and events". \textit{International Journal of Event and Festival Management}. Vol. 1 Nº 1, 2010. Pp. 75-76.}.
 And not to focus on measuring variables, but rather on \textit{collecting and constructing new variables} to build up ever more complex concepts: this adds nuance to the understanding of the phenomenon, and provides material to suggest new questions and aspects to be worked on%
 \footnote{BECKER, Howard S. \textit{What About Mozart? What About Murder? Reasoning From Cases}. The University of Chicago Press, Chicago, 2014. Pp. 13-14, 18.}.

\end{enumerate}




%%%%%%%%%%%%%%%%%%%%%%%%%%%%%%%%%

% benjaminjuarez.com/ARCHIVO/2016.02.29.OutlineProjectPhD-extended.html 

%Dear Benjamin,  I very much enjoyed reading your draftPhD proposal. I think it is an exciting project. 

%I wonder if you could expand a bit more on Section 3 Materials and Methods. This would also need that you specify a couple of research questions which affect your methodological account, your methods etc (ho will you research the lifeworld of runners?)...

%You may also say a bit more about your conceptual approach? 

%Kind regards
%Michael. 
%\section*{Research questions }