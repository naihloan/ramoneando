\subsubsection*{Breves palabras sobre la etología del ser humano}
\label{breves-palabras-sobre-la-etologia-del-ser-humano}
% \addcontentsline{toc}{subsubsection}{line added to TOC: subsubsection one}
\addcontentsline{toc}{subsubsection}{\protect\numberline{\thesubsubsection} line added to TOC: subsubsection one}
\addcontentsline{toc}{section}{Breves palabras sobre la etología del ser humano}

\noindent
Introduccion para la edición Bulgara del año 2002

~ ~ ~ A la psicología tradicional, así como otras ciencias que estudian
al ser humano, siempre le interesó la relación de la parte biológica y
no biológica en el comportamiento humano. En distintas épocas se
estudiaba como predominante ora la influencia biológica, ora no
biológica. En el siglo 19 y a principios del siglo 20 la opinión más
aceptada era la predominancia del componente biológico. Zigmund Freud
era el más famoso de los que expresaban tal opinión (aunque el no era el
único). Pero hablando siempre prácticamente de instintos (``libido'' y
``mortido'') y su influencia sobre el ser humano, el nunca hizo ningún
esfuerzo para estudiar su naturaleza física o génesis. No es de
sorprender que sus ideas no se percibían como muy convincentes, y
siempre se criticaban. Además, fundada en la paradigma muy similar,
eugenesia se desacreditó por la culpa de los regímenes despóticos que la
utilizaron como apoyo ideológico de la política de opresión. Por eso
desde los años 20 del siglo XX el péndulo se movió para otro lado,
pasando de largo, como suele suceder en estos casos, el justo medio.
Hasta hace poco reinó la teoría de la predominancia del componente
social en el comportamiento humano, algunas veces llamado concepto de la
``Tabula Rasa'', es decir; ``Hoja en Blanco''. Esta teoría supone que el
ser humano al nacer es como una hoja en blanco donde la sociedad y el
ambiente escriben unos u otros modos de comportamiento, y dependiendo de
qué va a escribirse así va a ser la persona. Pero con el transcurso del
tiempo se hizo más obvia la imposibilidad de explicar claramente y fácil
todos los matices de las reacciones del comportamiento humano. Al mismo
tiempo muchas inexplicables reacciones humanas se explican con mucha
naturalidad utilizando la hipótesis de la existencia en el ser humano de
una base muy amplia de esquemas innatos del comportamiento. Por eso ya a
finales del siglo XX los extremos empezaron a eliminarse, y el punto de
vista que el ser humano es un ser en gran parte biológico que nace con
una carga no despreciable de los esquemas predeterminados de conducta
empezó a obtener mayor apoyo. Quizá la aportación más grande en ello
realizó casi hace medio siglo la ciencia de etología y las asignaturas
que surgieron de ella. La etología estudia las bases instintivas de la
conducta de los seres vivos mediante el método de comparación del
comportamiento de distintas especies entre sí. El ser humano para un
etólogo es solamente un mamífero erguido, es decir, es uno entre los
mamiferos.\\
\hspace*{0.333em} ~ ~ Comparando entre sí la conducta de los
representantes de distintos tipos zoológicos desde los primitivos hasta
los superiores, los científicos descubren paralelismos sorprendentes que
atestiguan los principios de conducta comunes que conciernen a todos los
representantes del reino animal, entre ellos al ser humano. Los métodos
parecidos de estudiar el mundo son muy fructíferos y se utilizan
ampliamente en otras ciencias. Por ejemplo los astrónomos conocen mejor
la estructura interior del Sol que los geológos la estructura interior
de la Tierra. Y es que hay muchos astros y comparándolos entre sí se
pueden entender muchas cosas. La Tierra es única y no hay con qué
compararla. Eso pasa en los estudios del ser humano. Limitándose al
estudio de este mismo nos arriesgamos a limitarnos en su comprensión.\\
\hspace*{0.333em} ~ ~ Pero no es fácil estudiar la etologia. Además de
las dificultades objetivas que surgen de la enorme influencia de la
razon que enmascara y modifica muchas manifestaciones instintivas, los
científicos chocan con el rechazo del método etológico para el ser
humano. A muchos les parece inadmisible e inclusive ofensivo el hecho de
comparar la conducta humana con la animal. ¡Esto tiene su explicación
etológica! Se trata de la acción del instinto del aislamiento etológico
de las especies, el estudio detallado del cual sobrepasa los límites de
nuestro libro (los que quieran pueden leer el libro de V.Dolnik ``el
nino desobediente de la biósfera''). El principio de este instinto se
puede expresar con el lema: ``Ama al prójimo - odia al ajeno'', y
``ajenos'' en nuestro caso son los monos, la hostilidad a los cuales se
extiende a la tesis del parecido de nuestro comportamiento con su
comportamiento. Aunque la teoría de Darwin pese a todos los intentos
(debido a esta hostilidad) hasta hoy día de refutarla está admitida
irrevocablemente por la comunidad científica, y la mayoría de la gente
está de acuerdo de su procedencia de los monos, la idea de que uno u
otro sentimiento es la voz del instinto provoca en mucha gente protestas
airadas que no encuentran ninguna explicación racional. Pero la causa de
este rechazo está en el rechazo subconsciente de nuestro parentesco con
los monos. Recuerden eso, mis estimados lectores.\\
\hspace*{0.333em} ~ ~ ¿Qué más puede contar un etólogo sobre el ser
humano? Mucho. Sobre la agresividad, sobre la naturaleza del poder,
sobre la moralidad innata y las fuerzas motoras del nacionalismo e
inclusive sobre las extrañezas del amor. Y precisamente de las
extrañezas del amor vamos a hablar en este libro.
