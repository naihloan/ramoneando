\documentclass[a4paper, hidelinks]{twentysecondcv} 
% \documentclass[letter,hidelinks]{twentysecondcv} 
% \usepackage[pass,letterpaper]{geometry}
\usepackage[pass,paperwidth=8.5in,paperheight=2in]{geometry}
% \pdfpageheight=8cm
% \pdfpagewidth=8.5in
% a4paper for A4 % https://latex.org/forum/viewtopic.php?t=31329 CHANGING TITLES IN SECTIONS // ABOUT ME ... ETC.
\usepackage[spanish]{babel} 
\selectlanguage{spanish}
\usepackage[utf8]{inputenc} %\usepackage[colorlinks=true,linkbordercolor = {black}]{hyperref} \usepackage{xcolor} 
% \DeclareUnicodeCharacter{262D}{\hamsic}
% \hypersetup{colorlinks, linkcolor={red!50!black}, citecolor={blue!50!black}, urlcolor={blue!55!black} }
\usepackage{slantsc} % small caps
\usepackage{hyperref}
\usepackage{scrextend}
% \changefontsizes{12pt}

%----------------------------------------------------------------------------------------
%	 PERSONAL INFORMATION
%----------------------------------------------------------------------------------------

% If you don't need one or more of the below, just remove the content leaving the command, e.g. \cvnumberphone{}

\profilepic{avatar.png} % Profile picture
\cvname{Benjamín Juárez} 
\cvjobtitle{desarrollador web trilingüe $ $  $ $  $ $ [Ciudadano de EEUU]} %☭ 
\cvdate{\today} 
\cvaddress{Córdoba, Argentina %\newline % Crisol 305
} 
% Short address/location, use \newline if more than 1 line is required
\cvnumberphone{(+54 9 351) 153 104043} 
\cvsite{%
% https://www.linkedin.com/in/benjam%C3%ADn-ju%C3%A1rez-960649152/}
% \href{linkedin.com/in/benjamin-juarez-960649152}{linkedin.com [mi cv online]}
\href{https://www.linkedin.com/in/benjamín-juárez-960649152}{linkedin.com [mi cv online]}
} 
\cvmail{benjaminjuarezarlt@gmail.com}

%----------------------------------------------------------------------------------------
\begin{document}
%----------------------------------------------------------------------------------------
%	 ABOUT ME
%----------------------------------------------------------------------------------------

\aboutme{%
% Hice Sociología a nivel de Licenciatura y Maestría. 
% Recientemente trabajé fuera de la universidad. 
% Armé proyecto propio: un bar de jugos de fruta natural, \textit{Juice Bar}. 
% También conocí el mundo empresarial trabajando en \textit{Pearson} entre 2017 y 2018.} 
}
% To have no About Me section, just remove all the text and leave \aboutme{}

%----------------------------------------------------------------------------------------
%	 SKILLS
%----------------------------------------------------------------------------------------
 % Skill bar section, each skill must have a value between 0 an 6 (float)
 
\skills{%
{portugués (fluído)/4.7},
{\textsc{inglés (nativo)}/6}%
% {esfuerzo/5.9},
% {atención al detalle/5.7},
% {buenos modales/5.3}%
}

%------------------------------------------------

% Skill text section, each skill must have a value between 0 an 6
\skillstext{%
% {usuario linux/desde 2005 [hoy i3wm]},
% {teclado dvorak/desde 2004},
% {capacidad de aprendizaje/9,299},
% {relajado/6,5 \newline},
% {social / > $ $ \href{<https://ssbc.github.io/secure-scuttlebutt/>}{scuttlebutt} \newline},
% {música / > $ $ \href{<https://github.com/fermentation/ferment>}{ferment}}%,
}

\tech{%
% md 
vim  % ranger 
\LaTeX $ $ html css bootstrap \\
git github gitlab mysql \\ % mongodb express.js angular node.js 
mongodb \\ express.js \\ angular \\ node.js \newline
intereses: nest python django  scrum 
}

% \tech{%
% hola
% }
% - editing: vim ranger  \newline %
% - markup: md \LaTeX $ $ html css \newline %
% - web: php javascript vcs (git)\newline
% - programming: java lua node.js 
% }

%----------------------------------------------------------------------------------------

\makeprofile % Print the sidebar

\vfill 
%----------------------------------------------------------------------------------------
%	 INTERESTS
%----------------------------------------------------------------------------------------

% \section{Intereses}

% \test{DejaVu Sans}
% \test{FreeSans}
% \test{Segoe UI Symbol}

\section{Perfil General} % Symbol: ☭

Me crié en Córdoba desde los 7 años. Antes viví en EEUU. 
Manejo excelente el inglés y muy bien el portugués. Viví en Brasil 2012–2013.\\
% Desarrollé desde temprana edad gusto por el idioma: estudié inglés hasta nivel nativo C1 [CEFR] y lo uso a diario para leer y escribir. 
% También leo, escucho y hablo portugués fluidamente, estuve en Brazil más de dos años. 
% Programación: varios años autodidacta. Ahora en curso de 
Cursé Desarrollo Web en UTN.
Cursando Analista en la Escuela Superior de Comercio Manuel Belgrano [Terminando primer año completo].
% Tengo buena capacidad de adaptación, interés en áreas nuevas y la constancia de hacer continuadamente.\\

% \section{Reciente}

% \subsection{Review}

En los últimos años vengo trabajando en áreas de %fuera del ámbito académico: en 
editorial, dando clases de inglés, y aprendiendo sobre producción editorial. 
En el último tiempo con foco en desarrollo web, base de datos y programación.
% Entre 2017-2018 trabajé en la editorial \textit{Pearson} como Asesor de Servicios de aprendizaje: me reunía con docentes, coordinadores y directivos de institutos y colegios para hacer seguimiento y recomendación de materiales de trabajo. 
% Antes aprendí en sociedad comercial a gestionar \textit{Juice Bar}.\\
% Tengo buena capacidad de adaptación, interés en áreas nuevas y la constancia para empujar de manera continua. Me crié en Córdoba desde los 7 años. Antes viví en EEUU. Desarrollé desde edad gusto por el idioma y estudié inglés hasta un nivel C1 [CEFR] y lo uso a diario para leer y escribir. También aprendí a leer, escuchar y hablar portugués de manera fluida, teniendo una estancia en Brazil durante más de dos años.\\
% En los últimos cinco años además trabajé académicamente: supervisando proyectos de investigación, dando clases de sociología y cursos de conversación en inglés. En todos los casos me interesa trabajar con un ambiente relacionado al mundo editorial, de programación/gestión y de servicio.

% Alice approaches Wonderland as an anthropologist, but maintains a strong sense of noblesse oblige that comes with her class status. She has confidence in her social position, education, and the Victorian virtue of good manners. Alice has a feeling of entitlement, particularly when comparing herself to Mabel, whom she declares has a ``poky little house," and no toys. Additionally, she flaunts her limited information base with anyone who will listen and becomes increasingly obsessed with the importance of good manners as she deals with the rude creatures of Wonderland. Alice maintains a superior attitude and behaves with solicitous indulgence toward those she believes are less privileged.



%----------------------------------------------------------------------------------------
%	 EDUCATION
%----------------------------------------------------------------------------------------

\vspace{0.5cm}

\section{Formación}

\begin{twenty} % Environment for a list with descriptions
 
% \twentyitemshort{2019-2021}{Analista Universitario de Sistemas Informáticos [ESCMB, CBA]}%{}
\twentyitem{2019-2021}{Analista Universitario de Sistemas Informáticos}{[ESCMB%, Córdoba
]}{Primer Año Completo}
\twentyitem{2018-2019}{Curso de Desarrollo Web}{[UTN%, Córdoba, Argentina
]}{
	Javascript Jquery Angular ApiRest Mongodb Node.js Express.js\\ 
	HTML5 CSS3 Bootstrap}
\twentyitem{2018-2019}{Otros Cursos}{[Udemy]}{
	JavaScript: Understanding the Weird Parts\\ 
	React Basic in just 1 hour\\
	SQLite for Beginners | Learn SQL from Scratch\\
	Absolute Introduction to Object Oriented Programming in Java}
%  \twentyitemshort{desde 2017}{Candidato doctoral en Sociología [University of Exeter, UK]}%{El æfecto de los atletas de distancia en el ritmo urbano: cómo corren los ultramaratonistas en ciudades automatizadas?}
 \twentyitemshort{desde 2013}{Lenguajes de marcación-%programación-
edición [hechos: libro + sitio web]}%{Autopubliqué libro [\textit{exhalaciones}] y sitio web [\texttt{ramoneando.com}]}
%  \twentyitemshort{2012-2014}{Maestría en Sociología [UNICAMP, S\~{a}o Paulo. Brazil]}%{Arte Urbano [graffiti y pixaç\~{a}o]}
%  \twentyitemshort{2002-2009}{Licenciado en Sociología [Universidad de Buenos Aires, Argentina]}%
% 	 %\twentyitem{<dates>}{<title>}{<location>}{<description>}
\end{twenty}

% \section{Publicaciones/Presentaciones}
% 
% \begin{twenty} % Environment for a list with descriptions
% 	\twentyitemshort{2017}{being ralphy wiggum}
% 	\twentyitemshort{2016}{The visual and social indeterminacy of pixação: the inextricable moods of São Paulo’s inscriptions}
% % 	\twentyitemshort{2016}{exhalaciones}
% 	\twentyitemshort{2014}{Espacios abiertos: la calle como hábitat}
% 	\twentyitemshort{2013}{Arte urbano paulistano: degradación urbana y paisajismo turístico}
% 	\twentyitemshort{2012}{Urban Geopolitics: Recongurations in Art, Activism, and Research}
% 	\twentyitemshort{2011}{(No) es tan fácil ser grafitero}
% 	\twentyitemshort{2008}{El legado de Durkheim en Schutz: hacia un horizonte en diálogo}
% 	\twentyitemshort{2008}{Observaciones sobre la opinión pública a partir de El ciudadano bien informado}
% 	%\twentyitem{<dates>}{<title>}{<location>}{<description>}
% \end{twenty}
% 
% 
% \section{Traducciones}
% 
% \begin{twenty} % Environment for a list with descriptions
% 	\twentyitemshort{2019... }{23 ideas sobre la juventud [Howard Becker]}
% 	\twentyitemshort{2011}{La influencia de las cosmovisiones en el pasado y en el presente según Max Weber [Stephen Kalberg]}
% 	\twentyitemshort{2005}{Los tipos de racionalidad de Max Weber: piedras angulares para el análisis del proceso de racionalización de la historia [S. Kalberg]}
% 	%\twentyitem{<dates>}{<title>}{<location>}{<description>}
% \end{twenty}

\vspace{0.5cm}

\section{Trabajo}

% \begin{document}

%------------------------------------------------------------
%	 ABOUT ME
%------------------------------------------------------------

\aboutme{% 
\textit{Native English Speaker} \newline %
Born/raised USA [age 7]. Has VISA.\newline
Language Level [CEFR]: \newline
% \hspace*{8} 
% \textit{
English.C1|Spanish.C2|Portuguese.B2
% \hspace*{8} \textit{English C1} \newline
% \hspace*{8} \textit{Spanish C2} \newline 
% \hspace*{8} \textit{Portuguese B2}
}

%------------------------------------------------------------
%	 SKILLS
%------------------------------------------------------------

\skills{% Skill bar section, each skill must have a value between 0 an 6 (float)
{effort/5.9},
{atention to detail/5.7},
{proper manners/5.3}%
}

% ------------------------------------------------

% Skill text section, each skill must have a value between 0 an 6
\skillstext{%
{linux user/since 2005 [now i3wm]},
{dvorak keyboard user/since 2004}%,
% {learning thrust/9,2 \newline},
% {relaxed/6,5},
}
% }

\tech{
html css vim \LaTeX $ $ \\ 
bootstrap javascript\\ %\bigskip% md ranger 
git github gitlab \\
agile kanban %methodology with 
jira trello \\
os: microsoft apple gnu/linux \\ \bigskip

mongodb mariaDB mysql \\
express.js \\
angular \\
node.js \\ \bigskip

interests:
python 
C\# 
vue react 
php laravel symfony
docker %\bigskip
IoT
big data
% Raspberry Pi
% nest gatsby 

% scrum C$\#$ \\ %\bigskip
% interests: 
% django  jquery % nest %C++ 
% - dvcs: % - en curso: % ApiRest % Microsoft SQL Server


% - web: javascript php \newline
% - editing: vim vcs (git) ranger  \newline %
% % - programming: node.js lua java \newline 
}

%----------------------------------------------------------------------------------------

\makeprofile % Print the sidebar

\vfill 

%------------------------------------------------------------
%	 INTERESTS
%------------------------------------------------------------

\section{Short Bio} % \section{Intereses}

I am a sociologist by studies and research + %\& %in process of becoming a 
an aspiring Developer.
% programmer. % at heart.

% 2020: second year student as Systems Analyst.
% 
% I am adaptable, curious and tenacious. 
% I was born in Chicago, USA; and raised from age 7 in Córdoba, Argentina.
% I always kept studying English, raising my level to a C1 [CEFR] standard.
% Also, I use English on a daily basis by reading and writing. 
% I have also learned to read, listen and speak Portuguese in a fluent manner, living in Brazil through nearly 2 years. 

I’ve worked in academic environments: researching, supervising research projects, giving sociology classes, and English conversation. I feel at ease in contexts relating to publishing, film production, programming, and non-profit.

%------------------------------------------------------------
%	 WORK
%------------------------------------------------------------

\section{Recent work}

\begin{twentyshort} 
	\twentyitem{2020--2020}{Customer Dev Backoffice}{[Cintelink]} 
% 	Fuel Business Intelligence
% 	]} % Córdoba, Argentina]}
        {IoT sql python postman} %php, nodejs, vue, docker. 
%         We work in an agile environment. 
%         I use source code versioning and Kanban methodology [jira] on a daily basis.
%         We deliver results in small increments: my first solo project used the Knime platform to manage big data sets with SQL-type queries.} % docker
	\twentyitem{2019--2020}{Back End Web Developer --nodejs}{[Ross Outside the Box]}
        {%Back End | Full Stack [nodejs express mariaDB %| (react angular)
        %] \\ 
%         This was my first experience in a tech/agile environment. 
%         I used source code versioning and 
        %We deliver results in small increments: 
        My first solo project used the Knime platform to manage big data sets with SQL-type queries.
        testing: e2e + performance.
%         Kanban methodology with jira. %on a daily basis.
        } % docker   
	
	\twentyitemshort{2018--2019}{Research Assistant -- Project \textit{In the Name of Wild} | Phillip Vannini}
\twentyitemshort{2018--2019}{Translations and Reviews: Spanish to English}
	% 	\twentyitemshort{2019}{Translation: How to make sense of precariousness? \\
% 	\textit{Bíos}-precarious and sensitive life [Martín De Mauro]}
% 	\twentyitemshort{2020}{Translation: 23 ideas sobre la juventud [Howard Becker]}
	\twentyitemshort{2015--2018}{English Conversation Classes} %[Córdoba, Argentina]
	\twentyitemshort{2017-2018}{Pearson}  %[Córdoba, Argentina]
\end{twentyshort}

%------------------------------------------------------------
%	 EDUCATION
%------------------------------------------------------------

\section{Education}

\begin{twenty} 
        \twentyitem{2019-2021}{Systems Analyst}
        {[ESCMB, Córdoba, Argentina]}{
        Second year student [First year Completed]}
        
        \twentyitem{2018--2020}{Courses} % Several Courses [Some Complete / Others Ongoing]
        {[Udemy, online]}{bash git sql javascript jquery angular php java node} %laravel symfony
 	
 	\twentyitem{2018--2019}{Web Development Course}{[UTN, Córdoba, Argentina]}
 	{js %Javascript 
 	Jquery Angular ApiRest Mongodb % Node.js Express.js 
 	node express  	%\\ 
 	% HTML5 CSS3 
 	Bootstrap}
 	\twentyitemshort{2012--2014}{Master in Sociology [UNICAMP, S\~{a}o Paulo. Brazil]}
	\twentyitemshort{2002--2009}{Graduate in Sociology [Universidad de Buenos Aires, Argentina]}{}%
 	\twentyitemshort{since $  $ 2009}{Markup-editing languages [Hands-on]}
\end{twenty}


%------------------------------------------------------------
%	 OTHER INFORMATION
%------------------------------------------------------------


\section{Other Projects}

\begin{twenty} % Environment for a list with descriptions
	\twentyitemshort{2018--today}{Writer + Editor: \texttt{greendrinkscba.org}}
	\twentyitemshort{2014--today}{Personal web site: \texttt{ramoneando.com}}
	\twentyitemshort{2016--2019}{\textit{inextricable publisher} [independent book publishing]}
% 	\twentyitemshort{2012--2014}{Master in Sociology [UNICAMP, S\~{a}o Paulo. Brazil]}
% 	\twentyitemshort{2002--2009}{Graduate in Sociology [Universidad de Buenos Aires, Argentina]}{}%
	\twentyitemshort{2009--today}{text and design editing tool: \LaTeX}

% 	\twentyitem%
% 	{since 2017}
% 	{PhD [distance] Sociology}
% 	{[University of Exeter, UK]}
% 	{Distance athletes in urban rhythm | How do ultra-runners run through and out of cities?}
 	%Distance athlete’s æffect on urban rhythm | How do ultra-runners run in automatized cities?
%  	{Self-published book [\textit{exhalations}] and web site [\texttt{ramoneando.com}]}
% 	\twentyitem%
% 	{2014}
% 	{Master in Sociology [UNICAMP, S\~{a}o Paulo. Brazil]}{}
% 	{I researched urban art: graffiti \& pixaç\~{a}o}
\end{twenty}

\section{General Profile} % \section{Intereses}

% I am a sociologist by studies and research + 
% Also, I am in process of becoming a curious programmer at heart.

2020: second-year student as Systems Analyst.

I am adaptable, curious, and tenacious. 
I was born in Chicago, USA, and raised from age 7 in Córdoba, Argentina. 
I always kept studying English, raising my level to a C1 [CEFR] standard.
Also, I use English daily by reading and writing. 
I have also learned to read, listen, and fluently speak Portuguese, living in Brazil for nearly two years.

% I've worked in academic environments: supervising research projects, giving sociology classes and English conversation.
% I feel at ease in environments relating to publishing, film production, programming and service.

\bigskip

% \epigraph{ “Nada de lo que hacemos o decimos se pierde en el vacío: el aire está lleno del pensamiento de todos.” }

\twentyitemshort{ }{> “None of what we do or say is lost into a void: \\ 
the air is full of everyone's thoughts.” \\ 
–  Pedro Bonifacio Palacios}

\twentyitemshort{ }{> “Nada de lo que hacemos o decimos se pierde en el vacío: \\ 
el aire está lleno del pensamiento de todos.” \\ 
– ALMAFUERTE}

% \begin{minipage}[r]{0.8\textwidth}
% %  × lalalal
% \begin{quote}
%  “Nada de lo que hacemos o decimos se pierde en el vacío: el aire está lleno del pensamiento de todos.” 
% – ALMAFUERTE
% \end{quote}
% \end{minipage}

\vfill 

% \section{Translations}
% 
% \begin{twenty} % Environment for a list with descriptions
% % 	\twentyitemshort{2020?}{Off the Grid -- Re-Assembling Domestic Life [Phillip Vannini]}
% 	\twentyitemshort{2019}{Research Assistant -- Project \textit{In the Name of Wild} [Phillip Vannini]}
% 	\twentyitemshort{2019}{23 ideas sobre la juventud [Howard Becker]}
% 	\twentyitemshort{2019}{%
% 	How to make sense of precariousness? \\
% 	\textit{Bíos}-precarious and sensitive life
% % 	Taedium Vitae: Precariety and Affects in Porteña Night 
% 	[Martín De Mauro]}
% % 	\twentyitemshort{2011}{La influencia de las cosmovisiones en el pasado y en el presente según Max Weber [Stephen Kalberg]}
% % 	\twentyitemshort{2005}{Los tipos de racionalidad de Max Weber: piedras angulares para el análisis del proceso de racionalización de la historia [S. Kalberg]}
% 	%\twentyitem{<dates>}{<title>}{<location>}{<description>}
% \end{twenty}


\begin{twentyshort} % Environment for a short list with no descriptions
	\twentyitem{2019--}{Desarrollador Web}{[Ross | Outside the Box. Córdoba, Argentina]}
        {Back End | Full Stack [nodejs express angular docker]}
	\twentyitemshort{2018–2019}{Productor Documental [Argentina-Canadá]} 
	\twentyitemshort{2018–2018}{IICANA Valle Escondido [Córdoba, Argentina]} 
	\twentyitemshort{2017–2018}{Pearson Editorial[Córdoba, Argentina]} 
	\twentyitemshort{2015–2018}{Cursos de conversación en inglés [Córdoba, Argentina]}
\end{twentyshort}

% 
%----------------------------------------------------------------------------------------
%	 OTHER INFORMATION
%----------------------------------------------------------------------------------------

\vspace{0.5cm}

\section{Otros proyectos}

\begin{twenty} % Environment for a list with descriptions
	\twentyitemshort{2016-hoy}{\textit{ediciones inextricables} [publicación independiente en papel]}
	\twentyitemshort{2014-hoy}{Sitio web personal: \url{ramoneando.com}}
% 	\twentyitemshort{2013-hoy}{herramientas de programación: html [+ css javascript md vim]}
	\twentyitemshort{2009-hoy}{herramientas de edición de texto y diseño: \LaTeX}
% 	\twentyitemshort{2005-hoy}{Sistemas operativos: microsoft, apple, gnu/linux}
	%\twentyitem{<dates>}{<title>}{<location>}{<description>}
\end{twenty}

\vfill 

\end{document} 

%----------------------------------------------------------------------------------------
%	 SECOND PAGE EXAMPLE
%----------------------------------------------------------------------------------------

%\newpage % Start a new page

%\makeprofile % Print the sidebar

%\section{Other information}

%\subsection{Review}

%Alice approaches Wonderland as an anthropologist, but maintains a strong sense of noblesse oblige that comes with her class status. She has confidence in her social position, education, and the Victorian virtue of good manners. Alice has a feeling of entitlement, particularly when comparing herself to Mabel, whom she declares has a ``poky little house," and no toys. Additionally, she flaunts her limited information base with anyone who will listen and becomes increasingly obsessed with the importance of good manners as she deals with the rude creatures of Wonderland. Alice maintains a superior attitude and behaves with solicitous indulgence toward those she believes are less privileged.

%\section{Other information}

%\subsection{Review}

%Alice approaches Wonderland as an anthropologist, but maintains a strong sense of noblesse oblige that comes with her class status. She has confidence in her social position, education, and the Victorian virtue of good manners. Alice has a feeling of entitlement, particularly when comparing herself to Mabel, whom she declares has a ``poky little house," and no toys. Additionally, she flaunts her limited information base with anyone who will listen and becomes increasingly obsessed with the importance of good manners as she deals with the rude creatures of Wonderland. Alice maintains a superior attitude and behaves with solicitous indulgence toward those she believes are less privileged.

%----------------------------------------------------------------------------------------

% \vfill 

% \end{document} 

%%%%%%%%%%%%%%%%%%%%%%%%%%%%%%%%%%%%%%%%%
% Twenty Seconds Resume/CV
% LaTeX Template
% Version 1.1 (8/1/17)
%
% This template has been downloaded from:
% http://www.LaTeXTemplates.com
%
% Original author:
% Carmine Spagnuolo (cspagnuolo@unisa.it) with major modifications by 
% Vel (vel@LaTeXTemplates.com)
%
% License:
% The MIT License (see included LICENSE file)
%
%%%%%%%%%%%%%%%%%%%%%%%%%%%%%%%%%%%%%%%%%

%----------------------------------------------------------------------------------------
%	PACKAGES AND OTHER DOCUMENT CONFIGURATIONS
%----------------------------------------------------------------------------------------

% \usepackage{contour}% small caps
% \usepackage{fontspec} % pico y pala

% Symbol
% \usepackage{tikz}
% \usetikzlibrary{svg.path}
% 
% % SVG path from Sarang:
% % https://upload.wikimedia.org/wikipedia/commons/a/ac/U%2B262D.svg
% \newcommand*{\hamsic}{%
%   \begingroup
%     \settoheight{\dimen0 }{H}%
%     \resizebox{!}{\dimen0 }{%
%       \tikz\fill svg[yscale=-1]{%
%         M67,80l13,19L130,65L108,52z%
%         M54,103A60,60 0 1,0 55,23A50,50 0 1,1 55,101z%
%         m0,0L14,160l19,10L64,110z%
%         M95,87l46,79l17,-12L108,79z%
%       };%
%     }%
%   \endgroup
% }

% \newcommand*{\test}[1]{%
%   #1:&\fontspec{#1}\symbol{"262D}\\%
% }
