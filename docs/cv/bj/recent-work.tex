%------------------------------------------------------------
%	 WORK
%------------------------------------------------------------

% \section{\en{Work}\de{Trabajo} }

\cvsection{\en{Work}\de{Trabajo}}

% \cvsubsection

\begin{cvtable}

\cvitem{2016\de{ -- hoy }\en{--today}}% date
{\en{Sociology Teacher}\de{Profesor de Sociología}}%Description
{[ {\scriptsize CESM--CAE} ]}%Location
{}
	
\cvitem{2018\de{ -- hoy }\en{--today}}%
{
  \en{Translations and Reviews }
  \de{Traducciones y Revisiones}
}
{
  \en{[ Spanish to English ]}
  \de{[ De Español a Inglés ]}
}{} \smallskip 

	\cvitem{2019--2020}{\textcolor{darkgray}{%
	  \de{Desarrollador Web }%
	  {Back End}
	  %
	  \en{Web Developer} 
	}}%
	{[ R{\scriptsize OSS} O{\scriptsize UTSIDE THE} B{\scriptsize OX} ]}{
	  \en{nodejs/authentication. On my 1st solo project I used Knime with big data sets with SQL-type queries. testing: e2e + performance.} 
	  \de{nodejs/ authenticación. En el 1er proyecto a mi cargo usé Knime con big data y consultas tipo SQL. testing: e2e + performance.} 
        }  \smallskip 

	\cvitem{2018--2019}%date
	{\textcolor{darkgray}{\en{Research Assistant}\de{Productor Etnografía Visual} }}%description
	{\en{\hfill [Project \textit{In the Name of Wild}}\de{[\textit{En Nombre de Lo Salvaje} }%location
	| \en{Phillip}\de{P.} Vannini]}
% 	{\de{Proyecto documental mundial}\en{World Documentary Project}
% 	\hfill 
% 	{%
% 	[\de{Entrevistas en}\en{Interviews in} Chaltén, Santa Cruz -- Argentina ]}
	{\en{Unesco Natural World Heritage}\de{Patrimonios Natural Unesco} \hfill %}
	[U{\scriptsize NIV. OF} B{\scriptsize RITISH} C{\scriptsize OLUMBIA} -- \en{Canada}\de{Canadá}]}
	 %\smallskip
	
	\cvitem{2015--2018}{%
	  \en{English Conversation Classes \hfill }%
	  \de{Grupos de Conversación en Inglés \hfill}%
	}%
	{
	\en{[ Private Lessons ]}%
	\de{[ Clases Privadas ]}%
	}{}
	
	\cvitem{2017--2020}{%
	  \en{Ethnography on Ultra Running \hfill}%
	  \de{Etnografía de Ultra Running \hfill }%
	}
	{
	  \en{ [ Field Research CBA, Argentina ]}%
	  \de{[ Investigación en CBA -- Argentina ]}%
	}
	{}

	\cvitem{2012--2014}{%
	  \en{Ethnography on Graffiti \& Pixação \hfill }%
	  \de{Etnografía sobre Graffiti \& Pixação \hfill}%
	}%
	{
	  \en{[ Field Research in SP -- Brazil ]}%
	  \de{[ Investigación en SP -- Brasil ]}%
	}
	{}
	
\end{cvtable}
	
% \cvsubsection{One-line without description}
% \begin{cvtable}
% 	\cvitem{Award}{One-line description}{Sponsor}{}
% 	\cvitem{Award}{One-line description}{Sponsor}{}
% 	\cvitem{Award}{One-line description}{Sponsor}{}
% \end{cvtable}
	
	
% \end{twentyshort}

% 	\twentyitem{2020--2020}%
% 	{\en{Customer Dev}\de{Desarrollador} Backoffice}%
% 	{[ C{\scriptsize INTELINK} ]}%
% 	{IoT sql python postman} %php, nodejs, vue, docker. 

% \de{
% \begin{twentyshort} 
% 	\twentyitem{2020--2020}{Desarrollador Backoffice}{[C{\scriptsize INTELINK}]}{webApp ssh IoT sql python postman} %php, nodejs, vue, docker. 
% 	\twentyitem{2019--2020}{Back End Web Developer --nodejs}{[Ross Outside the Box]}
%         {nodejs/ authenticación. En el 1er proyecto a mi cargo usé Knime con big data y consultas tipo SQL. testing: e2e + performance.} 
% 	\twentyitemshort{2018--2019}{Productor -- Proyecto \textit{En Nombre de Lo Salvaje} | Phillip Vannini}
% 	\twentyitemshort{2018--2019}{Traducciones y revisiones: de español a inglés}
% 	\twentyitemshort{2015--2018}{Clases de conversación de inglés} 
% 	\twentyitemshort{2017-2018}{Editorial Pearson}
% \end{twentyshort}
% }