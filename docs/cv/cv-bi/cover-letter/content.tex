I’m looking to join a great team, are you it? That would be awesome.

I’ve worked as a Technical Writer for a year. 
And then less than a year as a Project Manager.
Now I am looking to work directly
with the product area, and even closer with developers. 
In the past I did research as an American
Master's Sociologist based in Argentina and Brazil. 
Soon, I look forward to 2022, I’ll be completing my Thesis on System's Analysis.

Are you in need of a Product Manager? 


I look forward to having a conversation with you.
% Best regards

% yada yada

% En el año 2016 empecé la adscripción en la cátedra de Metodología de la Investigación. Asistí al horario de las clases prácticas de los jueves, que estaban a cargo de Angela Alessio y en la que se planteaban los tópicos para el desarrollo de un proyecto de investigación. Los alumnos se ordenaron por grupos temáticos y cada grupo desarrollaba un texto de pre-proyecto para ir revisando clase a clase. En esta instancia de revisión los adscriptos tomábamos grupos específicos haciendo el acompañamiento de los alumnos señalando posibles direcciones para continuar el trabjo redactado y modificar y/o agregar secciones. 
% 
% En mi caso trabajé con dos grupos de alumnos. El primero de los grupos eran alumnas que trataron el tema de arte autogestivo con el caso empírico cordobés de un nodo central que reunía grupos de artistas de diferentes medios como ser teatro, publicaciones independientes, músicos y artistas visuales. El segundo grupo de alumnos trataron el tema de la educación Waldorf en una escuela de Villa Allende y plantearon el tema de la disciplina y la libertad creativa que este tipo de método de enseñanza facilitaba a los alumnos, a los padres, y a toda la comunidad educativa.
% 
% En ambos casos se hicieron las revisiones semana a semana, acompañando a los alumnos en las clases y también por medio del intercambio de los materiales vía e-mail. Todos los grupos trabajaban con conciencia de estar aprendiendo una metodología general para todo el curso porque además de los temas dictadas y las revisiones por grupo en las clases siempre se hablaba del trabajo de los grupos de manera que cada alumno conociera el tema y modo de trabajar de sus compañeros.
% 
% Del lado de adscripto fue muy productivo revisar y seguir estudiando la metodología en las ciencias sociales y fue además muy interesante conocer las inquietudes de los alumnos, sus maneras de trabajo, y los contenidos desarrollados en sus temas empíricos seleccionados.

%Lorem ipsum dolor sit amet, consectetur adipiscing elit. Duis ullamcorper neque sit amet lectus facilisis sed luctus nisl iaculis. Vivamus at neque arcu, sed tempor quam. Curabitur pharetra tincidunt tincidunt. Morbi volutpat feugiat mauris, quis tempor neque vehicula volutpat. Duis tristique justo vel massa fermentum accumsan. Mauris ante elit, feugiat vestibulum tempor eget, eleifend ac ipsum. Donec scelerisque lobortis ipsum eu vestibulum. Pellentesque vel massa at felis accumsan rhoncus.

% Suspendisse commodo, massa eu congue tincidunt, elit mauris pellentesque orci, cursus tempor odio nisl euismod augue. Aliquam adipiscing nibh ut odio sodales et pulvinar tortor laoreet. Mauris a accumsan ligula. Class aptent taciti sociosqu ad litora torquent per conubia nostra, per inceptos himenaeos. Suspendisse vulputate sem vehicula ipsum varius nec tempus dui dapibus. Phasellus et est urna, ut auctor erat. Sed tincidunt odio id odio aliquam mattis. Donec sapien nulla, feugiat eget adipiscing sit amet, lacinia ut dolor. Phasellus tincidunt, leo a fringilla consectetur, felis diam aliquam urna, vitae aliquet lectus orci nec velit. Vivamus dapibus varius blandit.

% Duis sit amet magna ante, at sodales diam. Aenean consectetur porta risus et sagittis. Ut interdum, enim varius pellentesque tincidunt, magna libero sodales tortor, ut fermentum nunc metus a ante. Vivamus odio leo, tincidunt eu luctus ut, sollicitudin sit amet metus. Nunc sed orci lectus. Ut sodales magna sed velit volutpat sit amet pulvinar diam venenatis.

% Albert Einstein discovered that $e=mc^2$ in 1905.

% \[ e=\lim_{n \to \infty} \left(1+\frac{1}{n}\right)^n \]

\makeletterclosing

%\clearpage\end{CJK*}                              % if you are typesetting your resume in Chinese using CJK; the \clearpage is required for fancyhdr to work correctly with CJK, though it kills the page numbering by making \lastpage undefined

% \input{cv}
