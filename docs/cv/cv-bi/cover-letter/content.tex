I’m looking to join a great team, are you it? That would be awesome.

I’ve worked as a Technical Writer for a year. 
And then less than a year as a Project Manager.
Now I am looking to work directly
with the product area, and even closer with developers. 
In the past I did research as an American
Master's Sociologist based in Argentina and Brazil. 
Soon, I look forward to 2022, I’ll be completing my Thesis on System's Analysis.

Are you in need of a Product Manager? 


I look forward to having a conversation with you.
% Best regards

% yada yada

% En el año 2016 empecé la adscripción en la cátedra de Metodología de la Investigación. Asistí al horario de las clases prácticas de los jueves, que estaban a cargo de Angela Alessio y en la que se planteaban los tópicos para el desarrollo de un proyecto de investigación. Los alumnos se ordenaron por grupos temáticos y cada grupo desarrollaba un texto de pre-proyecto para ir revisando clase a clase. En esta instancia de revisión los adscriptos tomábamos grupos específicos haciendo el acompañamiento de los alumnos señalando posibles direcciones para continuar el trabjo redactado y modificar y/o agregar secciones. 
% 
% En mi caso trabajé con dos grupos de alumnos. El primero de los grupos eran alumnas que trataron el tema de arte autogestivo con el caso empírico cordobés de un nodo central que reunía grupos de artistas de diferentes medios como ser teatro, publicaciones independientes, músicos y artistas visuales. El segundo grupo de alumnos trataron el tema de la educación Waldorf en una escuela de Villa Allende y plantearon el tema de la disciplina y la libertad creativa que este tipo de método de enseñanza facilitaba a los alumnos, a los padres, y a toda la comunidad educativa.
% 
% En ambos casos se hicieron las revisiones semana a semana, acompañando a los alumnos en las clases y también por medio del intercambio de los materiales vía e-mail. Todos los grupos trabajaban con conciencia de estar aprendiendo una metodología general para todo el curso porque además de los temas dictadas y las revisiones por grupo en las clases siempre se hablaba del trabajo de los grupos de manera que cada alumno conociera el tema y modo de trabajar de sus compañeros.
% 
% Del lado de adscripto fue muy productivo revisar y seguir estudiando la metodología en las ciencias sociales y fue además muy interesante conocer las inquietudes de los alumnos, sus maneras de trabajo, y los contenidos desarrollados en sus temas empíricos seleccionados.

%Lorem ipsum dolor sit amet, consectetur adipiscing elit. Duis ullamcorper neque sit amet lectus facilisis sed luctus nisl iaculis. Vivamus at neque arcu, sed tempor quam. Curabitur pharetra tincidunt tincidunt. Morbi volutpat feugiat mauris, quis tempor neque vehicula volutpat. Duis tristique justo vel massa fermentum accumsan. Mauris ante elit, feugiat vestibulum tempor eget, eleifend ac ipsum. Donec scelerisque lobortis ipsum eu vestibulum. Pellentesque vel massa at felis accumsan rhoncus.

% Suspendisse commodo, massa eu congue tincidunt, elit mauris pellentesque orci, cursus tempor odio nisl euismod augue. Aliquam adipiscing nibh ut odio sodales et pulvinar tortor laoreet. Mauris a accumsan ligula. Class aptent taciti sociosqu ad litora torquent per conubia nostra, per inceptos himenaeos. Suspendisse vulputate sem vehicula ipsum varius nec tempus dui dapibus. Phasellus et est urna, ut auctor erat. Sed tincidunt odio id odio aliquam mattis. Donec sapien nulla, feugiat eget adipiscing sit amet, lacinia ut dolor. Phasellus tincidunt, leo a fringilla consectetur, felis diam aliquam urna, vitae aliquet lectus orci nec velit. Vivamus dapibus varius blandit.

% Duis sit amet magna ante, at sodales diam. Aenean consectetur porta risus et sagittis. Ut interdum, enim varius pellentesque tincidunt, magna libero sodales tortor, ut fermentum nunc metus a ante. Vivamus odio leo, tincidunt eu luctus ut, sollicitudin sit amet metus. Nunc sed orci lectus. Ut sodales magna sed velit volutpat sit amet pulvinar diam venenatis.

% Albert Einstein discovered that $e=mc^2$ in 1905.

% \[ e=\lim_{n \to \infty} \left(1+\frac{1}{n}\right)^n \]

\makeletterclosing

%\clearpage\end{CJK*}                              % if you are typesetting your resume in Chinese using CJK; the \clearpage is required for fancyhdr to work correctly with CJK, though it kills the page numbering by making \lastpage undefined

% \documentclass[a4paper,hidelinks]{twentysecondcv} % a4paper for A4 % https://latex.org/forum/viewtopic.php?t=31329 CHANGING TITLES IN SECTIONS // ABOUT ME ... ETC.
\usepackage[spanish]{babel} \selectlanguage{spanish}
\usepackage[utf8]{inputenc} \usepackage[colorlinks=true,linkbordercolor = {black}]{hyperref} \usepackage{xcolor}
\DeclareUnicodeCharacter{262D}{\hamsic}
\hypersetup{colorlinks, linkcolor={red!50!black}, citecolor={blue!50!black}, urlcolor={blue!55!black} }
\usepackage{contour}% small caps
% \usepackage{fontspec} % pico y pala

% Symbol
\usepackage{tikz}
\usetikzlibrary{svg.path}

% SVG path from Sarang:
% https://upload.wikimedia.org/wikipedia/commons/a/ac/U%2B262D.svg
\newcommand*{\hamsic}{%
  \begingroup
    \settoheight{\dimen0 }{H}%
    \resizebox{!}{\dimen0 }{%
      \tikz\fill svg[yscale=-1]{%
        M67,80l13,19L130,65L108,52z%
        M54,103A60,60 0 1,0 55,23A50,50 0 1,1 55,101z%
        m0,0L14,160l19,10L64,110z%
        M95,87l46,79l17,-12L108,79z%
      };%
    }%
  \endgroup
}

%----------------------------------------------------------------------------------------
%	 PERSONAL INFORMATION
%----------------------------------------------------------------------------------------
% If you don't need one or more of the below, just remove the content leaving the command, e.g. \cvnumberphone{}
\profilepic{avatar.png} % Profile picture
\cvname{Benjamín Juárez} \cvjobtitle{☭ Sociólogo} \cvdate{\today} \cvaddress{Córdoba, Argentina} % Short address/location, use \newline if more than 1 line is required
\cvnumberphone{(+54 9 351) 153 104043} \cvsite{http://ramoneando.com/} \cvmail{benjij$\_$1980@yahoo.com}
% benjaminjuarezarlt@gmail.com
%----------------------------------------------------------------------------------------
\begin{document}
%----------------------------------------------------------------------------------------
%	 ABOUT ME
%----------------------------------------------------------------------------------------
% \newcommand*{\test}[1]{%
%   #1:&\fontspec{#1}\symbol{"262D}\\%
% }
\aboutme{%
Hice Sociología a nivel de Licenciatura y Maestría. Recientemente trabajé fuera de la universidad. Armé proyecto propio: un bar de jugos de fruta natural, \textit{Juice Bar}. También conocí el mundo empresarial trabajando en \textit{Pearson} entre 2017 y 2018.
} % To have no About Me section, just remove all the text and leave \aboutme{}
%----------------------------------------------------------------------------------------
%	 SKILLS
%----------------------------------------------------------------------------------------
 % Skill bar section, each skill must have a value between 0 an 6 (float)
\skills{%
{portugués (fluído)/4.7},
{\textsc{INGLÉS (NATIVO)}/6},
{esfuerzo/5.9},
{atención al detalle/5.7},
{buenos modales/5.3}%
}

%------------------------------------------------

% Skill text section, each skill must have a value between 0 an 6
\skillstext{%
{usuario linux/desde 2005 [hoy i3wm]},
{teclado dvorak/desde 2004},
{capacidad de aprendizaje/9,299},
{relajado/6,5 \newline},
{social / > $ $ \href{<https://ssbc.github.io/secure-scuttlebutt/>}{scuttlebutt} \newline},
{música / > $ $ \href{<https://github.com/fermentation/ferment>}{ferment}}%,
}

\tech{%
- web: html css md javascript \newline
% web: html css md javascript 
% - \normalsize{\textb{web: html css md javascript}} 
% \textbf{\normalsize{E}\scriptsize{ITHER} \normalsize{O}\scriptsize{NE} \normalsize{B}\scriptsize{UT} \normalsize{N}\scriptsize{OT} \normalsize{B}\scriptsize{OTH}} \newline
- edición-tecnología: \LaTeX $ $ vim ranger
- crm: salesforce \newline
- programación: php lua node.js java \newline
% - editing: vim ranger  \newline %
% - markup: md \LaTeX $ $ html css \newline %
% - web: php javascript vcs (git)\newline
% - programming: java lua node.js 
}
% \tech{%
% hola
% }
% - editing: vim ranger  \newline %
% - markup: md \LaTeX $ $ html css \newline %
% - web: php javascript vcs (git)\newline
% - programming: java lua node.js 
% }



%----------------------------------------------------------------------------------------

\makeprofile % Print the sidebar

\vfill 

%----------------------------------------------------------------------------------------
%	 INTERESTS
%----------------------------------------------------------------------------------------

% \section{Intereses}
% 
% The heroine and the dreamer of Wonderland; Alice is the principal character.

% \test{DejaVu Sans}
% \test{FreeSans}
% \test{Segoe UI Symbol}

\section{Perfil General}

% Symbol: ☭

Tengo buena capacidad de adaptación, interés en áreas nuevas y la constancia para empujar de manera continua. Me crié en Córdoba desde los 7 años. Antes viví en EEUU. Desarrollé desde esa temprana edad gusto por el idioma y estudié inglés hasta un nivel nativo C1 [CEFR] y lo uso a diario para leer y escribir. También aprendí a leer, escuchar y hablar portugués de manera fluida, teniendo una estancia en Brazil durante más de dos años. Programación: autodidacta.\\

\section{Trabajo reciente}

\begin{twentyshort} % Environment for a short list with no descriptions
	\twentyitemshort{desde 2018}{IICANA Valle Escondido [Córdoba, Argentina]} 
	\twentyitemshort{desde 2015}{Cursos de conversación en inglés [Córdoba, Argentina]}
	\twentyitemshort{desde 2012}{Clases de Sociología [Argentina-Brasil]}
	\twentyitemshort{2017-2018}{Pearson [Córdoba, Argentina]} 
	\twentyitemshort{2017-2018}{Juice Bar [Córdoba, Argentina]}
	\twentyitemshort{2012-2014}{Beca de investigación \textsc{CNPq} en Sociología [S\~{a}o Paulo, Brasil]}
	\twentyitemshort{2009--2010}{Librería Borders [Atlanta, EEUU]}
	%\twentyitemshort{<dates>}{<title/description>}
\end{twentyshort}


%----------------------------------------------------------------------------------------
%	 EDUCATION
%----------------------------------------------------------------------------------------

\section{Formación}

\begin{twenty} % Environment for a list with descriptions
 	\twentyitem{desde 2017}{Candidato doctoral en Sociología [a distancia]}{University of Exeter, UK}{El æfecto de los atletas de distancia en el ritmo urbano: cómo corren los ultramaratonistas en ciudades automatizadas?}
 	\twentyitem{desde 2013}{Lenguajes de programación-edición}{aprendiendo al hacer}{Autopubliqué libro [\textit{exhalaciones}] y sitio web [\texttt{ramoneando.com}]}
	\twentyitem{2012-2014}{Investigación de Maestría en Sociología}{UNICAMP, S\~{a}o Paulo. Brazil}{Arte Urbano [graffiti y pixaç\~{a}o]}
	\twentyitem{2002-2009}{Licenciado en Sociología}{Universidad de Buenos Aires, Argentina}%
	{%Estudios de graduación
	}
	%\twentyitem{<dates>}{<title>}{<location>}{<description>}
\end{twenty}

\section{Publicaciones/Presentaciones}

\begin{twenty} % Environment for a list with descriptions
	\twentyitemshort{2017}{being ralphy wiggum}
	\twentyitemshort{2016}{The visual and social indeterminacy of pixação: the inextricable moods of São Paulo’s inscriptions}
% 	\twentyitemshort{2016}{exhalaciones}
	\twentyitemshort{2014}{Espacios abiertos: la calle como hábitat}
	\twentyitemshort{2013}{Arte urbano paulistano: degradación urbana y paisajismo turístico}
	\twentyitemshort{2012}{Urban Geopolitics: Recongurations in Art, Activism, and Research}
	\twentyitemshort{2011}{(No) es tan fácil ser grafitero}
	\twentyitemshort{2008}{El legado de Durkheim en Schutz: hacia un horizonte en diálogo}
	\twentyitemshort{2008}{Observaciones sobre la opinión pública a partir de El ciudadano bien informado}
	%\twentyitem{<dates>}{<title>}{<location>}{<description>}
\end{twenty}


\section{Traducciones}

\begin{twenty} % Environment for a list with descriptions
	\twentyitemshort{2018... }{23 ideas sobre la juventud [Howard Becker]}
	\twentyitemshort{2011}{La influencia de las cosmovisiones en el pasado y en el presente según Max Weber [Stephen Kalberg]}
	\twentyitemshort{2005}{Los tipos de racionalidad de Max Weber: piedras angulares para el análisis del proceso de racionalización de la historia [S. Kalberg]}
	%\twentyitem{<dates>}{<title>}{<location>}{<description>}
\end{twenty}

%----------------------------------------------------------------------------------------
%	 OTHER INFORMATION
%----------------------------------------------------------------------------------------

\section{Información general}

% \subsection{Review}

Entre 2017-2018 trabajé en la editorial \textit{Pearson} como Asesor de Servicios de aprendizaje: me reunía con docentes, coordinadores y directivos de institutos y colegios para hacer seguimiento y recomendación de materiales de trabajo. 
Antes aprendí en sociedad comercial a gestionar \textit{Juice Bar}.
\\
% Tengo buena capacidad de adaptación, interés en áreas nuevas y la constancia para empujar de manera continua. Me crié en Córdoba desde los 7 años. Antes viví en EEUU. Desarrollé desde edad gusto por el idioma y estudié inglés hasta un nivel C1 [CEFR] y lo uso a diario para leer y escribir. También aprendí a leer, escuchar y hablar portugués de manera fluida, teniendo una estancia en Brazil durante más de dos años.\\
En los últimos cinco años además trabajé académicamente: supervisando proyectos de investigación, dando clases de sociología y cursos de conversación en inglés. En todos los casos me interesa trabajar con un ambiente relacionado al mundo editorial, de programación/gestión y de servicio.

% materiales de texto que sirvan no solamente en los ambientes especializados de sociología y de inglés, sino para el aprendizaje del público general.
% Alice approaches Wonderland as an anthropologist, but maintains a strong sense of noblesse oblige that comes with her class status. She has confidence in her social position, education, and the Victorian virtue of good manners. Alice has a feeling of entitlement, particularly when comparing herself to Mabel, whom she declares has a ``poky little house," and no toys. Additionally, she flaunts her limited information base with anyone who will listen and becomes increasingly obsessed with the importance of good manners as she deals with the rude creatures of Wonderland. Alice maintains a superior attitude and behaves with solicitous indulgence toward those she believes are less privileged.

\section{Otros proyectos}

\begin{twenty} % Environment for a list with descriptions
	\twentyitemshort{2016-hoy}{\textit{ediciones inextricables} [publicación independiente]}
	\twentyitemshort{2014-hoy}{Sitio web personal: \texttt{ramoneando.com}}
% 	\twentyitemshort{2013-hoy}{herramientas de programación: html [+ css javascript md vim]}
	\twentyitemshort{2009-hoy}{herramientas de edición de texto y diseño: \LaTeX}
% 	\twentyitemshort{2005-hoy}{Sistemas operativos: microsoft, apple, gnu/linux}
	%\twentyitem{<dates>}{<title>}{<location>}{<description>}
\end{twenty}

%----------------------------------------------------------------------------------------
%	 SECOND PAGE EXAMPLE
%----------------------------------------------------------------------------------------

%\newpage % Start a new page

%\makeprofile % Print the sidebar

%\section{Other information}

%\subsection{Review}

%Alice approaches Wonderland as an anthropologist, but maintains a strong sense of noblesse oblige that comes with her class status. She has confidence in her social position, education, and the Victorian virtue of good manners. Alice has a feeling of entitlement, particularly when comparing herself to Mabel, whom she declares has a ``poky little house," and no toys. Additionally, she flaunts her limited information base with anyone who will listen and becomes increasingly obsessed with the importance of good manners as she deals with the rude creatures of Wonderland. Alice maintains a superior attitude and behaves with solicitous indulgence toward those she believes are less privileged.

%\section{Other information}

%\subsection{Review}

%Alice approaches Wonderland as an anthropologist, but maintains a strong sense of noblesse oblige that comes with her class status. She has confidence in her social position, education, and the Victorian virtue of good manners. Alice has a feeling of entitlement, particularly when comparing herself to Mabel, whom she declares has a ``poky little house," and no toys. Additionally, she flaunts her limited information base with anyone who will listen and becomes increasingly obsessed with the importance of good manners as she deals with the rude creatures of Wonderland. Alice maintains a superior attitude and behaves with solicitous indulgence toward those she believes are less privileged.

%----------------------------------------------------------------------------------------

\vfill 
\end{document} 

%%%%%%%%%%%%%%%%%%%%%%%%%%%%%%%%%%%%%%%%%
% Twenty Seconds Resume/CV
% LaTeX Template
% Version 1.1 (8/1/17)
%
% This template has been downloaded from:
% http://www.LaTeXTemplates.com
%
% Original author:
% Carmine Spagnuolo (cspagnuolo@unisa.it) with major modifications by 
% Vel (vel@LaTeXTemplates.com)
%
% License:
% The MIT License (see included LICENSE file)
%
%%%%%%%%%%%%%%%%%%%%%%%%%%%%%%%%%%%%%%%%%

%----------------------------------------------------------------------------------------
%	PACKAGES AND OTHER DOCUMENT CONFIGURATIONS
%----------------------------------------------------------------------------------------

