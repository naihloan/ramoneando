    
    
			\ecvsection{Presentaciones}
  
  \ecvblueitem{2017}{\textit{Calzado deportivo como tecnología: trayectorias socio-técnicas del talón elevado/acolchonado}. \textsc{XXXI Congreso ALAS.}
  2017 – Montevideo. December 3 – 8. }
%   la Asociación Latinoamericana de Sociología
%   Sport shoes as technology: socio-technical trajectories of the elevated/cushioned support

%   \ecvblueitem{2017}{\textit{Urban inscription is not only about crisis. The case of Yellow Stars as a legalized protest against deadly traffic accidents.}
%   In preparation for \textsc{XII Reunión de Antropología del Mercosur}  RAM 2017 – Posadas. December 4 – 7. }

  \ecvblueitem{2014}{\textit{
%   Open spaces: the street as habitat.%   %
  Espacios abiertos: la calle como hábitat.
  }
  \textsc{I Congreso de Epistemologías Críticas en el campo del Habitat}. Facultad de Arquitectura y Urbanismo [FAU] – Córdoba, Argentina.}
 
  \ecvblueitem{2013}{\textit{1978-2013 
  Arte urbano paulistano: entre degradación urbana y paisajismo turístico
%   Paulistano urban art: between urban degradation and turistic landscaping
  }. At \textsc{X Reunión de Antropología del Mercosur}. RAM 2013 – Córdoba, Argentina.}

  \ecvblueitem{2012}{Workshop: \textit{Urban Geopolitics: Recongurations from Art, Activism, and Research. Hemispheric Institute of Performance \& Politics / Convergence. The geo/body politics of emancipation}. Duke University – Durham, USA.}

  \ecvblueitem{2011}{``¿Hay un orden jurídico? Escudriñando el terreno, las fundaciones de lo social-legal%   Is there a juridical order? Looking back at the terrain, foundations of the socio-legal
''.  \textit{1er Encuentro-Debate en Estudios de Posgrado en Sociología Jurídica}. Córdoba, Argentina. }
  
  \ecvblueitem{2011}{\textit{
%   It is (not) so easy to be a graffiti artist
  (No) es tan fácil ser grafitero}. \textsc{III Jornadas de Antropología Social del Centro}. Campus Olavarría – Provincia de Buenos Aires, Argentina.}