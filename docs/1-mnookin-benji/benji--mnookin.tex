
\documentclass[10pt, a4paper, twocolumn]{article}
\input{structure.tex} % Specifies the document structure and loads requires packages
\showhyphens{syllable breaking algorithm opportunities}


\title{Benji J} % The article title

\author{
	\authorstyle{Product Manager | SaaS, % 	and Mobile App,
	Music, Web3/Blockchain/Crypto
	} \\ \\
	\noindent\fbox{%
    \parbox{\textwidth}{%
%         The quick brown fox jumps right over the lazy dog. the quick brown fox jumps right over the lazy dog. the quick brown fox jumps right over the lazy dog. the quick brown fox jumps right over the lazy dog. the quick brown fox jumps right over the lazy dog. the quick brown fox jumps right over the lazy dog. the quick brown fox jumps right over the lazy dog. the quick brown fox jumps right over the lazy dog.
% %
\textbf{This Document Goal}
is to better understand:
1) what I have to offer, how the market sees me;
2) opportunities I haven’t considered; and
3) connect with people you think I should talk with. \\ \\
% %
\textbf{Candidate-Market Fit} \\ \textit{%
Seeking a Product Manager remote role with attention to UX at a Series-A to C SaaS-based tech company in Crypto, ideally with social impact, B2C or B2B2C.
Ideal: based in Argentina or Latin America, in Spanish or Portuguese.
\\
Location preference: Remote, NYC timezone. OK travel 3/4 times per year
% \\ \\
}
% \textbf{This Document Goal}
% is to better understand:
% 1) what I have to offer, how the market sees me;
% 2) opportunities I haven’t considered; and
% 3) connect with people you think I should talk with.
    }%
}
\\ \\
%
\textbf{Summary} \\
With a background in Project Management and Sociology, Benji has recently been in Product Management roles, building with design and development teams.
He works in small teams to create small impactful launches from ideation to post-launch.
Attends to UX flows, copywriting, research, data analytics, strategy and growth.
Likes Web3/Crypto as a needed personal tool for stable coins, and learning about finance and IP rights. \\ \\
\textbf{Career Goals} \\ % % % % % % % % % % % %
\textit{Short-Term (6 months)} \\
Write user stories to gain understinding across teams.
Tools: Use operational data to measure, build and launch.
\\ \\ % % % % % % % % % % % %
  \textit{Mid-Term (6+ months --- 3 years)} \\
Develop strategy skills, focus on user/s journey,
discovery, becoming data+metric fluent, positive team impact
\\ \\ % % % % % % % % % % % %
\textit{Long-Term (5+ years)} \\
Clarifying and resolving ambiguity at a high business level.
Nurture a vision, mission, new market segments %\\ \\
% % % % % % % % % % % %
% \hrule
% }
%
% \noindent\fbox{%
%     \parbox{\textwidth}{%
% \textbf{This Document Goal}
% is to better understand:
% 1) what I have to offer, how the market sees me;
% 2) opportunities I haven’t considered; and
% 3) connect with people you think I should talk with.
% %         The quick brown fox jumps right over the lazy dog. the quick brown fox jumps right over the lazy dog. the quick brown fox jumps right over the lazy dog. the quick brown fox jumps right over the lazy dog. the quick brown fox jumps right over the lazy dog. the quick brown fox jumps right over the lazy dog. the quick brown fox jumps right over the lazy dog. the quick brown fox jumps right over the lazy dog.
%     }%
% }
}

\date{ }

%----------------------------------------------------------------------------------------

\begin{document}
% \includepdf[pages={1}]{main.pdf}
%
\maketitle % Print the title
\thispagestyle{firstpage} % IMPORTANT FOOTER
% \thispagestyle{headings}
% \thispagestyle{empty}
% \thispagestyle{plain}
% Apply the page style for the first page (no headers and footers)
% \vfill\eject
% -------------------------------------------------------------
% SUMMARY % ABSTRACT % --------------------------------------------------------------
%
% \vfill\eject
%
%
\section*{Love Doing}

\begin{description}
\item[Building useful product.] Better lives for people. % Good!
% % % % % % % % % % % % % % % % % % % % % % %
\item[Getting into customers' shoes.]
Understanding pain points.
Solving them, hopefully!
% % % % % % % % % % % % % % % % % % % % % % %
\item[Envisioning user flows.] Setting up (un)happy paths
% % % % % % % % % % % % % % % % % % % % % % %
% \item[Building trust.] With teammates, other teams, users
% % % % % % % % % % % % % % % % % % % % % % %
\item[Listening to all parties.] Resolving real needs: %,
% both external to the company as well as internally
talking to customers,
working with UXR,
Design,
Data Science,
Development,
Sales,
Marketing,
QA.
% C-Level
\item[Writing requirements.] Understanding needs, %requirements,
taking notes, setting technical specifications
\item[Expectation setting.] Ideating and communicating solutions across teams, resolving dependencies
\item[Delivering.] What impacts most, not late.
\item[Learning.]
From others', %experiences and areas, diving into depth of
keeping up to date on industry and role, testing things out.
% % % % % % % % % % % % % % % % % % % % % % %
% Creating an impact on people’s lives, for the better. %- ideally working for a mission-driven company.
\end{description}


\section*{Must Have}

\begin{description}
 \item[Strong leadership.] Experienced + Product Vision
%  Or less Experienced but with Drive and a compelling Roadmap
  \item[Team guidance.] Product informed cycle: CPO-CTO
 \item[Useful product.] B2C ideally. B2B might work
  \item[Industry.]
  Crypto,
  Urban,
  Travel,
  Social Impact,
  Wellness and Fitness,
  Cyber Security,
  UX and UI Design \\
  \item[Nice to have.]
%  \item[Good culture.]
 Good culture: People gel, meet, retreats %  Great Place to Work is amazing
%   Creative,
%   Entertainment,
%   Health,
\end{description}


% \subsection*{Nice To Have}
% \begin{description}
%  \item lalala
% %  \item lalala
% %  \item lalala
% %  \item lalala
% \end{description}


% \end{document}







% This sentence requires citation \citep{Reference1}. This sentence requires multiple citations to imply that it is better supported \citep{Reference2,Reference3}. Finally, when conducting an appeal to authority, it can be useful to cite a reference in-text, much like \cite{Reference1} do quite a bit. Oh, and make sure to check out the bear in Figure \ref{bear}.

% \onecolumn

% \vfill\eject

% \end{twocolumn}
%
% \begin{twocolumn}


% \section*{Not Good}
%
%
% Lorem ipsum dolor sit amet, consectetur adipiscing elit. Fusce maximus nisi ligula. Morbi laoreet ex ligula, vitae lobortis purus mattis vel. Vestibulum ante ipsum primis in faucibus orci luctus et ultrices posuere cubilia Curae; Donec ac metus ut turpis mollis placerat et nec enim. Duis tristique nibh maximus faucibus facilisis. Praesent in consequat leo. Maecenas condimentum ex rhoncus, elementum diam vel, malesuada ante. Fusce pulvinar, mauris pretium placerat venenatis, lectus ex tempus lacus, id suscipit libero lorem eu augue. Interdum et malesuada fames ac ante ipsum primis in faucibus.
%

\vfill\eject

\section*{Hate Doing}

\begin{description}
 \item[Telling others to do things, pushing.]
%  In a team effort,
 I seek %we are looking for space for
 collaboration, %but I'm
 not to be my teammates' boss.
 \item[Overworking.] Except one sprint at night or weekend once a quarter, or once a year.
 Not a given.
 \item[Useless meetings.] Goal: <2 per day,  <5/6 per week. %  No unlimited meeting invites please.
 \item[Reporting.]
%  Stakeholders need to
 I communicate progress, %with focus on
 and delivering value. %, %not presentation.
 Else can become micro-management.
  \item[Bureaucracy.] Good PMs negotiate well.
  % good at handling several stakeholders %can have good outcomes or bad ones % depends on many factors, including company direction:
  But stakeholders overstepping hurts the building process.
% % % % % % % % % % % %
%  I dislike delaying results and delivery:
%  It's OK to chill
%  but  make the product grow, deliver value, on time;
%  for users, team, business.
\end{description}

% \smallskip

%   \vspace*{-20pt}

\section*{Must Not Have}

\begin{description}
  \item[Unbalanced work-life.] Management has no OOO
  \item[Engineering-only lead.]
  Ownership and drive are needed.
  But please: %building decided by
  devs, don't build alone.
%   only may not lead to good product.
%   with product decisions, not going over the side.
 \item[Underpaid.] Regional wages are lesser than global
 \item[Lacks vision.] Direction is not set by product, nor users. Organizational Rigidity.
  \item[Industry.] Government, Taxes, Ads, Sales % Marketing
%  \item[Good Culture.] People actually gel, meet, retreats %  Great Place to Work is amazing
%  \item[Useful Product.] B2C ideally. B2B might work
%   \item[Industry.]
%   Crypto,
%   Urban,
%   Travel,
%   Social Impact
%   Creative,
%   Entertainment,
%   Health,
\end{description}


% This sentence requires citation \citep{Reference1}. This sentence requires multiple citations to imply that it is better supported \citep{Reference2,Reference3}. Finally, when conducting an appeal to authority, it can be useful to cite a reference in-text, much like \cite{Reference1} do quite a bit. Oh, and make sure to check out the bear in Figure \ref{bear}.

% \end{document}


% \section*{Personality}
%
% \subsection*{Myers Briggs}
% \begin{description}
%
% %  \item[INTJ: Architect.] Architects are imaginative and strategic thinkers, with a plan for everything.
%  \item[INTJ: Architect] %Architects are
%  \item[] Imaginative + Strategy / Planning
% %  thinkers, with a plan for everything.
% \end{description}
%
%
%  \subsection*{Enneagram}
%  \begin{description}
%
%  \item[Type 5 / Wing 4 -- The Iconoclast] %\newline
%  \item[] creativeness + sensitivity
% %  is a subtype that results from the encounter of types 5 and 4, which means that the sensitivity of the 4s is added to the natural creativeness of 5s.
%  \end{description}

%  -%Architects are imaginative and strategic thinkers, with a plan for everything.


% \end{document}


\clearpage

\section*{Strengths \& Weaknesses}

My Strengths are based on my Gallup Strengths:
\begin{description}
 \item [Believer]
 \item [Philomath]
 \item [Coach]
 \item [Self-Believer]
 \item [Strategist]
\end{description}

\noindent
My areas to improve come from Personality overall.

\subsection*{Strengths}

\begin{description}
 \item[Belief] in: people's capabilities and that they can bring a lot to the table, from both a work and personal view.
 self, others, team, company.
 My view %of the company and team
 makes me %super
 high drive, and persistent.
\item[Curious.] I love learning, but my fuel is in placing questions forward and not staying stagnant with a simple eternal answer. Iteration and nuance are key.
\item[Supportive.] I push causes to happen, and I believe we can all give something out to the world. wagmi: we're all gonna make it.
\item[Ownership.] I believe to have both confidence and certainty to move forward, and I strive for other people to gain them as well.
\item[Strategist.] Always aiming for the big picture: am I going to be happy about this work in 20 to 50/200 years? I hope so. Let's plan steps and execute!
 \end{description}

\subsection*{Weaknesses}

\begin{description}
 \item[Contrarian:] sometimes wrongly, :P
 \item[] \> \> \>
%  \hrule
%  $ $ $ $
 I believe in people's intent and their heart-felt beliefs and actions. At the same time,
 I can disagree with their ideas and perspectives,
 and it's a challenge to communicate both angles.
%  at the same time
 \item[Interruptor:] a fine art that ought never be learned :P \\
 I get excited about a conversation and want to contribute. Video calls do help with this:
 because you have to unmute, and really
 check if the other person is OK and wrapped their idea.
%  pay attention to pauses in conversation before speaking.
 \item[Overly Imaginative:] going off rail to derivatives can be unproductive for working on immediate goals. Let's better plan one step at a time %so we can
 to stay in sync
 \item[Data insufficient:] I'm biased to teamwork and intuition. It's OK to build quickly and iterate, but %it would  be
 it's best to plan strategies with hard-data. % based decisions.
 It's one of my next steps, but not there yet.
%  \item[Unexperienced Professional Background:] no well-known, big-name company on my resume as an employer. Primary experience in unstructured companies, mostly regional, in LatinAmerica, and in a promising Music Startup (with HQ in US and Europe) but that hasn't yet gained traction.
%  \item[]
\item[Stakeholder Influence.] Even though I feel in general to have good relationships across teams, I may need to review what my leverage is, and how much I can or should influence across company.
 \end{description}

\vfill\eject


\section*{Personality}

% \subsection*{Gallup Strengths (*)}
% \begin{description}
%  \item [Believer]
%  \item [Philomath]
%  \item [Coach]
%  \item [Self-Believer]
%  \item [Strategist]
% \end{description}



\subsection*{Myers Briggs}
\begin{description}

%  \item[INTJ: Architect.] Architects are imaginative and strategic thinkers, with a plan for everything.
 \item[INTJ: Architect] %Architects are
 \item[] Imaginative + Strategy / Planning
%  thinkers, with a plan for everything.
\end{description}


 \subsection*{Enneagram}
 \begin{description}

 \item[Type 5 / Wing 4 -- The Iconoclast] %\newline
 \item[] creativeness + sensitivity
%  is a subtype that results from the encounter of types 5 and 4, which means that the sensitivity of the 4s is added to the natural creativeness of 5s.
 \end{description}

%  -%Architects are imaginative and strategic thinkers, with a plan for everything.

\noindent\fbox{%
    \parbox{0.47 \textwidth \fboxsep \fboxsep}{% the quick brown fox jumps right over the lazy dog.
%     }%
% }

% \fbox{%
% \begin{minipage}{25em}
% lalala
% \end{minipage}
% }
%
\section*{Extended Summary}
% \textit{%
% Seeking a Product Manager remote role
% with attention to UX
% at a Series-A/C SaaS-based tech company in Crypto,
% ideally with social impact, B2C. \\
% Ideal:
% Argentina/Latin America,
% Spanish/Portuguese.
% }
% \\ \\
I enjoy user feedback roles, discovery and beta testing.
My background is in sociology and systems analysis: now inclining towards data analytics to inform product decisions.
% \\
% Open to growth roles focused on customer experience.
% Prefer a role that includes other Product Manager, or near: above or below.
% Interested in joining a start-up in an in-demand industry. % % % % % % % % %
% If given a choice, my
\subsection*{Preferred industries: crypto, or not}
\begin{description}
%  \item[Crypto]
\item[Social:]
Urban,
Travel,
Social Impact,
Fundraising
\item[Entertainment:]
Publishing,
Gaming,
Ed Tech,
Music
\item[Wellbeing:]
Fitness,
Meditation,
Sleep
% \item[B2B/B2B2C:]
% Cyber Security,
% UX and UI Design,
% Task Management Systems
\end{description}
\\ \\
\subsection*{Company stage preference}     \\
Series A: Product-Market Fit \\
Series B/C: Growth
% I am open to other opportunities as I do not have direct industry experience in these fields.
% My background is in sociology and systems analysis. I enjoy user feedback roles, discovery and beta testing.


% This sentence requires citation \citep{Reference1}. This sentence requires multiple citations to imply that it is better supported \citep{Reference2,Reference3}. Finally, when conducting an appeal to authority, it can be useful to cite a reference in-text, much like \cite{Reference1} do quite a bit. Oh, and make sure to check out the bear in Figure \ref{bear}.

% \onecolumn
% \end{minipage}
% }
    }%
}


\subsection*{Dream Idea}

\begin{description}
 \item[Earthworms everywhere:] %sharing, %in a chain,
 like %as in %the movie
 \href{https://www.imdb.com/title/tt0223897/}
 {Pay It Forward (2000)}
 \item \> \> \>
 I already have the know-how to give to one person at a time, and have been doing so for almost a decade. But how to create the chain reaction?
\end{description}


\vfill\eject

% \end{twocolumn}
%
% \begin{twocolumn}


% \section*{Not Good}
%
%
% Lorem ipsum dolor sit amet, consectetur adipiscing elit. Fusce maximus nisi ligula. Morbi laoreet ex ligula, vitae lobortis purus mattis vel. Vestibulum ante ipsum primis in faucibus orci luctus et ultrices posuere cubilia Curae; Donec ac metus ut turpis mollis placerat et nec enim. Duis tristique nibh maximus faucibus facilisis. Praesent in consequat leo. Maecenas condimentum ex rhoncus, elementum diam vel, malesuada ante. Fusce pulvinar, mauris pretium placerat venenatis, lectus ex tempus lacus, id suscipit libero lorem eu augue. Interdum et malesuada fames ac ante ipsum primis in faucibus.
%
% \vfill\eject
%
% \section*{Hate}
%
% \begin{description}
%  \item[Underpaid.] Regional wages are lesser than global
%   \item[Unbalanced work-life.] Management has no OOO
%     \item[Engineering Lead.] Building decided by devs only
%  \item[Lacks Vision.] Direction is not set by product, nor users
%   \item[Industry.] Government, Taxes, Marketing, Ads, Sales
% %  \item[Good Culture.] People actually gel, meet, retreats %  Great Place to Work is amazing
% %  \item[Useful Product.] B2C ideally. B2B might work
% %   \item[Industry.]
% %   Crypto,
% %   Urban,
% %   Travel,
% %   Social Impact
% %   Creative,
% %   Entertainment,
% %   Health,
% \end{description}
%
% \smallskip
%
% \section*{Must Not Have}
%
% \begin{description}
%  \item[Underpaid.] Regional wages are lesser than global
%   \item[Unbalanced work-life.] Management has no OOO
%     \item[Engineering Lead.] Building decided by devs only
%  \item[Lacks Vision.] Direction is not set by product, nor users
%   \item[Industry.] Government, Taxes, Marketing, Ads, Sales
% %  \item[Good Culture.] People actually gel, meet, retreats %  Great Place to Work is amazing
% %  \item[Useful Product.] B2C ideally. B2B might work
% %   \item[Industry.]
% %   Crypto,
% %   Urban,
% %   Travel,
% %   Social Impact
% %   Creative,
% %   Entertainment,
% %   Health,
% \end{description}



\end{document}

%%%%%%%%%%%%%%%%%%%%%%%%%%%%%%%%%%%%%%%%%
% Wenneker Article
% LaTeX Template
% Version 2.0 (28/2/17)
%
% This template was downloaded from:
% http://www.LaTeXTemplates.com
%
% Authors:
% Vel (vel@LaTeXTemplates.com)
% Frits Wenneker
%
% License:
% CC BY-NC-SA 3.0 (http://creativecommons.org/licenses/by-nc-sa/3.0/)
%
%%%%%%%%%%%%%%%%%%%%%%%%%%%%%%%%%%%%%%%%%

%----------------------------------------------------------------------------------------
%	PACKAGES AND OTHER DOCUMENT CONFIGURATIONS
%----------------------------------------------------------------------------------------


%----------------------------------------------------------------------------------------
%	ABSTRACT
%----------------------------------------------------------------------------------------

% \lettrineabstract{Lorem ipsum dolor sit amet, consectetur adipiscing elit. Fusce maximus nisi ligula. Morbi laoreet ex ligula, vitae lobortis purus mattis vel. Vestibulum ante ipsum primis in faucibus orci luctus et ultrices posuere cubilia Curae; Donec ac metus ut turpis mollis placerat et nec enim. Duis tristique nibh maximus faucibus facilisis. Praesent in consequat leo. Maecenas condimentum ex rhoncus, elementum diam vel, malesuada ante.}

%----------------------------------------------------------------------------------------
%	ARTICLE CONTENTS
%----------------------------------------------------------------------------------------

% \section*{Good}
%
% \begin{description}
%  \item[Strong Leadership.] Experienced + Product Vision
% %  Or less Experienced but with Drive and a compelling Roadmap
%  \item[Good Culture.] People actually gel, meet, retreats %  Great Place to Work is amazing
%  \item[Useful Product.] B2C ideally. B2B might work
%   \item[Industry.] Entertainment, Creative, Travel, Health
% \end{description}

% This sentence requires citation \citep{Reference1}. This sentence requires multiple citations to imply that it is better supported \citep{Reference2,Reference3}. Finally, when conducting an appeal to authority, it can be useful to cite a reference in-text, much like \cite{Reference1} do quite a bit. Oh, and make sure to check out the bear in Figure \ref{bear}.


% % % % % % % % % % % % % % % % %

% 10pt font size (11 and 12 also possible), A4 paper (letterpaper for US letter) and two column layout (remove for one column)
% \usepackage[1]{pagesel}
% https://tex.stackexchange.com/questions/96256/compiling-only-a-page-range-or-page-selection
