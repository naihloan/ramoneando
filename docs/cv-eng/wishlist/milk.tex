\documentclass[twoside]{article} %twocolumn

\usepackage{blindtext} % Package to generate dummy text throughout this template 

\usepackage[sc]{mathpazo} % Use the Palatino font
\usepackage[T1]{fontenc} % Use 8-bit encoding that has 256 glyphs
\linespread{1.05} % Line spacing - Palatino needs more space between lines
\usepackage{microtype} % Slightly tweak font spacing for aesthetics

% \usepackage[english]{babel} % Language hyphenation and typographical rules
\usepackage[spanish]{babel} 
\selectlanguage{spanish}
\usepackage[utf8]{inputenc} %\usepackage[colorlinks=true,linkbordercolor = {black}]{hyperref} \usepackage{xcolor} 

\usepackage[hmarginratio=1:1,top=32mm,columnsep=20pt]{geometry} % Document margins
\usepackage[hang, small,labelfont=bf,up,textfont=it,up]{caption} % Custom captions under/above floats in tables or figures
\usepackage{booktabs} % Horizontal rules in tables

\usepackage{lettrine} % The lettrine is the first enlarged letter at the beginning of the text

\usepackage{enumitem} % Customized lists
\setlist[itemize]{noitemsep} % Make itemize lists more compact

\usepackage{abstract} % Allows abstract customization
\renewcommand{\abstractnamefont}{\normalfont\bfseries} % Set the "Abstract" text to bold
\renewcommand{\abstracttextfont}{\normalfont\small\itshape} % Set the abstract itself to small italic text

\usepackage{titlesec} % Allows customization of titles
\renewcommand\thesection{\Roman{section}} % Roman numerals for the sections
\renewcommand\thesubsection{\roman{subsection}} % roman numerals for subsections
\titleformat{\section}[block]{\large\scshape\centering}{\thesection.}{1em}{} % Change the look of the section titles
\titleformat{\subsection}[block]{\large}{\thesubsection.}{1em}{} % Change the look of the section titles

\usepackage{fancyhdr} % Headers and footers
\pagestyle{fancy} % All pages have headers and footers
\fancyhead{} % Blank out the default header
\fancyfoot{} % Blank out the default footer
\fancyhead[C]{Running title $\bullet$ May 2016 $\bullet$ Vol. XXI, No. 1} % Custom header text
\fancyfoot[RO,LE]{\thepage} % Custom footer text

\usepackage{titling} % Customizing the title section

\usepackage{hyperref} % For hyperlinks in the PDF

%----------------------------------------------------------------------------------------
%	TITLE SECTION
%----------------------------------------------------------------------------------------

\setlength{\droptitle}{-4\baselineskip} % Move the title up

\pretitle{\begin{center}\Huge\bfseries} % Article title formatting
\posttitle{\end{center}} % Article title closing formatting
\title{Technical User Support (Part Time)} % Article title
\author{%
\textsc{Benjamin Juarez}
% \thanks{A thank you or further information} \\[1ex] % Your name
% \normalsize University of California \\ % Your institution
% \normalsize \href{mailto:john@smith.com}{john@smith.com} % Your email address
%\and % Uncomment if 2 authors are required, duplicate these 4 lines if more
%\textsc{Jane Smith}\thanks{Corresponding author} \\[1ex] % Second author's name
%\normalsize University of Utah \\ % Second author's institution
%\normalsize \href{mailto:jane@smith.com}{jane@smith.com} % Second author's email address
}
\date{November 7, 2018} % Leave empty to omit a date
\renewcommand{\maketitlehookd}{%
% \begin{abstract}
% \noindent \blindtext % Dummy abstract text - replace \blindtext with your abstract text
% \end{abstract}
}

%----------------------------------------------------------------------------------------

\begin{document}


% Print the title
\maketitle

\subsection*{Writing Samples}

\begin{itemize}
 \item \textit{Explain in 3 paragraphs or less what Smart Lists are, and why a user might be interested in using them.}
 
Smart Lists are interactive To Do lists. They have the capability of making further anotations once you have moved on with activities, accomplished set goals, and change priorities if needed. One of the interesting points about using Smart Lists is that you can keep track of which are the points of the list that are properly being attended to, and those that are running slowly behind. There are plenty of reasons to use Smart Lists, and they vary widely. We hope that you find several uses to them.
 
 \item \textit{A user is experiencing a problem with the web app only partially loading. Respond to the user suggesting that they clear their cache (feel free to include any other troubleshooting steps you think might help them).}

Hello! We are happy that you are using the Remember The Milk app! If the web app does not load fully you can make it work by cleaning the cache. This is quite simple, in most web browsers it is a matter of accessing through settings/advanced settings and set *Clear browsing data*. Please let us know if this solved the problem. As a shortcut it is also possible to try simply shutting down the browser and then reopening the app, this may be helpful to reset the app. 
 
 \item \textit{A user wants to use our iPhone app, but only has iOS 10. Respond to the user that iOS 11 or above is required.}

We are grateful that you have chosen Remember The Milk! To give our best cutting edge technology we have raised our standard to the latest releases by Apple's Operative Systems. In order to use our app the system used should be iOS 11 or superior. This is a possibility for apple users since hardware is prepared to allow the upgrade from iOS 10 to iOS 11, or above. We hope that you can make that shift and enjoy using our app. Please let us know if this instruction is of help for you to use Remember The Milk on your iPhone.
 
 \end{itemize}


\end{document}
