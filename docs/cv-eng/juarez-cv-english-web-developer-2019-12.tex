%----------------------------------------------------------------------------------------
% \begin{document}
%----------------------------------------------------------------------------------------
%	 ABOUT ME
%----------------------------------------------------------------------------------------

\aboutme{%
% hello
% Hice Sociología a nivel de Licenciatura y Maestría. 
% Recientemente trabajé fuera de la universidad. 
% Armé proyecto propio: un bar de jugos de fruta natural, \textit{Juice Bar}. 
% También conocí el mundo empresarial trabajando en \textit{Pearson} entre 2017 y 2018.} 
}
% To have no About Me section, just remove all the text and leave \aboutme{}

%----------------------------------------------------------------------------------------
%	 SKILLS
%----------------------------------------------------------------------------------------
 % Skill bar section, each skill must have a value between 0 an 6 (float)
 
\skills{%
{portugués (fluído)/4.7},
{\textsc{inglés (nativo)}/6}%
% {esfuerzo/5.9},
% {atención al detalle/5.7},
% {buenos modales/5.3}%
}

%------------------------------------------------

% Skill text section, each skill must have a value between 0 an 6
\skillstext{%
% {usuario linux/desde 2005 [hoy i3wm]},
% {teclado dvorak/desde 2004},
% {capacidad de aprendizaje/9,299},
% {relajado/6,5 \newline},
% {social / > $ $ \href{<https://ssbc.github.io/secure-scuttlebutt/>}{scuttlebutt} \newline},
% {música / > $ $ \href{<https://github.com/fermentation/ferment>}{ferment}}%,
}

\tech{%
- web: html css bootstrap javascript node.js express.js md  \newline
- dvcs: git github \\
- edición: vim \LaTeX \\
- en curso: Jquery Angular ApiRest Mongodb 
Microsoft SQL Server
\newline
- intereses: Python Django Scrum C++ C$\#$ React
% - programación: node.js java  \newline %php lua
% - edición-tecnología: \LaTeX vim %$ $ vim ranger
% - crm: salesforce \newline
}
% web: html css md javascript 
% - \normalsize{\textb{web: html css md javascript}} 
% \textbf{\normalsize{E}\scriptsize{ITHER} \normalsize{O}\scriptsize{NE} \normalsize{B}\scriptsize{UT} \normalsize{N}\scriptsize{OT} \normalsize{B}\scriptsize{OTH}} \newline
% - editing: vim ranger  \newline %
% - markup: md \LaTeX $ $ html css \newline %
% - web: php javascript vcs (git)\newline
% - programming: java lua node.js 

% \tech{%
% hola
% }
% - editing: vim ranger  \newline %
% - markup: md \LaTeX $ $ html css \newline %
% - web: php javascript vcs (git)\newline
% - programming: java lua node.js 
% }

%----------------------------------------------------------------------------------------

\makeprofile % Print the sidebar

\vfill 
%----------------------------------------------------------------------------------------
%	 INTERESTS
%----------------------------------------------------------------------------------------

% \section{Intereses}

% \test{DejaVu Sans}
% \test{FreeSans}
% \test{Segoe UI Symbol}

\section{Perfil General} % Symbol: ☭

Me crié en Córdoba desde los 7 años. Antes viví en EEUU. 
Manejo excelente el inglés y muy bien el portugués. Viví en Brasil 2012–2013.\\
% Desarrollé desde temprana edad gusto por el idioma: estudié inglés hasta nivel nativo C1 [CEFR] y lo uso a diario para leer y escribir. 
% También leo, escucho y hablo portugués fluidamente, estuve en Brazil más de dos años. 
% Programación: varios años autodidacta. Ahora en curso de 
Cursé Desarrollo Web en UTN.
Cursando Analista en la Escuela Superior de Comercio Manuel Belgrano [Terminando primer año completo].
% Tengo buena capacidad de adaptación, interés en áreas nuevas y la constancia de hacer continuadamente.\\

% \section{Reciente}

% \subsection{Review}

En los últimos años vengo trabajando en áreas de %fuera del ámbito académico: en 
editorial, dando clases de inglés, y aprendiendo sobre producción editorial. 
En el último tiempo con foco en desarrollo web, base de datos y programación.
% Entre 2017-2018 trabajé en la editorial \textit{Pearson} como Asesor de Servicios de aprendizaje: me reunía con docentes, coordinadores y directivos de institutos y colegios para hacer seguimiento y recomendación de materiales de trabajo. 
% Antes aprendí en sociedad comercial a gestionar \textit{Juice Bar}.\\
% Tengo buena capacidad de adaptación, interés en áreas nuevas y la constancia para empujar de manera continua. Me crié en Córdoba desde los 7 años. Antes viví en EEUU. Desarrollé desde edad gusto por el idioma y estudié inglés hasta un nivel C1 [CEFR] y lo uso a diario para leer y escribir. También aprendí a leer, escuchar y hablar portugués de manera fluida, teniendo una estancia en Brazil durante más de dos años.\\
% En los últimos cinco años además trabajé académicamente: supervisando proyectos de investigación, dando clases de sociología y cursos de conversación en inglés. En todos los casos me interesa trabajar con un ambiente relacionado al mundo editorial, de programación/gestión y de servicio.

% Alice approaches Wonderland as an anthropologist, but maintains a strong sense of noblesse oblige that comes with her class status. She has confidence in her social position, education, and the Victorian virtue of good manners. Alice has a feeling of entitlement, particularly when comparing herself to Mabel, whom she declares has a ``poky little house," and no toys. Additionally, she flaunts her limited information base with anyone who will listen and becomes increasingly obsessed with the importance of good manners as she deals with the rude creatures of Wonderland. Alice maintains a superior attitude and behaves with solicitous indulgence toward those she believes are less privileged.



%----------------------------------------------------------------------------------------
%	 EDUCATION
%----------------------------------------------------------------------------------------

\vspace{0.5cm}

\section{Formación}

\begin{twenty} % Environment for a list with descriptions
 
% \twentyitemshort{2019-2021}{Analista Universitario de Sistemas Informáticos [ESCMB, CBA]}%{}
\twentyitem{2019-2021}{Analista Universitario de Sistemas Informáticos}{[ESCMB%, Córdoba
]}{Primer Año Completo}
\twentyitem{2018-2019}{Curso de Desarrollo Web}{[UTN%, Córdoba, Argentina
]}{
	Javascript Jquery Angular ApiRest Mongodb Node.js Express.js\\ 
	HTML5 CSS3 Bootstrap}
\twentyitem{2018-2019}{Otros Cursos}{[Udemy]}{
	JavaScript: Understanding the Weird Parts\\ 
	React Basic in just 1 hour\\
	SQLite for Beginners | Learn SQL from Scratch\\
	Absolute Introduction to Object Oriented Programming in Java}
%  \twentyitemshort{desde 2017}{Candidato doctoral en Sociología [University of Exeter, UK]}%{El æfecto de los atletas de distancia en el ritmo urbano: cómo corren los ultramaratonistas en ciudades automatizadas?}
 \twentyitemshort{desde 2013}{Lenguajes de marcación-%programación-
edición [hechos: libro + sitio web]}%{Autopubliqué libro [\textit{exhalaciones}] y sitio web [\texttt{ramoneando.com}]}
%  \twentyitemshort{2012-2014}{Maestría en Sociología [UNICAMP, S\~{a}o Paulo. Brazil]}%{Arte Urbano [graffiti y pixaç\~{a}o]}
%  \twentyitemshort{2002-2009}{Licenciado en Sociología [Universidad de Buenos Aires, Argentina]}%
% 	 %\twentyitem{<dates>}{<title>}{<location>}{<description>}
\end{twenty}

% \section{Publicaciones/Presentaciones}
% 
% \begin{twenty} % Environment for a list with descriptions
% 	\twentyitemshort{2017}{being ralphy wiggum}
% 	\twentyitemshort{2016}{The visual and social indeterminacy of pixação: the inextricable moods of São Paulo’s inscriptions}
% % 	\twentyitemshort{2016}{exhalaciones}
% 	\twentyitemshort{2014}{Espacios abiertos: la calle como hábitat}
% 	\twentyitemshort{2013}{Arte urbano paulistano: degradación urbana y paisajismo turístico}
% 	\twentyitemshort{2012}{Urban Geopolitics: Recongurations in Art, Activism, and Research}
% 	\twentyitemshort{2011}{(No) es tan fácil ser grafitero}
% 	\twentyitemshort{2008}{El legado de Durkheim en Schutz: hacia un horizonte en diálogo}
% 	\twentyitemshort{2008}{Observaciones sobre la opinión pública a partir de El ciudadano bien informado}
% 	%\twentyitem{<dates>}{<title>}{<location>}{<description>}
% \end{twenty}
% 
% 
% \section{Traducciones}
% 
% \begin{twenty} % Environment for a list with descriptions
% 	\twentyitemshort{2019... }{23 ideas sobre la juventud [Howard Becker]}
% 	\twentyitemshort{2011}{La influencia de las cosmovisiones en el pasado y en el presente según Max Weber [Stephen Kalberg]}
% 	\twentyitemshort{2005}{Los tipos de racionalidad de Max Weber: piedras angulares para el análisis del proceso de racionalización de la historia [S. Kalberg]}
% 	%\twentyitem{<dates>}{<title>}{<location>}{<description>}
% \end{twenty}

\vspace{0.5cm}

\section{Trabajo}

\begin{twentyshort} % Environment for a short list with no descriptions
	\twentyitemshort{2018–2019}{Productor Documental [Argentina-Canadá]} 
	\twentyitemshort{2018–2018}{IICANA Valle Escondido [Córdoba, Argentina]} 
	\twentyitemshort{2017–2018}{Pearson Editorial[Córdoba, Argentina]} 
	\twentyitemshort{2015–2018}{Cursos de conversación en inglés [Córdoba, Argentina]}
\end{twentyshort}

% 
%----------------------------------------------------------------------------------------
%	 OTHER INFORMATION
%----------------------------------------------------------------------------------------

\vspace{0.5cm}

\section{Otros proyectos}

\begin{twenty} % Environment for a list with descriptions
	\twentyitemshort{2016-hoy}{\textit{ediciones inextricables} [publicación independiente en papel]}
	\twentyitemshort{2014-hoy}{Sitio web personal: \url{ramoneando.com}}
% 	\twentyitemshort{2013-hoy}{herramientas de programación: html [+ css javascript md vim]}
	\twentyitemshort{2009-hoy}{herramientas de edición de texto y diseño: \LaTeX}
% 	\twentyitemshort{2005-hoy}{Sistemas operativos: microsoft, apple, gnu/linux}
	%\twentyitem{<dates>}{<title>}{<location>}{<description>}
\end{twenty}

\vfill 

% \end{document} 

%----------------------------------------------------------------------------------------
%	 SECOND PAGE EXAMPLE
%----------------------------------------------------------------------------------------

%\newpage % Start a new page

%\makeprofile % Print the sidebar

%\section{Other information}

%\subsection{Review}

%Alice approaches Wonderland as an anthropologist, but maintains a strong sense of noblesse oblige that comes with her class status. She has confidence in her social position, education, and the Victorian virtue of good manners. Alice has a feeling of entitlement, particularly when comparing herself to Mabel, whom she declares has a ``poky little house," and no toys. Additionally, she flaunts her limited information base with anyone who will listen and becomes increasingly obsessed with the importance of good manners as she deals with the rude creatures of Wonderland. Alice maintains a superior attitude and behaves with solicitous indulgence toward those she believes are less privileged.

%\section{Other information}

%\subsection{Review}

%Alice approaches Wonderland as an anthropologist, but maintains a strong sense of noblesse oblige that comes with her class status. She has confidence in her social position, education, and the Victorian virtue of good manners. Alice has a feeling of entitlement, particularly when comparing herself to Mabel, whom she declares has a ``poky little house," and no toys. Additionally, she flaunts her limited information base with anyone who will listen and becomes increasingly obsessed with the importance of good manners as she deals with the rude creatures of Wonderland. Alice maintains a superior attitude and behaves with solicitous indulgence toward those she believes are less privileged.

%----------------------------------------------------------------------------------------

% \vfill 

% \end{document} 

%%%%%%%%%%%%%%%%%%%%%%%%%%%%%%%%%%%%%%%%%
% Twenty Seconds Resume/CV
% LaTeX Template
% Version 1.1 (8/1/17)
%
% This template has been downloaded from:
% http://www.LaTeXTemplates.com
%
% Original author:
% Carmine Spagnuolo (cspagnuolo@unisa.it) with major modifications by 
% Vel (vel@LaTeXTemplates.com)
%
% License:
% The MIT License (see included LICENSE file)
%
%%%%%%%%%%%%%%%%%%%%%%%%%%%%%%%%%%%%%%%%%

%----------------------------------------------------------------------------------------
%	PACKAGES AND OTHER DOCUMENT CONFIGURATIONS
%----------------------------------------------------------------------------------------

% \usepackage{contour}% small caps
% \usepackage{fontspec} % pico y pala

% Symbol
% \usepackage{tikz}
% \usetikzlibrary{svg.path}
% 
% % SVG path from Sarang:
% % https://upload.wikimedia.org/wikipedia/commons/a/ac/U%2B262D.svg
% \newcommand*{\hamsic}{%
%   \begingroup
%     \settoheight{\dimen0 }{H}%
%     \resizebox{!}{\dimen0 }{%
%       \tikz\fill svg[yscale=-1]{%
%         M67,80l13,19L130,65L108,52z%
%         M54,103A60,60 0 1,0 55,23A50,50 0 1,1 55,101z%
%         m0,0L14,160l19,10L64,110z%
%         M95,87l46,79l17,-12L108,79z%
%       };%
%     }%
%   \endgroup
% }

% \newcommand*{\test}[1]{%
%   #1:&\fontspec{#1}\symbol{"262D}\\%
% }
