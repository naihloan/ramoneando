\documentclass[a4paper,hidelinks]{twentysecondcv} % a4paper for A4
% https://latex.org/forum/viewtopic.php?t=31329 CHANGING TITLES IN SECTIONS // ABOUT ME ... ETC.
% \usepackage[spanish]{babel}
\selectlanguage{spanish}
\usepackage[utf8]{inputenc}
\usepackage{slantsc} % small caps

























%----------------------------------------------------------------------------------------
%	 PERSONAL INFORMATION
%----------------------------------------------------------------------------------------

% If you don't need one or more of the below, just remove the content leaving the command, e.g. \cvnumberphone{}

\profilepic{avatar.png} % Profile picture

\cvname{Benjamin Juarez} % Your name
\cvjobtitle{trilingual web developer $ $  $ $  $ $ [US Citizen]} % Job title/career
% \cvname{Benjamín Juárez} % Your name
% \cvjobtitle{Sociólogo} % Job title/career

\cvdate{\today} % Date of birth
\cvaddress{Córdoba, Argentina} % Short address/location, use \newline if more than 1 line is required
\cvnumberphone{(+54 9 351) 153 104043} % Phone number
\cvsite{http://ramoneando.com/} % Personal website
\cvmail{benjaminjuarezarlt@gmail.com} % Email address

%----------------------------------------------------------------------------------------

\begin{document}

%----------------------------------------------------------------------------------------
%	 ABOUT ME
%----------------------------------------------------------------------------------------

\aboutme{% To have no About Me section, just remove all the text and leave \aboutme{}
% I studied and researched in Sociology as a Bachelor as well as a Master.
% My recent jobs have been non academic.
- Born and raised in USA until age 7. \newline %
- Master degree in Sociology: Brasil. \newline %
- I co-founded \textit{Juice Bar}. \newline %
- Worked in \textit{Pearson} Publishing. \newline
- \textit{Fluent Spanish Portuguese English}*.\newline %
* [Born USA, has VISA]
} 

%----------------------------------------------------------------------------------------
%	 SKILLS
%----------------------------------------------------------------------------------------

\skills{% Skill bar section, each skill must have a value between 0 an 6 (float)
{effort/5.9},
{atention to detail/5.7},
{proper manners/5.3}%
}
% }

% ------------------------------------------------

% Skill text section, each skill must have a value between 0 an 6
\skillstext{%
{linux user/since 2005 [now i3wm]},
{dvorak keyboard user/since 2004},
{learning thrust/9,2 \newline},
{relaxed/6,5},
}
% }

\tech{
- markup: html css md \LaTeX $ $ \newline %
- web: javascript php \newline
- editing: vim vcs (git) ranger  \newline %
- programming: lua node.js java \newline 
- ongoing: Jquery Angular ApiRest Mongodb Express.js Bootstrap
}
% }

%----------------------------------------------------------------------------------------

\makeprofile % Print the sidebar

\vfill 

%----------------------------------------------------------------------------------------
%	 INTERESTS
%----------------------------------------------------------------------------------------

% \section{Intereses}
% 
% The heroine and the dreamer of Wonderland; Alice is the principal character.

\section{General Profile}

I have good adaptability, interest in new areas and the tenacity to give a continuous throttle. I was born in Chicago, USA; and raised from age 7 in Córdoba, Argentina. 
I have always kept %developed a taste for 
studying %language, raising my 
English, raising my level to a C1 [CEFR] standard. Also, I use English on a  daily basis by reading and writing. I have also learned to read, listen and speak Portuguese in a fluent manner, living in Brazil through 2 nearly years. Programming: mostly self-taught, now taking a course.
% Tengo buena capacidad de adaptación, interés en áreas nuevas y la constancia para empujar de manera continua. Me crié en Córdoba desde los 7 años. Antes viví en EEUU. Desarrollé desde esa temprana edad gusto por el idioma y estudié inglés hasta un nivel nativo C1 [CEFR] y lo uso a diario para leer y escribir. También aprendí a leer, escuchar y hablar portugués de manera fluida, teniendo una estancia en Brazil durante más de dos años. Programación: autodidacta.\\

\section{Recent work}

\begin{twentyshort} % Environment for a short list with no descriptions
	\twentyitemshort{2018--2019}{Research Assistant [Royal Roads, British Columbia, Canada]} 
% 	\twentyitemshort{2018}{IICANA Valle Escondido [Córdoba, Argentina]} 
	\twentyitemshort{2015--2018}{English Conversation Classes [Córdoba, Argentina]}
	\twentyitemshort{since 2012}{Sociology Classes [Argentina-Brasil]}
	\twentyitemshort{2017-2018}{Pearson [Córdoba, Argentina]} 
	\twentyitemshort{2017-2018}{Juice Bar [Córdoba, Argentina]}
	\twentyitemshort{2012-2014}{\textsc{CNPq} Research Grant in Sociology [S\~{a}o Paulo, Brasil]}
	\twentyitemshort{2009--2010}{Borders Bookstore [Atlanta, USA]}
	%\twentyitemshort{<dates>}{<title/description>}
\end{twentyshort}


%----------------------------------------------------------------------------------------
%	 EDUCATION
%----------------------------------------------------------------------------------------

\section{Education}

\begin{twenty} % Environment for a list with descriptions
% 	\twentyitemshort{since 2018}{IICANA Valle Escondido [Córdoba, Argentina]} 
 	\twentyitem{since 2018}{Web Development Course}{[UTN, Córdoba, Argentina]}{
 	Javascript Jquery Angular ApiRest Mongodb Node.js Express.js\\
 	HTML5 CSS3 Bootstrap
 	}
 	\twentyitemshort{since 2017}{PhD [distance] Sociology [University of Exeter, UK]}%{}
%  	{%
% %  	Distance athlete’s æffect on urban rhythm | How do ultra-runners run in automatized cities?
% 	Distance athletes in urban rhythm | How do ultra-runners run through and out of cities?
% }
 	\twentyitemshort{since 2013}{Programming-editing languages [Hands-on]}%{}
%  	{Self-published book [\textit{exhalations}] and web site [\texttt{ramoneando.com}]}
	\twentyitemshort{2012-2014}{Master %Degree Research 
	in Sociology [UNICAMP, S\~{a}o Paulo. Brazil]}%{}
% 	{Urban Art [graffiti \& pixaç\~{a}o]}
	\twentyitemshort{2002-2009}{Graduate in Sociology [Universidad de Buenos Aires, Argentina]}%
% 	{%Estudios de graduación
% 	}
	%\twentyitem{<dates>}{<title>}{<location>}{<description>}
\end{twenty}

\section{Recent Papers/Presentations}

\begin{twenty} % Environment for a list with descriptions
	\twentyitemshort{2017}{being ralphy wiggum}
	\twentyitemshort{2017}{Sport shoe wear as technology: \\ socio-technical trajectories of high-heel cushioning}
	\twentyitemshort{2016}{The visual and social indeterminacy of pixação: \\ the inextricable moods of São Paulo’s inscriptions}
% 	\twentyitemshort{2016}{exhalaciones}
	\twentyitemshort{2014}{Espacios abiertos: la calle como hábitat}
	\twentyitemshort{2013}{Arte urbano paulistano: degradación urbana y paisajismo turístico}
	\twentyitemshort{2012}{Urban Geopolitics: Recongurations in Art, Activism, and Research}
	\twentyitemshort{2011}{(No) es tan fácil ser grafitero}
% 	\twentyitemshort{2008}{El legado de Durkheim en Schutz: hacia un horizonte en diálogo}
% 	\twentyitemshort{2008}{Observaciones sobre la opinión pública a partir de El ciudadano bien informado}
	%\twentyitem{<dates>}{<title>}{<location>}{<description>}
\end{twenty}


\section{Translations}

\begin{twenty} % Environment for a list with descriptions
% 	\twentyitemshort{2018}{Off the Grid -- Re-Assembling Domestic Life [Phillip Vannini]}
	\twentyitemshort{2019}{23 ideas sobre la juventud [Howard Becker]}
	\twentyitemshort{2019}{Research Assistant -- Project \textit{In the Name of Wild} [Phillip Vannini]}
	\twentyitemshort{2011}{La influencia de las cosmovisiones en el pasado y en el presente según Max Weber [Stephen Kalberg]}
	\twentyitemshort{2005}{Los tipos de racionalidad de Max Weber: piedras angulares para el análisis del proceso de racionalización de la historia [S. Kalberg]}
	%\twentyitem{<dates>}{<title>}{<location>}{<description>}
\end{twenty}

%----------------------------------------------------------------------------------------
%	 OTHER INFORMATION
%----------------------------------------------------------------------------------------

\section{Quick view}
% \subsection{Review}

% For six months I worked at \textit{Pearson} Publishing as Learning Services Consultant: I had reunions with teachers, coordinators and directors to understand needs and preferences, and then guide and provide English materials accordingly, also making presentations when needed. 
% En los últimos seis meses trabajé en la editorial \textit{Pearson} como Asesor de Servicios de aprendizaje: me reunía con docentes, coordinadores y directivos de institutos y colegios para hacer seguimiento y recomendación de materiales de trabajo. 
% Before that 
% I learned some comercial and management skills co-founding and working at our \textit{Juice Bar}.
% Antes aprendí en sociedad comercial a gestionar \textit{Juice Bar}.
\\
% Tengo buena capacidad de adaptación, interés en áreas nuevas y la constancia para empujar de manera continua. Me crié en Córdoba desde los 7 años. Antes viví en EEUU. Desarrollé desde edad gusto por el idioma y estudié inglés hasta un nivel C1 [CEFR] y lo uso a diario para leer y escribir. También aprendí a leer, escuchar y hablar portugués de manera fluida, teniendo una estancia en Brazil durante más de dos años.\\
% In the past five years 
I've worked in academic environments: supervising research projects, giving sociology classes and English conversation. %In all cases 
I feel at ease in environments relating to publishing, film production, programming and service.
% En los últimos cinco años además trabajé académicamente: supervisando proyectos de investigación, dando clases de sociología y cursos de conversación en inglés. En todos los casos me interesa trabajar con un ambiente relacionado al mundo editorial y de servicio.

% materiales de texto que sirvan no solamente en los ambientes especializados de sociología y de inglés, sino para el aprendizaje del público general.
% Alice approaches Wonderland as an anthropologist, but maintains a strong sense of noblesse oblige that comes with her class status. She has confidence in her social position, education, and the Victorian virtue of good manners. Alice has a feeling of entitlement, particularly when comparing herself to Mabel, whom she declares has a ``poky little house," and no toys. Additionally, she flaunts her limited information base with anyone who will listen and becomes increasingly obsessed with the importance of good manners as she deals with the rude creatures of Wonderland. Alice maintains a superior attitude and behaves with solicitous indulgence toward those she believes are less privileged.

\section{Other Projects}

\begin{twenty} % Environment for a list with descriptions
	\twentyitemshort{2016-today}{\textit{inextricable publisher} [independent publishing]}
	\twentyitemshort{2014-today}{Personal web site: \texttt{ramoneando.com}}
	\twentyitemshort{2013-today}{programming tools: html [+ css javascript md vim]}
	\twentyitemshort{2009-today}{text and design editing tool: \LaTeX}
% 	\twentyitemshort{2005-hoy}{Sistemas operativos: microsoft, apple, gnu/linux}
	%\twentyitem{<dates>}{<title>}{<location>}{<description>}
\end{twenty}

%----------------------------------------------------------------------------------------
%	 SECOND PAGE EXAMPLE
%----------------------------------------------------------------------------------------

%\newpage % Start a new page

%\makeprofile % Print the sidebar

%\section{Other information}

%\subsection{Review}

%Alice approaches Wonderland as an anthropologist, but maintains a strong sense of noblesse oblige that comes with her class status. She has confidence in her social position, education, and the Victorian virtue of good manners. Alice has a feeling of entitlement, particularly when comparing herself to Mabel, whom she declares has a ``poky little house," and no toys. Additionally, she flaunts her limited information base with anyone who will listen and becomes increasingly obsessed with the importance of good manners as she deals with the rude creatures of Wonderland. Alice maintains a superior attitude and behaves with solicitous indulgence toward those she believes are less privileged.

%\section{Other information}

%\subsection{Review}

%Alice approaches Wonderland as an anthropologist, but maintains a strong sense of noblesse oblige that comes with her class status. She has confidence in her social position, education, and the Victorian virtue of good manners. Alice has a feeling of entitlement, particularly when comparing herself to Mabel, whom she declares has a ``poky little house," and no toys. Additionally, she flaunts her limited information base with anyone who will listen and becomes increasingly obsessed with the importance of good manners as she deals with the rude creatures of Wonderland. Alice maintains a superior attitude and behaves with solicitous indulgence toward those she believes are less privileged.

%----------------------------------------------------------------------------------------

\vfill 
\end{document} 

%%%%%%%%%%%%%%%%%%%%%%%%%%%%%%%%%%%%%%%%%
% Twenty Seconds Resume/CV
% LaTeX Template
% Version 1.1 (8/1/17)
%
% This template has been downloaded from:
% http://www.LaTeXTemplates.com
%
% Original author:
% Carmine Spagnuolo (cspagnuolo@unisa.it) with major modifications by 
% Vel (vel@LaTeXTemplates.com)
%
% License:
% The MIT License (see included LICENSE file)
%
%%%%%%%%%%%%%%%%%%%%%%%%%%%%%%%%%%%%%%%%%

%----------------------------------------------------------------------------------------
%	PACKAGES AND OTHER DOCUMENT CONFIGURATIONS
%----------------------------------------------------------------------------------------
