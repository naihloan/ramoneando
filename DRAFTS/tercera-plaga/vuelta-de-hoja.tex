\subsection{Expectativa}

Algunos días pensamos lo peor, incluso lo que parece menos probable.
Un cometa cae del cielo, o nos revienta de adentro. O no hay cometa.
% Existen mundos en los que están nuestros pies y otros mundos paralelos
% de los que sólo podemos imaginar: pero, ¿en qué mundo estamos?
% (Mi hija de menos de dos años \sout{dice }grita que quiere
% pintar así que vamos a divertirnos con cosas pavas. A lo mejor amasamos
% masa con harina agua y sal. Vamos a distraernos sin prestar atención al
% tiempo, sea el de hoy o del 2050, lo tengamos o no).

% Serenity now
Pero a no desesperar amigo mío.

% Buen Clima Literalmente % Mientras tanto 
El mundo no se cae exactamente a pedazos. O al menos eso
pensamos los días en que nos levantamos optimistas. Hay días también en
que no conseguimos ignorar que los agentes científicos del IPCC [Panel
Intergubernamental sobre Cambio Climático de Naciones Unidas] nos avisan
que todas las simulaciones por computadora indican que si no cambiamos
nuestra forma de actuar como especie, estamos fritos en pocas décadas, en 2050 para ser exacto.

% there is no spoon, no covid??? AUGI
Por ahora, pensaremos que no hay un cometa viniendo del espacio exterior. 
No hay tampoco un cometa virósico en nuestra población local 
escalando exponencialmente sin cesar (porque somos cautos, al menos eso esperamos creer). 
Caerán humanos, a la misma taza de mortalidad que siempre o mayor, cómo saberlo. 
Mientras tanto nuestro país hace, como mi dice me esposa a la que le tomo la palabra, 
lo que van haciendo otros países. 
España permitió una salida diaria de una hora, y una semana después Argentina hace lo mismo. 
En Canadá quieren que los niños puedan retomar las clases, acá muy pronto veremos qué decisiones nuevas hay. 

% \begin{quote}
% \emph{¿Qué ocurre con los derechos laborales? ¿Y con esa otra partecita de vida que no tiene que ver con el trabajo?} 
% \href{https://www.hoydia.com.ar/cultura/68839-bajo-sospecha.html}{Voloj}
% \end{quote}