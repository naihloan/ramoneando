\begin{itemize}
 \item \emph{``Si llegara el día en que mis palabras vayan contra la ciencia,
elijan ciencia.'' Mustafa Kemal Atatürk
 \item  El virus de la gripe se transmite en una relación de 1:1 y el sar-cov2 en una relación de 1:5. Ahí está la gran diferencia
 \item  Las infraestructuras de hospitales son distintos en cada país ¿Cuál es
tu plan Elon? ¿Cómo vas a lograr que mi hospital no quede totalmente
sobrepasado? ¿Vas a mandar enfermeras cuando las nuestras se enfermen?
¿Le vas a mandar flores a los fallecidos? ¿Les vas a pagar la
universidad a sus hijos?
 \item  LIBEREN A EEUU YA no es una iniciativa muy inteligente viniendo de
alguien como vos. Pero siempre es bueno escuchar distintos puntos de
vista}
\end{itemize}

Importante aclaración, al día de la fecha el video del Médico de California quitó, quedó offline.
Ahora se puede encontrar, la explicación de alguien que critica el video, a partir del 1º de mayo, desde el 
\href{https://www.youtube.com/watch?v=59JwT08mhFI}{[Washington Post]} | What Dan Erickson and Artin Massihi get wrong about coronavirus. Los comentarios destacados a este video:

\begin{itemize}
 \item \emph{%
 No permitir que circule desinformación es simplemente ser responsable.
 Estas personas estaban basando sus conclusiones en estadísticas defectuosas.
%  Not allowing misinformation during a pandemic is just being responsible.  These guys were basing their info on flawed statistics.
 \item ¿Quién define qué es desinformación? ¿Nazis liberales?
 %Who defines "misinformation"? Liberal Nazis?
 \item Gente educada.
 %People with education.
 \item No. No usaron estadísticas defectuosas.
 %No. They did not use flawed statistics.
 }
\end{itemize}
%  \item % So their actual interview was “banned” because it was against YouTube community guidelines. So this is on the debate against them. So tell me how YouTube is not politically biased??
%  \item EXACTLY!!!! Let us see their video, and  let us conclude for ourselves.