\subsection{Aceptaciones}

Estuvimos encerrados en nuestras casas la mayoría de los argentinos.
Desde fines de marzo esto parecía una idea prudente, o exagerada, según cómo se vea.
Pero la cautela y la distancia social eran decisiones que en los hechos, se cumplían.
Las preguntas son, ¿hasta cuándo, cómo, y ante qué?

La vida la pasamos estos meses como si todo estuviera a punto de estallar.
Algunos lo vivimos con preocupación, otros con miedo, otros
con aceptación. Habrá quienes lleguen a formar un nudo en la garganta
(no literalmente) y con una sensación de densa y pesada prensa de metal
sobre el pecho, como si nos estuviéramos por asfixiar de angustia y en
dolor de paro cardíaco. ¿Pero todo esto de dónde viene: de un agente
inocuo, o mortífero, o normal, o de la información que circula día a día?

Una primera consideración es que cada día hay una noticia nueva sobre el estado de pandemia:
¿cuáles fueron las fortalezas y debilidades del virus SARS-CoV-2 / COVID-19 en el último periodo de tiempo?
¿cuáles son las expectativas de desarrollo de vacunas y medidas preventivas?
¿Qué se sabe del virus? Bueno, se saben algunas cosas y otras no. 
Acá no nos vamos a enfocar ni en unas, ni en otras, sino en tratar de dar vuelta patas arriba el tema, que no es solamente de salud, sino de qué pasa con los mecanismos sociales al recibir información,
al acatar obedientemente las reglas legales, 
y a nivel de salud también pero no solamente física sino psíquica, a escala individual y social.

La realidad que se vivió en cada casa (y en cada no casa) es que no se puede, o no sería ideal, salir a la calle:
esto sea por motivos legales, de prevención social, o directamente en algunos casos por salud y previsión personal.
Pero el hecho patente en controles urbanos es que la circulación de personas y tránsito está restringida
y que la situación no es tanto, o meramente de pandemia, sino de cuarentena.

El encierro podría tomar diferentes connotaciones, según el color con que se mira.
Encierro voluntario sería hacerse responsable por no adquirir nada del exterior, ni transmitir nada hacia afuera.
Hacer home office, gente trabajando desde las computadoras en sus casas, puede mostrarse como un beneficio de la era digital y una manera de salir adelante.
También prisión domiciliaria, pero se podría considerar este término demasiado incendiario.

Es legítimo que se oriente a la población a encerrarse un tiempo limitado como medida preventiva.
Pero, ¿es legítimo dar y recibir permisos de circulación, 
y el plan a futuro cercano de que todas las personas se registren en una aplicación digital a nivel público-estatal? 
%
Es más, ¿podríamos tomar el encierro y el control de circulación como una medida anti-constitucional?
La fuerza de la norma en Argentina podría haber parecido demasiado en Europa, 
y no todos los países del viejo continente tuvieron buenos resultados, pero otros sí.

Algunas personas se encierran por una creencia casi religiosa en la ciencia: 
por creer en la medicina, la infectología, y por aceptar la epidemiología se da
por sentado que hay un agente real en el ambiente del que tenemos que tener cuidado. 
¿Es un elemento fuera de lo común, o es lo mismo de siempre?.
Pero, ¿hay otros intereses en juego?

%

% Escepticismo
También vivimos a veces con escepticismo, no tenemos certeza de cómo funciona el peligro real, 
o si mueren más personas por COVID o por \sout{accidentes} choques de autos.
Pero tenemos la creencia en un estado legal y legítimo, y en esa inercia hacemos lo que el estado manda: 
hacemos caso a la ley porque sino nos mandan a la penitencia para grandes. 
%
Ahora uno tranquilamente se podría preguntar:
¿China oculta información y Occidente es transparente? 
Suena un poco a \sout{cuento chino} maniqueísmo. 
``Son todos narcos'' decía la banda Las manos de Filippi hace cerca de dos décadas. 
¿Esa sospecha, y acusación, aplica hoy a nivel nacional, como internacionalmente a lo que decidimos meter en nuestras venas físicas y mentales? ¿O sería una sospecha infundada?

% Venta de humo
% ¿Qué nos quieren vender con este encierro? 
% ¿Qué más quieren comprar o ganar con este toque de queda? 
% % ¿Cómo decirlo en términos menos conspiracionistas? % ¿Quién gana con esta situación, cómo y por qué? % Respuesta corta, nunca sabremos. % Respuesta más espaciada y pensada: 
% Los ricos son más ricos hoy que hace dos meses, y se hacen más ricos por
% satisfacer necesidades que ellos cubren, porque el sistema mundial
% (económica, financiero, político y quizás sobretodo
% \textasciitilde{}espiritual, pero en un sentido putrefacto de la
% palabra\textasciitilde{} mental) lleva hacia ese lado. 
