\subsection{Sospechas}

Y después de todo esto contexto único de esta cepa de COVID cabe la pregunta: 
¿algo de todo esto cambia la manera en que venimos viviendo? 
En algunos sentidos para peor, estamos todos un poco más locos, o más sedados, o más ignorantes, o más sospechosos.

Pocos viven hoy igual que en tiempos de pre-encierro. 
Sea real o ficticia la biología de lo que está pasando;
sea real o desproporcionado lo que aparece en medios globales; 
sea real o drástico, o poco impactante epidemiológicamente; 
en casas comunitarias o de personas aisladas (pero en archipiélagos planetarios); 
sea serio o disparatado lo que estamos viviendo como civilización;
estamos alertas a los cambios de escala y seriedad de lo que vivimos más de 7 millones juntos, 
o al menos al unísono.

% Números
¿Puede ser que todo se reduzca a una cuestión de números? Si los números
valen, y nos tenemos que aplicar una prisión domiciliaria voluntaria o
por código penal, pues bueno. Lo acepto. O mejor dicho una parte de mí
lo acepta, como diciendo que capaz el confinamiento a escala planetaria
sea una manera en que el planeta, y no solo la sociedad, se pueda
reiniciar. Reinicio parcial al menos, pero shock es shock.

% Apoyo y desapoyo
¿Podemos apoyar que Argentina, y su representante, hayan sido cautos y
tomar severas medidas para que la gente se quede en sus casas? Sí, en
principo, y por un tiempo. Pero ¿hasta cuándo y para qué? Ya deberíamos
tener más información y ángulos estratégicos. Y con todo, nos quedamos
quietos, demasiado quietos, hay una revuelta en nuestro interior, algo
se está por descontrolar. ¿Pesimismo? Siempre presente, pero no, te
cuento mañana (o el día después de ese) sobre pesimismo.

% Timing de salida >> los comos
Para \href{https://twitter.com/elonmusk/status/1255380013488189440}{Elon
Musk [en twitter]} ya hay que sacar la gente de la casa al trabajo.
% ¿Fundamentos? 
Lo acusan de querer ganar más billones, y no cuidar la vida de la gente. 
% ¿Pero qué otros otros intereses hay por detrás? %No lo sé. 
% Hay mucha información. 
% ¿Teorías conspiracionistas, cuánto crédito darle?
% Para los Hombres de Negro de ficción la información más fiel puede estar
% en lugares descabellados, inesperados, o simplemente pasados por alto.
Los datos para responder esto están (pienso en un día en que me siento positivo), hay que atravesarlos. Menuda tarea. 
Todavía necesitamos bucear a través. Y respirar. 
Hoy el optimismo me dice que, gracias a la quietud reinante, hay más aire\sout{ puro}.

% Escepticismo de escala
% Doctores aprueban salida y desacredita números
Para el \href{https://www.youtube.com/watch?v=_F_bgOEMFQg\&feature=youtu.be}{Dr. Dan Erikson}, en un video difundido estos días desde el 22 de abril de 2020
la cuarentena no amerita encerrar a los sanos. 
Lo interesante de las partes en juego es que las respuestas tanto a Musk como a Erikson, en las plataformas de publicación, son sensatas. 
