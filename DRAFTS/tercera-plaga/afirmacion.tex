\subsection{Afirmación}
% OTRAS FRASES PARA DECIR SUERTE POR ECHAR
Sí hay cometas impredecibles en nuestras creencias colectivas y los
nortes cambiantes de nuestras decisiones y hábitos. Pero si tenemos
suerte, como dice un amigo fatalista contra la especie humana, quizás el
planeta verde tiene una buena chance. Y si tenemos más suerte, quizás
todos tenemos una chance de salir adelante con sanidad interior:
fisiológica, mental y emotiva. 
Siempre habrá cometas cayendo o haciendo implosión. ¿Cara o seca?

% \begin{center}\rule{0.5\linewidth}{\linethickness}\end{center}

% NO SPOON? % AUGI
% Hay que resetear la matrix. No hay cuchara. Con esto la intención no es
% ser insensible ante quien ya pasó el virus de la peor manera, o la gente
% adulta que sí está en mayor situación de riesgo. Pero estas
% consideraciones valen en todo tiempo, no solo \textasciitilde{}en
% momento, época\textasciitilde{} cuando los diarios \textsubscript{dicen}
% comunicamos que hay una situación de alerta pandémica.

% CERO ALCOHOL / CERO CONTACTO
% Hay un caso ejemplar, una imagen que ilustra cómo exageramos a veces en
% Argentina, sobretodo en Córdoba: en las rutas de Córdoba la ley dicta
% que quien conduce no puede tomar una gota de alcohol. A quien se detecte
% en esas condiciones se le aplica severa multa. Pero uno se podría
% preguntar: ¿Cuántas y cuáles medidas se pueden tomar para que la
% circulación de autos produzca menos \sout{accidentes} muertes
% prevenibles? ¿Es posible que el alcohol sea la única o mayor causa de
% que haya \sout{accidentes} problemas y muertes al volante?

% RESPIRAR % Moraleja. 
No hay virus, o tal vez sea de otra escala y/o podemos
tratarla de otra manera a cómo lo venimos haciendo. O sí hay virus, pero
necesitamos tratarlo con más conocimiento y estrategia que simplemente
seguir la ley. 

Con el contacto diario desde la pantalla hacemos bastante catarsis, tal vez a veces demasiado,
la tecnología puede ser una herramienta salvadora, pero también otro encierro más. 
Ya nos avisaba el sociólogo Charles Wright Mills que no tenemos que tomar los hitos de nuestra vida solamente a título personal, 
más bien anticipaba que ``muchos problemas no se pueden resolver meramente como problemas,
sino que los tenemos que entender en términos de temas públicos, y en términos de cómo se hace historia''.

¿Nos ayudan hoy las comunicaciones digitales en nuestros diálogos a escala personal, 
socialmente, políticamente, y hasta en la escala planetaria?

¿Qué reglas tenemos para vivir? Mucho pedir, pero al
menos respiremos hondo y nos olvidemos un rato de todo esto y pensemos
en una playa con arena. O al menos imaginemos estar en una playa de
estacionamiento, o un baldío, o una plaza, con un poco de sol y al aire\sout{ libre}.
