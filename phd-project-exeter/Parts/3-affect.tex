\section*{3. Specific concepts}

\section*{Affect}

A main concern underlying this project is that in day-to-day life there is a prevalent automatism and standardization of ways of using transport, working and perceiving the body and the city as a whole. Each one of these elements that composes urban rhythms is more or less taken for granted, and considered as mere objects and routines that function under fixed ways with little change. However, the city concert also has a number of potentials %that are being permanently  
that undergo permanent changes. The city has a way of constantly creating its own character: that is, people and objects create a full working network and at the same time the city itself becomes a living being. There is no intention (perhaps only slightly) to humanize a non-human, but to understand the influence under which it is subjected and how it creates effects upon \textit{all citizens}.

Emotions are also boiling on the surface of city activity, but it is not this alone that imprints a distinction on how urban flows act, react, change directions and sizes, stop, and come back under different ways. There are certain properties and activities that make a city closer to "all that it can be", or rather those that just keep it at a minimum. \textit{Emotions are not the same as affections} repeat Deleuzian texts and their readers: while the first are needed for a qualitative detection of lively experiences and themes, they are not the points that show how far an entity can reach to the infinite of possibilities to be developed. Each atom of this "life of associations", to use a Tardean-Latourean-ANT idea, can have a strength and ability to propagate itself, or just die by itself from internal implosion. Affect can be on one extreme the potential for sadness, but most commonly cited as the contrary potential for joy, expansion, and freedom to do what is desired and desirable.

\begin{quote}
 Affectivity is understood as intrinsically positive: it is the force that aims at fulfilling the subject's capacity for interaction and freedom. [...] The positivity of this desire to express one's innermost and constitutive freedom can be termed as conatus, potentia or becoming.%
 \footnote{BRAIDOTTI, Rosi. \textit{Transpositions: On Nomadic Ethics}. Polity, Cambridge, 2006. P. 148.}
\end{quote}

One of the primary  driving forces that allows for  these realizations is desire, highlighted as the prime trigger that enables subsequent processes in human society. Hence, one can make a direct relation by linking affections and desire to produce societal outcomes, be them productive for liberation or, on the other hand, even for alienating, controlling and impeding the liberation of desire:

\begin{quote}
 Undeniably,  a romantic concept within his discussion of the regulation and production of desire and energy within a social field, Deleuze’s writings on affect and affection nevertheless enable a material, and therefore political critique of capital and its operations. Within a Deleuzian framework, affect operates as a dynamics of desire within any assemblage to manipulate meaning and relations, inform and fabricate desire, and generate intensity – yielding different affects in any given situation or event. [...] In Deleuze’s singular and collaborative work with Guattari, affective forces are depicted as reactive or active (following Nietzsche), tacit or performed. As Deleuze portrays it, affective power can be utilized to enable ability, authority, control and creativity.%
 \footnote{PARR, Adrian. \textit{The Deleuze Dictionary} [2005]. Edinburgh University Press, Edinburgh, 2010. P. 13}
 \end{quote}

The city and its components, humans and non-humans, have a number of potentials working at several levels of  interconnected networks. And these network points even have the capacity to interchange properties and labels: a passive human-car annexed together can be more of a block in the system of flows than an active non-human river that transports water to the industries and to quench people’s thirst. Even separating humans from non-humans is fictional: they are structured ways of understanding agents that are not actually very easily separated, a cyborg human becomes indistinguishable from a pos-human, or even a simple human.

The research suggested here intends to explore the capabilities in a side of urban life and seek how it may unfold: it is a search that goes to several components of a network and tries to see how the latter may render multiple layers of possible outcomes.