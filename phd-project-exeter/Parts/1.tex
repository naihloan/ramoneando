\section*{1. Runners \& The City}

The total number  of runners moving through any space seems to be an independent flow from the rest of the city's circulation. This is true in the sense that green spaces are mainly designed for leisure and as traffic-free zones. On the  other hand,  however, it is not quite true that runners are independent of other flows because car-traffic and other types of non-running traffic cross the runners’ way and, hence, make them stop: breaking runners' momentum%
\footnote{ETTEMA, Dick. "Runnable Cities. How Does the Running Environment Influence Perceived Attractiveness, Restorativeness, and Running Frequency?" \textit{Environment and Behavior}. Pp. 1-21. 2015. P. 17.}.
Runners, just as all others, depend on getting available paths as they go. This has two major implications:

\begin{itemize}
 \item The need for paths to move freely in a city
 \item No two objects/people can be in the same place at the same time 
\end{itemize}

\subsection*{The need for paths to circulate in}

This has a huge dimension in which non-humans get into play. For each space that is used in the city one could follow a science studies method: to determine all the objects and people that come into action to deliver a single object. The generic city as a civilized construction always has a set of layers upon which it has been built: be it an arid,  rocky, or damp or even forest-like, or any other kind of environment there are ways of setting in. Humans have customized spaces for millenniums. Only the past couple of centuries, at the most, have taken into account the use of delimited areas of public space for new purposes such as leisure.

\subsection*{The need to share space}

True as it may be, this last point seems to be overlooked in today's flawed auto-mobility system%
\footnote{SHELLER, Mimi; URRY, John (eds.). "The new mobilities paradigm". In \textit{Environment and Planning}. volume 38, pages 207-226, 2006.
%SHELLER, Mimi; URRY, John (eds.). \textit{Mobile Technologies of the City}. Routledge. 2006.
}:
not only do cars (and drivers) burn fuels, and leave a lasting carbon footprint, but  private vehicles can become  quite impractical with the normalcy and abundance of traffic jams as well. LeCorbusier, in his Athens Letter (1933), settled the four main modern uses of urban space: inhabiting, working, circulating and recreating. Granted that this view has a somewhat non-layering of
functionalities, and an oversimplification of uses; however, it was intended to take into account city livability for human beings, hence prioritizing the housing and green areas on urban planning. Also, transportation was the least considered element, in a period where automobile expansion numbers (an \textit{overpopulation} of non-humans, so to speak) had only just recently begun. In the XXIst  century, this seems to be a much more critical issue, where these old proposed functions have %, at least generally speaking, 
nearly collapsed. How do runners find non-occupied paths in such an overflowed system?

\subsection*{The need to share times of use}

The physical environment is not used at all times in the same way. Social space has areas in which one acts among other people; and others, in which this presentation is left aside: this is what has for long been called the front and back regions of human conduct, also well known as front-stage and backstage%
\footnote{GOFFMAN in HANNERZ, Ulf. “The City as Theater: Tales of Goffmann”. In Exploring the city: inquiries toward an urban anthropology. New York, NY: Columbia University Press, 1980. P. 206.}.
So attention is shifted from one \textit{stage} %situation 
to the other. 

It could be arguable that, of the classic functions presented by LeCorbusier, three of them are to be pursued as part of social and even animal life: working, sleeping and wandering. Transportation, even if exaggerating and stretching the argument a bit too far, as a means to an end has no real function. It seems that all time lost in traffic is time in the backstage with no actual point. However, runners do seek to transport themselves, but with a whole other meaning, closer to leisure in their free time (even \textit{serious leisure}), or even the mental-rest aspect of sleep time.
