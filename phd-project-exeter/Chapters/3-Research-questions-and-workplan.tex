\pagebreak
\subsection*{Research questions and workplan} 

The above leaves the way closer to attempt a series of questions.
Would it be possible then to see freeways and cities as something else than a mere containment of controlled flows?
Or said in a more positive way and onto a slightly other physical direction: \medskip
%

\textit{Can we consider roads, paths in general, and other technological artifacts as enablers that shape 
human experience and social relationships?} 
%\textsc{Can we consider roads, paths in general, and other technological artifacts as enablers that shape human experience and social relationships?} 
And this can even be considered as a double track proposal:
%
\textit{%could we consider certain 
%Are 
%In which ways 
How do human experiences and social relationships %as well
%shapers of roads, paths in general, 
shape pathways, views, resources and technologies?
%and other technological artifacts?
}
Even if two seemingly separate sides merge to form a socio-technical assemblage, they in fact hibridize: 
hence the importance to rescue all agents, humans and non humans, involed with simetrical weight. 
The double sided view separates analitically what actually forms a network of dependencies.
\medskip
%%%%%%%%%%%%%%%%%%

There are many available resources to study the ultrarunning scene even from a distance.
Secondary material involves both texts%
\footnote{FIXX, James F.; JUREK, Scott; KOSTRUBALA, Thaddeus; LEONARD, George; McDOUGALL, Christopher; MURAKAMI, Haruki; NUNES, Valmir; ROHÉ, Fred. ROLL, Rich.}  
%En segundo lugar, las fuentes secundarios de campo, que tiene que ver con los materiales que los propios ultracorredores y fue citada a lo largo de este proyecto y en la bibliografía. Adicionalmente hay una serie de autores y textos disponibles vía web, habría que tener presente una multiplicidad de actores: los estudios de vibram en zapatillas minimalistas, las revisiones y comentarios del médico Mark Cucuzella, textos varios de Peter Larson, Jason Robillard y en general el acceso infinito que da la red rizomática del sitio web run100s. Los corredores más experimentados tienen en cuenta un abanico de variadades en lo que hace a la técnica. 
written by participants, and journalists; as well as videos\footnote{BENNA J.B; COEMAN, Tom; DUNHAM, Jon; EHRLICH, Judd; FRANKEL, Davey Frankel \& LAKEW, Rasselas; HEISENBERG, Benjamin. RICHARDSON, Tony; ROTHWELL, Jerry; STEWART, Rob; STUART, Mel.}.
Part of this material has already been read and viewed. 
%The following step is to connect main themes with the Theoretical Framework and to a ethnographic/qualitative axis.

Field work allows for direct contact with the ultra world and for day to day updates on normal practices and non-structured interviews. The first-hand material is expected to be a strength of the proposal since the candidate is a long-time runner, with more than two decades of experience in several distances. Having already completed the marathon distance the candidate is highly likely to fulfill races of at least 50 kms. Longer distances (80, 100, 150) could and are expected to be attemped later on. 
Regardless of the kind of participation, be it by running or simply attending to events as observer, the contacts have already began: I already gained information on specific yearly ultra-distance races with different attractives:
the german Rennsteiglauf with an average of 15000 participants, the chilean Rapa Nui trophy at the exotic Easter Island,
and the important NGO that prepare races for awareness to fight aganst human trafficking: Muskathlon, both in South Africa and also crossing the border from Bulgaria into Greece.
%http://www.a21.org/campaigns/content/muskathlon/glarjd?permcode=glarjd

The relevance of what appears in the environment to ultrarunners is not always obvious in a third party written description, or even in conversation. The possibility to participate in the same training and competitions is to be part of the “same capsule of events” as other ultrarunners. What changes is not only the events but rather their at-handedness, which allow for a closer possibility of involvement.

Two possible outcomes of the study involve: on the one hand, the chance to get in-depth insight on the technological analysis of these practices and events. On this matter, time for research at the Sciences Po would be a gain. 
The direction of the project would benefit from the perspective of considering the mechanical-chemical aspects of ultra in relation to scientific humanities, specialty of Bruno Latour’s team, with whom the candidate has taken an online MOOC course, early 2014. On the other hand, the second possibility should aim at spending time together with specific communities of ultrarunners.

