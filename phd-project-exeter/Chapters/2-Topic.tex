\chapter*{Research topic and subject of analysis} 

It would seem at a first glance that all the social and biological mechanics
of the functioning of running just work on auto%-%matic 
pilot. And however,
several controversies open up from different angles. The popular book “Born
to Run” suggests that the Mexican Indians, the Raramuri of the Tarahamura
mountains, run today in the same way they have been  doing so for  the last four
centuries.
%Pareciera a primera vista que toda la mecánica biológica y social de funcionamiento en torno al correr funciona en piloto automático. Y sin embargo hay controversias que se abren por varios frentes. El popularizado libro Nacidos para correr sugiere que los indios mexicanos, los raramuri de las sierras Tarahamura, corren de la misma manera hace cuatro siglos. 

Running as a trend arises half a century ago, together with sport gear
and recommendations on the use of special shoes to that end. In a wider
specter, these customs are inserted in a world of vast shortages and excesses.
On the one hand, shortages of activity-involving options for the dormant body
---in an economic system in which desk jobs prevail as well as  bodily passivity and mass
consumption. On the other hand, excesses in the search of vivid attention, fun,
fatigue and the exploration of the limits of the body when activated. The bodily
biology opens up to new sensations. The body becomes a centrifugal force
making it necessary to analyze the outcoming bodily fluids, the salinity of sweat,
the color of urine indicative of hydration, and of feces that show the gastric
processes. The body as the center of centripetal forces seeks for a nourishment
that allows running for hundreds of kilometers, and the knowledge it will not
die in the attempt. In sum, \textit{a social body as a machine of singularization} that
goes beyond the standarized solutions provided to all.

%La costumbre/moda de correr surge hace medio siglo, así como accesorios deportivos y recomendaciones de por qué usar calzado especial. En un espectro más amplio estas actividades están insertas en un mundo de carencias y excesos. De un lado, carencias de opciones activas para el cuerpo aletargado en un sistema económico en el que predominan los trabajos de escritorio, y los pasatiempos de consumos pasivos. De otro lado, excesos de búsquedas en la fatiga, la atención, la diversión y la búsqueda de límites del cuerpo cuando se activa. La biología corporal se abre a sensaciones nuevas. El cuerpo se vuelve máquina centrífuga: se vuelve necesario analizar los fluidos corporales salientes, la salinidad de la transpiración, el color de la orina indicativo de la hidratación, materia fecal que indica procesos gástricos. El cuerpo como máquina centrípeta busca los consumos alimenticios que permitan correr por centenas de kilómetros, y los conocimientos para no morir en el intento. 

The individual and collective bodies may face the challenge of becoming a couch
potato or not: and thus affect their ability and desire to act. There are even
those who hold that it is not even necessary to wear shoes to run, claiming it
would be enough to simply use your legs, and barefoot feet. This is supported
by several athletes, academics and even a small fraction of the industry looking
for innovation (going back to basics) with minimalist footwear: with low or no
heel height.
%Los cuerpos individuales y colectivos toman en este entorno el desafío de afectar su capacidad y deseo de actuar. Hay quienes proponen que ni siquiera hace falta usar zapatillas para correr, bastaría solamente con usar las piernas, y ¡descalzos!. Esto lo sostienen algunos atletas, académicos y hasta una fracción de la industria que busca la innovación con calzados minimalistas: sin acolchonamiento en el talón. ¿Cuánto hay en el correr de destreza aprendida o innata? 

% http://www.lanacion.com.ar/1716787-susana-saulquin-el-sistema-de-la-moda-sigue-operando-aunque-la-tendencia-sea-salirse-de-lo-masivo
%-LN: Digamos que, en todo el planeta, habría dos grandes problemas con lo textil: la sospecha de trabajo esclavo o infantil al comienzo de la cadena de producción, y las denuncias de impacto medioambiental hacia el final.

Is it really necessary to wear shoes? Some studies suggest that certain barefoot movements can prevent injuries: for example landing with the ball of the feet (metatharsi) while running, instead of using the heel. This type of bodily
mechanics is one of the topics that the paleoanthropologist Daniel Lieberman
has researched for over a decade. Runners learn technical resources and make
them their own from different sources: nutritional, mechanic, motivational. Yet
each person uses them, develops them, and tailors these resources to their own
knowledge: they \textit{singularize} them, they learn how to run in their own unique
way. In this point, the experimentation of athletes becomes key. In a broader
sense, the whole of the runners’ world would also affect the urban rhythms,
slowing them down and accelerating them, intervening in the physical city and
the way spaces are used. To that purpose the goals of this research are as
follows.


\begin{itemize}
 \item \textsc{General Objective}\\
Researching into how people learn to manage resources/knowledge and take risks to do activities that the majority ignores. Runners of ultramarathon are not superhumans: they develop a know-how and find interest in the methods of running to an extreme extent.
  \item \textsc{Specific Objectives}\\
    - Taking into account the progressive information that a community of ultrarunners have access to and handle through specific events. \\
    - Reviewing the impact of running styles related to injury/health, and the relationship with (or the lack of) running gear.\\
    - Considering the motivations to run beyond a health goal (even risking life) and the mental and spiritual levels that get into motion.\\
    - Detailing the particularities and differences of an ultramarathon event in comparison to shorter races.\\ %\SubItem 
    - Seeking the paths of the athletes through public space as an opposite
      direction against standarized and massive urban rhythms, placing the
      general mobility as a limitation to the body and social development.\\
      % yo le preguntarìa tambièn a los ultrarunners què piensan de la ciudad en donde "corren" 
      %y entonces: pondrìa un objetivo especìfico sobre esta cuestiòn
      %la pregunta entonces: què tipo de practicas y saberes se generan / desarrollan en los ultrarunners? 
      %la anticipaciòn de sentido: los saberes y pràcticas condicionadas por el contexto y por la industria...por ejemplo!
    - Attending to how ultrarunners cope with the cities and spaces they live in, work and run through. It is assumed that runners and their surroundings are conditioned by the trends of massive behaviour \ldots but at the same time athlete's behaviour reshape production by making the input of new demands for new equipment, a renewal in set of mind and for nuances in ways to be in urban placements.
\end{itemize}

%¿Hace falta realmente usar calzado? Algunos estudios sugieren que hay movimientos que evitan lesiones en el cuerpo: por ejemplo correr apoyando el metatarso, en vez del talón. Este tipo de mecánica corporal es uno de los temas que investiga hace más de una década el paleoantropólogo Daniel Lieberman. En todo caso los corredores aprenden recursos técnicos y las hacen propias a partir de distintas fuentes: nutricionales, mecánicas, motivacionales. Pero cada persona las usa y hace a su manera: las singulariza, aprende cómo correr de una manera propia y única. En esto se vuelve clave el factor de la experimentación de los atletas. El conjunto de las actividades de ultracorredores además afectaría a los ritmos urbanos, alternativamente frenándolos o acelerándolos, interviniendo en la ciudad física y en la manera de usar los espacios. 
