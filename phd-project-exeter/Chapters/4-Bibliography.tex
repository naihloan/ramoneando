\backmatter

%\chapterstyle{default} % Reset the chapter style back to the default used for non-content chapters
\bibliographystyle{unsrt}
%\end{document} 

%----------------------------------------------------------------------------------------
%	BIBLIOGRAPHY
%----------------------------------------------------------------------------------------


\def\bibindent{1em}
\begin{thebibliography}{99\kern\bibindent}
\makeatletter
\let\old@biblabel\@biblabel
\def\@biblabel#1{\old@biblabel{#1}\kern\bibindent}
\let\old@bibitem\bibitem
\def\bibitem#1{\old@bibitem{#1}\leavevmode\kern-\bibindent}
\makeatother

\bibitem{} ARRIETA, Ezequiel. 
\textit{Vegetarianismo en el debate político}. Córdoba, Ed. Del autor. 2014.

\bibitem{} ASCHIERI, Patricia. "Hacia una etnografía encarnada: La corporalidad del etnógrafo/a como dato en la investigación". X RAM- Reunión de Antropología del Mercosur. Córdoba, Argentina, 2013. 

\bibitem{} BARBARINI, Tatiana A. 
\textit{O controle da infância = caminhos da medicalização}. 2011. 192 p. Dissertação (mestrado) - Universidade Estadual de Campinas, Instituto de Filosofia e Ciências Humanas, Campinas, SP.

%\bibitem{} BECKER, Howard S. % Chapter 12
%``Drama and Multivocality: Shaw, Churchill, and Shawn''. In 
%\textit{Telling About Society.} Pp. 204-222. Chicago: University of Chicago Press, 2007.

\bibitem{} BECKER, Howard S. \textit{What About Mozart? What About Murder? Reasoning From Cases}. The University of Chicago Press, Chicago, 2014.

\bibitem{} BECKER, H. S. \& PESSIN, A. 
``A Dialogue on the Ideas of `World' and `Field' with Alain Pessin''. \textit{Sociological Forum}, 21, pp. 275-86. 2006.

\bibitem{} BOTTERO, Wendy \& CROSSLEY, Nick. 
“Worlds, Fields and Networks: Becker, Bourdieu and the Structures of Social Relations”. \textit{Cultural Sociology} 5(1) 99–119. 2011.

\bibitem{} BRAIDOTTI, Rosi. \textit{Transpositions: On Nomadic Ethics}. Polity, Cambridge, 2006.

\bibitem{} BRIGHENTI, Andrea. ``Walled Urbs to Urban Walls – and Return? On the social life of walls'' in 
Andrea Mubi Brighenti (ed.), \textit{The Wall and the City}, Trento: professionaldreamers, 2009.


%\bibitem{} CORBANEZI, Elton. 
%\textit{O poder médico-psiquiátrico como forma de subjetivação e controle}. 2014. Exame de qualificação (Doutorando em Sociologia) - Universidade Estadual de Campinas.
%\bibitem{} CRUZ, Leonardo Ribeiro da. 
%\textit{Internet e arquiteturas de controle: as estratégias de repressão e inserção do mercado fonográfico digital}. 2014. 255 p. Tese (doutorado) - Universidade Estadual de Campinas, Instituto de Filosofia e Ciências Humanas, Campinas, SP.

\bibitem{} DELEUZE, G.; GUATTARI, F. 
\textit{Anti-Oedipus. Capitalism and Schizophrenia} [1972].
University of Minnesota Press Minneapolis, Minneapolis, 2000.
%\textit{O anti-Edipo: capitalismo e esquizofrenia}. SP, SP: Ed. 34, 2010.

\bibitem{} EDENSOR, Tim. ``Defamiliarizing the Mundane Roadscape''. In \textit{Space \& Culture}. Vol. 6. no 2.
May. 2003.

\bibitem{} ETTEMA, Dick. "Runnable Cities. How Does the Running Environment Influence Perceived Attractiveness, Restorativeness, and Running Frequency?" \textit{Environment and Behavior}. Pp. 1-21. 2015. P. 17.

\bibitem{} FIXX, James F. \textit{The complete book of running}. Random House, New York, 1977.

%\bibitem{} FERREIRA, Pedro Peixoto. “Máquinas sociais: o Filo Maquínico e a Sociologia da Tecnologia”. \textit{CTeMe}. 2004.
\bibitem{} GOFFMAN, Erving. \textit{Relations in public: microstudies of the public order}. New York: Harper \& Row, 1972. 

\bibitem{} HAMMERSLEY and ATKINSON. \textit{Ethnography: principles in practice}. 3rd ed. London; New York, NY: Routledge, 2007. 

\bibitem{} HANNERZ, Ulf. ``City as theater: tales of Goffman''. In 
\textit{Exploring the city: inquiries toward an urban anthropology}. Pp. 202-241. New York: Columbia University Press, 1980. 

\bibitem{} HENNION, Antoine. ``Those Things That Hold Us Together: Taste and Sociology''. \textit{Cultural Sociology} Volume 1(1): 97-114. 2007.

\bibitem {} HOLLOWAY, Imma; BROWN, Lorraine; and SHIPWAY, Richard. "Meaning not measurement: Using ethnography to bring a deeper understanding to the participant experience of festivals and events". \textit{International Journal of Event and Festival Management}. Vol. 1 Nº 1, 2010. Pp. 75-76.

%%%%%%%%%%%%%%%%%
\bibitem{} INGOLD, Tim. ``The Temporality of the Landscape'', \textit{World Archaeology}, 25(2): pp. 152-174, 1993. 

\bibitem{} JUREK, Scott. \textit{Eat & Run: My Unlikely Journey to Ultramarathon Greatness}. Mariner,
New York, 2013.

\bibitem{} KOSTRUBALA, Thaddeus. \textit{The joy of running}. New York, Lippincott, 1976.
%%%%%%%%%%%%%%%%%

\bibitem{} LATHAM, Alan. 
“The history of a habit: jogging as a palliative to sedentariness in 1960s America”. \textit{Cultural geographies} 2015, Vol. 22(1) 103–126.

\bibitem{} LATOUR, Bruno. “How to Talk About the Body? the Normative Dimension of Science Studies”. \textit{Body \& Society}. 2004.

\bibitem{} LEONARD, George. Mastery: The Keys to Success and Long-Term Fulfillment. Penguin
Books, New York, 1992.

\bibitem{} LIEBERMAN, Daniel E, etal. 
``Foot strike patterns and collision forces in habitually barefoot versus shod runners''. \textit{Nature}. 28 January 2010. Pp. 1-6.
%\bibitem{} INGOLD, Tim. ``Traces, threads and surfaces''. In \textit{Lines: a brief history.} Pp. 39-71. New York: Routledge, 2007.


\bibitem{} McDOUGALL, Christopher. \textit{Born to Run: A Hidden Tribe, Superathletes, and the Greatest Race the World Has Never Seen}. Knopf, New York, 2009.

\bibitem{} NUNES, Valmir. \textit{Segredos de um ultramaratonista. Histórias do campeão}. Hemus, Brasil,
2009.

\bibitem{} PARR, Adrian. \textit{The Deleuze Dictionary} [2005]. Edinburgh University Press, Edinburgh, 2010.

\bibitem{} ROHÉ, Fred. \textit{The Zen of Running} (1974). Pan-American, Middletown, 2000.

\bibitem{} ROLL, Rich. \textit{Finding Ultra: Rejecting Middle Age, Becoming One of the World’s Fittest
Men, and Discovering Myself}. Random House, New York, 2012.

\bibitem{} SHELLER, Mimi; URRY, John (eds.). "The new mobilities paradigm". In \textit{Environment and Planning}. volume 38, pages 207-226, 2006. %\textit{Mobile Technologies of the City}. Routledge. 2006.

\bibitem{} TARDE, Gabriel. 
\textit{Monadology and sociology} [1895]. Melbourne, re.press, 2012.

\bibitem{} YOUNG, di Eugene B. (auth.); GENOSKO, Gary (auth.); WATSON, Jannell. 
\textit{The Deleuze and Guattari Dictionary}. Norfolk: Bloomsbury, 2013.

\end{thebibliography}
