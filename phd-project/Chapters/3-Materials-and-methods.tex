\chapter*{Materials and methods}
%\chapter{Materiales y métodos de investigación}

This work looks to a wider spectrum of escape mechanisms from the inertia of
the productive system, a production of machinic uselessness, where cars and
public transportation \textit{circulate, overpopulate and congest}, all of which control
movement in favor of an economic and political order and power.

%\begin{quote} (Deleuze y Guattari, 2010: 527). %\end{quote}

Ultrarunners are the case study. However, the implications of the topic
go beyond this social world. A good number of ultrarunners are so aware of
the need of natural food that they are more inclined to eating more fruits and
vegetables than average people and many of them become long term vegetarians.
This is related, but differently, to massive consumption patterns: of avoiding
the dangers of canned foods, with preservatives, refined sugars, processed flour;
and a political view that requires that the land  be distributed and
cultivated to feed people and not animals.
%(Arrieta, 2013). 

The proposal is of qualitative research. On one side, it will nurture from
secondary material in texts and videos made by and about the participants of
ultramarathon races. On the other side, first hand material will be collected
 from fieldwork. As study material, the general training method shall be
reviewed and at least, the ethnography of one specific competition. These
elements pursue to give new life to the ideas of how people move, beyond
a mere transportation function, and how urban spaces can be circulated,
expanding their uses.


%%%%%%%%%%%%%%%%%%%%%%%%%%%%%%%%%

% benjaminjuarez.com/ARCHIVO/2016.02.29.OutlineProjectPhD-extended.html 

%Dear Benjamin,  I very much enjoyed reading your draftPhD proposal. I think it is an exciting project. 

%I wonder if you could expand a bit more on Section 3 Materials and Methods. This would also need that you specify a couple of research questions which affect your methodological account, your methods etc (ho will you research the lifeworld of runners?)...

%You may also say a bit more about your conceptual approach? 

%Kind regards
%Michael. 
%\section*{Research questions }