\section*{Materiality and Body}

One take on materiality could be that physical elements of the city could not have any ability whatsoever to \textit{act proper} by themselves onto other objects or people. An object is by definition inert. But this doesn't mean that objects don't have a weight in social relationships. Any piece of technology involves in itself a cristalized work that absolves the user from extra effort. Roads mean that each person does not need to clean a path through the wild jungle... each time intending to move across.

The notion that any matter has potential of, and active non-material, social relationships is even fair with the most unexpected.
The most apparently inactive of objects have life by interchange of elements and as mediums through which other elements of a network travel. Runners and hitchhikers need to know their ways. The ability to understand how to run through a city is in fact, part of a taken for granted training that goes beyond the physical and relates with how to understand and grasp control and develop a route by moving through obstacles, gates and pathways. There is a certain learning of the senses on how to understand the environment or the objects we are confronted with. Perceiving, or tasting as Hennion exemplifies, is a "passivity actively sought". That which applies in general, applies to running as well as to several other fields. Even a rock climber and a rock can have a certain interchange, in quite a direct manner:

\begin{quote}
 What climbing shows is not that the geological rock is a social construction, but that it is a reservoir of differences that can be brought into being. The climber makes the rock as the rock makes the climber. The differences are indeed in the rock, and not in the ‘gaze’ that is brought to it. But these are not brought to bear without the activity of the climb which makes them present. There is co-formation. Differences emerge, multiply and are projected. The ‘object’ is not an immobile mass against which our goals are thrown. It is in itself a deployment, a response, an infinite reservoir of differences that can be apprehended and brought into being%
 \footnote{HENNION, Antoine. ``Those Things That Hold Us Together: Taste and Sociology''. \textit{Cultural Sociology} Volume 1(1): 97-114. 2007. Pp. 100-101}.
\end{quote}

To learn how to feel the city, and our own body, does not come without effort. One may not know how to move, resist, relax and stretch, or when it is time to rest and sleep. All physical skills are also developed by a training not only of the body but also of the ability to perceive. What can one perceive if one does not know what to be aware of? The problem raised by Latour (2004: 210) regarding the potentials of the body is that a body that doesn't perceive anymore, gets closer and closer to an inactive body, whose ultimate projection is towards the life of a dead corpse.

\begin{quote}
 An inarticulate subject is someone who whatever the other says or acts always feels, acts and says the same thing. [Articulation, on the other hand, means the ability to] being affected by differences.
\end{quote}

Through the concepts above mentioned, a set of ideas deploy a range of possibilities for society and in which directions they can mutate. The case study of ultra-runners may highlight a number a variables that look to a perhaps different system that is not based on consumption or production, but  to other horizons. Aside the critiques of the automobile systems, there is a parallel consideration of practices of slow mobility and de-growth. If one could summarise a cycle of interests, that would run in the following steps:

%\begin{center}
% affect $ \> $ --> $ \> $desire $ \> $--> $ \> $effects $ \> $--> $ \> $changes
%\end{center}

\begin{center}
 %\begin{minipage}{0.8\textwidth}
 affect $ \> $ --> $ \> $desire $ \> $--> $ \> $effects $ \> $--> $ \> $changes
%\end{minipage}
\end{center}

On both ends of this transition, one can realize that much of the foundation is built on how affects are dealt with, what they produce at a desired machine level, and what material and corporeal effects they have upon the collective body.

The mobility of running is not that slow per se. Rather, speed is defined by different scales of technology: runners would be more of organic machines hybridized with "simple" technologies, such as shoes, (in the same way that cyclists use a metabolic energy for propulsion) instead of the bigger scale of solely mechanical cars that depend on exterior fuels such as gas and oil.

On the side of "de-growth" something similar happens. If a life style is measured comparing the use of gas, or lack of it, of course living without them feels like a decrease. But on the other hand, when everyday life is related to a more green life style,  a different sense of growth can appear, that of life which entails another time frame: like worms that eat slower than humans, of bacteria in fermentation which serve humans to deliver pre-digested food. Finally the active, non-sedentary, human body awakens, gets a more active, quicker, metabolism and flourishes. Perhaps a new system may be devised not under the comparative term of being slower than that of automobiles, but one that it may grow in a different sense and direction.