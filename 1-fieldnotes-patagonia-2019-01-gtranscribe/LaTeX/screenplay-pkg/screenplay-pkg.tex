% !TEX TS-program = pdflatexmk

\documentclass[11pt]{article}
\title{\textbf{The \textsf{screenplay-pkg} package}}
\author{\textbf{Alan Munn}\\Department of Linguistics and Languages\\Michigan State University\\\texttt{\href{mailto:amunn@msu.edu}{amunn@msu.edu}}}
\date{Version 1.1\\August 5, 2017}
\usepackage[T1]{fontenc}
\usepackage[margin=1in]{geometry}
\usepackage{titling}
\usepackage[utf8]{inputenc}
\usepackage{array, booktabs, multicol, fancyhdr, xspace,tabularx}
\usepackage{enumitem}
\usepackage{fancyvrb,listings,url}
\usepackage[sf,compact]{titlesec}
\usepackage{screenplay-pkg}
\usepackage[colorlinks=true]{hyperref}


\DefineShortVerb{\|}
\newcommand*\bs{\textbackslash}

\IfFileExists{luximono.sty}%
{%
  \usepackage[scaled]{luximono}%
}
{%
  \IfFileExists{beramono.sty}%
  {%
    \usepackage[scaled]{beramono}%
  }{}
}

  
\lstset{%
    basicstyle=\ttfamily\small,
    commentstyle=\itshape\ttfamily\small,
    showspaces=false,
    showstringspaces=false,
    breaklines=true,
    breakautoindent=true,
    captionpos=t
    language=TeX
}
  
\newcommand*{\pkg}[1]{\texttt{#1}\xspace}
\setitemize[1]{label={}}
\setitemize[2]{label={}}
\setdescription{font={\normalfont}}
\setlength{\droptitle}{-1in}

\lhead{}
\chead{}
\rhead{}
\lfoot{\emph{}}
\cfoot{\thepage}
\rfoot{}
\renewcommand{\headrulewidth}{0pt}
\renewcommand{\footrulewidth}{0pt}
\pagestyle{fancy}


\begin{document}
\maketitle
\thispagestyle{empty}
\renewcommand{\abstractname}{\sffamily Abstract}

\abstract{\noindent\begin{quote}This is a package version of the \pkg{screenplay} document class. The class version is designed to produce a properly formatted screenplay manuscript. This package version allows portions of screenplays formatted using the class methods to be included into other document classes.
\end{quote}
\section{Introduction}
This package arose out of a question asked on the StackExchange website: \href{http://tex.stackexchange.com/q/26227/}{Converting document classes into environments: is it possible?} The question asked how easy it would be convert the class functionality into an environment.  This package is the result of that discussion.
\section{Package use}
To use the package just load it like any other package:

\begin{lstlisting}
    \usepackage{screenplay-pkg}
\end{lstlisting}

\subsection{New commands}
The package implements one new environment and two formatting hooks for it.
The |screenplay| environment provides an environment to wrap a screenplay fragment.  Within this environment, you should use any of the macros defined in the \href{https://www.ctan.org/pkg/screenplay}{\pkg{screenplay}} class. Please consult its  documentation for details.
\begin{lstlisting}
    \begin{screenplay}
     
    \end{screenplay} 
\end{lstlisting}

 Two additional user commands are added to allow for user adjustment of the
 fragment font and linespacing (the \pkg{setspace} package is used for spacing unless the \pkg{memoir} class is loaded). American spellings of the centring commands have also been created. British spellings of these commands are retained:

\begin{center}

\begin{tabular}{l>{\raggedright\arraybackslash}p{3.5in}}
\toprule
\textsf{Command}       & \textsf{Explantion}\\
    |\screenspacing{}| & sets the linespacing for screenplay fragments\\
    				   & default is |\onehalfspacing|\\
				       & (if \pkg{memoir} is loaded, |\OnehalfSpacing|)\\
    |\screenfont{}|    & sets the font for screenplay fragments\\
    				   & default is |\ttfamily|\\
   |\sccenter |	       & American spellings of centring commands now possible\\
   |\centertitle|      & \\
\bottomrule
\end{tabular}


\end{center}
\subsection{Eliminated commands}
All commands from the \pkg{screenplay} class that relate to creating the title page have been removed from the package version of the class.
\section{Troubleshooting and package dependencies}
\subsection{Package dependencies}
Note that the package uses the \pkg{setspace} package for linespacing. If you are using the \pkg{memoir} class which provides its own linespacing methods the package will use its linespacing commands instead.

\subsection{Bugs}
You are welcome to report bugs and submit feature requests, but I should warn you that this package is extremely low priority for me in terms of maintenance, as I do not use it at all.  If you are interested in taking it over, please get in touch with me.

\section{Sample document}
The following is a sample document showing how the package is used. It can be found in the documentation folder for the package.
\lstinputlisting{screenplay-pkg-example.tex}

\end{document}



