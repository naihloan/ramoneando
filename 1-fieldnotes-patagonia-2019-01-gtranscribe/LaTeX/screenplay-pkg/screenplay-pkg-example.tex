%%
%% This is file `pkg-example.tex',
%% +=+=+=+=+=+=+=+=+=+=+=+=+=+=+=+=+=+=+=+=+=+=+=+=+=+=+=+=+=+=+=+=+=+=+=+=+=
%% 
%% Original file Authored by and Copyright (C)2006 by
%% John Pate <johnny@dvc.org.uk>
%% http://dvc.org.uk
% This file modified by Alan Munn to test the standalone package version
%
%
%
%This package may be distributed and/or modified under the conditions of
%the LaTeX Project Public License, either version 1.3 of this license or
%any later version. The latest version of this license is in
%http://www.latex-project.org/lppl.txt and version 1.3 or later is part
%of all distributions of LaTeX version 2005/12/01 or later.
%
%This package has the LPPL maintenance status `maintained'.
%
%

\documentclass{article}
\usepackage{screenplay-pkg}
\usepackage{lipsum}
\begin{document}

%% TeX allows quite a lot of leeway in whitespace, so I've messed this
%% up a bit.  I find keeping the format structured helps me a lot tho.
%% Note: but don't have blank lines *inside* the body of text in
%% a dialogue environment.
%% I use vi (Elvis) with macros to make a lot of typing disappear.
%% You can, of course, define LaTeX macros to shorten the command names.
%%
%% Anyhoo, on with the show ...

%% for some reason this always happens at the start ...
This is some normal text in the document. \lipsum[1]

% To insert a screenplay fragment, use the screenplay environment
\begin{screenplay}
\fadein

\intslug[illumination]{example sample -- screenplay.cls}

In space, nobody knows what time of day it is.  Wait, there is no
day.

So BOB, a cross-dressing Republican, and BROWN, a Christian
fundamentalist Democrat, are talking nonsense instead.

\begin{dialogue}{Bob}
That means that someone
sabotaged the unit and killed the
President! Was it one of us?
\end{dialogue}
\begin{dialogue}{Brown}
Who else is mad but us, Condi\ldots
 \paren{beat}
and Bliar?
\end{dialogue}

Bob buries his head in his hands.

\intslug{Atlantis -- somewhere ANyway}

JOHN and MARK are at adjacent consoles. FRED is with them. TOM
is at another console slightly further away.

\begin{dialogue}{John}
The planetoid seems to have a thin crust
covering a nickel-iron core. Could have
been an Earth-like planet at one time.
\end{dialogue}

\begin{dialogue}{Mark}
We're coming up on the radio source now.
\end{dialogue}

Brown walks in and goes to a console.

He has a PARROT on his shoulder.

The Parrot has an air of quiet insouciance.

\begin{dialogue}{Fred}
Switch the visual to main screen so we
can get a good look.\end{dialogue}

They look up at the main screen.

\begin{dialogue}[to John and Mark]{Fred}
Lock on to that.
\dialbreak[to Tom]{Fred}
Establish planetary orbit.
\end{dialogue}

\intextslug[day]{in or out}
Apparently some people do this.

\intercut
\extintslug[night]{out or in}
Or even this.

\extslug[day or nite]{NO WARRANTY -- EXPRESS OR IMPLIED}

\pov\ I made the slugline DAY/NIGHT optional 'cause in space no-one can
tell the time.  You probably will need to specify.

Don't put in pagebreaks by hand until you're really, really
finished editing!

It isn't the done thing to hyphenate for formatting purposes.
\centretitle{http://dvc.org.uk/sacrific.txt/}

That was a centred titleover.
\begin{titleover}There's a titleover environment for dialogue-like layout if you're
doing the "Star Wars" thing.\end{titleover}\extslug[All Hail Discordia!]{where to find us}
http://dvc.org.uk/sacrific.txt/screenplay.zip

Use the source, Luke.

\extslug[illumination]{for definitive info on layout}

http://www.oscars.org/nicholl/format.html

%% and this always happens at the end ...
\fadeout

\theend
\end{screenplay}
\end{document}
%% 
%% Hail Eris!  All Hail Discordia!
%%
%% End of file `pkg-example.tex'.
