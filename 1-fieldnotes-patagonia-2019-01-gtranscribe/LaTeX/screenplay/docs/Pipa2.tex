{[}Pipa goes up to get the rope out of the clips{]} {[}Talks of
substitution techniques{]}{[}It's 11:30{]}

12:45 Benji: What does ``the wild'' mean to you?

13:11~PIPA: I think that the wild (savage/salvaje), considering mountain
activities, is what describes Patagonia. In some way Patagonia still is
wild (savage/salvaje) in the sense that there is still very little
information about what one would like do, and climbing in Patagonia is
rebuilding what climbing involved from a long time ago: a lot of
research on the target,

13:38~the climate conditions, which are very variable, information on
the routes, information on the access that have a lot of variation. In
that sense Patagonia involves the wild (savage/salvaje) of climbing,
with a certain degree of accessibility, because they are still some
places that are more wild (savage/salvaje) in Antarctica, but Patagonia
has a nearby town, it's easy to get over,

14:02~but the climbing up there is still wild (savage/salvaje).

Benji: Regarding Patagonia, you've given some attributes, and yet
Patagonia seems to be somewhat unknown inside Argentina, and people from
abroad have impressions. How would you present Patagonia to a local and
foreign public?

14:28 PIPA: I believe that foreigner have more knowledge about what
Patagonia is than Argentineans. For Argentinean, Patagonia has been
represented by the area of Bariloche, that is, North Patagonia. Little
by little Argentineans are getting to know this part of Patagonia. And
regardless of the most well known spots like Chaltén, Calafate, Ushuaia,
there is a lot of parts that are totally wild (savage/salvaje).

15:02 There is all to be discovered. All to be done. Actually over here,
in terms of climbing, there are still mountains that don't have names
because they haven't been climbed, simply two days of walking distance
away. If you would try to find such a situation in other parts of the
world it would imply a much larger effort.

15:26~April: Could he introduce himself, who he is, what he does?

Benji: Name, job, could you also explain your influences to come over
here, whatever that may imply, from your childhood up to how it
developed, and how it changed for you to end up living here.

16:05 PIPA: My name is Juan Manuel Raselli, they call me Pipa. I wasn't
born in Patagonia. I was born in Mar del Plata, Buenos Aires Province. I
think my first contact with nature was there, with my parents, not
actually climbing but in the fact of going out to the mountain, or to
the beach and do camping in places that had some wild into it.

16:34~And I believe that it then where the sense of adventure was born.
And climbing always represented that: adventure. So I began rock
climbing, then I started to going higher into the mountains, to travel
to that. To do some ice climbing, to undertake courses, and try to get a
solid preparation in that, and then I tried to link that, the
sports-side, up to a way of life and being able to create some income
with that.

17:01 and being able to be in permanent contact with nature. And I
started working in the mountain, then I began to prepare as a guide and
I believe that's the way I ended up in Patagonia. Because I began to
come as a sports person, wanting to climb these mountains, and then to
work on them. And now I believe that my sport aims are complete but I
have other aims as a guide: to take other people to learn about places
over here.

17:33 And they're not very easily accessible. And trying to go over
there in a safe way. So, that's why I got prepared as a guide, and I got
the complete level guide {[}GM International{]}, which also gave me the
chance to travel to other places in the world.

Phil: Do you normally teach children to climb? How often?

17:58 PIPA: It is not very common to have children. It is more common to
have families who introduce their children to climbing, and that I did
have. Locally, in Chaltén, there´s a climbing school for children, and
from kindergarten they go out to the mountain. The teachers take them
over to walk, they recognize the climbing places, they know the climbing
places' names and the routes' names.

18:33 And I believe that this will change in the next generation. It
didn't happen to my generation, but I believe that in the next one
there's going to be a very good basis of children with a relationship
with the mountain.

Phil: Is it good for children today to learn how to be outdoors and how
to climb?

18:53 PIPA: {[}Laughter{]} Yes, I absolutely believe that. It's another
way to live, relating to the mountain since you're a child. And all of
these challenges that the mountain puts onto you at any level, from
being able to go through a route up to planning a trip, or an ascension.
There's the trick.

19:21~Phil: If there is one lesson that mountains of the world have
taught you, what is it?

19:32 PIPA: I believe that it would be to live simple. I believe that in
the mountains everything reduces to that. All is so simple that at the
same time it can become complicated, because we're so used to
complicating things that when we have the simple answer we refuse to
accept it. But I think that when you're in the mountain, even more so in
hard situations, in which I've been,~ of life or death: survival and
simplicity.

20:01 April: Earlier, before climbing you explained a difference between
sport climbing and alpine climbing. Would you like to elaborate more?

20:10 PIPA: I like everything. I learned to live with all mountain'
specialties, from skiing, to climbing. I believe that all is so specific
that they're children that would only do sport climbing and they find it
hard to go to the mountain for a walk, because they have the sport
climbing very very easy, by car. But the essence of the mountaineer is
in everything, in being able to combine all the activities.

20:45 April: What is your most memorable climb through Patagonia? More
challenging, or a story about it?

20:58 PIPA: I believe that the climbing that brings the most memories is
the one from Cerro Torre, which I did a couple of times, but when I went
through the West Side making a crossing that has not yet been repeated,
and climbing a mountain that had not been approached before, it was
several days, and I could share the moment with my brother and a friend,
that was my biggest challenge and the one that gave me the highest
satisfaction. 21:25~Sometimes, in climbing the satisfaction doesn't come
from the difficulty level but rather in something that actually
represents something important to yourself. And I believe that that
climb combined the both things.

21:35 Benji: Do you feel, emotionally, anything different about going
with someone as close as a brother? More risky \ldots{}? More positive
in some sense \ldots{}

21:44 PIPA: No. I believe that the important point is to choose your
climbing partner and being able to know him completely, not only in a
technical aspect. But rather all the emotional side, and how to react
right there when your up there. But you can also get that with a friend,
a fellow climber, with whom you share a lot of time climbing together.

22:10 It doesn't have to be your brother. Obviously, I think it's easier
and perhaps more gratifying, but I have had very good feelings with
friends, while climbing.
