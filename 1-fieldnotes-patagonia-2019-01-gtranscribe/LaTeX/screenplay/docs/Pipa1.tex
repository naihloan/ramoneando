\section*{Pipa @ Mountainside / Chaltén {[}2019
01{]}}\label{pipa-mountainside-chaltuxe9n-2019-01}
\addcontentsline{toc}{subsubsection}{Pipa @ Mountainside / Chaltén 2019 01}


01:07 PIPA: All here there are climbing routes. This is pretty high, it
is 120 meters. We call that multi-pitch. Because you can make different
stations. Now we're going away to a shorter one.~~

April: How long have you been climbing?

01:30 25 years probably.

02:45~Benji: Are there any mountains near Mar del Plata?

PIPA: Yes! Near Mar del Plata, where I am from, there are some
mountains, and climbing: very good climbing.

03:12 Now we can see the Fitz Roy range and Torre behind, which is the
little one here. The one with the mushroom on top, that's the hardest
peak in the area, even if the Fitz Roy is higher, Torre is harder. It
has an interesting climbing history.~

April: Have you climbed it?

03:35 PIPA: Yes, I've climbed it. Two times a couple of years ago, 4
years ago.

April: Why did you climb it?

PIPA: {[}Laughter{]} I don't know. There is no answer for that.
{[}pause{]} Why not? Yeah.

April: How long did it take you to climb?

04:05 PIPA: Well, it depends which side you climb. This is the east side
but behind is the west side, which is actually a great adventure
actually to get there because you need to look at the range to get to
the ice field so that takes like two/three days to approach, then
climbing, then coming back the same way. And then you need good weather
window for that.

04:44 One time I climbed it I needed a traverse over all this range,
like climbing and un-climbing a peak behind this one. Climb all the
white peaks in the background and then Torre. So it was something
different. All the range.

April: Have you climbed Fitz Roy as well?

05:08 Yeah. Fitz Roy is a bit easier, uh?

06:12 Just yesterday it was cold, the temperature lowered quite much.
Now it raised a bit. It fluctuates a lot here. You can have snow in
January, and be hot in the same month. Not humid at all, very dry.

07:42 Well stop around here.

April: And so how long have you lived in Chaltén?

08:11 PIPA: Now 13 years. About that.

April: And what made you come over here?

PIPA: Mountains!! {[}Laughter{]} Yeah! First time I got here I just came
for a bike trip from North Patagonia down to Ushuaia. Just came into
here and saw the peaks and said: I want to climb this one, one day.

08:38 And then I came back and I climbed it and then I stayed.

April: Has it changed a lot since you've lived here?

PIPA: Yeah!! Absolutely, yeah. It's more houses, more people living,
more tourists coming, more climbers! It's pretty crowded right now. It
began to become famous after some movies, you know, more information and
a good guide book about climbing stuff. Yeah, and a lot of climbers. All
overcoming here.~

April: And I guess, do you get \ldots{} I made an observation, and there
seems to be a lot of trekkers and climbers, and a lot of people here who
are just to eat. Or just to enjoy the scenery.~ What do you think?

09:25 PIPA: No. More hikers I would say. There is not much to do for
people that don't like to hike. I mean, sometimes they come but they
stay a day and then they leave. But, yeah. I think that those people
stay more in Calafate or even in some other places in Patagonia. So the
ones that get here are more hikers or climbers, yeah.

April: How did you learn how to climb?

10:06 I took courses, like everything. First I started rock climbing.
Because were I was born there are rocks, no ice at all. So I started
rock climbing and then you climb a little bit you want to go higher. So
I went up to somewhere in Patagonia to do a mountain, an easy one. Then
I took an ice climbing course.

10:33 And then I started going higher, and then I started skiing.

Phil: What kind of mountains to you begin with?

PIPA: In Bariloche you have Lanín, which is the easy one.

Phil: That's the easy one? One side of Lanín.

PIPA: Yes, the normal route.

Benji: Because the other one is quite difficult, right?

10:54~PIPA: Sí! The South face is here in Argentina are the hard ones to
climb. All the North faces are easier. Then Tronador in Bariloche is
another option. There are some big ones. Even here. If you want to start
in mountaineering there are some options. Like this peak here, Eléctrico
it's an easy one to start.~

11:57 PIPA: Do you want to start? We can begin putting the gear on.

12:22 First we're going to put the harnesses on.

13:35 April: We have a friend who is an ice climber. We interviewed him
for this project.

13:44 PIPA: Glaciar or waterfall ice?

April: Waterfall!!

15:06 Benji: How much is your feet size?

15:06 PIPA: 41\ldots{} Depends on the type of climbing the shoe size
you're going to use.

15:57 PIPA: Helmet on. It should match the size. It should be tight. You
can adjust it. It shouldn't be extra tight because it can be
uncomfortable.

16:56 April: Have you climbed other places in the world?

16:57 PIPA: I've been in Europe. All over the Alps: France, Austria,
Italy, Germany. In the Dolomites. And then there is like a little town
called Larco?, that is very famous for rock climbing.~

April: Is it different climbing here than in those places?

17:21 PIPA: Like \ldots{} Technically speaking it's about the same. Here
is there a lot less infrastructure so it is like the Alps a hundred
years ago. So the approaches are very long. The forecast is not that
reliable as in the Alps. So that makes for an interesting climbing here,
more challenging. You also don't have either Fitz Roy, nor Torre in the
Alps.

17:50 So these peaks are unique. Specially Torre. Like the West face is
famous all over the world. Because it's all ice climbing, and like the
strong winds that we have here make like mushroom-ice, tastes like
lime-rice? It's very soft ice, so it is very hard to place anchors. The
ice (croups)? came off so it's quite exposed to climb.

18:19 So you need to look for tunnels, that the wind makes into the ice
actually. Somehow you are climbing inside a tunnel, which is crazy.
Crazy landscape. And you go there, inside.

Phil: Two years ago for this project we were in the Dolomites. We were
in Valgardena? I grew up in Italy and I spent all my summers there.

18:57~So I probably have walked some of the trails you've been at. The
rock there is very different than it is here. How do you feel about this
rock, what is unique about this rock?

19:11 PIPA: There is a big difference between the rock that we have here
that is volcanic and it's older, and the other one we have up there in
Cerro Torre and Fitz Roy. That's granite. That's the rock that we are
looking for climbing. This rock is kind of soft, so it's not the best
rock we can find.

19:32 But for training or sport climbing it's good. And in Dolomites,
the formations, the peaks, the needles are very nice but the rock is
also soft. It breaks. It is quite exposed to climb there. But every rock
is interesting. To climb different rocks is like climbing in different
places. So the climbing itself is different. The places where you place
your protection, I mean, how you feel about climbing there, how exposed
you are is different. Every rock, every place is different.

20:14 Phil: When people normally go to mountains they use their eyes,
sense of sight is everything for landscape. When you climb it's the
sense of touch.

20:24 PIPA: Even in the States there's a place in Utah, called Fisher
Towers, it is very soft rock. Even the anchors you place come off. It's
super hard to climb that, but people like that, to get that feeling.

23:40 Well, generally speaking there are two different kinds of climb.
We have here all that is set up, so this is called sport climbing, with
anchors. It means that someone came before us, placed the bolts, and on
top there are two that make a relay station. So it is quite safe in
terms of safety.

24:05~But if we talk about climbing Fitz Roy the situation is completely
different. We won't find any of these. So it is more like an adventure
climbing and we need to place our own protection. So we use this kind of
thing and we could track climbing. There are xxxx? and xxxx?, that you
place in a crack and then they became stuck there and then if you push
you can take them off and bring them back.

24:32 Of course this is not as solid as a bolt but it is what we use for
climbing on Fitz Roy range. So alpine climbing with this stuff and sport
climbing with this stuff here. \ldots{} There are two different types of
climbing. When you start from the very bottom and you need to get to the
top is called leading, it's what I am going to do right now.

25:00 So I need to go up, place some protection with these bolts, then
get to the end and come back down. Once a rope is on the top, we are
going to be climbing what we call top roping. So it means that there is
no potential fall, because the rope is already on top. So you will be
climbing on one side and I will be belaying you on the other side. So
top roping and leading. When you lead you can fall and if you fall you
are going to fall a couple of meters but of course, all the material we
use is prepared for that.

25:32 Like the ropes are dynamic so they are going to absorb the energy
of a potential fall. So, once I set the rope above I need someone that
breaks. That's what we call the belayer. So I need you to belay me.

26:11 So as belay device we are going to use this one. They're are
different types. This is called ATC. So the idea is just to pass the
rope through and by making friction you're gonna stop me eventually if I
fall. So we use the ATC, a karabiner, and we are going to connect this
to your harness.

26:35 So basically, you are going to use one hand on the breaking side,
and one hand on the climbers side. And there are two situations: when
you need to give slack and when you need to take slack. So let's
practice that.

28:48 Now I am going to get to the top and I need the rope to be tight
and you lower me down, little by little.

30:23 If the angle between you and the wall is bigger, if I fall,
\emph{{[}pack{]}} I can bring you against the wall.

30:50 You have to let the rope loose when I go up so I can move freely.
More rope.

33:49 Take! {[}\ldots{}{]} So I can come down.

36:46 {[}Shoe trying{]}

37:12 The hardest part is trusting in all the material, because it's all
about that. When you need to hang over and I need to bring you down,
that can be a little frightening.

April: We did some indoor climbing wall once.

37:40 PIPA: But with rope or just wall.

April: So maybe you remember a little bit.

PIPA: Right now try to focus more on your feet. Where you put your feet.

38:28 We also use this, which is chalk. We put this back behind in case
your hands are sweaty.

40:17 Follow the rope and un-clip once you get to each point.

41:00 You will always be secured to the rope so if you fall it will not
be a big fall for you.

41:29 Kudos {[}Autumn{]}!

42:47 Untie!

45:19 Right hand! Yes!

48:18 Be careful with those plants. They are spiky.

April: So when you're climbing, do you put your hand to the spot where
you want to move your legs next?

49:14 Pipa: Depending on the wall and how steep the wall is, is the
technique you're going to use. These kind of walls are vertical or even
like it's a positive slope. You want to move your feet first. So feet,
two or three steps and then hands. If it's overhanging it is difficult,
it's different. You need to reach with your arms, and you need arm push.
But,

49:43 This kind of wall is all on your feet.

50:03 Let's see Benji now.

50:55 No socks on. Be careful with the insect, it's a tábano. The shoe
should be very tight, toes cramped.

53:01 When you reach the end, to the express, you un-clip the rope.

55:45 Make a deep breath and relax.

57:56 Another breath and check for options, you can reach with your
feet.

59:34 Hardest part all done!!

1:00:29 Now you can release the rope from the clip. And come down.

1:01:17 Now you can come down, I'll do that slowly. No hands! Only feet.

1:02:03 Just hang from the harness. Trust!! And lift your feet flat
against the wall, walking.

1:03:20 Hurray!!

Autumn up!

1:04:18 The shoes are only for when you're going upwards. Otherwise
they're so tight you just remove them. {[}Moving on to other angle of
the rock{]}.

1:06:26 It's important to breathe. Most routes have ideal places to
rest, and then go on.

1:10:05 Handholds there! {[}To Autumn{]} Match your hands. Good!! Just
relax!

1:13:27 Well done!! You can come down now. A: Drop? P: Yes! Walking

1:16:23 Claps!!!

1:17:23~(Pipa cleans the 1st path to build the 2nd one) 1:21:18 (Coming
back down)

1:24:22~Benji 2nd time up (preparation).

1:25:57 PIPA: Use the flat part of shoe to smear against the wall.

1:26:58 Short steps and hips away from the wall, that way you create
friction.

1:28:07 All these wall have rating. The first one was 4+, and this one
is 5+.

April: How high does the rating go?

PIPA: We use the French scale: 9C. It changes all the time.

1:28:30 Two years ago it used to be 9B. So now it's a bit harder.

April: What makes them higher? What changes, type of rock\ldots{}?

PIPA: Type of handholds you have. How negative the wall is. How long the
wall is.

Phil: So if it's a French scale, it depends how much fun you have on
top. {[}Laughter{]}

1:29:02 (Benji ready to go up 2nd time).

1:29:27 April: Benja has Pearl Jam on his back now!! (where the chalk
is)

1:33:15 PIPA: Few places for hands. Lift your legs quite a bit. And
place them wide open.

1:36:11 Adherence step there. 1:38:07 Esa!! 1:38:35 Breath! And finish.

1:39:02 Just another try!! Then it gets easy.

1:40:55 Buena!! Just a short distance. Phil: You're getting paid! Benji:
Enjoying it!! {[}Laughs{]}

1:41:39 Listo!! Joya!! Just hang.

1:45:11~Benji celebrates getting back down and tired.

1:43:19 Where are you running the project? Besides here in Patagonia.

April: Belize, Tazmania, New Zealand, Dolomites, Galapagos, Japan,
Okasara Islanda, all the sites in Canada, one more to go. And then after
here to Island, and most likely Vietnam, and two sites in Africa. Maybe
Tanzania. And Madagascar. Depending on the political climate. Five year
project, three and a half years into it now.

1:45:30 (Autumn up again 2nd time!) 1:49:12 PIPA: Nice!!

1:50:01 I studied under a Bachelor in Environment Management, and I
liked sociology and anthropology. In Tandil, at UNICEN.

1:52:26 Yes!! Excellent!!!

1:55:17 {[}Today I am going to a needle (aguja) called Aguja Guillaumet,
in the range of Fitz Roy.{]}
